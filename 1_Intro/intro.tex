\chapter{Introduction}
\label{chapter:Introduction}
\thispagestyle{myheadings}

\section{Introduction}

%% why dependent types?
Writing correct programs is difficult. While many formal methods approaches
make some errors rare or impossible, they often require programmers
learn additional syntax and semantics. Dependent type systems can
offer a simpler approach. In dependent type systems, proofs and properties
use the same language and meaning already familiar to functional programmers.

While the type systems of mainstream programing languages allows tracking
simple properties, like $\mathtt{7:int}$. Dependent types allow complicated
properties to be assumed and verified, such as a provably correct
sorting function 

\[
sort\,:\,\left(input:List\,\mathbb{N}\right)\rightarrow\Sigma ls:List\,\mathbb{N}.IsSorted\,input\,ls
\]

by providing an appropriate term of that type. From the programmer's
perspective, the function arrow and the implication arrow are the
same. The proof $IsSorted$ is no different then any other term of
a datatype like $List$ or $\mathbb{N}$. 

The power of dependent types has been recognized for decades. Dependent
types form the back bone of several poof system, such as Coq\cite{Coq12},
Lean\cite{10.1007/978-3-030-79876-5_37}, and Agda\cite{norell2007towards}
\todo{not clear these are the best citations, there seems to be little consensus
online about how to cite software}. They have have been proposed as a foundation for mathematics\cite{Martin-Lof-1972,HoTTbook}.
Dependent types are directly used in several programming languages
such as ATS\cite{DependentMLAnapproachtopracticalprogrammingwithdependenttypes}
and Idris\cite{brady2013idris}, while influencing many other programing
languages such as Haskell and Scala\todo{cite the languages/ the dependent features both? neither?}. 

Unfortunately, dependent types have not yet become mainstream in the
software industry. Many of the usability issues with dependent types
can trace their root to the the conservative nature of dependently
typed equality. This thesis illustrates a new way to deal with equality
constraints by delaying them until runtime.

A fragment of the system is proven correct according to a modified
view of type soundness, and several of the proofs have been validated
in Coq \footnote{available at \url{https://github.com/marklemay/dtest-coq}{https://github.com/marklemay/dtest-coq},
most work is due to Qiancheng Fu}. The system has been prototyped\footnote{available at \url{https://github.com/marklemay/dDynamic}{https://github.com/marklemay/dDynamic}}.

\todo[inline]{revise last paragraph}

\subsection{Example}

For example, dependent type systems can prevent an index-out-of-bounds
error when trying to read the first element a list. A version of the
following type checks in virtually all dependent type systems:

\todo[inline]{expand this example? perhaps at the head of a constant vector first?
``and this reasoning can be abstracted under functions'' }

\begin{align*}
\mathtt{Bool} & :*,\\
\mathtt{Nat} & :*,\\
\mathtt{Vec} & :*\rightarrow\mathtt{Nat}\rightarrow*,\\
\mathtt{add} & :\mathtt{Nat}\rightarrow\mathtt{Nat}\rightarrow\mathtt{Nat},\\
\mathtt{rep} & :\left(A:*\right)\rightarrow A\rightarrow\left(x:\mathtt{Nat}\right)\rightarrow\mathtt{Vec\,}A\,x,\\
\mathtt{head} & :\left(A:*\right)\rightarrow\left(x:\mathtt{Nat}\right)\rightarrow\mathtt{Vec}\,A\,\left(\mathtt{add}\,1\,x\right)\rightarrow A
\end{align*}
\[
\vdash\lambda x.\mathtt{head}\,\mathtt{Bool}\,x\,\left(\mathtt{rep}\,\mathtt{Bool}\,\mathtt{true}\,\left(\mathtt{add}\,1\,x\right)\right)\,:\,\mathtt{Nat}\rightarrow\mathtt{Bool}
\]

\todo[inline]{make this a ``code'' example?}

Where $\rightarrow$ is a function and $*$ means that the function
results in a type. $\mathtt{Vec}$ is a list indexed by the type of
element it contains and its length, it is a type that depends on its
length. $\mathtt{rep}$ is a dependent function that produces a list
containing a type with a given length, by repeating its input that
number of times. $\mathtt{head}$ is a dependent function that expects
a list of length $\mathtt{add}\,1\,x$, retuning the first element
of that non-empty list. 

There is no risk that $\mathtt{head}$ inspects an empty list. Luckily
in the example the $\mathtt{\mathtt{rep}\,\mathtt{Bool}\,\mathtt{true}\,\left(\mathtt{add}\,1\,x\right)}$
function will return a list of length $\mathtt{add}\,1\,x$, exactly
the type that is required.

%% This example only scratch the surface of what is possible with dependent types.

Unfortunately, programmers often find dependent type systems difficult
to learn and use. This resistance has limited the ability of dependent
types reach their full potential to help eliminate the bugs that pervade
software systems. One of the deepest underling reasons for this frustration
is the way dependent type systems handle equality.

For example, the following will not type check in any conventional
dependent type system with user defined addition,

\[
\cancel{\vdash}\lambda x.\mathtt{head}\,\mathtt{Bool}\,x\,\left(\mathtt{rep}\,\mathtt{Bool}\,\mathtt{true}\,\left(\mathtt{add}\,x\,1\right)\right)\,:\,\mathtt{Nat}\rightarrow\mathtt{Bool}
\]

While ``obviously'' $1+x=x+1$, in the majority of dependently typed
languages, $\mathtt{add}\,1\,x\equiv\mathtt{add}\,x\,1$ is not a
``definitional'' equality. ``Definitional equality'' is the name
for the conservative approximation of equality used by dependent type
systems for when two types are ``obviously'' the same. This prevents
the use of a term of type $\mathtt{Vec}\,\mathtt{Bool}\,\left(\underline{\mathtt{add}\,1\,x}\right)$
where a term of type $\mathtt{Vec}\,\mathtt{Bool}\,\left(\underline{\mathtt{add}\,x\,1}\right)$
is expected. Usually when dependent type systems encounter situations
like this, they will give an error message and block evaluation until
the ``mistake'' is resolved.

In programming, types are used to avoid bad behavior, for instance
they are often used to avoid ``stuck'' terms. If it is the case
that $\mathtt{add}\,1\,x\,=\,\mathtt{add}\,x\,1$ the program will
never get stuck. However, if there is a mistake in the implementation
of $\mathtt{add}$, the program might get stuck. For instance, if
the $\mathtt{add}$ function incorrectly computes $\mathtt{add}\,8\,1=0$
the above function will ``get stuck'' on the input $8$. 

While the intent and properties of the $\mathtt{add}$ function are
clear to programmers from its name and type, this information is unavailable
to the type system. If the programmer made a mistake in the definition
of addition, such that for some $x$, $\mathtt{add}\,1\,x\,\cancel{=}\,\mathtt{add}\,x\,1$,
the system will not provide hints on which $x$ witnesses this inequality.
Worse, the type system may even disallow experimenting with the $\mathtt{add}$
function until the ``error'' is removed.

Why block programmers when there is a type ``error''? \todo{Awk, in HM it makes sense to block}

There appears to be no reason! Alternatively, we can track unclear
equalities and if the program ``gets stuck'', we are able to stop
the program execution and provide a concrete witness for the inequality.
If that application is encountered at runtime we can give a runtime
error stating $\mathtt{add}\,1\,8=9\,\neq\,0=\mathtt{add}\,8\,1$.
Which is exactly the kind of specific feedback programers want when
correcting code.

\section{A Different Workflow}

This thesis advocates an alternative usage of types. In most types
systems a programmer can't run programs until the type system is convinced
of their correctness \footnote{often requiring a graduate degree and uncommon patients}.
Where this thesis argues ``the programer is always right (until proven
wrong)''. This philosophy will likely go over better with programmers.

More concretely, whenever possible, static errors should be replaced
with 
\begin{itemize}
\item static warnings containing the same information, 
\item and more concrete and clear runtime errors that correspond to one
of the warnings
\end{itemize}
Figure \ref{fig:intro-standard-workflow} illustrates the standard
workflow from the perspective of programers in most typed languages.
Figure \ref{fig:intro-thesis-workflow} shows the workflow that is
explored in this thesis. 

\begin{figure}
\begin{lstlisting}
edit program
 |      ^
 |      | type errors
 v      |
 Type checks
 |
 | no type errors
 v
run program
\end{lstlisting}

\todo[inline]{better graphics}

\caption{Standard Typed Programming Workflow}
\label{fig:intro-standard-workflow}
\end{figure}

\begin{figure}
\begin{lstlisting}
edit program
 |      
 |      
 v
 Elaborates                     ^
 |              |               | runtime error
 | no warnings  | type warnings |
 v              v               |
    run program   ---------------
\end{lstlisting}

\todo[inline]{better graphics}

\caption{Workflow for this Thesis}
\label{fig:intro-thesis-workflow}
\end{figure}

These diagrams make it clear why there is so much pressure for type
errors to be better in dependently typed programming\cite{eremondi2019framework}.
Type errors block programmers from running programs! However complaints
about the type errors are probably better addressed by resolving mismatch
between the expectations of the programmer and the design of the underling
type theory. Better worded error messages are unlikely to bridge this
gap when the type system doubts $x+1=1+x$.

The standard workflow seems sufficient for type systems in many mainstream
typed programing languages. Though there is experimental evidence
that even OCaml can be easier to learn and use with the proposed workflow
\cite{10.1145/2951913.2951915}. In the presence of dependent types
the standard workflow is challenging for both beginners and experts,
making a new approach much more critical. 

\todo[inline]{What is new here? talk about some of the prior dependent type attempts,
always a subset of syntax, or changes the syntax}

By switching to the proposed workflow, type errors become type warnings,
and the programer is free to run their program and experiment, while
still presented with the all the information they would have gotten
from a type error in the form of warnings. If there are no warnings,
the programmer could call their program a proof along the lines of
the Curry-Howard correspondence \todo{cite?} \footnote{In the system presented here the programmer will need to manually
verify termination}. If there is value in a type error it comes from the message itself
and not the hard stop it puts to programming.

The proposed workflow is necessary, since often the type system is
too conservative and the programmer is correct in implicitly asserting
an equality that is not provable within the system. That the programmer
may need to go outside the conservative bounds of definitional equality
has been recognized since the earliest dependent type theories \cite{Martin-Lof-1972}
and difficulties in dependently typed equality have motivated many
research projects \cite{HoTTbook,sjoberg2015programming,cockx2021taming}.
However, these impressive efforts are still only usable by experts,
since they frequently require the programer prove their equalities
explicitly\cite{HoTTbook,sjoberg2015programming}, or add custom rules
into the type system \cite{cockx2021taming}. Further, since program
equivalence is undecidable in general, no system will be able to statically
verify every ``obvious'' equality for arbitrary user defined data
types and functions. In practice, every dependently typed language
has a way to assume equalities, even though these assumptions will
result in computationally bad behavior (the program may ``get stuck'').

The proposed workflow presented in this thesis is justified by: 
\begin{itemize}
\item The strict relation between warnings and runtime errors. A runtime
error will always correspond exactly to a reported warning, always
adding specificity to the the warning that was presented.
\item A form of type soundness holds, programs will never ``get stuck''
unless a concrete counter example to a type assertion is found.
\item Programs that type check against a model type system will not have
warnings, and therefore cannot have errors.
\item Other then warnings and errors the runtime behavior is identical to
a conventional type theory.
\end{itemize}

\subsection{Example}

While the primary benefit of this system is the ability to experiment
more freely with dependent types, while still getting the full feedback
of a dependent type system, it is also possible to encode examples
that would be unfeasible in existing systems. This comes from accepting
warnings that are justified with external mathematical or programatic
intuition, even while being theoretically thorny in dependent type
theory.

For instance, here is part of an interpreter for interpreter for System
F\todo{cite?}\footnote{System F is one of the foundational systems used to study programming
languages. It is possible to fully encode evaluation and proofs into
Agda, but it is difficult if substitution computation happens in a
type. In our system, it is possible to start with the ideal type indexed
encoding and build an interpreter, without proving any properties
of substitution.} that encodes the type of the term at the type level. The step function
asserts type preservation in its function signature,

\begin{figure}
\begin{lstlisting}[basicstyle={\ttfamily\tiny}]
Ctx : * ;
Ctx = Var -> Ty;
 
data Ty : * {
| tv : tVar -> Ty
| arr : Ty -> Ty -> Ty
| forall : Ty -> Ty
};
 
data Term : Ctx -> Ty -> * {
| V : (ctx : (Var -> Ty)) -> (x : Var) ->
 Term ctx (ctx x)
| lam : (ctx : Ctx) ->
 (targ : Ty) -> (tbod : Ty) ->
 Term (ext ctx targ) tbod ->
 Term ctx (arr targ tbod)
| app : (ctx : Ctx) ->
 (arg : Ty) -> (bod : Ty) ->
 Term ctx (arr arg bod) ->
 Term ctx arg ->
 Term ctx bod
| tlam : (ctx : Ctx) ->
 (bod : Ty) ->
 Term ctx bod ->
 Term ctx (forall bod)
| tapp : (ctx : Ctx) ->
 (targ : Ty) -> (tbod : Ty) ->
 Term ctx (forall tbod) ->
 Term (tSubCtx targ ctx) (tSubt targ tbod)
};
 
step : (ctx : Ctx) -> (ty : Ty) -> Term ctx ty ->
 Term ctx ty ;
step ctx ty trm =
case trm <_ => Term ctx ty > {
| (app _ targ tbod (lam _ _ _ bod) a) =>
  sub ctx targ a tbod bod
| x => x
};
\end{lstlisting}

\todo[inline]{clean up, write out sub?}\caption{System F}
\label{fig:ex-sysf}
\end{figure}

It will generate warnings like the following \todo{``It will generate the following warnings'' need to clean up some
bugs here}
\begin{itemize}
\item $\mathtt{tbod}$ in $\mathtt{Term\ (tSubCtx\ targ\ ctx)\ (tSubt\ targ\ \underline{tbod})}$
may have the wrong type 
\end{itemize}
First note that the program has assumed several of the standard properties
of substitution. Formalizing substitution in a dependent type theory
is a substantial\todo{this might be really underselling it, we are talking months of effort}
task\cite{10.1145/3293880.3294101}\todo{cite some of the libs}.
Informally substitution and binding is usually considered obvious
and uninteresting, and little explanation is usually given\footnote{A convention that will be followed in this thesis}.

Second, the type contexts have been encoded as functions. This would
be a reasonable encoding in a mainstream functional language since
it hides the uninteresting lookup information. This encoding would
be unthinkable in other dependently typed languages since equality
over functions is so fraught. Here we can rest on our intuition that
functions that act the same are the same.

Finally it is perfectly possible that is a bug in the code invalidating
one of the assumptions. There are two options for the programmer:
\begin{itemize}
\item reformulate the above code so that there are no warnings, formally
proving all the required properties in the language (this is possible
but would take prohibitive effort with substantial changes)
\item exercise the $step$ function using standard software testing techniques.
If there interpreter does not preserve types, then a concrete counter
example can be found
\end{itemize}
The programmer is free to choose how much effort should go into removing
warnings. But even if the programmer wanted a fully formally correct
interpreter, it would still be wise to test the functions first before
attempting such a proof.

\todo[inline]{For instance, if the following error is introduced, }

\todo[inline]{Then it will be possible to get the runtime error }

\section{Design Decisions}

There are many flavors of dependent types that can be explored, this
thesis attempts to always us the simplest and most programmer friendly
formulations. Specifically, 
\begin{itemize}
\item The type system in this thesis is a \textbf{full spectrum} dependent
type system. The full-spectrum approach is the most uniform approach
to dependent type theory, computation behaves the same at the term
and type level \cite{10.1145/289423.289451,norell2007towards,brady2013idris,sjoberg2012irrelevance}.
This is contrasted with a leveled theory where terms embedded in types
may have different or limited behavior\footnote{this is the approach taken in ATS, and other refinement type systems}.
While the full spectrum approach offers tradeoffs (it is harder to
deal with effects), it seems to be the most predictable from the programmer's
perspective.
\item Data types and pattern matching are essential to practical programming.
While it is theoretically possible to simulate data types via church
encodings, they are too awkward for programmers to work with, and
would complicate the runtime errors this system hopes to deliver.
To provide a better programing experience data types are built into
the system and pattern matching is supported.
\item The theories presented in this thesis will allow unrestricted general
recursion and thus allow non-termination. While there is some dispute
about how essential general recursion is\todo{that mcbride paper? ``Church\textquoteright s Thesis and Functional
Programming'' (2004)}, there is no mainstream general prepose programing language without
it. Allowing nontermination weakens the system when considered as
a logic, (any proposition can be given a nonterminating inhabitant).
This removes any justification for a type universe hierarchy, so our
theories will have type-in-type. Similarly non-strict data definitions
will be allowed.
\item Aside from the non-termination mentioned above, effects will not be
considered. Even though effects seem essential to mainstream programing
they are a very complicated area of active research that will not
be considered here. In this sense the language will be a ``pure''
functional language like Haskell. As in Haskell, effects can be treated
though a functional interface.
\end{itemize}
It is possible to imagine a system where a wide range of properties
are held optimistically and tested at runtime. However the bulk of
this thesis will only deal with equality, since that relation is uniquely
fundamental to dependent type systems. Since computation can appear
at the type level, and types must be checked for equality, dependent
type theories must define what computations they intend to equate
for type checking. It would be premature to deal with any other properties
until equality is dealt with.

\section{Issues}

Inserting runtime equality checks into a dependent types system is
easier said then done.
\begin{itemize}
\item In the presence of Dependent types, equality checks may drift into
types. What does it mean when a term is a list of length $\mathtt{Bool}$?
\item Terms can ``get stuck'' in new way. What happens when an equality
check is used as a function by being applied an argument? What happens
when a check blocks a pattern match?
\item Equality is not decidable at many types even in the empty context.
For instance, functions from $\mathtt{Nat}\rightarrow\mathtt{Nat}$
do not have decidable equality.
\end{itemize}
These problems are solved by extending dependent type theory with
a cast checking operator. This cast operator will ``get stuck''
if their is a discrepancy, and we can show that a program will always
resolve to a value or get stuck in such a way that a counterexample
can be reported.
\begin{itemize}
\item A system is needed to localize casts, these can be generated by extending
a \textbf{bidirectional} typing procedure to insert checks when it
would statically check equality. Without a way to localize errors,
the work in this thesis would be infeasible.
\item Once the casts are inserted, evaluations are possible. Lazy evaluation
means checking only has to happen up to the outermost type constructor,
avoiding issues of undecidability.
\item When an argument is applied to a cast operator, new casts can be computed
using similar logic to that of contracts and monitors\cite{10.1145/581478.581484}.
\item Pattern matching can be extended to support checks, and redirect blame
as needed. 
\end{itemize}

\section{The work in this thesis}

\begin{figure}
\begin{lstlisting}
Surface language (ch. 2,4)
 {syntax {typed syntax {bidirectionally typed syntax}       }           }
                       |                            |               |
elaboration (ch. 3)    |without warnings            | with warnings |
                       |                            |               |
                       vvvvvvvvvvvvvvvvvvvvvvvvvvvvvvvvvvvvvvvvvvvvvv
    {                   cast system (ch. 3)                             }

\end{lstlisting}

\todo[inline]{better graphics}

\caption{Systems in This Thesis}
\label{fig:intro-thesis-workflow-1}
\end{figure}

While apparently a simple idea, the technical details required to
manage checks that delay until runtime in a dependently typed language
is fairly involved. 

\todo[inline]{This should probably be expanded}
\begin{itemize}
\item Chapter 2 describes a dependently typed language intended to model
standard dependent type theories (called the\textbf{ Surface Language}),
proves \textbf{type soundness}, and presents a bidirectional type
checking procedure system intended to model standard type checking. 
\item Chapter 3 describes a dependently typed language with embedded equality
checks, called the \textbf{cast language}. The cast language has it's
own version of type soundness, called \textbf{cast soundness}, which
is proven correct. An Elaboration procedure takes most terms of the
surface syntax into terms in the cast language. Several desirable
properties for elaboration are presented and explored.
\item Chapter 4 reviews how dependent data and pattern matching can be added
to the surface language.
\item Chapter 5 shows how to extend the cast language with dependent data
and pattern matching. 
\item Chapter 6 discusses other ideas related to usability, such as automated
testing and runtime proof search.
\end{itemize}
Versions of the proof of type soundness in Chapter 2, and the cast
soundness in Chapter 3 have been formally proven in Coq. Properties
in chapter 4 and 5 are only conjectured correct.

Those interested in exploring the meta-theory of a ``standard''
dependent type theory can read Chapters 2 and 4 which can serve as
a self contained tutorial.

% \bibliographystyle{alpha}
% \bibliography{/Users/stephaniesavir/thesis/bibliography/dtest}

