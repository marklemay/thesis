\section{Examples}

The surface system is extremely expressive.
Several example surface language constructions can be found in \Fref{surface-examples}.
Turnstile notion is abused slightly so that examples can be indexed by other expressions that obey type rules.
For instance, we can say $refl_{2_{c}:\mathbb{N}_{c}}\ :\ 2_{c}\doteq_{\mathbb{N}_{c}}2_{c}$ since $\mathbb{N}_{c}:\star$ and $2_{c}:\mathbb{N}_{c}$.

\begin{sidewaysfigure}
\begin{tabular}{lllll}
  & $\vdash\perp_{c}$ & $\coloneqq\left(X:\star\right)\rightarrow X$ & $:\star$ & \makecell[l]{Void, ``empty'' type,\\ logical false}\tabularnewline
  & $\vdash Unit_{c}$ & $\coloneqq\left(X:\star\right)\rightarrow X\rightarrow X$ & $:\star$ & Unit, logical true\tabularnewline
  & $\vdash tt_{c}$ & $\coloneqq\lambda-\,x\Rightarrow x$ & $:Unit_{c}$ & \makecell[l]{trivial proposition,\\ polymorphic identity}\tabularnewline
  & $\vdash\mathbb{B}_{c}$ & $\coloneqq\left(X:\star\right)\rightarrow X\rightarrow X\rightarrow X$ & $:\star$ & booleans\tabularnewline
  & $\vdash true_{c}$ & $\coloneqq\lambda-\,then\,-\Rightarrow then$ & $:\mathbb{B}_{c}$ & boolean true\tabularnewline
  & $\vdash false_{c}$ & $\coloneqq\lambda-\,-\,else\Rightarrow else$ & $:\mathbb{B}_{c}$ & boolean false\tabularnewline
$x:\mathbb{B}_{c}$ & $\vdash!_{c}x$ & $\coloneqq x\,\mathbb{B}_{c}\,false_{c}\,true_{c}$ & $:\mathbb{B}_{c}$ & boolean not\tabularnewline
$x:\mathbb{B}_{c},y:\mathbb{B}_{c}$ & $\vdash x\,\&_{c}\,y$ & $\coloneqq x\,\mathbb{B}_{c}\,y\,false_{c}$ & $:\mathbb{B}_{c}$ & boolean and\tabularnewline
  & $\vdash\mathbb{N}_{c}$ & $\coloneqq\left(X:\star\right)\rightarrow(X\rightarrow X)\rightarrow X\rightarrow X$ & $:\star$ & natural numbers\tabularnewline
  & $\vdash0_{c}$ & $\coloneqq\lambda-\,-\,z\Rightarrow z$ & $:\mathbb{N}_{c}$ & \tabularnewline
  & $\vdash1_{c}$ & $\coloneqq\lambda-\,s\,z\Rightarrow s\,z$ & $:\mathbb{N}_{c}$ & \tabularnewline
  & $\vdash2_{c}$ & $\coloneqq\lambda-\,s\,z\Rightarrow s\left(s\,z\right)$ & $:\mathbb{N}_{c}$ & \tabularnewline
  & $\vdash n_{c}$ & $\coloneqq\lambda-\,s\,z\Rightarrow s^{n}\,z$ & $:\mathbb{N}_{c}$ & \tabularnewline
$x:\mathbb{N}_{c},y:\mathbb{N}_{c}$ & $\vdash x+_{c}y$ & $\coloneqq\lambda X\,s\,z\Rightarrow x\,X\,s\,\left(y\,X\,s\,z\right)$ & $:\mathbb{N}_{c}$ & \tabularnewline
$X:\star,Y:\star$ & $\vdash X\times_{c}Y$ & $\coloneqq\left(Z:\star\right)\rightarrow(X\rightarrow Y\rightarrow Z)\rightarrow Z$ & $:\star$ & pair, logical and\tabularnewline
$X:\star,Y:\star$ & $\vdash Either_{c}\,X\,Y$ & $\coloneqq\left(Z:\star\right)\rightarrow(X\rightarrow Z)\rightarrow(Y\rightarrow Z)\rightarrow Z$ & $:\star$ & either, logical or\tabularnewline
$X:\star$ & $\vdash\lnot_{c}X$ & $\coloneqq X\rightarrow\perp_{c}$ & $:\star$ & logical negation\tabularnewline
$x:\mathbb{N}_{c}$ & $\vdash Even_{c}\,x$ & $\coloneqq\mathbb{N}_{c}\,\star\,\left(\lambda x\Rightarrow\lnot_{c}x\right)\,Unit_{c}$ & $:\star$ & $x$ is an even number\tabularnewline
$X:\star,Y:X\rightarrow\star$ & $\vdash\exists_{c}x:X\Rightarrow Y\,x$ & $\coloneqq\left(C:\star\right)\rightarrow\left((x:X)\rightarrow Y\,x\rightarrow C\right)\rightarrow C$ & $:\star$ & \makecell[l]{dependent pair,\\ logical exists}\tabularnewline
$X:\star,x_{1}:X,x_{2}:X$ & $\vdash x_{1}\doteq_{X}x_{2}$ & $\coloneqq\left(C:\left(X\rightarrow\star\right)\right)\rightarrow C\,x_{1}\rightarrow C\,x_{2}$ & $:\star$ & Leibniz equality\tabularnewline
$X:\star,x:X$ & $\vdash refl_{x:X}$ & $\coloneqq\lambda-\,cx\Rightarrow cx$ & $:x\doteq_{X}x$ & reflexivity\tabularnewline
% $X:\star,x_{1}:X,x_{2}:X$ & $\vdash sym_{x_{1},x_{2}:X}$ & $\coloneqq\lambda p\,C\Rightarrow p\left(\lambda x\Rightarrow C\,x\rightarrow C\,x_{1}\right)\,\left(\lambda x\Rightarrow x\right)$ & $:x_{1}\doteq_{X}x_{2}\rightarrow x_{2}\doteq_{X}x_{1}$ & symmetry\tabularnewline
\end{tabular}

\todo[inline]{use tagged union notation}
\todo[inline]{break out eq for a better table}
\todo[inline]{suc, pred as an example of an unpleasant encoding, also citation}
\todo[inline]{trans, cong}
\todo[inline]{list, vec, singleton}

\caption{Example Surface Language Expressions}
\label{fig:surface-examples}
\end{sidewaysfigure}

\subsection{Church encodings}

Data types are expressible using Church encodings, (in the style of System F).
Church encodings embed the elimination principle of a data type into higher order functions.
For instance, boolean data is eliminated against true and false, two tags with no additional data.
This can also be recognized as the familiar if-then-else construct. \todo{awk}
So $\mathbb{B}_{c}$ encodes the possibility of choice between two elements, $true_{c}$ picks the $then$ branch, and $false_{c}$ picks the $else$ branch.

Natural numbers\footnote{called \textbf{church numerals} in this scheme.} are encodable with two tags, zero and successor.
Where the successor tag also contains the result of the preceding number.
So $\mathbb{N}_{c}$ encodes those two choices, $(X\rightarrow X)$ handles the recursive result of the prior number in the successor case, and the $X$ argument specifies how to handle the base case of $0$.
This can be viewed as a simple looping construct with temporary storage.

Parameterized data types such as pairs and the $Either$ type can also be encoded in this scheme.
A pair type can be used in any way the two terms it contains can, so a pair is defined as the curried input to a function.
The $Either$ type is handled if both possibilities are handled, so it is defined as a higher order function that will return an output if both possibilities are handled for input.

\todo{church encoding citation?}

% not necessarily convenient
Church encodings provide a theoretically lightweight way of working with data in a minimal lambda calculus.
However, they are inconvenient.
For instance, the predecessor function on natural numbers is not as simple as its behavior would imply.
To make the system easier for programmers, data types will be added directly in Chapter 4.

\subsection{Proposition encodings}

In general we associate the truth value of a proposition with the inhabitation of a type by a meaningful value.
This meaningful inhabitant corresponds to a proof.
So, $\perp_{c}$, the ``empty'' type, can be interpreted as a false proposition.
While $Unit_{c}$ can be interpreted as a trivially true proposition, since it has only one good inhabitant.

Several of the church encoded data types we have seen can also be interpreted as logical predicates.
For instance, the tuple type can be interpreted as logical $and$.
$X\times_{c}Y$ can be inhabited when both $X$ and $Y$ are inhabited.
The $Either$ type can be interpreted as logical $or$.
$Either_{c}\,X\,Y$ can be inhabited when either $X$ or $Y$ is inhabited.

With dependent types, more interesting logical predicates can be encoded.
For instance, we can characterize when a number is even with $Even_{c}\,x$.
We can show that $2$ is even by showing that $Even_{c}\,2_{c}$ is inhabited with the term $\lambda s\Rightarrow s\,tt_{c}$.

Other predicates are encodable in the style of Calculus of Constructions\cite{10.1016/0890-5401(88)90005-3}.
For instance, we can encode the existential as $\exists_{c}$ as shown in \Fref{surface-examples}.
Then if we want to show $\exists_{c}x:\mathbb{N}_{c}\Rightarrow\,Even_{c}\,x$ we need to find a suitable inhabitant of that type.
$0$ is clearly an even number, so our inhabitant could be $\lambda f\Rightarrow f\,0_{c}\,tt_{c}$.
Note that the existential is equivalent to the tuple if $Y$ does not depend on the value of $X$.

One of the most interesting propositions is the proposition of equality.
$\doteq$ is referred to as \textbf{Leibniz equality} since two terms are equal when they behave the same on all predicates\footnote{
  Originally, Leibniz assumed a metaphysical notion of identification of ``substance''s, not a mathematical notion of equality\cite[Section 9]{Leibniz1686}.
  Over time the principle evolved into the current notion of ``the identification of indiscernibles'', and is referred to as \textbf{Leibniz's law}.}.
  \todo{cite the ency of philosophy}
We can prove $\doteq$ is an equivalence within the system by proving it is reflexive, symmetric, and transitive.
\todo{give example}
% Additionally we can prove congruence.
\todo[inline]{talk more about congruence}
\todo[inline]{footnote on Leibniz equality for alt encoding}

\subsection{Large Eliminations}

\todo[inline]{double check large elimination def.
consistent with the notes % https://github.com/RobertHarper/hott-notes/blob/5339576f55a4b7f5d04734370a5117491c44b1fe/notes_week5.tex#L155 .
would like better explanation}

It is useful for a type to depend specifically on term level data, this is called \textbf{large elimination}.
Large elimination can be simulated with \tit{}.

\begin{tabular}{llll}
  $toLogic$ & $\coloneqq\lambda b\Rightarrow b\,\star\,Unit_{c}\,\perp_{c}$ & $:$ & $\mathbb{B}_{c}\rightarrow\star$\tabularnewline
  $isPos$ & $\coloneqq\lambda n\Rightarrow n\,\star\,(\lambda-\Rightarrow Unit_{c})\,\perp_{c}$ & $:$ & $\mathbb{N}_{c}\rightarrow\star$\tabularnewline
\end{tabular}
  
For instance, $toLogic$ can convert a $\mathbb{B}_{c}$ term into its corresponding logical type, $toLogic\ true_{c}\,\equiv\, Unit_{c}$ while $toLogic\ false_{c}\, \equiv\, \perp_{c}$.
The expression $isPos$ has similar behavior, going to $\perp_{c}$ at $0_{c}$ and $Unit_{c}$ otherwise.

Note that such functions are not possible in the Calculus of Constructions.

\subsection{Inequalities}

Large eliminations can be used to prove inequalities that can be hard or impossible to express in other minimal dependent type theories such as the Calculus of Constructions.
For instance,

\begin{tabular}{lcll}
  $\lambda pr\Rightarrow pr\,\left(\lambda x\Rightarrow x\right)\,\perp_{c}$ & : & $\lnot_{c}\star\doteq_{\star}\perp_{c}$ & \makecell{the type universe is distinct\\ from Logical False}\tabularnewline
  $\lambda pr\Rightarrow pr\,\left(\lambda x\Rightarrow x\right)\,tt_{c}$ & : & $\lnot_{c}Unit_{c}\doteq_{\star}\perp_{c}$ &  \makecell{Logical True is distinct\\ from Logical False}\tabularnewline
  $\lambda pr\Rightarrow pr\,toLogic\,tt_{c}$ & : & $\lnot true_{c}\doteq_{\mathbb{B}_{c}}false_{c}$ &  \makecell{boolean true and false\\ are distinct}\tabularnewline
  $\lambda pr\Rightarrow pr\,isPos\,tt_{c}$ & : & $\lnot1_{c}\doteq_{\mathbb{N}_{c}}0_{c}$ & 1 and 0 are distinct\tabularnewline
  \end{tabular}
  
\todo[inline]{1st not sensible in CC}

\todo[inline]{2 possibly in CC?}

Note that a proof of $\lnot1_{c}\doteq_{\mathbb{N}_{c}}0_{c}$ is not possible in the Calculus of Constructions\cite{10.2307/2274575}\footnote{
  Martin Hofmann excellently motivates the reasoning in the exercises of \cite{hofmann_1997}}.
  % For any encoding of Nats [Guevers?]
\todo[inline]{citation is actually for an MLTT, but if good enough for Hoff good enough for me}

\subsection{Recursion}

Additionally, the syntax of functions build in unrestricted recursion.
Though not always necessary, recursion can be very helpful for writing programs.
For instance, here is (an inefficient) function that calculates Fibonacci numbers.

$\mathsf{fun}\,f\,x\Rightarrow case_{c}\,x\,0_{c}\left(\lambda px\Rightarrow case_{c}\,px\,1_{c}\left(\lambda-\Rightarrow f\left(x-_{c}1\right)+_{c}f\left(x-_{c}2\right)\right)\right)$

Assuming appropriate definitions for $case_{c}$, and subtraction.

Recursion can also be used to simulate induction. 
We won't see much of recursion until Chapter 4, when data types are introduced and larger examples are easier to express.
\todo[inline]{GCD, recursive types?}