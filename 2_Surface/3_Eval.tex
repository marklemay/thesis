\section{\SLang{} Evaluation}
As a programming language we should have some way to evaluate terms in the \slang{}.
We can define a conventional \cbv{} system of reductions on top of the syntax already defined.
\Cbv{} is a popular execution strategy that reflects the prototype implementation.
It works by selecting some expressions as \textbf{value}s, expressions that have been computed enough;
 and a \textbf{reduction} relation that will compute terms.
 
Values are characterized by the sub-grammar in \Fref{surface-value-syntax}.
As usual, functions with any body are values.
Additionally, the type universe ($\star$) is a value, and function types are values.
 
\begin{figure}
\begin{tabular}{lcl}
\multicolumn{3}{l}{values,}\tabularnewline
v & $\Coloneqq$ & $\star$\tabularnewline
 & $|$ & $\left(x:M\right)\rightarrow N$\tabularnewline
 & $|$ & $\mathsf{fun}\,f\,x\Rightarrow m$\tabularnewline
\end{tabular}\caption{\SLang{} Value Syntax}
\label{fig:surface-value-syntax}
\end{figure}
 
For instance, $\mathbb{B}_{c}$  and $true_{c}$ are values. % since the outer most syntax is .
However, $!_{c}x$ is not a value since it is defined as an application to a variable.
 
A \cbv{} relation is defined in \Fref{surface-reduction-step}.
The reductions are standard for a \cbv{} lambda calculus, except that type annotations are only removed from values.
We will write $\rightsquigarrow_{\ast}$ as the transitive reflexive closure of the $\rightsquigarrow$.
 
\begin{figure}
\[
\frac{\,}{\left(\mathsf{fun}\,f\,x\Rightarrow m\right)v\rightsquigarrow m\left[f\coloneqq\mathsf{fun}\,f\,x\Rightarrow m,x\coloneqq v\right]}
\]
\[
\frac{m\rightsquigarrow m'}{m\,n\rightsquigarrow m'\,n}
\]
\[
\frac{n\rightsquigarrow n'}{v\,n\rightsquigarrow v\,n'}
\]
\[
\frac{m\rightsquigarrow m'}{m::M\rightsquigarrow m'::M}
\]
\[
\frac{\,}{v::M\rightsquigarrow v}
\]
\[
\frac{\,}{m\rightsquigarrow_{\ast}m}
\]
\[
\frac{m\rightsquigarrow_{\ast}m'\quad m'\rightsquigarrow m''}{m\rightsquigarrow_{\ast}m''}
\]
 
\caption{\SLang{} \CbV{} Reductions}
\label{fig:surface-reduction-step}
\end{figure}
 
For example, the expression $!_{c}\,\vdash true_{c}$ reduces as follows
\begin{tabular}{ll}
 & $!_{c}\,\vdash true_{c}$ \tabularnewline
 $=$ & $ true_{c}\,\mathbb{B}_{c}\,false_{c}\,true_{c}$ \tabularnewline
 $=$ & $ \left(\lambda-\,then\,-\Rightarrow then\right)\,\mathbb{B}_{c}\,false_{c}\,true_{c}$ \tabularnewline
 $\rightsquigarrow$ & $ \left(\lambda\,then\,-\Rightarrow then\right)\,false_{c}\,true_{c}$ \tabularnewline
 $\rightsquigarrow$ & $ \left(\lambda-\Rightarrow false_{c}\right)\,true_{c}$ \tabularnewline
 $\rightsquigarrow$ & $ false_{c}$ \tabularnewline
\end{tabular}
 
This system of reductions raises the question, what happens if an expression is not a value but also not cannot be reduced?
These terms will be called $\textbf{Stuck}$.
Formally, $m\ \textbf{Stuck}$ if $m$ is not a value and there does not exist $m'$ such that $m \rightsquigarrow m'$. \todo{definition env?}
For instance, $\star\star\star\star\ \textbf{Stuck}$.

Though the \cbv{} system described here is standard, other systems of reduction are also possible.
% The advantage of the system here is that it is deterministic.
We will see a non-deterministic system of reductions in \Fref{surface-reduction}.
Occasionally examples will be easier to demonstrate using a \whnf{} style of reduction, or with custom reduction rules.
\todo{weak in the sense that not everything lands on the exact same syntax}

\todo{say the words "small step"?}