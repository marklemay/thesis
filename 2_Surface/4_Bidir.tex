\section{\Bidir{} \SLang{}}
 
% Annotate all the vars
There are many possible ways to localize the type checking process.
We could ask that all variables be annotated at binders.
This is enticing from a theoretical perspective, since it matches how type contexts are built up.
 
However note that, our proof of $\lnot1_{c}\doteq_{\mathbb{N}_{c}}0_{c}$ will look like:

\[
\lambda pr\underline{:1_{c}\doteq_{\mathbb{N}_{c}}0_{c}}\Rightarrow
\]
\[
\left(\lambda n \underline{:\left(C:\left(\mathbb{N}_{c}\rightarrow\star\right)\right)\rightarrow C\,1_{c}\rightarrow C\,0_{c}}\Rightarrow n\,\star\,(\lambda - \underline{:\star\Rightarrow Unit_{c}})\,\perp_{c}\right)\,tt_{c}
% : \lnot1_{c}\doteq_{\mathbb{N}_{c}}0_{c}
\]
\todo{better newlining}

More than half of the term is type annotations!
Annotating every binding site requires a lot of redundant information.
% Further the theory would need to deal with all the binders evaluating
Luckily there's a better way.
 
\subsection{\Bidir{} Type Checking}
 
\textbf{\Bidir{} type checking} is a popular form of lightweight type inference, which strikes a good compromise between the required type annotations and the simplicity of the procedure, while allowing for localized errors\footnote{
  \cite{christiansen2013bidirectional} is a good tutorial, \cite{10.1145/3450952} is a survey of the technique.
}.
In the usual \bidir{} typing schemes, annotations are only needed at the top-level, or around a function that is directly applied to an argument\footnote{
  More generally when an elimination reduction is possible.
}.
For example $(\lambda x\Rightarrow x+x)7$ would need to be written $\left((\lambda x\Rightarrow x+x)::\mathbb{N}\rightarrow\mathbb{N}\right)7$ to type check \bidir{}ly.
Since programers rarely write functions that are immediately evaluated, this style of type checking usually only needs top-level functions to be annotated\footnote{
  Even in Haskell, with full Hindley-Milner type inference, top level type annotations are encouraged.
}\todo{ref style guide. the point was similarly made in Agda thesis.}.
In fact, almost every example in \Fref{surface-examples} has enough annotations to type check \bidir{}ly without further information.
 
\todo{note popularity}
 
\Bidir{} type checking is accomplished by separating the \ac{TAS} typing judgments into two mutual judgments:
\begin{itemize}
\item \textbf{Type Inference} where type information propagates out of a term, $\overrightarrow{\,:\,}$ in our notation.
\item \textbf{Type Checking} judgments where a term is checked against a type, $\overleftarrow{\,:\,}$ in our notation.
\end{itemize}
This allows typing information to flow from the ``outside in'' for type checking judgments and ``inside out'' for the type inference judgments.
Check mode can be induced by the programmer with a type annotation.
When an inference meets a check, a conversion verifies that the types are definitionally equal.
This has the advantage of precisely limiting where the $\rulename{ty-conv}$ rule can be used, since conversion checking is usually an inefficient part of dependent type checking.
 
This enforced flow of information results in a system that localizes type errors.
If a type was inferred, it was unique, so it can be used freely.
Checking judgments force terms that could have multiple typings in the \ac{TAS} to have at most one type.
\todo{and that localizes, how?}
 
\begin{figure}
\[
\frac{x:M\in\Gamma}{\Gamma\vdash x\overrightarrow{\,:\,}M}
\rulename{\overrightarrow{ty}-var}
\]
\[
\frac{\,}{\Gamma\vdash\star\overrightarrow{\,:\,}\star}
\rulename{\overrightarrow{ty}-\star}
\]
\[
\frac{\Gamma\vdash m\overleftarrow{\,:\,}M\quad\Gamma\vdash M\overleftarrow{\,:\,}\star}{\Gamma\vdash m::M\overrightarrow{\,:\,}M}
\rulename{\overrightarrow{ty}-::}
\]
\[
\frac{\Gamma\vdash M\overleftarrow{\,:\,}\star\quad\Gamma,x:M\vdash N\overleftarrow{\,:\,}\star}{\Gamma\vdash\left(x:M\right)\rightarrow N\overrightarrow{\,:\,}\star}
\rulename{\overrightarrow{ty}-\mathsf{fun}-ty}
\]
\[
\frac{\Gamma\vdash m\overrightarrow{\,:\,}\left(x:N\right)\rightarrow M\quad\Gamma\vdash n\overleftarrow{\,:\,}N}{\Gamma\vdash m\,n\overrightarrow{\,:\,}M\left[x\coloneqq n\right]}
\rulename{\overrightarrow{ty}-\mathsf{fun}-app}
\]
\[
\frac{\Gamma,f:\left(x:N\right)\rightarrow M,x:N\vdash m\overleftarrow{\,:\,}M}{\Gamma\vdash\mathsf{fun}\,f\,x\Rightarrow m\overleftarrow{\,:\,}\left(x:N\right)\rightarrow M}
\rulename{\overleftarrow{ty}-\mathsf{fun}}
\]
\[
\frac{\Gamma\vdash m\overrightarrow{\,:\,}M\quad M\equiv M'}{\Gamma\vdash m\overleftarrow{\,:\,}M'}
\rulename{\overleftarrow{ty}-conv}
\]
 
\caption{\SLang{} \Bidir{} Typing Rules}
\label{fig:surface-bityping-rules}
\end{figure}
 
% review the typing rule
The \slang{} supports \bidir{} type-checking over the syntax with the rules in \Fref{surface-bityping-rules}.
The rules are almost the same as before, except that typing direction is now explicit in the judgment.
 
As mentioned, \bidir{} type checking handles higher order functions very well.
For instance, the expression $\vdash(\lambda x\Rightarrow x\,(\lambda y\Rightarrow y)\,2)\overleftarrow{\,:\,}\left(\left(\mathbb{N}\rightarrow\mathbb{N}\right)\rightarrow\mathbb{N}\rightarrow\mathbb{N}\right)\rightarrow\mathbb{N}$ checks because $\vdash(\lambda y\Rightarrow y)\overleftarrow{\,:\,}\left(\mathbb{N}\rightarrow\mathbb{N}\right)$ and $\vdash2\overleftarrow{\,:\,}\mathbb{N}$.
 
Unlike the undirected judgments of the type assignment system, the inference rule of the \bidir{} system does not associate definitionally equivalent types.
The inference judgment is unique up to syntax!
For example $x:Vec\,3\vdash x\overrightarrow{\,:\,}Vec\,3$, but $x:Vec\,3\, \cancel{\vdash}\, x\overrightarrow{\,:\,}Vec\,\left(1+2\right)$.
This could cause unexpected behavior around function applications.
For instance, if $\Gamma\vdash m\overrightarrow{\,:\,}\mathbb{N}\rightarrow\mathbb{N}$ then $\Gamma\vdash m\,7\overrightarrow{\,:\,}\mathbb{N}$ will infer, but only because the $\rightarrow$ is in the head position of the type $\mathbb{N}\underline{\rightarrow}\mathbb{N}$.
If $\Gamma\vdash m\overrightarrow{\,:\,}\left((\mathbb{N}\rightarrow\mathbb{N})::\star\right)$ then $::$ is in the head position of $(\mathbb{N}\rightarrow\mathbb{N})\underline{::}\star$ and $\Gamma\cancel{\vdash}m\,7\overrightarrow{\,:\,}\mathbb{N}$ will will not infer.
 
A similar issue exists with check rules around function definitions.
For instance, $\vdash\left((\lambda x\Rightarrow x)::\mathbb{N}\rightarrow\mathbb{N}\right)\ \overrightarrow{\,:\,}\mathbb{N}\rightarrow\mathbb{N}$ will infer, but if computation blocks the $\rightarrow$ from being in the head position, inference will be impossible.
As in the expression, $\left((\lambda x\Rightarrow x)::\left((\mathbb{N}\rightarrow\mathbb{N})\underline{::}\star\right)\right)$ which will not infer.
 
For these reasons, many presentations of \bidir{} dependent typing will evaluate the types needed for $\overleftarrow{ty}-\mathsf{fun}$, and $\rulename{\overrightarrow{ty}-\mathsf{fun}-app}$ into \whnf{}\footnote{
  As in \cite{COQUAND1996167}.
}.
With that caveat, this document opts for an unevaluated version of the rules to make some properties easier to present and prove.
Specifically, this will allow us to present a terminating elaboration procedure in \ch{3}\footnote{
 The prototype implementation uses the more conventional \whnf{} check up to some ``time bound'', which avoids the issues above from the programmer's perspective but is more messy in theory.
}.
% the simplest possible presentation of \bidir{} type checking.

\todo[inline]{alternative listed in appendix?}
\todo[inline]{what else does the algorithm infer not listed here}
% \todo[inline]{More about extending the system so constraint solving can happen under a check judgment }
\todo[inline]{Clearly explain why this is needed for the \csys{}, annotating every var is cumbersome, constraint solving is iffy when things may be undecidable}

Though this chapter opts for a simple presentation of \bidir{} type checking, it is possible to take the ideas of \bidir{} typing very far.
More advanced \bidir{} implementations such as Agda\cite{norell2007towards} even perform unification as part of their \bidir{} type checking.
% There is nothing that prevents doning quite a lot more clever work inside the check judgment.
% There will always be ways to make type inference more powerful, at the cost of complexity.

\subsection{The \Bidir{} System is Type Sound}
 
It is possible to prove \bidir{} type systems are type sound directly\cite{nanevski2005dependent}.
But it would be difficult for the system described here since type annotations evaluate away, contradicting a potential preservation lemma.
Alternatively we can show that a \bidir{} typing judgment implies a type assignment system typing judgment.
 
\begin{thm} \Bidir{} implies \ac{TAS}.
 
If $\Gamma\vdash m\overrightarrow{\,:\,}M$ then $\Gamma\vdash m\,:\,M$.
 
If $\Gamma\vdash m\overleftarrow{\,:\,}M$ then $\Gamma\vdash m\,:\,M$.
\end{thm}
 
\begin{proof}
by mutual induction on the \bidir{} typing derivations.
\end{proof}
Therefore the \bidir{} system is also type sound.
 
\subsection{The \ac{TAS} System Is Weakly Annotatable by the \Bidir{} System}
 
In \bidir{} systems, \textbf{annotatability}\footnote{
  Also called \textbf{completeness}.
} is the property that any expression that types in a \ac{TAS} will type in the \bidir{} system with only additional annotations. \todo{does this actually hold?}
% This property doesn't exactly hold for the \bidir{} system presented here.\todo{it actually might}
% For instance, $\vdash\left((\lambda x\Rightarrow x)::\left(\mathbb{N}\rightarrow\mathbb{N}::\star\right)\right)$ type checks in the \ac{TAS} system, but no amount of annotations will make it check in the \bidir{} system. % that is incorrect since you can wedge a cast under the cast
To save space we can instead show that the \bidir{} system does not preclude any computation available in the \ac{TAS}, though annotations may need to be added (or removed).% no need to remove if properly \bidir{}).
We will call this property \textbf{weak annotatability}.
\begin{thm} Weak annotatability.
 
If $\Gamma\vdash m\,:\,M$ then $\Gamma\vdash m'\overleftarrow{\,:\,}M'$, $m\equiv m'$ and $M\equiv M'$.
 
If $\Gamma\vdash m\,:\,M$ then $\Gamma\vdash m'\overrightarrow{\,:\,}M'$, $m\equiv m'$ and $M\equiv M'$.
\end{thm}
 
\begin{proof}
By induction on the typing derivation, adding and removing annotations at each step that are convertible with the original term.
\end{proof}
\todo[inline]{slight changes have been made, double check this}