\chapter{Data in the Surface Language}
\label{chapter:SurfaceData}
\thispagestyle{myheadings}

\todo{more on parameters /indicies}
\todo{flex vars, could probly make a s.r. system if unification is limited}

User defined data is an important part of a realistic programming language.
Programmers need to be able to define concrete types that are meaningful for the problems they are trying to solve.

Dependent data types allow these user defined types, while also unifying many types that are handled as special cases in most mainstream languages.
For instance, ``primitive'' data types like \Nat{} and \Bool{} are degenerate forms of dependent data.
Dependent data can represent mathematical predicates like equality or the evenness of a number.
Dependent data can also be used to preserve invariants, like the length of a list in \Vect{}, or the ``color'' of a node in a red-black-tree.

The encoding scheme for data presented in \ch{2} could handle all of these cases.
However, such an encoding is inconvenient in practice.
Since ``ease of use'' is the overriding concern for the system developed in this thesis, those encoding are an unrealistic way to use data.
% However, the combination of user defined data and dependent types can be subtle.

In this chapter we will show two different ways to add data to the \slang{} and \bidir{} system.
The first, a direct eliminator scheme, is meta-theoretically well behaved but cumbersome for programmers.
The second is based on pattern matching, and is extremely convenient, though its meta-theory is much more difficult.

\section{Data}

%What data defs/ constructors
% The langugae defined in this thesus uses data definitions like those found in systems like Agda and Coq. 

A dependent data type is defined by a type constructor indexed\footnote{
 In more developed systems such as Coq and Agda data types may also have parameters.
 These are indices that apply uniformly over the type and every term construtor.
 For instance, the first argument of \Vect{} is often given by a parameter.
% \begin{lstlisting}[basicstyle={\ttfamily\tiny}]
% data Vec (A : *) : Nat -> * {
% | Nil  : Vec A Z
% | Cons : A -> (x : Nat) -> Vec A x -> Vec A (S x)
% };
% \end{lstlisting}
 Parameters help remove clutter, help inference and erasure, and presumably have parametric properties, in the sense of \cite{10.1145/99370.99404}.
 However parameters are easy to simulate with indexes, so for simplicity we will only deal with \emph{indexed} dependent data.
 Parameters can simulate indexes with a suitable equality type\cite{sjoberg2012irrelevance}, though this is not always possible\cite[Section 8]{bertrand:hal-02896776}.
} by arguments, and a set of data constructors that tag data and refine those arguments.
Several familiar data types are defined in Figure \ref{fig:data-defs}.
For example, the data type of natural numbers is defined with the type constructor $\mathtt{Nat}$ (which has no type arguments), the data constructors $\mathtt{Z}$ which takes no further information and the data constructor $\mathtt{S}$ which is formed with the prior number.
The data type $\mathtt{Vec}$ has two type arguments corresponding to the type contained in the vector and its length; it has two data constructors that allow building an empty vector, or to add an element to the front of an existing vector.

Data defined in this style is simple to build and reason about, since data can only be created from its constructors.
Unfortunately the details of data elimination are a little more involved.

\begin{figure}
\begin{lstlisting}[basicstyle={\ttfamily\small}]
data Void : * {};

data Unit : * {
| tt : Unit
};

data Bool : * {
| True : Bool
| False : Bool
};
 
data Nat : * {
| Z : Nat
| S : Nat -> Nat
};
three : Nat;
three = S (S (S Z)));
-- Syntactic sugar allows 3 = S (S (S Z)))

data Vec : (A : *) -> Nat -> * {
| Nil  : (A : *) -> Vec A Z
| Cons : (A : *) -> A -> (x : Nat)
        -> Vec A x -> Vec A (S x)
};

someBools : Vec Bool 2;
someBools = Cons Bool True 1 (Cons Bool False 0 (Nil Bool));

data Id : (A : *) -> A -> A -> * {
| refl  : (A : *) -> (a : A) -> Id A a a
};

threeEqThree : Id Nat three three;
threeEqThree = refl Nat three;
\end{lstlisting}
% include if notation is used
% -- Syntactic sugar expands list notation,
% -- for example
% -- [True, False]<Bool> =
% -- Cons Bool True 1 (Cons Bool False 1 (Nil Bool))
 \caption{Definitions of Common Data Types}
\label{fig:data-defs}
\end{figure}

\section{Direct Elimination}

How should a program observe data?
Since a term of a given data type can only be created with one of the constructors from its definition, we can completely handle a data expression if each possible constructor is accounted for.
For instance, $\mathtt{Nat}$ has the two constructors $\mathtt{Z}$ and $\mathtt{S}$ (which holds the preceding number), so the expression $\mathsf{case}\,n\,\left\{ |\,Z\Rightarrow Z\,|\,S\,x\Rightarrow x\right\}$ will extract the proceeding number from $n$(or $0$ if $n=0$).
In this light, boolean case elimination corresponds to the if-then-else expression found in many mainstream languages. 

We will need to extend the syntax of $\mathsf{case}$s to support dependent type checking.
Specifically, we will need to add a \textbf{motive} annotation that allows the type checker to compute the output type of the branches if they vary in terms of the input.
For instance, recursively generating a vector of a given length\todo{give explicit example, note the motive}.
We may also want to use some values of the type level argument to calculate the motive, and type the branches.
This will be allowed with additional bindings in the motive and in each branch\todo{give explicit of data eliminations that uses these annotations}.
In general, motive annotations will be treated like the typing annotations in \ch{2}, in that the \ac{TAS} will only allow correct motives in a well typed term, and that the motive will be definitionally irrelevant.

This version of data can be given by extending the surface language syntax in \ch{2}, as in \Fref{data-defs}.
Data Type constructors and Data Term constructors are treated like functions.
This direct eliminator scheme, is roughly similar to how Coq handles data in it's core language.

\begin{figure}
\begin{tabular}{lcll}
\multicolumn{4}{l}{list of $O$, separated with $s$}\tabularnewline
$\overline{sO}$,$\overline{Os}$ & $\Coloneqq$ & $.$ & empty list\tabularnewline
 & $|$ & $sO\overline{sO}$ & extend list\tabularnewline
$\Delta,\varTheta$ & $\Coloneqq$ & $\overline{\left(x:M\right)\rightarrow}$ & telescope\tabularnewline
\multicolumn{4}{l}{data type identifier,}\tabularnewline
$D$ &  &  & \tabularnewline
\multicolumn{4}{l}{data constructor identifier,}\tabularnewline
$d$ &  &  & \tabularnewline
\multicolumn{4}{l}{contexts,}\tabularnewline
$\Gamma$ & $\Coloneqq$ & ... & \tabularnewline
 & $|$ & $\Gamma,\mathsf{data}\,D\,:\,\Delta\rightarrow\star\,\left\{ \overline{|\,d\,:\,\varTheta\rightarrow D\overline{m}}\right\} $ & data def extension\tabularnewline
 & $|$ & $\Gamma,\mathsf{data}\,D\,:\,\Delta\rightarrow\star$ & abstract data extension\tabularnewline
 \multicolumn{4}{l}{expressions,}\tabularnewline
$m,...$ & $\Coloneqq$ & ... & \tabularnewline
 & $|$ & $D$ & type cons.\tabularnewline
 & $|$ & $d$ & data cons.\tabularnewline
 & $|$ & $\mathsf{case}\,\overline{N,}n\,\left\{ \overline{|\overline{x\Rightarrow}(d\,\overline{y})\Rightarrow m}\right\} $ & data elim. without motive\tabularnewline
 & $|$ & $\mathsf{case}\,\overline{N,}n\,\left\langle \overline{x\Rightarrow}\,y:D\,\overline{x}\Rightarrow M\right\rangle \left\{ \overline{|\overline{x\Rightarrow}(d\,\overline{y})\Rightarrow m}\right\} $ & data elim. with motive\tabularnewline
\multicolumn{4}{l}{values,}\tabularnewline
$v$ & $\Coloneqq$ & ... & \tabularnewline
 & $|$ & $D\,\overline{v}$ & \tabularnewline
 & $|$ & $d\,\overline{v}$ & \tabularnewline
\end{tabular}

\caption{Surface Language (Direct Eliminator) Data}
\label{fig:surface-data-min}
\end{figure}

\todo{abreviations}
% \begin{figure}
% \begin{tabular}{llllll}
%  & syntax & written & when & for example & is written\tabularnewline
% leading separator & $sO\overline{sO}$ & $O\overline{sO}$ & clear from context & $,1,2,3$ & $1,2,3$\tabularnewline
% trailing separator & $\overline{sO}$ & $\overline{sO}s$ & clarifies intent & $\left(x\rightarrow y\rightarrow z\right)Id\,x\,y\,z$ & $x\rightarrow y\rightarrow z\rightarrow Id\,x\,y\,z$\tabularnewline
% non dependent telescope binder & $\left(x:M\right)\rightarrow$ & $M\rightarrow$ & $x$is not intended to bind  & $\left(x:Nat\right)\rightarrow\left(y:IsEven\,x\right)\rightarrow Nat$ & $\left(x:Nat\right)\rightarrow IsEven\,x\rightarrow Nat$\tabularnewline
% repeated application & $m\,n_{0}\,n_{1}\,...$ & $m\,\overline{n}$ & $\overline{n}=n_{0}\,n_{1}\,...$ &  & \tabularnewline
% \end{tabular}

% \todo[inline]{review if there are more abbreviations around?}

% \caption{Surface Language Abbreviations}
% \label{fig:surface-pre-syntax-data-abrev-1}
% \end{figure}

\todo[inline]{motive should not need to insist on the type info of the binder? grey out? Grey out things that are surface syntax but not needed for theory?}

The case eliminator first takes the explicit type arguments, followed by a \textbf{scrutinee}\footnote{also called a \textbf{discriminee}} of correct type.
Then optionally a motive that characterizes the output type of each branch with all the type arguments and scrutinee abstracted and in scope.
For instance, this case expression checks if a vector $x$ is empty,

\[
x:Vec\,\mathbb{B}\,1\vdash\mathsf{case}\,\mathbb{B},1,x\,\left\langle y\Rightarrow z\Rightarrow s:Vec\,y\,z\Rightarrow\mathbb{B}\right\rangle \left\{ |y\Rightarrow z\Rightarrow Nil\,-\Rightarrow True\,|\,y\Rightarrow z\Rightarrow Cons\,----\Rightarrow False\right\} 
\]

We include a little more syntax then is strictly necessary, since the $\mathbb{B},1,$ list could be inferred and the $y\Rightarrow z\Rightarrow$ binders are not needed in the branch.
This slightly verbose case eliminator syntax is designed to be forward compatible with the pattern matching system defined in the rest of this Chapter. 

Additionally we define telescopes, which generalize zero or more typed bindings\todo{Cite De Bruijn for telescopes}.
This allows a much cleaner definition of data then is otherwise possible.
Also we define syntactic lists that allow zero or more pieces of syntax.
Expressions in a list can be used to generalize dependent pairs, and can be type checked against a telescope.
For instance, the list $\mathtt{Nat},2,2,refl\,\mathtt{Nat}\,2$ type checks against $\left(X:\star\right)\rightarrow\left(y:X\right)\rightarrow\left(z:X\right)\rightarrow\left(-:Id\ X\,y\,z\right)$.
This becomes helpful in several situations, but especially when we need work with the listed arguments of the data type constructor.
We will allow several syntactic puns, such as treating telescopes as prefixes for function types. For instance, 
  if $\Delta=\left(y:\mathtt{Nat}\right)\rightarrow\left(z:\mathtt{Nat}\right)\rightarrow\left(-:Id\ \mathtt{Nat}\,y\,z\right)$
  then writing $f:\Delta\rightarrow\mathtt{Nat}$
  will be short hand for $f:\left(y:\mathtt{Nat}\right)\rightarrow\left(z:\mathtt{Nat}\right)\rightarrow Id\ \mathtt{Nat}\,y\,z\rightarrow\mathtt{Nat}$.
\todo{move this down to the typing section?}

In the presence of general recursion case elimination is powerful.
Well-founded recursion can be used to make structurally inductive computations that can be interpreted as proofs.\todo{say with rep example}.

Adding data allows for two additional sources of bad behavior.
Incomplete matches, and nontermination from non-strictly positive data. 

\subsubsection{Incomplete Eliminations}

Consider the match 

\[
x:\mathbb{N}\vdash\mathsf{case}\,x\,\left\langle s:\mathbb{N}\Rightarrow\mathbb{B}\right\rangle \left\{ |S-\Rightarrow True\right\} 
\]

This match will ``get stuck'' if $0$ is substituted for $x$.
Recall that the key theorem of the surface language is type soundness, ``well typed terms don't get stuck''.
Since verifying every constructor has a branch is relatively easy, the surface language \ac{TAS} will require every constructor to be matched in order to type check type check with direct elimination.
This is in contrast to most programming languages, which do allow incomplete patterns, though usually a warning is given, and a runtime error is raised if the scrutinee cannot be matched.

% Since this is a "well behaved" failure

This thesis already has a system for handling warnings and runtime errors through the cast language.
When we get to the cast language data in \ch{5}, we will allow non-exhaustive data to be reported as a warning and that will allow ``unmatched'' errors to be observed at runtime.

For similar reasons, in the direct eliminator scheme, we will insist that each constructor is matched at most once, so there is no ambiguity for how an case eliminated when using direct eliminations.

\subsubsection{(non-)Strict Positivity}

A more subtle concern is posed by data definitions that are not strictly positive.
Consider the following definition, \todo{think this can be simplified}

\begin{lstlisting}[basicstyle={\ttfamily\small}]
data Bad : * {
| C : (Bad -> Bad) -> Bad
};

selfApply : Bad -> Bad;
selfApply = \ b =>
  case b {
    | C f => f b
  };

loop : Bad;
loop = selfApply (C selfApply)
\end{lstlisting}

The $\mathtt{C}$ constructor in the definitions of $\mathtt{Bad}$ has a self reference in a negative position, $(\mathtt{Bad}\rightarrow\underline{\mathtt{Bad}})\rightarrow\mathtt{Bad}$. 

Non-strictly positive data definitions can cause non-termination, independent of the two other sources of non-termination already considered (general recursion and type-in-type).
Dependent type systems usually require a strictness check on data definitions to avoid this possibility.
However, this would disallow some useful constructions like higher order abstract syntax.
Since non-termination is already allowed in the surface \ac{TAS}, we will not restrict the surface language to strictly positive data.

\todo{who cam up with this first? Martin Lof?}

% co-inductive uses of data

\subsection{Type assignment System}

Before the typing rules for data can be considered, first some meta rules must be presented that will allow the simultaneous type-checking of lists and telescopes.
These rules are listed in \ref{fig:surface-data-meta-ty}, and are standard.
Telescopes are $\mathbf{ok}$ when they extend the context in an $\mathbf{ok}$ way.
Lists of expressions can be said to have the type of the telescope if every expression in the list types successively.

\begin{figure}
\[
\frac{\Gamma\,\mathbf{ok}}{\Gamma\vdash.\,\mathbf{ok}}\operatorname{\mathbf{ok}-Tel-empty}
\]

\[
\frac{\Gamma\vdash M:\star\quad\Gamma,x:M\vdash\Delta\,\mathbf{ok}}{\Gamma\vdash\left(x:M\right)\rightarrow\Delta\,\mathbf{ok}}\operatorname{\mathbf{ok}-Tel-ext}
\]

\[
\frac{\Gamma\,\mathbf{ok}}{\Gamma\vdash,:.}\operatorname{ty-ls-empty}
\]

\[
\frac{\Gamma,x:M\vdash\Delta\quad\Gamma\vdash m:M\quad\Gamma\vdash\overline{n,}\left[x\coloneqq m\right]:\Delta\left[x\coloneqq m\right]}{\Gamma\vdash m\overline{,n}\,:\,\left(x:M\right)\rightarrow\Delta}\operatorname{ty-ls-ext}
\]

\caption{Meta rules}
\label{fig:surface-data-meta-ty}
\end{figure}

Data definitions can be added to contexts if all of their constituents are well typed and \textbf{ok}.
The rules are listed in Figure \ref{fig:surface-data-ok}.
The $\operatorname{\mathbf{ok}-abs-data}$ rule allows data to be considered abstractly if it is formed with a plausible telescope.
$\operatorname{\mathbf{ok}-data}$ checks a full data definition with an abstract reference to a data definition in context, which allows recursive data definitions such as $\mathtt{Nat}$ which needs $\mathtt{Nat}$ to be in scope to define the $\mathtt{S}$ constructor.
This thesis does not formalize a syntax that adds data to context, though a very simple module system has been implemented in the prototype.
It is taken for granted that any well formed data in context is fine.\todo{awk}
This presentation of data definitions largely follows \cite{sjoberg2012irrelevance}. 

\begin{figure}
\[
\frac{\Gamma\vdash\Delta\,\mathbf{ok}}{\Gamma\vdash\mathsf{data}\,D\,\Delta\,\mathbf{ok}}\operatorname{\mathbf{ok}-abs-data}
\]

\[
\frac{\Gamma\vdash\mathsf{data}\,D\,\Delta\,\mathbf{ok}}{\Gamma,\mathsf{data}\,D\,\Delta\,\mathbf{ok}}\operatorname{\mathbf{ok}-data-ext}
\]
\[
\frac{\Gamma\vdash\mathsf{data}\,D\,\Delta\,\mathbf{ok}\quad\forall d.\Gamma,\mathsf{data}\,D\,\Delta\vdash\varTheta_{d}\,\mathbf{ok}\quad\forall d.\:\Gamma,\mathsf{data}\,D\,\Delta,\varTheta_{d}\vdash\overline{m}_{d}:\Delta}{\Gamma\vdash\mathsf{data}\,D\,:\,\Delta\left\{ \overline{|\,d\,:\,\varTheta_{d}\rightarrow D\overline{m}_{d}}\right\} \,\mathbf{ok}}\operatorname{\mathbf{ok}-data}
\]

\[
\frac{\Gamma\vdash\mathsf{data}\,D\,:\,\Delta\left\{ \overline{|\,d\,:\,\varTheta\rightarrow D\overline{m}}\right\} \,\mathbf{ok}}{\Gamma,\mathsf{data}\,D\,:\,\Delta\left\{ \overline{|\,d\,:\,\varTheta\rightarrow D\overline{m}}\right\} \,\mathbf{ok}}\operatorname{\mathbf{ok}-data-ext}
\]

\caption{Surface Language Data ok}
\label{fig:surface-data-ok}
\end{figure}

The type assignment system with direct elimination must be extended with the rules in \ref{fig:surface-data-ty}.
These rules make use of several convenient shorthands: $\mathsf{data}\,D\,\Delta\in\Gamma$ and $d\,:\,\varTheta\rightarrow D\overline{m}\in\Gamma$ extract the type constructor definitions and data constructor definitions from the context respectively.
$\operatorname{ty-TCon}$ and $\operatorname{ty-Con}$ allow type and data constructors to be used as functions of appropriate type.
The $\operatorname{ty-}\mathsf{case}<>$ rule types a case expression by ensuring that the correct data definition for $D$ is in context, the scrutinee $n$ has the correct type, the motive $M$ is well formed under the type arguments and the scrutinee, % The motive requires checking in the case of empty data
  finally every data constructor is verified to have a corresponding branch.
$\operatorname{ty-}\mathsf{case}$ allows for the same typing logic, but does not require the motive be annotated in syntax.
In both rules we allow telescopes to rename their variables with the shorthand $\overline{x}:\Delta$.

\begin{figure}
\[
\frac{\Gamma\,\mathbf{ok}\quad\mathsf{data}\,D\,\Delta\in\Gamma}{\Gamma\vdash D\,:\,\Delta\rightarrow\star}\operatorname{ty-TCon}
\]

\[
\frac{\Gamma\,\mathbf{ok}\quad d\,:\,\varTheta\rightarrow D\overline{m}\in\Gamma}{\Gamma\vdash d\,:\,\varTheta\rightarrow D\overline{m}}\operatorname{ty-Con}
\]

\[
\frac{\begin{array}{c}
\mathsf{data}\,D\,\Delta\left\{ ...\right\} \in\Gamma\\
\Gamma\vdash n:D\overline{N}\\
\Gamma,\overline{x}:\Delta,z:D\,\overline{x}\vdash M:\star\\
\forall\:d\,:\,\varTheta\rightarrow D\overline{o}\in\Gamma.\\
\Gamma,\overline{y}_{d}:\varTheta\vdash m_{d}\left[\overline{x}\coloneqq\overline{o}\left[\varTheta\coloneqq\overline{y}_{d}\right]\right]:M\left[\overline{x}\coloneqq\overline{o}\left[\varTheta\coloneqq\overline{y}_{d}\right],z\coloneqq d\,\overline{y}_{d}\right]\\
\mathrm{No\ duplicate\ branches}
\end{array}}{\begin{array}{c}
\Gamma\vdash\mathsf{case}\,\overline{N,}n\,\left\langle \overline{x\Rightarrow}z:D\,\overline{x}\Rightarrow M\right\rangle \left\{ \overline{|\,\overline{x\Rightarrow}\,d\overline{y}_{d}\,\Rightarrow m_{d}}\right\} \\
:M\left[\overline{x}\coloneqq\overline{N},z\coloneqq n\right]
\end{array}}\operatorname{ty-}\mathsf{case}<>
\]

\[
\frac{\begin{array}{c}
\mathsf{data}\,D\,\Delta\left\{ ...\right\} \in\Gamma\\
\Gamma\vdash n:D\overline{N}\\
\Gamma,\overline{x}:\Delta,z:D\,\overline{x}\vdash M:\star\\
\forall\:d\,:\,\varTheta\rightarrow D\overline{o}\in\Gamma.\\
\Gamma,\overline{y}_{d}:\varTheta\vdash m_{d}\left[\overline{x}\coloneqq\overline{o}\left[\varTheta\coloneqq\overline{y}_{d}\right]\right]:M\left[\overline{x}\coloneqq\overline{o}\left[\varTheta\coloneqq\overline{y}_{d}\right],z\coloneqq d\,\overline{y}_{d}\right]
\end{array}}{\begin{array}{c}
\Gamma\vdash\mathsf{case}\,\overline{N,}n\,\left\{ \overline{|\,\overline{x\Rightarrow}\,d\overline{y}_{d}\,\Rightarrow m_{d}}\right\} \\
:M\left[\overline{x}\coloneqq\overline{N},z\coloneqq n\right]
\end{array}}\operatorname{ty-}\mathsf{case}
\]

\caption{Surface Language Data Typing}
\label{fig:surface-data-ty}
\end{figure}

\todo[inline]{suspect this also hinges on regularity, which should be addressed more directly}

Extensions to the parallel reduction rules are listed in Figure \ref{fig:surface-data-red}.
They follow the scheme of parallel reductions laid out in \ch{2}.
The $\textrm{\ensuremath{\Rrightarrow}-\ensuremath{\mathsf{case}}-red}$ rule\footnote{Also called $\iota$, or Iota reduction}\todo{mention beta in footnote in CH2} reduces a case expression by choosing the appropriate branch.
The $\textrm{\ensuremath{\Rrightarrow}-\ensuremath{\mathsf{case}}<>-red}$ rule removes the motive annotation, much like the annotation rule in \ch{2}.
The rules $\textrm{\ensuremath{\Rrightarrow}-\ensuremath{\mathsf{case}}<>}$, $\textrm{\ensuremath{\Rrightarrow}-}D$, and $\textrm{\ensuremath{\Rrightarrow}-}d$ keep the $\Rrightarrow$ relation reflexive.
The reduction relation is be generalized to lists in the expected way.

\begin{figure}
\[
\frac{\begin{array}{c}
\overline{N}\Rrightarrow\overline{N'}\quad\overline{m}\Rrightarrow\overline{m'}\\
\exists\overline{x\Rightarrow}(d\,\overline{y}_{d})\Rightarrow m_{d}\in\left\{ \overline{|\,\overline{x\Rightarrow}(d'\,\overline{y}_{d'})\Rightarrow m_{d'}}\right\} \\
m_{d}\Rrightarrow m_{d}'
\end{array}}{\mathsf{case}\,\overline{N,}\,d\overline{m}\,\left\{ \overline{|\,\overline{x\Rightarrow}(d'\,\overline{y}_{d'})\Rightarrow m_{d'}}\right\} \Rrightarrow m_{d}\left[\overline{x}\coloneqq\overline{N'},\overline{y}_{d}\coloneqq\overline{m'}\right]}\,\textrm{\ensuremath{\Rrightarrow}-\ensuremath{\mathsf{case}}-red}
\]

\[
\frac{\begin{array}{c}
\overline{N}\Rrightarrow\overline{N'}\quad m\Rrightarrow m'\\
\forall\overline{x\Rightarrow}(d\,\overline{y}_{d})\Rightarrow m_{d}\in\left\{ \overline{|\,\overline{\Rightarrow x}\Rightarrow(d'\,\overline{y}_{d'})\Rightarrow m_{d'}}\right\} .\:m_{d}\Rrightarrow m_{d}'
\end{array}}{\mathsf{case}\,\overline{N,}\,m\,\left\langle ...\right\rangle \left\{ \overline{|\,\overline{\Rightarrow x}\Rightarrow(d'\,\overline{y}_{d'})\Rightarrow m_{d'}}\right\} \Rrightarrow\mathsf{case}\,\overline{N',}\,m'\left\{ \overline{|\,\overline{\Rightarrow x}\Rightarrow(d'\,\overline{y}_{d'})\Rightarrow m_{d'}'}\right\} }\,\textrm{\ensuremath{\Rrightarrow}-\ensuremath{\mathsf{case}}<>-red}
\]

\[
\frac{\begin{array}{c}
\overline{N}\Rrightarrow\overline{N'}\quad m\Rrightarrow m'\\
M\Rrightarrow M'\\
\forall\overline{x\Rightarrow}(d'\,\overline{y}_{d'})\Rightarrow m_{d'}\in\left\{ \overline{|\,\overline{\Rightarrow x}\Rightarrow(d\,\overline{y}_{d})\Rightarrow m_{d}}\right\} .\:m_{d'}\Rrightarrow m'_{d'}
\end{array}}{\begin{array}{c}
\mathsf{case}\,\overline{N,}\,m\,\left\langle \overline{x\Rightarrow}z:D\,\overline{x}\Rightarrow M\right\rangle \left\{ \overline{|\,\overline{\Rightarrow x}\Rightarrow(d\,\overline{y}_{d})\Rightarrow m_{d}}\right\} \Rrightarrow\\
\mathsf{case}\,\overline{N,}\,m'\,\left\langle \overline{x\Rightarrow}z:D\,\overline{x}\Rightarrow M'\right\rangle \left\{ \overline{|\,\overline{\Rightarrow x}\Rightarrow(d\,\overline{y}_{d})\Rightarrow m'_{d}}\right\} 
\end{array}}\,\textrm{\ensuremath{\Rrightarrow}-\ensuremath{\mathsf{case}}<>}
\]

\[
\frac{\,}{D\Rrightarrow D}\,\textrm{\ensuremath{\Rrightarrow}-}D
\]
\[
\frac{\,}{d\Rrightarrow d}\,\textrm{\ensuremath{\Rrightarrow}-}d
\]

\caption{Surface Language Data Reduction}
\label{fig:surface-data-red}
\end{figure}

We are now in a position to select a sub relation of $\Rrightarrow$ reductions that will be used to characterize call-by-value evaluation.
This relation could be used used to prove type safety, and is close to the reduction used in the implementation.
The rules are listed in \Fref{surface-data-cbv}.\todo[inline]{extend cbv over lists}

\begin{figure}
\[
\frac{\,}{\mathsf{case}\,\overline{N,}\,n\,\left\langle ...\right\rangle \left\{ \overline{|\,\overline{\Rightarrow x}\Rightarrow(d'\,\overline{y}_{d'})\Rightarrow m_{d'}}\right\} \rightsquigarrow\mathsf{case}\,\overline{N,}\,n\left\{ \overline{|\,\overline{\Rightarrow x}\Rightarrow(d'\,\overline{y}_{d'})\Rightarrow m_{d'}}\right\} }\textrm{\ensuremath{\rightsquigarrow}-\ensuremath{\mathsf{case}}<>}
\]

\[
\frac{\exists\overline{x\Rightarrow}(d\,\overline{y}_{d})\Rightarrow m_{d}\in\left\{ \overline{|\,\overline{x\Rightarrow}(d'\,\overline{y}_{d'})\Rightarrow m_{d'}}\right\} }{\mathsf{case}\,\overline{V,}\,d\overline{v}\,\left\{ \overline{|\,\overline{x\Rightarrow}(d'\,\overline{y}_{d'})\Rightarrow m_{d'}}\right\} \rightsquigarrow m_{d}\left[\overline{x}\coloneqq\overline{V},\overline{y}_{d}\coloneqq\overline{v}\right]}\,\textrm{\ensuremath{\rightsquigarrow}-\ensuremath{\mathsf{case}}-red}
\]

\[
\frac{\overline{N}\rightsquigarrow\overline{N'}}{\mathsf{case}\,\overline{N,}\,n\,\left\{ \overline{|\,\overline{x\Rightarrow}(d'\,\overline{y}_{d'})\Rightarrow m_{d'}}\right\} \rightsquigarrow\mathsf{case}\,\overline{N',}\,n\,\left\{ \overline{|\,\overline{x\Rightarrow}(d'\,\overline{y}_{d'})\Rightarrow m_{d'}}\right\} }\,
\]

\[
\frac{n\rightsquigarrow n'}{\mathsf{case}\,\overline{V,}\,n\,\left\{ \overline{|\,\overline{x\Rightarrow}(d'\,\overline{y}_{d'})\Rightarrow m_{d'}}\right\} \rightsquigarrow\mathsf{case}\,\overline{V,}\,n'\,\left\{ \overline{|\,\overline{x\Rightarrow}(d'\,\overline{y}_{d'})\Rightarrow m_{d'}}\right\} }\,
\]

\caption{Surface Language Data CBV}
\label{fig:surface-data-cbv}
\end{figure}

\todo[inline]{extend step over lists}

Finally we characterize what it means for a context to be empty in the presence of data in \Fref{surface-data-empty}.

\begin{figure}
\[
\frac{\ }{\lozenge\,\mathbf{Empty}}\operatorname{Empty-ctx}
\]

\[
\frac{\Gamma\,\mathbf{Empty}\quad\Gamma\vdash\mathsf{data}\,D\,:\,\Delta\left\{ \overline{|\,d\,:\,\varTheta_{d}\rightarrow D\overline{m}_{d}}\right\} \,\mathbf{ok}}{\Gamma,\mathsf{data}\,D\,:\,\Delta\left\{ \overline{|\,d\,:\,\varTheta_{d}\rightarrow D\overline{m}_{d}}\right\} \,\mathbf{Empty}}\operatorname{Empty-ctx}
\]

\caption{Surface Language Empty}
\label{fig:surface-data-empty}
\end{figure}

While a system with a similar presentation has proven type soundness in \cite{sjoberg2012irrelevance}\todo{switch to vilhelms thesis?}, we will not prove the type soundness of the system here.
For clarity we will list the important properties as conjectures.
\begin{conjecture}
The surface language extended with data and elimination preserves types over reduction.
\end{conjecture}

% ...
\begin{conjecture}
The surface language extended with data and elimination has progress
if $\Gamma\,\mathbf{Empty}$, \textup{$\Gamma\vdash m:M$}, then $m$
is a value, or $m\rightsquigarrow m'$ .
\end{conjecture}

% ...
\begin{conjecture}
The surface language extended with data and elimination is type sound.
\end{conjecture}


\subsection{Bidirectional Type Checking}

A bidirectional type checking procedure exists for the type assignment rules listed above.
An outline of these rules is in \Fref{surface-data-empty}.
As noted in \cite{10.1145/3450952}, the bidirectional rules around data are open to some interpretation.
The dependent case simplifies these questions since only a few rules are sensable.

The type of data constructors and type constructors can always be inferred.
If the motive does not depend on the scrutinee or type arguments, it can be used to check against the type of the branches.
An unmotivated $\mathsf{case}$ will be type checked.
A $\mathsf{case}$ with a motive will have its type inferred.
\todo{actually more complicated then that}

\begin{figure}
\[
\frac{\mathsf{data}\,D\,\Delta\in\Gamma}{\Gamma\vdash D\overrightarrow{\,:\,}\Delta\rightarrow*}\operatorname{\overrightarrow{ty}-TCon}
\]

\[
\frac{d\,:\,\varTheta\rightarrow D\overline{m}\in\Gamma}{\Gamma\vdash d\overrightarrow{\,:\,}\varTheta\rightarrow D\overline{m}}\operatorname{\overrightarrow{ty}-Con}
\]

\[
\frac{\begin{array}{c}
\mathsf{data}\,D\,\Delta\left\{ ...\right\} \in\Gamma\\
\Gamma\vdash\overline{N}\overleftarrow{\,:\,}\Delta\quad\Gamma\vdash n\overleftarrow{\,:\,}D\overline{N}\\
\Gamma,\overline{x}:\Delta,z:D\,\overline{x}\vdash M\overleftarrow{\,:\,}\star\\
\forall\:d\,:\,\varTheta\rightarrow D\overline{o}\in\Gamma.\quad\Gamma,\overline{y}_{d}:\varTheta\vdash m_{d}\left[\overline{x}\coloneqq\overline{o}\right]\overleftarrow{\,:\,}M\left[\overline{x}\coloneqq\overline{o},z\coloneqq d\,\overline{y}_{d}\right]
\end{array}}{\begin{array}{c}
\Gamma\vdash\mathsf{case}\,\overline{,N},n\,\left\langle \overline{x\Rightarrow}z:D\,\overline{x}\Rightarrow M\right\rangle \left\{ \overline{|\,\overline{x\Rightarrow}(d\,\overline{y}_{d})\Rightarrow m_{d}}\right\} \\
\overrightarrow{\,:\,}M\left[\overline{x}\coloneqq\overline{N},z\coloneqq n\right]
\end{array}}\operatorname{\overrightarrow{ty}-}\mathsf{case}<>
\]

\[
\frac{\begin{array}{c}
\mathsf{data}\,D\,\Delta\left\{ ...\right\} \in\Gamma\\
\Gamma\vdash\overline{N}\overleftarrow{\,:\,}\Delta\quad\Gamma\vdash n\overrightarrow{\,:\,}D\overline{N}\\
\Gamma\vdash M\overleftarrow{\,:\,}\star\\
\forall\:d\,:\,\varTheta\rightarrow D\overline{o}\in\Gamma.\quad\Gamma,\overline{y}_{d}:\varTheta\vdash m_{d}\left[\overline{x}\coloneqq\overline{o}\right]\overleftarrow{\,:\,}M
\end{array}}{\Gamma\vdash\mathsf{case}\,\overline{N,}n\,\left\{ \overline{|\overline{x\Rightarrow}\,(d\,\overline{y}_{d})\Rightarrow m_{d}}\right\} \overleftarrow{\,:\,}M}\operatorname{\overleftarrow{ty}-}\mathsf{case}<>
\]

\todo[inline]{reparam o in $\left[\overline{x}\coloneqq\overline{o}\right]$ also}

\caption{Surface Language Bidirectional type checking }
\label{fig:surface-data-bi-ty}
\end{figure}

We can confidently conjecture that the desired bidirectional properties hold.
\begin{conjecture}
The data extension to the bidirectional surface language is type sound.
\end{conjecture}

% ...
\begin{conjecture}
The data extension to the bidirectional surface language is weakly annotatable from the data extension of the surface language.\todo{symbolically}
\end{conjecture}

This is a minimal (and somewhat crude) accounting of bidirectional data.
It is possible to imagine syntactic sugar that doesn't require the $\overline{N,}$ and $\overline{x\Rightarrow}\,$ the in case expression of the $\operatorname{\overleftarrow{ty}-}\mathsf{case}<>$ rule.
In the dependent rule $\operatorname{\overrightarrow{ty}-}\mathsf{case}<>$ it is also possible to imagine some type constructor arguments being inferred.
These features and more will be subsumed by the dependent pattern matching of the next section, though this will complicate the meta-theory.


\section{Pattern Matching}

Unfortunately, the direct eliminator style is cumbersome for programmers to deal with.
For instance, Figure \ref{fig:data-eliminators} shows how $\mathtt{Vec}$ data can be directly eliminated to extract the first element of a non-empty list in the definition of $\mathtt{head'}$.
The $\mathtt{head'}$ function needs to redirect unreachable vector inputs to a dummy type ($\mathtt{Unit}$) and requires several copies of the same $\mathtt{A}$ variable that are not identified automatically by the eliminator described in the last section.
The usual solution is to extend case elimination with \textbf{Pattern matching.}

Pattern matching is much more ergonomic than a direct eliminator $\mathsf{case}$.
In Figure \ref{fig:data-eliminators}, the $\mathtt{head}$ defined though pattern matching is simpler and clearer.
Nested constructor matching is now possible.
When pattern matching is extended to dependent types variables will be assigned their definitions as needed, and unreachable branches can be omitted from code.
For this reason, pattern matching has been considered an ``essential'' feature for dependently typed languages since \cite{coquand1992pattern} and is implemented implemented in most popular systems, such as Agda and the user facing language of Coq.

% This style of case elimination is pervasive in ML style languages and has become popular in more mainstream languages such as Python and Java (double check and perhaps not restrictions)

\begin{figure}
\begin{lstlisting}[basicstyle={\ttfamily\small}]
 -- direct eliminator style
 head' : (A : *) -> (n : Nat) ->
   Vec A (S n) ->
   A ;
 head' A n v =
   case A, (S n), v <
     A' => n' => _ : Vec A' n' =>
       case n' < _ => *> {
         | (Z  ) => Unit
         | (S _) => A'
       }
   >{
   | _ => (Z)   => (Nil _       ) => tt
   | _ => (S _) => (Cons _ a _ _) => a
   } ;

  -- pattern match style
 head : (A : *) -> (n : Nat) ->
   Vec A (S n) ->
   A ;
 head A n v =
   case v < _ => A > {
   | (Cons _ a _ _) => a
   } ;
\end{lstlisting}

\todo[inline]{match the single eliminator syntax}

\todo[inline]{clean when I get motive inference working}

\todo[inline]{example def by cases}

\caption{Eliminators vs. Pattern Matching}
\label{fig:data-eliminators}
\end{figure}

% Surface data syntax
Figure \ref{fig:surface-data} shows the extensions to the surface language for data and pattern matching.
Our case expression match a list of scrutinees, allowing us to be very precise about the typing of branches.
Additionally this style allows for syntactic sugar for easy definitions of functions by cases.
The syntax of the direct eliminator style $\mathsf{case}$s of the last section was designed to be a special case of pattern matching.
\todo{Example of why it was needed}

\begin{figure}
\begin{tabular}{lcll}
$m...$ & $\Coloneqq$ & ... & \tabularnewline
 & | & $\mathsf{case}\,\overline{n,}\,\left\{ \overline{|\,\overline{pat\Rightarrow}m}\right\} $ & data elim. without motive\tabularnewline
 & | & $\mathsf{case}\,\overline{n,}\,\left\langle \overline{x\Rightarrow}M\right\rangle \left\{ \overline{|\,\overline{pat\Rightarrow}m}\right\} $ & data elim. with motive\tabularnewline
\multicolumn{4}{l}{patterns,}\tabularnewline
$pat$ & $\Coloneqq$ & $x$ & match a variable\tabularnewline
 & | & $(d\,\overline{pat})$ & match a constructor\tabularnewline
\end{tabular}

\caption{Surface Language Data}
\label{fig:surface-data}
\end{figure}

Patterns correspond to a specific form of expression syntax.
When an expression matches a pattern it will capture the relevant subexpressions as variables.
For instance, the expression. 

$Cons\,\mathbb{B}\,true\,\left(S\left(S\left(S\left(Z\right)\right)\right)\right)\,\left(Cons\,\mathbb{B}\,false\,\left(S\left(S\left(Z\right)\right)\right)\,y'\right)$

will match the patterns
\begin{itemize}
\item $Cons\,w\,x\,y\,z$ with bindings $w=\mathbb{B}$, $x=true$, $y=3$, $z=Cons\,\mathbb{B}\,false\,\left(S\left(S\left(Z\right)\right)\right)\,y'$
\item $x$ with bindings $x=Cons\,\mathbb{B}\,true\,\left(S\left(S\left(S\left(Z\right)\right)\right)\right)\,\left(Cons\,\mathbb{B}\,false\,\left(S\left(S\left(Z\right)\right)\right)\,y'\right)$ 
\item $Cons\,-\,x\,-\,\left(Cons\,-\,y\,-\,-\right)$ with bindings $x=true$, $y=false$
\end{itemize}
When patterns are used in the case construct, the appropriate branch will reduce with the correct bindings in scope.
Therefore the expression 

$\mathsf{case}\,Cons\,\mathbb{B}\,true\,\left(S\left(S\left(S\left(Z\right)\right)\right)\right)\,\left(Cons\,\mathbb{B}\,false\,\left(S\left(S\left(Z\right)\right)\right)\,y'\right)\left\{ Cons\,-\,x\,-\,\left(Cons\,-\,y\,-\,-\right)\Rightarrow x\&y\right\} $
 reduces to $false$.

The explicit rules for pattern matching are listed in Figure \ref{fig:surface-data-match}, where $\sigma$ will hold a possibly empty set of assignments.

\begin{figure}
\[
\frac{\,}{x\ \mathbf{Match}_{\left\{ x\coloneqq m\right\} }\ m}
\]

\[
\frac{\overline{pat}\ \mathbf{Match}_{\sigma}\ \overline{m}}{d\overline{pat}\ \mathbf{Match}_{\sigma}\ d\overline{m}}
\]

\[
\frac{pat'\ \mathbf{Match}_{\sigma'}\ n\quad\overline{pat}\ \mathbf{Match}_{\sigma}\ \overline{m}}{pat',\overline{pat}\ \mathbf{Match}_{\sigma'\cup\sigma}\ n,\overline{m}}
\]
\[
\frac{\,}{.\,\mathbf{Match}_{\emptyset}\,.}
\]

\caption{Surface Language Match}
\label{fig:surface-data-match}
\end{figure}

It is now easier for case branches to overlap, which could allow nondeterministic reduction.
There are several plausible ways to handle this, such as requiring each branch to have independent patterns, or requiring patterns have the same behavior when they overlap \cite{10.1007/978-3-642-54833-8_6}.
For the purposes of this thesis, we will use the programatic convention that the first matching pattern takes precedence.
For example, we will be able to type check

\[
\mathsf{case}\,4\,\left\langle s:\mathbb{N}\Rightarrow\mathbb{B}\right\rangle \left\{ |S\left(S\,-\right)\Rightarrow True\,|\,-\Rightarrow False\right\} 
\]

and it will reduce to $True$.

% pattern matching is hard
While pattern matching is an extremely practical feature, typing these expressions tends to be messy.
To implement dependently typed pattern matching, a procedure is needed to resolve the equational constraints that arise within each pattern, and to confirm the impossibility of unwritten branches.\todo{ven generate impossible branches}

Since arbitrary computation can be embedded in the arguments of a type constructor\footnote{
  At least in a full spectrum theory, such as the one we study here.
  }, the equational constraints are undecidable in general.
Any approach to constraint solving will have to be an approximation that performs well enough in practice.
Usually this procedure takes the form of a first order unification. 

%Similarly it is undecidable when a given pattern cannot be inhabited, so coverage checking will also necessarily be a conservative approximation.

% Unfortunately, the unification procedure involves terms outside of the pattern,  hence it is hard to pinpoint the exact point of error to the pattern.
% Worryingly, it is easy to unintentionally validate or contradict principles like Streicher's axiom K, or the Law of the Excluded Middle.

\todo[inline]{talk about or formalize the more subtle inference in the actual system}

\subsection{First Order Unification}

When type checking the branches of the a case expression, the patterns are interpreted as expressions under bindings for each variable used in the pattern.
If these equations can be unified, then the brach will type-check under the variable assignments, with the additional typing information.
For instance,
\begin{example}
Type checking by unification

The pattern $Cons\,x\,\left(S\,y\right)\,2\,z$

Could be checked against the type $Vec\,Nat\,w$

This implies the typings $x:*,y:Nat,\left(S\,y\right):x,2:Nat,z:Vec\,x\,2,\left(Cons\,x\,\left(S\,y\right)\,2\,z\right):Vec\,Nat\,w$

Which in turn imply the equalities $x=Nat,w=3$
\end{example}

This is a very simple example, in the worst case we may have equations in the form $m\,n=m'\,n'$ which are hard to solve directly (but may become easy to solve if assignment of $m=\lambda x.x$, and $m'=\lambda-.0$ are discovered).

One advantage of the first order unification approach is that if the algorithm succeeds, it will succeed with a unique, most general solution.
Since assignments are maximal, we are sure that a unified pattern will still be able to match any well typed syntax.

A simplified version of a typical unification procedure is listed in Figure \ref{fig:surface-data-unification}.
Several variations are explored in \cite{cockx_devriese_2018}.
Unification is not guaranteed to terminate since it relies on definitional equalities, which are undecidable in the surface language.
The unification procedure does not exclude the possibly cyclic assignments that could occur, such as $x=S\,x$\todo{as a threat to soundness this should be corrected?}. 

\begin{figure}
\[
\frac{\,}{U\left(\emptyset,\emptyset\right)}
\]

\[
\frac{U\left(E,a\right)\quad m\equiv m'}{U\left(\left\{ m\sim m'\right\} \cup E,a\right)}
\]

\[
\frac{U\left(E\left[x\coloneqq m\right],a\left[x\coloneqq m\right]\right)}{U\left(\left\{ x\sim m\right\} \cup E,\left\{ a,x\coloneqq m\right\} \right)}
\]

\[
\frac{U\left(E\left[x\coloneqq m\right],a\left[x\coloneqq m\right]\right)}{U\left(\left\{ m\sim x\right\} \cup E,a\cup\left\{ x\coloneqq m\right\} \right)}
\]

\[
\frac{U\left(\overline{m}\sim\overline{m'}\cup E,a\right)\quad n\equiv d\overline{m}\quad n'\equiv d\overline{m'}}{U\left(\left\{ n\sim n'\right\} \cup E,a\right)}
\]

\[
\frac{U\left(\overline{m}\sim\overline{m'}\cup E,a\right)\quad N\equiv D\overline{m}\quad N'\equiv D\overline{m'}}{U\left(\left\{ N\sim N'\right\} \cup E,a\right)}
\]

\caption{Surface Language Unification}
\label{fig:surface-data-unification}
\end{figure}

After the branches have type checked we should makes sure that they are exhaustive, such that every possible branch will be covered.
Usually the is done by generating a set of patterns that would cover all combinations of constructors and proving that the unlisted branches are unreachable.
In general it is undecidable wether any given pattern is impossible or not, so a practical approximation must be chosen.
Usually a branch is characterized as unreachable if a contradiction is found in the unification procedure.
At least programmers have the ability to manually include non-obviously unreachable branches and prove their unreachability, (or direct those branches to dummy outputs).
Though there is a real risk that the unification procedure gets stuck in ways that are not clear to the programmer, and a clean error message may be very difficult.

But that set of patterns must still be generated, given the explicit branches the programmer introduced.
There is no clear best way to do this since a more fine devision of patterns may allow enough additional definitional information to show unsatisfiability, while a more coarse devision of patterns will be more efficient.
Agda uses a tree branching\todo{cite} approach, that is efficient, but generates course patterns.
The implementation of the language in this thesis generates patterns by a system of complements, this system seams slightly easier to implement, more uniform, and generates a finer set of patterns then the case trees used in Agda.
However this approach is exponentially less performant then Agda in the worse case.\todo{thsi is less true now} 

The bidirectional system can be extended with pattern matching with rules that look like 

\[
\frac{\begin{array}{c}
\Gamma\vdash\overline{n}\overrightarrow{\,:\,}\ \Delta\\
\Gamma,\Delta\vdash M\overleftarrow{\,:\,}\star\\
\forall\:i\,\left(\Gamma\vdash\overline{pat}_{i}:_{E}?\Delta\quad U\left(E,\sigma\right)\quad\sigma\left(\Gamma,|\overline{pat}_{i}|\right)\vdash\sigma m\overleftarrow{\,:\,}\sigma\left(M\left[\Delta\coloneqq\overline{pat}_{i}\right]\right)\right)\\
\Gamma\vdash\overline{\overline{pat}}:\Delta\ \mathbf{complete}
\end{array}}{\begin{array}{c}
\Gamma\vdash\mathsf{case}\,\overline{n,}\,\left\langle \Delta_{?}\Rightarrow M\right\rangle \left\{ \overline{|\,\overline{pat\Rightarrow}m}\right\} \\
\overrightarrow{\,:\,}M\left[\Delta_{?}\coloneqq\overline{n}\right]
\end{array}}
\]

\todo{more detail}

\[
\frac{\begin{array}{c}
\Gamma\vdash\overline{n}\overrightarrow{\,:\,}\ \Delta\\
\forall\:i\,\left(\Gamma\vdash\overline{pat}_{i}:_{E}?\Delta\quad U\left(E,\sigma\right)\quad\sigma\left(\Gamma,|\overline{pat}_{i}|\right)\vdash\sigma m\overleftarrow{\,:\,}\sigma\left(M\right)\right)\\
\Gamma\vdash\overline{\overline{pat}}:\Delta\ \mathbf{complete}
\end{array}}{\Gamma\vdash\mathsf{case}\,\overline{n,}\,\left\{ \overline{|\,\overline{pat\Rightarrow}m}\right\} \overleftarrow{\,:\,}M}
\]

Where $\Gamma\vdash\overline{pat}:_{E}?\Delta$ is shorthand for a set of equations that allow a list of patterns to type check under $\Delta$, 
  and $\Gamma\vdash\overline{\overline{pat}}:\Delta\ \mathbf{complete}$ is shorthand for the exhaustiveness check. 
\begin{conjecture}
There exists a suitable\footnote{supporting at least subject reduction, type soundness, and regularity} extension to the surface language \ac{TAS} that supports patten matching style elimination
\end{conjecture}

This conjecture is not obvious since pattern matching is not consistent under reduction, or even well typed substitution.
Probably the best way to work in that direction is to use explicit contextual equalities as in \cite{sjoberg2012irrelevance}. \todo{or careful use of flex vars}
\begin{conjecture}
The bidirectional extension listed here is weakly annotatable with that extension to the surface language.
\end{conjecture}

Additionally, it makes sense to allow some additional type annotations in the motive and for these annotations to switch the the type inference of the scrutinee into a type-check.
The implementation includes this along with a simple syntax for modules, and even mutually defined data types.
For simplicity these have been excluded from the presentation here.
\section{Discussion}

\todo{ECHO VIEW from the left to explore this more}

\todo{Inductive families? don't seem to make sense in this formulation}

Pattern matching seems simple, but is surprisingly subtle.

Even without dependent types, pattern matching is a strange programming construct. 
How important is it that patterns correspond exactly to a subset of expression syntax?
What about capture annotations or side conditions?
Restricting patterns to constructors and variables means that it is hard to encapsulate functionality, a problem noticed as early as \cite{10.1145/41625.41653}\todo{review}. 
This has led to making pattern behavior override-able in Scala via Extractor Objects.
An extension in GHC allows some computations to happen within a pattern match via the $\mathtt{ViewPatterns}$ extension\todo{font. lots of prior work https://gitlab.haskell.org/ghc/ghc/-/wikis/view-patterns}.
It seems unreasonable to extend patterns to arbitrary computation (though this is allowed in the Curry language\footnote{https://curry.pages.ps.informatik.uni-kiel.de/curry-lang.org/} as a syntax for its logical programming features). 

In the presence of full-spectrum dependent types, the perspective dramatically shifts.
Any terminating typing procedure will necessarily exclude some type-able patterns and be unable to exclude some unreachable branches.
Even though only data values are considered, dependent patterns are already attacking a much more difficult problem than in the non-dependent case.
It may make sense to extend the notion of pattern matching to include other useful but difficult features.
One such interesting feature is the $\mathsf{with}$ syntax of \cite{mcbride_mckinna_2004}.

\todo{discuss stuck state of unification explicitly, x = f a vs. f a = g b, and how it for instance makes it so you can't prove transitivity
of ID fa gb, ID gb hc -> ID fa hc. also calls into quesiton preservation entirely. }

Epigram, Agda and Idris make pattern matching more powerful using $\mathsf{with}$ syntax that allows further pattern based branching by attaching a computation to a branch.
This is justified as syntactic sugar that corresponds to helper functions that can be appropriately elaborated and type-checked.
The language described in this thesis does not use the $\mathsf{with}$ side condition since nested case expressions carry the same computational behavior, and the elaboration to the cast language will allow possibly questionable typing anyway.

More aggressive choices should be explored beyond the $\mathsf{with}$ construct.
In principle it seems that dependent case expressions could be extended with relevant proof search, arbitrary computation, or some amount of constraint solving, without being any theoretically worse than usual unification with conversion. 

\todo[inline]{Discus how stratified type systems like ATS handle things (additional equational information)}

% There are other questions that How should overlapping branches be handled? (think this got covered (HA!))

The details of pattern matching change the logical character of the system\cite{cockx_devriese_2018}.
Since non-termination is allowed in the language described here, the logical issues that arise from patterns are less of a concern than the immediate logical unsoundness that was discussed in \ch{2}.
However, it is worth noting that pattern matching as described here validates axiom k and thus appears unsuitable for Hott or CTT developments.\todo{acronyms}

This Chapter has glossed over the definitional behavior of $\mathsf{case}$s, since we plan to sidestep definitional issues entirely with the cast language.
It is worth noting that there are several ways to set up the definitional reductions.
Agda style case trees may result in unpredictable definitional equalities (in so far as definitional behavior is ever predictable).
\cite{10.1007/978-3-642-54833-8_6} advocates for a more conservative approach that makes function definitions by cases definitional (but shifts the difficulties to overlapping branches and does not allow the ``first match'' behavior programmers are used to).
Another extreme would be to only allow reductions when the scrutinee is a value, similar to the work in \cite{sjoberg2012irrelevance}\todo{confirm}.
Alternatively many partial reduction systems are possible, such that branches are eliminated as they are found unreachable and substitutions made as they are available.
This last approach is experimentally implemented for the language defined here.
However it is unclear how partial reduction could be handled in the meta-theory.

Pattern matching complicates the simple story from Chapter 2, where the bidirectional system made the TAS system checkable by only adjusting annotations.
Therefore we have only conjectured the existence of a suitable TAS system for pattern matching.
If the definitional equality that feeds the TAS is generated by a system of reductions, any of the reduction strategies listed above will generate a different TAS with subtly different characteristics.
For instance, insisting on a \cbv{} case reduction will leave many equivalent computations unassociated.
If the TAS system uses partial reductions it will need to inspect the constructors of the scrutinee in order to preserve typing when reduction eliminates branches.
Agda style reductions need to extend syntax under reduction to account for side conditions. 

% For this reason it is rare to see a fully formalized account of pattern matching.

Ideally the typing rule for pattern matching case expressions in the TAS should not use the notion of unification at all.
Instead the rule should characterize the behavior that is required directly and formally\footnote{\cite{coquand1992pattern} has a good informal description}.
An ideal rule might look like

\[
\frac{\begin{array}{cl}
\Gamma\vdash\overline{n}:\Delta' & (scrutinees\ type\ check)\\
\Gamma,\overline{x}:\Delta'\vdash M:\star & (motive\ exists\ and\ is\ well\ formed)\\
\forall i.\:? & (every\ branch\ is\ well\ typed\ over\ all\ possible\ instantiations)\\
? & (all\ scrutinees\ are\ handled)
\end{array}}{\Gamma\vdash\mathsf{case}\,\overline{n,}\,\left\{ \overline{|\,\overline{pat\Rightarrow}_{i}m_{i}}\right\} :M\left[\overline{x}\coloneqq\overline{n}\right]}
\]

% Where the last condition is optional if you're willing to modify type soundness to allow pattern match errors (again, they are no worse then the non-termination already allowed, and better behaved).


\section{Related work}

\subsection{Dependent Systems with Data}

Many systems that target data only formalize a representative collection of data types, expecting the reader to be able to generalize the scheme.
This data usually covers \Nat{}s (for recursive types) and dependent pairs (for dependent types). % and unit to end a chain of dependent pairs.
For example, Martin L{\"o}f's original paper treated data this way\todo{confirm, cite}, and is still a common approach to data. % (for instance in \cite{jia2010dependent}).

% Martin Lof generalized the notion of data to W types of well founded trees  and this still serves as a theoretical justification for data.

\todo[inline]{Abbreviations!!}
Unified Type Theory (UTT)\cite{luo1990extended,luo1994computation} is an extension to ECC that specifies a scheme to define strictly positive data types by way of a logical framework defined in \ac{MLTT}.
This scheme generates primitive recursors for schematized data, and does not inherently support pattern matching.

The Calculus of Inductive Constructions (CiC) is an extension to the calculus of constructions that includes a system of first class data definitions\todo{cite '93 COq report}.
It evolved form Calculus of Constructions with Inductive Definitions (CCID) which was first presented in, \cite{10.1007/BFb0037116} but the most complete formulation is maintained as part of the Coq manual\footnote{https://coq.github.io/doc/v8.9/refman/language/cic.html}.
A bidirectional account of CiC is given in \cite{lennonbertrand:LIPIcs.ITP.2021.24}, though it uses a different style of bidirectionally then discussed here to maintain compatibility with the existing Coq system.

\todo[inline]{CTT, higher inductive types, quotient types}

\subsection{Dependent Pattern matching}

\todo[inline]{Clean up this section}

The scheme for dependent pattern matching was first presented by Thierry Coquand in \cite{coquand1992pattern}.

Mcbride and Mckinna extended the power and theory of dependent pattern matching with several additional constructs such as $\mathsf{with}$ in \cite{mcbride_mckinna_2004}.

Ulf Norell greatly simplified the presentation of pattern matching in his thesis \cite{norell2007towards}.

\todo[inline]{A tutorial implementation of dynamic pattern unification Adam Gundry
and Conor McBride (2012) http://adam.gundry.co.uk/pub/pattern-unify/
(this links give you the choice to read a more detailed chapter of
Adam Gundry's thesis instead)}

The subtleties of dependently pattern matching are explored in \cite{cockx_devriese_2018}, which has many informative examples, some of which were motivated by long standing bugs in Agda.

\todo[inline]{https://research.chalmers.se/en/publication/519011 ?}

\todo[inline]{https://sozeau.gitlabpages.inria.fr/www/research/publications/Equations:\_A\_Dependent\_Pattern-Matching\_Compiler.pdf ?}

\todo[inline]{http://citeseerx.ist.psu.edu/viewdoc/download?doi=10.1.1.99.1405\&rep=rep1\&type=pdf ?}

\todo[inline]{barras's thesis data}

