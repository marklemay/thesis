\section{Direct Elimination}

How should a program observe data?
Since a term of a given data type can only be created with one of the constructors from its definition, we can completely handle a data expression if each possible constructor is accounted for.
For instance, \Nat{} has the two constructors $\mathtt{Z}$ and \Suc{} (which holds the preceding number), so the expression $\case{}\,n\,\left\{ |\,\mathtt{Z}\Rightarrow \mathtt{Z}\,|\,\Suc{}\,x\Rightarrow x\right\}$ will extract the proceeding number from $n$ (or $0$ if $n=0$).
We will call the term being inspected the \textbf{\scrut{}}\footnote{
  Also called a \textbf{discriminee}.
}, here $n$ is the \scrut{}.
% In this light, boolean case elimination corresponds to the if-then-else expression found in many mainstream languages.
 
This \Nat{} elimination is type checkable since we know the intended output type of each branch (\Nat{} in the above example), and can check that they are compatible.
But in the presence of dependent types the output may not be obvious.

We will need to extend the syntax of \case{}s to support dependent type checking.
Specifically, we will need to add a \textbf{motive} annotation that allows the type checker to compute the output type of the branches if they vary in terms of the input.
For instance, the \case{} in the the $\mathtt{rep}$ function in \Fref{data-elim-examples}, has the motive $\mathtt{n'\ :\ Nat\ \Rightarrow\ Vec\ A\ n'}$.
Each branch is typed knowing what it has observed about the input.
For instance, in the first branch $\mathtt{Nil\ A}$ :  $\mathtt{Vec\ A\ 0}$.
This allows the input to generalize to any appropriately typed term, even those that do not begin with a constructor.\todo{awK}
For instance, $\mathtt{rep\ Bool\ True\ }(f\,x)$ : $\mathtt{Vec\ Bool\ }(f\,x)$ if $(f\,x)$ : \Nat{}.

\begin{figure}
\begin{lstlisting}[basicstyle={\ttfamily\small}]
-- generate a vector of a given length
rep : (A : *) -> A -> (n : Nat) -> Vec A n ;
rep A a n = 
  case n < n' : Nat => Vec A n' >{
    | (Z)   => Nil A
    | (S p) => Cons A a p (rep A a p)
    } ;

trues : Vec Bool 3;
trues = rep Bool True 3;
-- = [True,True,True]

mapVec' : (A : *) -> (n : Nat) -> Vec A n 
 -> (B : *) -> (A -> B)
 -> Vec B n ;
mapVec' A n v =
  case A, n, v 
    < A : * => n : Nat => _ : Vec A n 
      => (B : *) -> (A -> B) -> Vec B n 
    >{
   | _ => _ => (Nil A)          
     => \ B => \ _ => Nil B
   | _ => _ => (Cons A a pn pv) 
     => \ B => \ f => Cons B (f a) pn (mapVec' A pn pv B f)
  };

falses : Vec Bool 3;
falses = mapVec' Bool 3 trues Bool not;
-- = [False,False,False]
\end{lstlisting}
\caption{Data Examples}
\label{fig:data-elim-examples}
\end{figure}

We may also want to use some values of the type level argument to calculate the motive, and type the branches.
This will be allowed with additional bindings in the motive and in each branch.
For example the $\mathtt{mapVec'}$ function in \Fref{data-elim-examples}.
In $\mathtt{mapVec'}$ the motive allows the output to vary along with the length argument $n$ that appears in the type position.
% For this language the \scrut{} list, motive and pattern
In general, the motive will be treated like the typing annotations in \ch{2}: we will only allow correct motives in a well typed term, motives are not always required, and the motive will be definitionally irrelevant.

This version of data can be given by extending the \slang{} syntax in \ch{2}, as in \Fref{surface-data-min}.
Data type constructors and data term constructors are presented as function like identifiers (in this syntax they reuse function application to collect arguments).
% In general thy are treated like functions.
% This direct eliminator scheme is roughly similar to how Coq handles data in its core language.

\begin{figure}
\begin{tabular}{lcll}
% \multicolumn{4}{l}{list of $O$, separated with $s$}\tabularnewline
% $\overline{sO}$,$\overline{Os}$ & $\Coloneqq$ & $.$ & empty list\tabularnewline
%  & $|$ & $sO\overline{sO}$ & extend list\tabularnewline

\multicolumn{4}{l}{telescopes,}\tabularnewline
$\Delta,\varTheta$ & $\Coloneqq$ & $\overline{\left(x:M\right)\rightarrow}$ & \tabularnewline
\multicolumn{4}{l}{data type identifier,}\tabularnewline
$D$ &  &  & \tabularnewline
\multicolumn{4}{l}{data constructor identifier,}\tabularnewline
$d$ &  &  & \tabularnewline
\multicolumn{4}{l}{contexts,}\tabularnewline
$\Gamma$ & $\Coloneqq$ & ... & \tabularnewline
 & $|$ & $\Gamma,\mathsf{data}\,D\,:\,\Delta\rightarrow\star\,\left\{ \overline{|\,d\,:\,\varTheta\rightarrow D\overline{m}}\right\} $ & data def.\tabularnewline
 & $|$ & $\Gamma,\mathsf{data}\,D\,:\,\Delta\rightarrow\star$ & abstract data\tabularnewline
 \multicolumn{4}{l}{expressions,}\tabularnewline
$m,...$ & $\Coloneqq$ & ... & \tabularnewline
 & $|$ & $D$ & type cons.\tabularnewline
 & $|$ & $d$ & data cons.\tabularnewline
 & $|$ & $\case{}\,\overline{N,}n\,\left\{ \overline{|\overline{x\Rightarrow}(d\,\overline{y})\Rightarrow m}\right\} $ & data elim.\tabularnewline
 & $|$ & $\case{}\,\overline{N,}n\,\left\langle \overline{x\Rightarrow}\,y:D\,\overline{x}\Rightarrow M\right\rangle \left\{ \overline{|\overline{x\Rightarrow}(d\,\overline{y})\Rightarrow m}\right\} $ & data elim. (motive)\tabularnewline
\multicolumn{4}{l}{values,}\tabularnewline
$v$ & $\Coloneqq$ & ... & \tabularnewline
 & $|$ & $D\,\overline{v}$ & \tabularnewline
 & $|$ & $d\,\overline{v}$ & \tabularnewline
\end{tabular}

\caption{\SLang{} (Direct Eliminator) Data}
\label{fig:surface-data-min}
\end{figure}

\todo{abreviations}
% \begin{figure}
% \begin{tabular}{llllll}
%  & syntax & written & when & for example & is written\tabularnewline
% leading separator & $sO\overline{sO}$ & $O\overline{sO}$ & clear from context & $,1,2,3$ & $1,2,3$\tabularnewline
% trailing separator & $\overline{sO}$ & $\overline{sO}s$ & clarifies intent & $\left(x\rightarrow y\rightarrow z\right)Id\,x\,y\,z$ & $x\rightarrow y\rightarrow z\rightarrow Id\,x\,y\,z$\tabularnewline
% non dependent telescope binder & $\left(x:M\right)\rightarrow$ & $M\rightarrow$ & $x$is not intended to bind  & $\left(x:Nat\right)\rightarrow\left(y:IsEven\,x\right)\rightarrow Nat$ & $\left(x:Nat\right)\rightarrow IsEven\,x\rightarrow Nat$\tabularnewline
% repeated application & $m\,n_{0}\,n_{1}\,...$ & $m\,\overline{n}$ & $\overline{n}=n_{0}\,n_{1}\,...$ &  & \tabularnewline
% \end{tabular}

% \todo[inline]{review if there are more abbreviations around?}

% \caption{\SLang{} Abbreviations}
% \label{fig:surface-pre-syntax-data-abrev-1}
% \end{figure}

\todo[inline]{motive should not need to insist on the type info of the binder? grey out? Grey out things that are surface syntax but not needed for theory? addforce a paren round the only intrestingin the \scrut{} list?}

\todo{as in the examples}
As in the examples the \case{} eliminator first takes the explicit type arguments, followed by a \scrut{} list. % of correct type.
Then optionally a motive that characterizes the output type of each branch with all the type arguments and arguments for each element of the \scruts{}. % an abstracted and in scope.

For instance, this \case{} expression checks if a vector $x:\Vect{}\,\Bool{}\,1$ is empty:

\[
  \begin{array}{l}
% x:\Vect{}\,\Bool{}\,1\vdash & \\
%  & 
  \case{}\,\Bool{},1,x\,\left\langle y\Rightarrow z\Rightarrow s:\Vect{}\,y\,z\Rightarrow\Bool{}\right\rangle \\
%  & 
 \left\{ 
    \begin{array}{llll}
    | y&\Rightarrow z&\Rightarrow \Nil{}\,-&\Rightarrow \True{} \\
    | y&\Rightarrow z&\Rightarrow \Cons{}\,----&\Rightarrow \False{}
    \end{array}
    \right\} 
\end{array}
\]

The grammar includes a little more syntax than is strictly necessary, since the $\Bool{},1$ part of the scrutinee list could be inferred from the type of $x$ and the $y\Rightarrow z\Rightarrow$ binders are not needed in the branch.
This slightly verbose \case{} eliminator syntax is designed to be forward compatible with the pattern matching system used in the rest of this thesis. 

Additionally we define telescopes, which generalize zero or more typed bindings\todo{Cite De Bruijn for telescopes}.
This allows a much cleaner definition of data than is otherwise possible.
% Also we define syntactic lists that allow zero or more pieces of syntax.
Expressions in a list can be type checked against a telescope. % can be used to generalize dependent pairs, and 
For instance, the list $\Nat{},2,2,\mathtt{refl}\,\Nat{}\,2$ type checks against $\left(X:\star\right)\rightarrow\left(y:X\right)\rightarrow\left(z:X\right)\rightarrow\left(-:\Id{}\ X\,y\,z\right)$.
This becomes helpful in several situations, but especially when we need work with the listed arguments of the data type constructor.
We will allow several syntactic puns, such as treating telescopes as prefixes for function types.
For instance, 
  if $\Delta=\left(y:\Nat{} \right)\rightarrow\left(z:\Nat{} \right)\rightarrow\left(-:\Id{}\ \Nat{} \,y\,z\right)$
  then writing $f:\Delta\rightarrow\Nat{} $
  will be shorthand for $f:\left(y:\Nat{} \right)\rightarrow\left(z:\Nat{} \right)\rightarrow \Id{}\ \Nat{} \,y\,z\rightarrow\Nat{}$.

In the presence of general recursion \case{} elimination is powerful.
For instance, all the functions in \Fref{data-elim-examples} use recursion.
Additionally, well-founded recursion can be used to represent inductive proofs.
\todo{example}

Adding data allows for two new potential sources of bad behavior: incomplete matches, and nontermination from non-strictly positive data. 

\subsubsection{Incomplete Eliminations}

Consider the match 

\[
x:\Nat{}\vdash\case{}\,x\,\left\langle s:\Nat{}\Rightarrow\Bool{}\right\rangle \left\{ |\Suc{}-\Rightarrow \True{}\right\} 
\]

This match will ``get stuck'' if $0$ is substituted for $x$.
Recall that the key theorem of the \slang{} is type soundness, ``well typed terms don't get stuck''.
Since verifying every constructor has a branch is relatively easy, the \slang{} \ac{TAS} will require every constructor to be handled in order to type check with direct elimination.
This is in contrast to most programming languages, which do allow incomplete patterns, though usually a warning is given, and a runtime error is raised if the \scrut{} cannot be handled.

% Since this is a "well behaved" failure

This thesis already has a philosophy for handling warnings and runtime errors through the \clang{}.
When we get to the \clang{} data in \ch{5}, we will allow non-exhaustive data to be reported as a warning and that will allow ``unmatched'' errors to be observed at runtime.

For similar reasons, in the direct eliminator scheme, we will insist that each constructor is handled at most once, so there is no ambiguity for how a \case{} is eliminated.

\subsubsection{(non-)Strict Positivity}

A more subtle concern is posed by data definitions that are not strictly positive.
Consider the following definition, \todo{think this can be simplified}

\begin{lstlisting}[basicstyle={\ttfamily\small}]
data Bad : * {
| C : (Bad -> Bad) -> Bad
};

selfApply : Bad -> Bad;
selfApply b =
  case b {
    | C f => f b
  };

loop : Bad;
loop = selfApply (C selfApply)
\end{lstlisting}

The $\mathtt{C}$ constructor in the definitions of $\mathtt{Bad}$ has a self reference in a negative position, $(\mathtt{Bad}\rightarrow\underline{\mathtt{Bad}})\rightarrow\mathtt{Bad}$. 

Non-strictly positive data definitions can cause non-termination, independent of the two other sources of non-termination already considered (general recursion and \tit{}).
Dependent type systems usually require a strictness check on data definitions to avoid this possibility.
However, this would disallow some useful constructions like higher order abstract syntax. % , or an encoding of department forms that must be complete three weeks before the date that they are needed to schedule.
Since non-termination is already allowed in the surface \ac{TAS}, we will not restrict the \slang{} to strictly positive data.

\todo{who cam up with this first? Martin Lof?}

% co-inductive uses of data

\subsection{Type Assignment System}

Before the typing rules for data can be considered, first some meta rules must be presented that will allow the simultaneous type-checking of lists and telescopes.
These rules are listed in \ref{fig:surface-data-meta-ty}, and are standard.
Telescopes are $\mathbf{ok}$ when they extend the context in an $\mathbf{ok}$ way.
Lists of expressions can be said to have the type of the telescope if every expression in the list type checks successively.

\begin{figure}
\[
\frac{
  % \Gamma\,\mathbf{ok}
  \,
  }{\Gamma\vdash.\,\mathbf{ok}}\rulename{\mathbf{ok}-Tel-empty}
\]

\[
\frac{
  \Gamma\vdash M:\star\quad\Gamma,x:M\vdash\Delta\,\mathbf{ok}
  }{\Gamma\vdash\left(x:M\right)\rightarrow\Delta\,\mathbf{ok}}\rulename{\mathbf{ok}-Tel-ext}
\]

\[
\frac{
  % \Gamma\,\mathbf{ok}
  \,
}{\Gamma\vdash,:.}\rulename{ty-ls-empty}
\]

\[
\frac{
  \Gamma,x:M\vdash\Delta\quad\Gamma\vdash m:M\quad\Gamma\vdash\overline{n,} % \left[x\coloneqq m\right]
  :\Delta\left[x\coloneqq m\right]
  }{\Gamma\vdash m\overline{,n}\,:\,\left(x:M\right)\rightarrow\Delta}\rulename{ty-ls-ext}
\]

\caption{Meta rules}
\label{fig:surface-data-meta-ty}
\end{figure}

Data definitions can be added to contexts if all of their constituents are well typed and \textbf{ok}.
The rules are listed in Figure \ref{fig:surface-data-ok}.
The $\rulename{\mathbf{ok}-abs-data}$ rule allows data to be considered abstractly if it is formed with a plausible telescope.
$\rulename{\mathbf{ok}-data}$ checks a full data definition with an abstract reference to a data definition in context, which allows recursive data definitions such as \Nat{} which needs \Nat{} to be in scope to define the \Suc{} constructor.
This thesis does not formalize a syntax that adds data to context, though a very simple module system has been implemented in the prototype.
It is taken for granted that any scheme to add well formed data into a type context is fine.
% This presentation of data definitions largely follows \cite{sjoberg2012irrelevance}. 

\begin{figure}
\[
\frac{\Gamma\vdash\Delta\,\mathbf{ok}}{\Gamma\vdash\mathsf{data}\,D\,\Delta\,\mathbf{ok}}\rulename{\mathbf{ok}-abs-data}
\]

% \[
% \frac{\Gamma\vdash\mathsf{data}\,D\,\Delta\,\mathbf{ok}}{\Gamma,\mathsf{data}\,D\,\Delta\,\mathbf{ok}}\rulename{\mathbf{ok}-abs-data-ext}
% \]
\[
\frac{\Gamma\vdash\mathsf{data}\,D\,\Delta\,\mathbf{ok}\quad\forall d.\Gamma,\mathsf{data}\,D\,\Delta\vdash\varTheta_{d}\,\mathbf{ok}\quad\forall d.\:\Gamma,\mathsf{data}\,D\,\Delta,\varTheta_{d}\vdash\overline{m}_{d}:\Delta}{\Gamma\vdash\mathsf{data}\,D\,:\,\Delta\left\{ \overline{|\,d\,:\,\varTheta_{d}\rightarrow D\overline{m}_{d}}\right\} \,\mathbf{ok}}\rulename{\mathbf{ok}-data}
\]

% \[
% \frac{\Gamma\vdash\mathsf{data}\,D\,:\,\Delta\left\{ \overline{|\,d\,:\,\varTheta\rightarrow D\overline{m}}\right\} \,\mathbf{ok}}{\Gamma,\mathsf{data}\,D\,:\,\Delta\left\{ \overline{|\,d\,:\,\varTheta\rightarrow D\overline{m}}\right\} \,\mathbf{ok}}\rulename{\mathbf{ok}-data-ext}
% \]

\caption{\SLang{} Data $\mathbf{ok}$}
\label{fig:surface-data-ok}
\end{figure}

The type assignment system with direct elimination must be extended with the rules in \ref{fig:surface-data-ty}.
The \rulename{ty-TCon} and \rulename{ty-Con} rules allow type and data constructors to be used as functions of appropriate type.
The \rulename{ty-\case{}<>} rule types a \case{} expression by ensuring that the correct data definition for $D$ is in context, the \scrut{} $n$ has the correct type, the motive $M$ is well formed under the type arguments and the \scrut{}, % The motive requires checking in the case of empty data
  finally every data constructor is verified to have a corresponding branch.
\rulename{ty-\case{}} allows for the same typing logic, but does not require the motive be annotated in syntax.
In both rules we allow telescopes to rename their variables with the shorthand $\overline{x}:\Delta$.

These rules make use of several convenient shorthands:
  $\mathsf{data}\,D\,\Delta\in\Gamma$ and $d\,:\,\varTheta\rightarrow D\overline{m}\in\Gamma$ extract the type constructor definitions and data constructor definitions from the context respectively;
  telescopes can be added to context, such as $\Gamma,\overline{x}:\Delta,z:D\,\overline{x}\vdash M:\star$;
  telescopes can be added to context, reparameterized by an existing list of variables, $\Gamma,\overline{y}_{d}:\varTheta$;
  and telescopes can be used as variable lists to substitute against, as in $\varTheta\coloneqq\overline{y}_{d}$.
  \todo{bullet?}

\begin{figure}
\[
\frac{\mathsf{data}\,D\,\Delta\in\Gamma}{\Gamma\vdash D\,:\,\Delta\rightarrow\star}\rulename{ty-TCon}
\]

\[
\frac{d\,:\,\varTheta\rightarrow D\overline{m}\in\Gamma}{\Gamma\vdash d\,:\,\varTheta\rightarrow D\overline{m}}\rulename{ty-Con}
\]

\[
\frac{\begin{array}{c}
\mathsf{data}\,D\,\Delta\left\{ ...\right\} \in\Gamma\\
\Gamma\vdash n:D\overline{N}\\
\Gamma,\overline{x}:\Delta,z:D\,\overline{x}\vdash M:\star\\
\forall\:d\,:\,\varTheta\rightarrow D\overline{o}\in\Gamma.\\
\Gamma,\overline{y}_{d}:\varTheta\vdash m_{d}\left[\overline{x}\coloneqq\overline{o}\left[\varTheta\coloneqq\overline{y}_{d}\right]\right]:M\left[\overline{x}\coloneqq\overline{o}\left[\varTheta\coloneqq\overline{y}_{d}\right],z\coloneqq d\,\overline{y}_{d}\right]\\
\mathrm{No\ duplicate\ branches}
\end{array}}{\begin{array}{c}
\Gamma\vdash \case{} \,\overline{N,}n\,\left\langle \overline{x\Rightarrow}z:D\,\overline{x}\Rightarrow M\right\rangle \left\{ \overline{|\,\overline{x\Rightarrow}\,d\overline{y}_{d}\,\Rightarrow m_{d}}\right\} \\
:M\left[\overline{x}\coloneqq\overline{N},z\coloneqq n\right]
\end{array}
}\rulename{ty-\case{}<>} 
\]
\todo[inline]{can remove "no dups..." if I can find a unique for all}

\[
\frac{\begin{array}{c}
\mathsf{data}\,D\,\Delta\left\{ ...\right\} \in\Gamma\\
\Gamma\vdash n:D\overline{N}\\
\Gamma,\overline{x}:\Delta,z:D\,\overline{x}\vdash M:\star\\
\forall\:d\,:\,\varTheta\rightarrow D\overline{o}\in\Gamma.\\
\Gamma,\overline{y}_{d}:\varTheta\vdash m_{d}\left[\overline{x}\coloneqq\overline{o}\left[\varTheta\coloneqq\overline{y}_{d}\right]\right]:M\left[\overline{x}\coloneqq\overline{o}\left[\varTheta\coloneqq\overline{y}_{d}\right],z\coloneqq d\,\overline{y}_{d}\right]
\end{array}}{\begin{array}{c}
\Gamma\vdash \case{} \,\overline{N,}n\,\left\{ \overline{|\,\overline{x\Rightarrow}\,d\overline{y}_{d}\,\Rightarrow m_{d}}\right\} \\
:M\left[\overline{x}\coloneqq\overline{N},z\coloneqq n\right]
\end{array}
}\rulename{ty-\case{}}
\]

\caption{\SLang{} Data Typing}
\label{fig:surface-data-ty}
\end{figure}

\todo[inline]{suspect this also hinges on regularity, which should be addressed more directly}

Extensions to the parallel reduction rules are listed in Figure \ref{fig:surface-data-red}.
They follow the scheme of parallel reductions laid out in \ch{2}.
The \rulename{\Rrightarrow-\case{}-red} rule\footnote{
  Also called $\iota$, or Iota reduction.
}\todo{mention beta in footnote in CH2} reduces a \case{} expression by choosing the appropriate branch.
The \rulename{\Rrightarrow-\case{}<>-red} rule removes the motive annotation, much like the annotation rule in \ch{2}.
The rules \rulename{\Rrightarrow-\case{}<>}, \rulename{\Rrightarrow-D}, and \rulename{\Rrightarrow -d} keep the $\Rrightarrow$ relation reflexive.
The reduction relation is generalized to lists in the expected way.

\begin{figure}
\[
\frac{\begin{array}{c}
\overline{N}\Rrightarrow\overline{N'}\quad\overline{m}\Rrightarrow\overline{m'}\\
\exists\overline{x\Rightarrow}(d\,\overline{y}_{d})\Rightarrow m_{d}\in\left\{ \overline{|\,\overline{x\Rightarrow}(d'\,\overline{y}_{d'})\Rightarrow m_{d'}}\right\}. \\
m_{d}\Rrightarrow m_{d}'
\end{array}}{ \case{} \,\overline{N,}\,d\overline{m}\,\left\{ \overline{|\,\overline{x\Rightarrow}(d'\,\overline{y}_{d'})\Rightarrow m_{d'}}\right\} \Rrightarrow m_{d}\left[\overline{x}\coloneqq\overline{N'},\overline{y}_{d}\coloneqq\overline{m'}\right]
} \rulename{\Rrightarrow-\case{}-red} 
\]

\[
\frac{\begin{array}{c}
\overline{N}\Rrightarrow\overline{N'}\quad m\Rrightarrow m'\\
\forall\overline{x\Rightarrow}(d\,\overline{y}_{d})\Rightarrow m_{d}\in\left\{ \overline{|\,\overline{\Rightarrow x}\Rightarrow(d'\,\overline{y}_{d'})\Rightarrow m_{d'}}\right\} .\:m_{d}\Rrightarrow m_{d}'
\end{array}}{
  \begin{array}{rl}
               & \case{} \,\overline{N,}\,m\,\left\langle ...\right\rangle \left\{ \overline{|\,\overline{\Rightarrow x}\Rightarrow(d'\,\overline{y}_{d'})\Rightarrow m_{d'}}\right\} \\
  \Rrightarrow & \case{} \,\overline{N',}\,m'\left\{ \overline{|\,\overline{\Rightarrow x}\Rightarrow(d'\,\overline{y}_{d'})\Rightarrow m_{d'}'}\right\} 
\end{array}
  } \rulename{\Rrightarrow-\case{}<>-red} 
\]

\[
\frac{\begin{array}{c}
\overline{N}\Rrightarrow\overline{N'}\quad m\Rrightarrow m'\\
M\Rrightarrow M'\\
\forall\overline{x\Rightarrow}(d'\,\overline{y}_{d'})\Rightarrow m_{d'}\in\left\{ \overline{|\,\overline{\Rightarrow x}\Rightarrow(d\,\overline{y}_{d})\Rightarrow m_{d}}\right\} .\:m_{d'}\Rrightarrow m'_{d'}
\end{array}}{\begin{array}{c}
  \case{} \,\overline{N,}\,m\,\left\langle \overline{x\Rightarrow}z:D\,\overline{x}\Rightarrow M\right\rangle \left\{ \overline{|\,\overline{\Rightarrow x}\Rightarrow(d\,\overline{y}_{d})\Rightarrow m_{d}}\right\} \Rrightarrow\\
  \case{} \,\overline{N,}\,m'\,\left\langle \overline{x\Rightarrow}z:D\,\overline{x}\Rightarrow M'\right\rangle \left\{ \overline{|\,\overline{\Rightarrow x}\Rightarrow(d\,\overline{y}_{d})\Rightarrow m'_{d}}\right\} 
\end{array}
}\rulename{\Rrightarrow-\case{}<>} 
\]

\[
\frac{\,}{D\Rrightarrow D}\rulename{\Rrightarrow-D}
\]
\[
\frac{\,}{d\Rrightarrow d}\rulename{\Rrightarrow-d}
\]
\todo[inline]{make array formatting consistent}
\caption{\SLang{} Data Reduction}
\label{fig:surface-data-red}
\end{figure}

We are now in a position to select a sub relation of $\Rrightarrow$ reductions that will be used to characterize \cbv{} evaluation.
This relation could be used to prove type safety, and is close to the reduction used in the implementation.
The rules are listed in \Fref{surface-data-cbv}.

\begin{figure}
\[
\frac{\,}{
  \begin{array}{rl}
   & \case{} \,\overline{N,}\,n\,\left\langle ...\right\rangle \left\{ \overline{|\,\overline{\Rightarrow x}\Rightarrow(d\,\overline{y})\Rightarrow m}\right\} \\
   \rightsquigarrow & \case{} \,\overline{N,}\,n\left\{ \overline{|\,\overline{\Rightarrow x}\Rightarrow(d\,\overline{y})\Rightarrow m}\right\} 
  \end{array}
   }\rulename{\rightsquigarrow-\case{}<>}
\]

\[
\frac{\exists\overline{x\Rightarrow}(d\,\overline{y}_{d})\Rightarrow m_{d}\in\left\{ \overline{|\,\overline{x\Rightarrow}(d'\,\overline{y}_{d'})\Rightarrow m_{d'}}\right\} }{\case{} \,\overline{V,}\,d\overline{v}\,\left\{ \overline{|\,\overline{x\Rightarrow}(d'\,\overline{y}_{d'})\Rightarrow m_{d'}}\right\} \rightsquigarrow m_{d}\left[\overline{x}\coloneqq\overline{V},\overline{y}_{d}\coloneqq\overline{v}\right]
}\rulename{\rightsquigarrow-\case{}-red}
\]
\todo[inline]{don't index the left side ones}
\[
\frac{\overline{N}\rightsquigarrow\overline{N'}}{\case{} \,\overline{N,}\,n\,\left\{ \overline{|\,\overline{x\Rightarrow}(d\,\overline{y})\Rightarrow m}\right\} \rightsquigarrow \case{} \,\overline{N',}\,n\,\left\{ \overline{|\,\overline{x\Rightarrow}(d\,\overline{y})\Rightarrow m}\right\} }\,
\]

\[
\frac{n\rightsquigarrow n'}{\case{} \,\overline{V,}\,n\,\left\{ \overline{|\,\overline{x\Rightarrow}(d\,\overline{y})\Rightarrow m}\right\} \rightsquigarrow \case{} \,\overline{V,}\,n'\,\left\{ \overline{|\,\overline{x\Rightarrow}(d\,\overline{y})\Rightarrow m}\right\} }\,
\]

\caption{\SLang{} Data \CbV{}}
\label{fig:surface-data-cbv}
\end{figure}

Finally we need to redefine what it means for a type context to be empty in the presence of data definitions (in \Fref{surface-data-empty}).
We will say a context is empty only if it contains concrete data definitions.

\begin{figure}
\[
\frac{\ }{\lozenge\,\mathbf{Empty}}\rulename{Empty-ctx}
\]

\[
\frac{\Gamma\,\mathbf{Empty}\quad\Gamma\vdash\mathsf{data}\,D\,:\,\Delta\left\{ \overline{|\,d\,:\,\varTheta\rightarrow D\overline{m}}\right\} \,\mathbf{ok}}{\Gamma,\mathsf{data}\,D\,:\,\Delta\left\{ \overline{|\,d\,:\,\varTheta\rightarrow D\overline{m}}\right\} \,\mathbf{Empty}}\rulename{Empty-ctx}
\]

\caption{\SLang{} Empty}
\label{fig:surface-data-empty}
\end{figure}

Since a system with a similar presentation has proven type soundness in \cite{sjoberg2012irrelevance}\todo{switch to vilhelms thesis? it's not the most simalar},
we will not prove the type soundness of the system here.

\begin{claim} The \slang{} extended with data and elimination preserves types over reduction.
\end{claim}

\begin{claim} The \slang{} extended with data and elimination has progress.

If $\Gamma\,\mathbf{Empty}$, \textup{$\Gamma\vdash m:M$}, then $m$ is a value, or $m\rightsquigarrow m'$ .
\end{claim}

\begin{claim}
The \slang{} extended with data and a direct elimnator scheme is type sound.
\end{claim}

\subsection{\Bidir{} Type Checking}

A \bidir{} type checking procedure exists for the type assignment rules listed above.
An outline of these rules is in \Fref{surface-data-bi-ty}.

The type of data constructors and type constructors can always be inferred.
% If the motive does not depend on the \scrut{} or type arguments, it can be used to check against the type of the branches.
A \case{} with a motive will have its type inferred, and the motive will be used to check every branch in the \rulename{\overrightarrow{ty}-\case{}<>} rule.
An unmotivated \case{} will be type checked by an argument ignoring type dependency with the \rulename{\overleftarrow{ty}-\case{}} rule.
% These rules rely on generalizing 

\begin{figure}
\[
\frac{\mathsf{data}\,D\,\Delta\in\Gamma}{\Gamma\vdash D\overrightarrow{\,:\,}\Delta\rightarrow*}\rulename{\overrightarrow{ty}-TCon}
\]

\[
\frac{d\,:\,\varTheta\rightarrow D\overline{m}\in\Gamma}{\Gamma\vdash d\overrightarrow{\,:\,}\varTheta\rightarrow D\overline{m}}\rulename{\overrightarrow{ty}-Con}
\]

\[
\frac{\begin{array}{c}
\mathsf{data}\,D\,\Delta\left\{ ...\right\} \in\Gamma\\
\Gamma\vdash\overline{N}\overleftarrow{\,:\,}\Delta\quad\Gamma\vdash n\overleftarrow{\,:\,}D\overline{N}\\
\Gamma,\overline{x}:\Delta,z:D\,\overline{x}\vdash M\overleftarrow{\,:\,}\star\\
\forall\:d\,:\,\varTheta\rightarrow D\overline{o}\in\Gamma.\quad\Gamma,\overline{y}_{d}:\varTheta\vdash m_{d}\left[\overline{x}\coloneqq\overline{o}'\right]\overleftarrow{\,:\,}M\left[\overline{x}\coloneqq\overline{o}',z\coloneqq d\,\overline{y}_{d}\right]
\end{array}}{\begin{array}{c}
\Gamma\vdash \case{} \,\overline{N,}n\,\left\langle \overline{x\Rightarrow}z:D\,\overline{x}\Rightarrow M\right\rangle \left\{ \overline{|\,\overline{x\Rightarrow}(d\,\overline{y}_{d})\Rightarrow m_{d}}\right\} \\
\overrightarrow{\,:\,}M\left[\overline{x}\coloneqq\overline{N},z\coloneqq n\right]
\end{array}
}\rulename{\overrightarrow{ty}-\case{}<>}
\]

% even though it is jenk to check M : *, this is needed to handle the case with empty types, otherwhise thesre is norestiction on teh syntax of M
%  it is fine  since this rule is entirely optional, and only exists to hint at how to infer simply typed things
\[
\frac{\begin{array}{c}
\mathsf{data}\,D\,\Delta\left\{ ...\right\} \in\Gamma\\
\Gamma\vdash\overline{N}\overleftarrow{\,:\,}\Delta\quad\Gamma\vdash n\overrightarrow{\,:\,}D\overline{N}\\
\Gamma\vdash M\overleftarrow{\,:\,}\star\\
\forall\:d\,:\,\varTheta\rightarrow D\overline{o}\in\Gamma.\quad\Gamma,\overline{y}_{d}:\varTheta\vdash m_{d}\left[\overline{x}\coloneqq\overline{o}'\right]\overleftarrow{\,:\,}M
\end{array}}{
  \Gamma\vdash \case{} \,\overline{N,}n\,\left\{ \overline{|\overline{x\Rightarrow}\,(d\,\overline{y}_{d})\Rightarrow m_{d}}\right\} \overleftarrow{\,:\,}M
}\rulename{\overleftarrow{ty}-\case{}}
\]

where 
\[
\overline{o}'= \overline{o}\left[\varTheta \coloneqq \overline{y}_{d} \right]
\]
\caption{\SLang{} \Bidir{} Type Checking }
\label{fig:surface-data-bi-ty}
\end{figure}

The desired \bidir{} properties hold.
\begin{claim}
The data extension to the \bidir{} \slang{} is type sound.
\end{claim}

\begin{claim}
The data extension to the \bidir{} \slang{} is weakly annotatable from the data extension of the \slang{}.\todo{symbolically}
\end{claim}

This is a minimal (and somewhat crude) accounting of \bidir{} data in the direct eliminator style.
It is possible to imagine syntactic sugar that doesn't require the $\overline{N,}$ and $\overline{x\Rightarrow}$ the in \case{} expression of the \rulename{\overleftarrow{ty}-\case{}} rule.
In the rule \rulename{\overrightarrow{ty}-\case{}<>} it is also possible to imagine some type constructor arguments being inferred.
These features and more will be subsumed by the dependent pattern matching of the next section. % , though this will complicate the meta-theory.
