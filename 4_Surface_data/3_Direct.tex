\section{Direct Elimination}

How should a program observe data?
Since a term of a given data type can only be created with one of the constructors from its definition, we can completely handle a data expression if each possible constructor is accounted for.
For instance, $\mathtt{Nat}$ has the two constructors $\mathtt{Z}$ and $\mathtt{S}$ (which holds the preceding number), so the expression $\mathsf{case}\,n\,\left\{ |\,Z\Rightarrow Z\,|\,S\,x\Rightarrow x\right\}$ will extract the proceeding number from $n$(or $0$ if $n=0$).
In this light, boolean case elimination corresponds to the if-then-else expression found in many mainstream languages. 

We will need to extend the syntax of $\mathsf{case}$s to support dependent type checking.
Specifically, we will need to add a \textbf{motive} annotation that allows the type checker to compute the output type of the branches if they vary in terms of the input.
For instance, recursively generating a vector of a given length\todo{give explicit example, note the motive}.
We may also want to use some values of the type level argument to calculate the motive, and type the branches.
This will be allowed with additional bindings in the motive and in each branch\todo{give explicit of data eliminations that uses these annotations}.
In general, motive annotations will be treated like the typing annotations in \ch{2}, in that the \ac{TAS} will only allow correct motives in a well typed term, and that the motive will be definitionally irrelevant.

This version of data can be given by extending the surface language syntax in \ch{2}, as in \Fref{data-defs}.
Data Type constructors and Data Term constructors are treated like functions.
This direct eliminator scheme, is roughly similar to how Coq handles data in it's core language.

\begin{figure}
\begin{tabular}{lcll}
\multicolumn{4}{l}{list of $O$, separated with $s$}\tabularnewline
$\overline{sO}$,$\overline{Os}$ & $\Coloneqq$ & $.$ & empty list\tabularnewline
 & $|$ & $sO\overline{sO}$ & extend list\tabularnewline
$\Delta,\varTheta$ & $\Coloneqq$ & $\overline{\left(x:M\right)\rightarrow}$ & telescope\tabularnewline
\multicolumn{4}{l}{data type identifier,}\tabularnewline
$D$ &  &  & \tabularnewline
\multicolumn{4}{l}{data constructor identifier,}\tabularnewline
$d$ &  &  & \tabularnewline
\multicolumn{4}{l}{contexts,}\tabularnewline
$\Gamma$ & $\Coloneqq$ & ... & \tabularnewline
 & $|$ & $\Gamma,\mathsf{data}\,D\,:\,\Delta\rightarrow\star\,\left\{ \overline{|\,d\,:\,\varTheta\rightarrow D\overline{m}}\right\} $ & data def extension\tabularnewline
 & $|$ & $\Gamma,\mathsf{data}\,D\,:\,\Delta\rightarrow\star$ & abstract data extension\tabularnewline
 \multicolumn{4}{l}{expressions,}\tabularnewline
$m,...$ & $\Coloneqq$ & ... & \tabularnewline
 & $|$ & $D$ & type cons.\tabularnewline
 & $|$ & $d$ & data cons.\tabularnewline
 & $|$ & $\mathsf{case}\,\overline{N,}n\,\left\{ \overline{|\overline{x\Rightarrow}(d\,\overline{y})\Rightarrow m}\right\} $ & data elim. without motive\tabularnewline
 & $|$ & $\mathsf{case}\,\overline{N,}n\,\left\langle \overline{x\Rightarrow}\,y:D\,\overline{x}\Rightarrow M\right\rangle \left\{ \overline{|\overline{x\Rightarrow}(d\,\overline{y})\Rightarrow m}\right\} $ & data elim. with motive\tabularnewline
\multicolumn{4}{l}{values,}\tabularnewline
$v$ & $\Coloneqq$ & ... & \tabularnewline
 & $|$ & $D\,\overline{v}$ & \tabularnewline
 & $|$ & $d\,\overline{v}$ & \tabularnewline
\end{tabular}

\caption{Surface Language (Direct Eliminator) Data}
\label{fig:surface-data-min}
\end{figure}

\todo{abreviations}
% \begin{figure}
% \begin{tabular}{llllll}
%  & syntax & written & when & for example & is written\tabularnewline
% leading separator & $sO\overline{sO}$ & $O\overline{sO}$ & clear from context & $,1,2,3$ & $1,2,3$\tabularnewline
% trailing separator & $\overline{sO}$ & $\overline{sO}s$ & clarifies intent & $\left(x\rightarrow y\rightarrow z\right)Id\,x\,y\,z$ & $x\rightarrow y\rightarrow z\rightarrow Id\,x\,y\,z$\tabularnewline
% non dependent telescope binder & $\left(x:M\right)\rightarrow$ & $M\rightarrow$ & $x$is not intended to bind  & $\left(x:Nat\right)\rightarrow\left(y:IsEven\,x\right)\rightarrow Nat$ & $\left(x:Nat\right)\rightarrow IsEven\,x\rightarrow Nat$\tabularnewline
% repeated application & $m\,n_{0}\,n_{1}\,...$ & $m\,\overline{n}$ & $\overline{n}=n_{0}\,n_{1}\,...$ &  & \tabularnewline
% \end{tabular}

% \todo[inline]{review if there are more abbreviations around?}

% \caption{Surface Language Abbreviations}
% \label{fig:surface-pre-syntax-data-abrev-1}
% \end{figure}

\todo[inline]{motive should not need to insist on the type info of the binder? grey out? Grey out things that are surface syntax but not needed for theory?}

The case eliminator first takes the explicit type arguments, followed by a \textbf{scrutinee}\footnote{also called a \textbf{discriminee}} of correct type.
Then optionally a motive that characterizes the output type of each branch with all the type arguments and scrutinee abstracted and in scope.
For instance, this case expression checks if a vector $x$ is empty,

\[
x:Vec\,\mathbb{B}\,1\vdash\mathsf{case}\,\mathbb{B},1,x\,\left\langle y\Rightarrow z\Rightarrow s:Vec\,y\,z\Rightarrow\mathbb{B}\right\rangle \left\{ |y\Rightarrow z\Rightarrow Nil\,-\Rightarrow True\,|\,y\Rightarrow z\Rightarrow Cons\,----\Rightarrow False\right\} 
\]

We include a little more syntax then is strictly necessary, since the $\mathbb{B},1,$ list could be inferred and the $y\Rightarrow z\Rightarrow$ binders are not needed in the branch.
This slightly verbose case eliminator syntax is designed to be forward compatible with the pattern matching system defined in the rest of this Chapter. 

Additionally we define telescopes, which generalize zero or more typed bindings\todo{Cite De Bruijn for telescopes}.
This allows a much cleaner definition of data then is otherwise possible.
Also we define syntactic lists that allow zero or more pieces of syntax.
Expressions in a list can be used to generalize dependent pairs, and can be type checked against a telescope.
For instance, the list $\mathtt{Nat},2,2,refl\,\mathtt{Nat}\,2$ type checks against $\left(X:\star\right)\rightarrow\left(y:X\right)\rightarrow\left(z:X\right)\rightarrow\left(-:Id\ X\,y\,z\right)$.
This becomes helpful in several situations, but especially when we need work with the listed arguments of the data type constructor.
We will allow several syntactic puns, such as treating telescopes as prefixes for function types. For instance, 
  if $\Delta=\left(y:\mathtt{Nat}\right)\rightarrow\left(z:\mathtt{Nat}\right)\rightarrow\left(-:Id\ \mathtt{Nat}\,y\,z\right)$
  then writing $f:\Delta\rightarrow\mathtt{Nat}$
  will be short hand for $f:\left(y:\mathtt{Nat}\right)\rightarrow\left(z:\mathtt{Nat}\right)\rightarrow Id\ \mathtt{Nat}\,y\,z\rightarrow\mathtt{Nat}$.
\todo{move this down to the typing section?}

In the presence of general recursion case elimination is powerful.
Well-founded recursion can be used to make structurally inductive computations that can be interpreted as proofs.\todo{say with rep example}.

Adding data allows for two additional sources of bad behavior.
Incomplete matches, and nontermination from non-strictly positive data. 

\subsubsection{Incomplete Eliminations}

Consider the match 

\[
x:\mathbb{N}\vdash\mathsf{case}\,x\,\left\langle s:\mathbb{N}\Rightarrow\mathbb{B}\right\rangle \left\{ |S-\Rightarrow True\right\} 
\]

This match will ``get stuck'' if $0$ is substituted for $x$.
Recall that the key theorem of the surface language is type soundness, ``well typed terms don't get stuck''.
Since verifying every constructor has a branch is relatively easy, the surface language \ac{TAS} will require every constructor to be matched in order to type check type check with direct elimination.
This is in contrast to most programming languages, which do allow incomplete patterns, though usually a warning is given, and a runtime error is raised if the scrutinee cannot be matched.

% Since this is a "well behaved" failure

This thesis already has a system for handling warnings and runtime errors through the cast language.
When we get to the cast language data in \ch{5}, we will allow non-exhaustive data to be reported as a warning and that will allow ``unmatched'' errors to be observed at runtime.

For similar reasons, in the direct eliminator scheme, we will insist that each constructor is matched at most once, so there is no ambiguity for how an case eliminated when using direct eliminations.

\subsubsection{(non-)Strict Positivity}

A more subtle concern is posed by data definitions that are not strictly positive.
Consider the following definition, \todo{think this can be simplified}

\begin{lstlisting}[basicstyle={\ttfamily\small}]
data Bad : * {
| C : (Bad -> Bad) -> Bad
};

selfApply : Bad -> Bad;
selfApply = \ b =>
  case b {
    | C f => f b
  };

loop : Bad;
loop = selfApply (C selfApply)
\end{lstlisting}

The $\mathtt{C}$ constructor in the definitions of $\mathtt{Bad}$ has a self reference in a negative position, $(\mathtt{Bad}\rightarrow\underline{\mathtt{Bad}})\rightarrow\mathtt{Bad}$. 

Non-strictly positive data definitions can cause non-termination, independent of the two other sources of non-termination already considered (general recursion and type-in-type).
Dependent type systems usually require a strictness check on data definitions to avoid this possibility.
However, this would disallow some useful constructions like higher order abstract syntax.
Since non-termination is already allowed in the surface \ac{TAS}, we will not restrict the surface language to strictly positive data.

\todo{who cam up with this first? Martin Lof?}

% co-inductive uses of data

\subsection{Type assignment System}

Before the typing rules for data can be considered, first some meta rules must be presented that will allow the simultaneous type-checking of lists and telescopes.
These rules are listed in \ref{fig:surface-data-meta-ty}, and are standard.
Telescopes are $\mathbf{ok}$ when they extend the context in an $\mathbf{ok}$ way.
Lists of expressions can be said to have the type of the telescope if every expression in the list types successively.

\begin{figure}
\[
\frac{\Gamma\,\mathbf{ok}}{\Gamma\vdash.\,\mathbf{ok}}\operatorname{\mathbf{ok}-Tel-empty}
\]

\[
\frac{\Gamma\vdash M:\star\quad\Gamma,x:M\vdash\Delta\,\mathbf{ok}}{\Gamma\vdash\left(x:M\right)\rightarrow\Delta\,\mathbf{ok}}\operatorname{\mathbf{ok}-Tel-ext}
\]

\[
\frac{\Gamma\,\mathbf{ok}}{\Gamma\vdash,:.}\operatorname{ty-ls-empty}
\]

\[
\frac{\Gamma,x:M\vdash\Delta\quad\Gamma\vdash m:M\quad\Gamma\vdash\overline{n,}\left[x\coloneqq m\right]:\Delta\left[x\coloneqq m\right]}{\Gamma\vdash m\overline{,n}\,:\,\left(x:M\right)\rightarrow\Delta}\operatorname{ty-ls-ext}
\]

\caption{Meta rules}
\label{fig:surface-data-meta-ty}
\end{figure}

Data definitions can be added to contexts if all of their constituents are well typed and \textbf{ok}.
The rules are listed in Figure \ref{fig:surface-data-ok}.
The $\operatorname{\mathbf{ok}-abs-data}$ rule allows data to be considered abstractly if it is formed with a plausible telescope.
$\operatorname{\mathbf{ok}-data}$ checks a full data definition with an abstract reference to a data definition in context, which allows recursive data definitions such as $\mathtt{Nat}$ which needs $\mathtt{Nat}$ to be in scope to define the $\mathtt{S}$ constructor.
This thesis does not formalize a syntax that adds data to context, though a very simple module system has been implemented in the prototype.
It is taken for granted that any well formed data in context is fine.\todo{awk}
This presentation of data definitions largely follows \cite{sjoberg2012irrelevance}. 

\begin{figure}
\[
\frac{\Gamma\vdash\Delta\,\mathbf{ok}}{\Gamma\vdash\mathsf{data}\,D\,\Delta\,\mathbf{ok}}\operatorname{\mathbf{ok}-abs-data}
\]

\[
\frac{\Gamma\vdash\mathsf{data}\,D\,\Delta\,\mathbf{ok}}{\Gamma,\mathsf{data}\,D\,\Delta\,\mathbf{ok}}\operatorname{\mathbf{ok}-data-ext}
\]
\[
\frac{\Gamma\vdash\mathsf{data}\,D\,\Delta\,\mathbf{ok}\quad\forall d.\Gamma,\mathsf{data}\,D\,\Delta\vdash\varTheta_{d}\,\mathbf{ok}\quad\forall d.\:\Gamma,\mathsf{data}\,D\,\Delta,\varTheta_{d}\vdash\overline{m}_{d}:\Delta}{\Gamma\vdash\mathsf{data}\,D\,:\,\Delta\left\{ \overline{|\,d\,:\,\varTheta_{d}\rightarrow D\overline{m}_{d}}\right\} \,\mathbf{ok}}\operatorname{\mathbf{ok}-data}
\]

\[
\frac{\Gamma\vdash\mathsf{data}\,D\,:\,\Delta\left\{ \overline{|\,d\,:\,\varTheta\rightarrow D\overline{m}}\right\} \,\mathbf{ok}}{\Gamma,\mathsf{data}\,D\,:\,\Delta\left\{ \overline{|\,d\,:\,\varTheta\rightarrow D\overline{m}}\right\} \,\mathbf{ok}}\operatorname{\mathbf{ok}-data-ext}
\]

\caption{Surface Language Data ok}
\label{fig:surface-data-ok}
\end{figure}

The type assignment system with direct elimination must be extended with the rules in \ref{fig:surface-data-ty}.
These rules make use of several convenient shorthands: $\mathsf{data}\,D\,\Delta\in\Gamma$ and $d\,:\,\varTheta\rightarrow D\overline{m}\in\Gamma$ extract the type constructor definitions and data constructor definitions from the context respectively.
$\operatorname{ty-TCon}$ and $\operatorname{ty-Con}$ allow type and data constructors to be used as functions of appropriate type.
The $\operatorname{ty-}\mathsf{case}<>$ rule types a case expression by ensuring that the correct data definition for $D$ is in context, the scrutinee $n$ has the correct type, the motive $M$ is well formed under the type arguments and the scrutinee, % The motive requires checking in the case of empty data
  finally every data constructor is verified to have a corresponding branch.
$\operatorname{ty-}\mathsf{case}$ allows for the same typing logic, but does not require the motive be annotated in syntax.
In both rules we allow telescopes to rename their variables with the shorthand $\overline{x}:\Delta$.

\begin{figure}
\[
\frac{\Gamma\,\mathbf{ok}\quad\mathsf{data}\,D\,\Delta\in\Gamma}{\Gamma\vdash D\,:\,\Delta\rightarrow\star}\operatorname{ty-TCon}
\]

\[
\frac{\Gamma\,\mathbf{ok}\quad d\,:\,\varTheta\rightarrow D\overline{m}\in\Gamma}{\Gamma\vdash d\,:\,\varTheta\rightarrow D\overline{m}}\operatorname{ty-Con}
\]

\[
\frac{\begin{array}{c}
\mathsf{data}\,D\,\Delta\left\{ ...\right\} \in\Gamma\\
\Gamma\vdash n:D\overline{N}\\
\Gamma,\overline{x}:\Delta,z:D\,\overline{x}\vdash M:\star\\
\forall\:d\,:\,\varTheta\rightarrow D\overline{o}\in\Gamma.\\
\Gamma,\overline{y}_{d}:\varTheta\vdash m_{d}\left[\overline{x}\coloneqq\overline{o}\left[\varTheta\coloneqq\overline{y}_{d}\right]\right]:M\left[\overline{x}\coloneqq\overline{o}\left[\varTheta\coloneqq\overline{y}_{d}\right],z\coloneqq d\,\overline{y}_{d}\right]\\
\mathrm{No\ duplicate\ branches}
\end{array}}{\begin{array}{c}
\Gamma\vdash\mathsf{case}\,\overline{N,}n\,\left\langle \overline{x\Rightarrow}z:D\,\overline{x}\Rightarrow M\right\rangle \left\{ \overline{|\,\overline{x\Rightarrow}\,d\overline{y}_{d}\,\Rightarrow m_{d}}\right\} \\
:M\left[\overline{x}\coloneqq\overline{N},z\coloneqq n\right]
\end{array}}\operatorname{ty-}\mathsf{case}<>
\]

\[
\frac{\begin{array}{c}
\mathsf{data}\,D\,\Delta\left\{ ...\right\} \in\Gamma\\
\Gamma\vdash n:D\overline{N}\\
\Gamma,\overline{x}:\Delta,z:D\,\overline{x}\vdash M:\star\\
\forall\:d\,:\,\varTheta\rightarrow D\overline{o}\in\Gamma.\\
\Gamma,\overline{y}_{d}:\varTheta\vdash m_{d}\left[\overline{x}\coloneqq\overline{o}\left[\varTheta\coloneqq\overline{y}_{d}\right]\right]:M\left[\overline{x}\coloneqq\overline{o}\left[\varTheta\coloneqq\overline{y}_{d}\right],z\coloneqq d\,\overline{y}_{d}\right]
\end{array}}{\begin{array}{c}
\Gamma\vdash\mathsf{case}\,\overline{N,}n\,\left\{ \overline{|\,\overline{x\Rightarrow}\,d\overline{y}_{d}\,\Rightarrow m_{d}}\right\} \\
:M\left[\overline{x}\coloneqq\overline{N},z\coloneqq n\right]
\end{array}}\operatorname{ty-}\mathsf{case}
\]

\caption{Surface Language Data Typing}
\label{fig:surface-data-ty}
\end{figure}

\todo[inline]{suspect this also hinges on regularity, which should be addressed more directly}

Extensions to the parallel reduction rules are listed in Figure \ref{fig:surface-data-red}.
They follow the scheme of parallel reductions laid out in \ch{2}.
The $\textrm{\ensuremath{\Rrightarrow}-\ensuremath{\mathsf{case}}-red}$ rule\footnote{Also called $\iota$, or Iota reduction}\todo{mention beta in footnote in CH2} reduces a case expression by choosing the appropriate branch.
The $\textrm{\ensuremath{\Rrightarrow}-\ensuremath{\mathsf{case}}<>-red}$ rule removes the motive annotation, much like the annotation rule in \ch{2}.
The rules $\textrm{\ensuremath{\Rrightarrow}-\ensuremath{\mathsf{case}}<>}$, $\textrm{\ensuremath{\Rrightarrow}-}D$, and $\textrm{\ensuremath{\Rrightarrow}-}d$ keep the $\Rrightarrow$ relation reflexive.
The reduction relation is be generalized to lists in the expected way.

\begin{figure}
\[
\frac{\begin{array}{c}
\overline{N}\Rrightarrow\overline{N'}\quad\overline{m}\Rrightarrow\overline{m'}\\
\exists\overline{x\Rightarrow}(d\,\overline{y}_{d})\Rightarrow m_{d}\in\left\{ \overline{|\,\overline{x\Rightarrow}(d'\,\overline{y}_{d'})\Rightarrow m_{d'}}\right\} \\
m_{d}\Rrightarrow m_{d}'
\end{array}}{\mathsf{case}\,\overline{N,}\,d\overline{m}\,\left\{ \overline{|\,\overline{x\Rightarrow}(d'\,\overline{y}_{d'})\Rightarrow m_{d'}}\right\} \Rrightarrow m_{d}\left[\overline{x}\coloneqq\overline{N'},\overline{y}_{d}\coloneqq\overline{m'}\right]}\,\textrm{\ensuremath{\Rrightarrow}-\ensuremath{\mathsf{case}}-red}
\]

\[
\frac{\begin{array}{c}
\overline{N}\Rrightarrow\overline{N'}\quad m\Rrightarrow m'\\
\forall\overline{x\Rightarrow}(d\,\overline{y}_{d})\Rightarrow m_{d}\in\left\{ \overline{|\,\overline{\Rightarrow x}\Rightarrow(d'\,\overline{y}_{d'})\Rightarrow m_{d'}}\right\} .\:m_{d}\Rrightarrow m_{d}'
\end{array}}{\mathsf{case}\,\overline{N,}\,m\,\left\langle ...\right\rangle \left\{ \overline{|\,\overline{\Rightarrow x}\Rightarrow(d'\,\overline{y}_{d'})\Rightarrow m_{d'}}\right\} \Rrightarrow\mathsf{case}\,\overline{N',}\,m'\left\{ \overline{|\,\overline{\Rightarrow x}\Rightarrow(d'\,\overline{y}_{d'})\Rightarrow m_{d'}'}\right\} }\,\textrm{\ensuremath{\Rrightarrow}-\ensuremath{\mathsf{case}}<>-red}
\]

\[
\frac{\begin{array}{c}
\overline{N}\Rrightarrow\overline{N'}\quad m\Rrightarrow m'\\
M\Rrightarrow M'\\
\forall\overline{x\Rightarrow}(d'\,\overline{y}_{d'})\Rightarrow m_{d'}\in\left\{ \overline{|\,\overline{\Rightarrow x}\Rightarrow(d\,\overline{y}_{d})\Rightarrow m_{d}}\right\} .\:m_{d'}\Rrightarrow m'_{d'}
\end{array}}{\begin{array}{c}
\mathsf{case}\,\overline{N,}\,m\,\left\langle \overline{x\Rightarrow}z:D\,\overline{x}\Rightarrow M\right\rangle \left\{ \overline{|\,\overline{\Rightarrow x}\Rightarrow(d\,\overline{y}_{d})\Rightarrow m_{d}}\right\} \Rrightarrow\\
\mathsf{case}\,\overline{N,}\,m'\,\left\langle \overline{x\Rightarrow}z:D\,\overline{x}\Rightarrow M'\right\rangle \left\{ \overline{|\,\overline{\Rightarrow x}\Rightarrow(d\,\overline{y}_{d})\Rightarrow m'_{d}}\right\} 
\end{array}}\,\textrm{\ensuremath{\Rrightarrow}-\ensuremath{\mathsf{case}}<>}
\]

\[
\frac{\,}{D\Rrightarrow D}\,\textrm{\ensuremath{\Rrightarrow}-}D
\]
\[
\frac{\,}{d\Rrightarrow d}\,\textrm{\ensuremath{\Rrightarrow}-}d
\]

\caption{Surface Language Data Reduction}
\label{fig:surface-data-red}
\end{figure}

We are now in a position to select a sub relation of $\Rrightarrow$ reductions that will be used to characterize call-by-value evaluation.
This relation could be used used to prove type safety, and is close to the reduction used in the implementation.
The rules are listed in \Fref{surface-data-cbv}.\todo[inline]{extend cbv over lists}

\begin{figure}
\[
\frac{\,}{\mathsf{case}\,\overline{N,}\,n\,\left\langle ...\right\rangle \left\{ \overline{|\,\overline{\Rightarrow x}\Rightarrow(d'\,\overline{y}_{d'})\Rightarrow m_{d'}}\right\} \rightsquigarrow\mathsf{case}\,\overline{N,}\,n\left\{ \overline{|\,\overline{\Rightarrow x}\Rightarrow(d'\,\overline{y}_{d'})\Rightarrow m_{d'}}\right\} }\textrm{\ensuremath{\rightsquigarrow}-\ensuremath{\mathsf{case}}<>}
\]

\[
\frac{\exists\overline{x\Rightarrow}(d\,\overline{y}_{d})\Rightarrow m_{d}\in\left\{ \overline{|\,\overline{x\Rightarrow}(d'\,\overline{y}_{d'})\Rightarrow m_{d'}}\right\} }{\mathsf{case}\,\overline{V,}\,d\overline{v}\,\left\{ \overline{|\,\overline{x\Rightarrow}(d'\,\overline{y}_{d'})\Rightarrow m_{d'}}\right\} \rightsquigarrow m_{d}\left[\overline{x}\coloneqq\overline{V},\overline{y}_{d}\coloneqq\overline{v}\right]}\,\textrm{\ensuremath{\rightsquigarrow}-\ensuremath{\mathsf{case}}-red}
\]

\[
\frac{\overline{N}\rightsquigarrow\overline{N'}}{\mathsf{case}\,\overline{N,}\,n\,\left\{ \overline{|\,\overline{x\Rightarrow}(d'\,\overline{y}_{d'})\Rightarrow m_{d'}}\right\} \rightsquigarrow\mathsf{case}\,\overline{N',}\,n\,\left\{ \overline{|\,\overline{x\Rightarrow}(d'\,\overline{y}_{d'})\Rightarrow m_{d'}}\right\} }\,
\]

\[
\frac{n\rightsquigarrow n'}{\mathsf{case}\,\overline{V,}\,n\,\left\{ \overline{|\,\overline{x\Rightarrow}(d'\,\overline{y}_{d'})\Rightarrow m_{d'}}\right\} \rightsquigarrow\mathsf{case}\,\overline{V,}\,n'\,\left\{ \overline{|\,\overline{x\Rightarrow}(d'\,\overline{y}_{d'})\Rightarrow m_{d'}}\right\} }\,
\]

\caption{Surface Language Data CBV}
\label{fig:surface-data-cbv}
\end{figure}

\todo[inline]{extend step over lists}

Finally we characterize what it means for a context to be empty in the presence of data in \Fref{surface-data-empty}.

\begin{figure}
\[
\frac{\ }{\lozenge\,\mathbf{Empty}}\operatorname{Empty-ctx}
\]

\[
\frac{\Gamma\,\mathbf{Empty}\quad\Gamma\vdash\mathsf{data}\,D\,:\,\Delta\left\{ \overline{|\,d\,:\,\varTheta_{d}\rightarrow D\overline{m}_{d}}\right\} \,\mathbf{ok}}{\Gamma,\mathsf{data}\,D\,:\,\Delta\left\{ \overline{|\,d\,:\,\varTheta_{d}\rightarrow D\overline{m}_{d}}\right\} \,\mathbf{Empty}}\operatorname{Empty-ctx}
\]

\caption{Surface Language Empty}
\label{fig:surface-data-empty}
\end{figure}

While a system with a similar presentation has proven type soundness in \cite{sjoberg2012irrelevance}\todo{switch to vilhelms thesis?}, we will not prove the type soundness of the system here.
For clarity we will list the important properties as conjectures.
\begin{conjecture}
The surface language extended with data and elimination preserves types over reduction.
\end{conjecture}

% ...
\begin{conjecture}
The surface language extended with data and elimination has progress
if $\Gamma\,\mathbf{Empty}$, \textup{$\Gamma\vdash m:M$}, then $m$
is a value, or $m\rightsquigarrow m'$ .
\end{conjecture}

% ...
\begin{conjecture}
The surface language extended with data and elimination is type sound.
\end{conjecture}


\subsection{Bidirectional Type Checking}

A bidirectional type checking procedure exists for the type assignment rules listed above.
An outline of these rules is in \Fref{surface-data-empty}.
As noted in \cite{10.1145/3450952}, the bidirectional rules around data are open to some interpretation.
The dependent case simplifies these questions since only a few rules are sensable.

The type of data constructors and type constructors can always be inferred.
If the motive does not depend on the scrutinee or type arguments, it can be used to check against the type of the branches.
An unmotivated $\mathsf{case}$ will be type checked.
A $\mathsf{case}$ with a motive will have its type inferred.
\todo{actually more complicated then that}

\begin{figure}
\[
\frac{\mathsf{data}\,D\,\Delta\in\Gamma}{\Gamma\vdash D\overrightarrow{\,:\,}\Delta\rightarrow*}\operatorname{\overrightarrow{ty}-TCon}
\]

\[
\frac{d\,:\,\varTheta\rightarrow D\overline{m}\in\Gamma}{\Gamma\vdash d\overrightarrow{\,:\,}\varTheta\rightarrow D\overline{m}}\operatorname{\overrightarrow{ty}-Con}
\]

\[
\frac{\begin{array}{c}
\mathsf{data}\,D\,\Delta\left\{ ...\right\} \in\Gamma\\
\Gamma\vdash\overline{N}\overleftarrow{\,:\,}\Delta\quad\Gamma\vdash n\overleftarrow{\,:\,}D\overline{N}\\
\Gamma,\overline{x}:\Delta,z:D\,\overline{x}\vdash M\overleftarrow{\,:\,}\star\\
\forall\:d\,:\,\varTheta\rightarrow D\overline{o}\in\Gamma.\quad\Gamma,\overline{y}_{d}:\varTheta\vdash m_{d}\left[\overline{x}\coloneqq\overline{o}\right]\overleftarrow{\,:\,}M\left[\overline{x}\coloneqq\overline{o},z\coloneqq d\,\overline{y}_{d}\right]
\end{array}}{\begin{array}{c}
\Gamma\vdash\mathsf{case}\,\overline{,N},n\,\left\langle \overline{x\Rightarrow}z:D\,\overline{x}\Rightarrow M\right\rangle \left\{ \overline{|\,\overline{x\Rightarrow}(d\,\overline{y}_{d})\Rightarrow m_{d}}\right\} \\
\overrightarrow{\,:\,}M\left[\overline{x}\coloneqq\overline{N},z\coloneqq n\right]
\end{array}}\operatorname{\overrightarrow{ty}-}\mathsf{case}<>
\]

\[
\frac{\begin{array}{c}
\mathsf{data}\,D\,\Delta\left\{ ...\right\} \in\Gamma\\
\Gamma\vdash\overline{N}\overleftarrow{\,:\,}\Delta\quad\Gamma\vdash n\overrightarrow{\,:\,}D\overline{N}\\
\Gamma\vdash M\overleftarrow{\,:\,}\star\\
\forall\:d\,:\,\varTheta\rightarrow D\overline{o}\in\Gamma.\quad\Gamma,\overline{y}_{d}:\varTheta\vdash m_{d}\left[\overline{x}\coloneqq\overline{o}\right]\overleftarrow{\,:\,}M
\end{array}}{\Gamma\vdash\mathsf{case}\,\overline{N,}n\,\left\{ \overline{|\overline{x\Rightarrow}\,(d\,\overline{y}_{d})\Rightarrow m_{d}}\right\} \overleftarrow{\,:\,}M}\operatorname{\overleftarrow{ty}-}\mathsf{case}<>
\]

\todo[inline]{reparam o in $\left[\overline{x}\coloneqq\overline{o}\right]$ also}

\caption{Surface Language Bidirectional type checking }
\label{fig:surface-data-bi-ty}
\end{figure}

We can confidently conjecture that the desired bidirectional properties hold.
\begin{conjecture}
The data extension to the bidirectional surface language is type sound.
\end{conjecture}

% ...
\begin{conjecture}
The data extension to the bidirectional surface language is weakly annotatable from the data extension of the surface language.\todo{symbolically}
\end{conjecture}

This is a minimal (and somewhat crude) accounting of bidirectional data.
It is possible to imagine syntactic sugar that doesn't require the $\overline{N,}$ and $\overline{x\Rightarrow}\,$ the in case expression of the $\operatorname{\overleftarrow{ty}-}\mathsf{case}<>$ rule.
In the dependent rule $\operatorname{\overrightarrow{ty}-}\mathsf{case}<>$ it is also possible to imagine some type constructor arguments being inferred.
These features and more will be subsumed by the dependent pattern matching of the next section, though this will complicate the meta-theory.
