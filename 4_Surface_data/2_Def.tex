\section{Data}

%What data defs/ constructors
% The langugae defined in this thesus uses data definitions like those found in systems like Agda and Coq. 

A dependent data type is defined by a type constructor indexed by arguments, and a set of data constructors that tag data and characterize their arguments.
Several familiar data types are defined in Figure \ref{fig:data-defs}.
For example, the data type $\mathtt{Nat}$ is defined with the type constructor $\mathtt{Nat}$ (which has no type arguments), the data constructors $\mathtt{Z}$ which takes no further information and the data constructor $\mathtt{S}$ which is formed with the prior number.
The data type $\mathtt{Vec}$ has two type arguments corresponding to the type contained in the vector and its length; it has two data constructors that allow building an empty vector, or to add an element to the front of an existing vector.

Data defined in this style is simple to build and reason about, since data can only be created from its constructors.
Unfortunately the details of data elimination are a little more involved.

\begin{figure}
\begin{lstlisting}[basicstyle={\ttfamily\small}]
data Void : * {};

data Unit : * {
| tt : Unit
};

data Bool : * {
| True : Bool
| False : Bool
};
 
data Nat : * {
| Z : Nat
| S : Nat -> Nat
};
three : Nat;
three = S (S (S Z)));
-- Syntactic sugar allows 3 = S (S (S Z)))

data Vec : (A : *) -> Nat -> * {
| Nil  : (A : *) -> Vec A Z
| Cons : (A : *) -> A -> (x : Nat)
        -> Vec A x -> Vec A (S x)
};

someBools : Vec Bool 2;
someBools = Cons Bool True 1 (Cons Bool False 0 (Nil Bool));

data Id : (A : *) -> A -> A -> * {
| refl  : (A : *) -> (a : A) -> Id A a a
};

threeEqThree : Id Nat three three;
threeEqThree = refl Nat three;
\end{lstlisting}
% include if notation is used
% -- Syntactic sugar expands list notation,
% -- for example
% -- [True, False]<Bool> =
% -- Cons Bool True 1 (Cons Bool False 1 (Nil Bool))
 \caption{Definitions of Common Data Types}
\label{fig:data-defs}
\end{figure}
