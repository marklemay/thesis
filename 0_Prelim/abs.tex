Dependent types offer a uniform foundation for both proof systems and programming languages.
While the proof systems built with dependent types have become relatively popular, dependently typed programming languages are far from mainstream.

One key issue with existing dependently typed languages is the overly conservative definitional equality that programmers are forced to use.
When combined with a traditional typing workflow, these systems can be quite challenging and require a large amount of expertise to master.

This thesis explores an alternative workflow and a more liberal handling of equality.
Programmers are given warnings that contain the same information as the type errors that would be given by an existing system.
Programmers can run these programs optimistically, and they will behave appropriately unless a direct contradiction confirming the warning is found.

This is achieved by localizing equality constraints using a new form of elaboration based on bidirectional type inference.
These local checks, or casts, are given a runtime behavior (similar to those of contract and monitors).
The elaborated terms have a weakened form of type soundness, they will not get stuck without an explicit counter example.

The language explored in this thesis will be a calculus of constructions like language with recursion, type-in-type, user defined data and pattern matching.

Several meta-theoretic results will be presented.
The key result is that the core language, called the \textbf{cast system}, ``will not get stuck without a counter example''; a result called \textbf{cast soundness}.
A proof of cast soundness is fully worked out for the fragment of the system without user defined data, and a Coq proof is available. Several other properties based on the gradual guarantees of gradual typing are also presented.
In the presence of user defined data and pattern matching these properties are conjectured to hold.

A prototype implementation of this work is available.

\todo[inline]{review of not new stuff}