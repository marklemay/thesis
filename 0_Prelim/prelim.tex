% This file contains all the necessary setup and commands to create
% the preliminary pages according to the buthesis.sty option.

\title{A Dependently Typed Programming Language with Dynamic Equality}

\author{Mark Lemay}

% Phd = 2
\degree=2

\prevdegrees{B.S., Rochester Institute of technology, 2010}

\department{Department of Computer Science}

\defenseyear{2022}
\degreeyear{2022}

% For each reader, specify appropriate label {First, Second, Third},
% then name, and title. IMPORTANT: The title should be:
%   "Professor of Electrical and Computer Engineering",
% or similar, but it MUST NOT be:
%   Professor, Department of Electrical and Computer Engineering"
% or you will be asked to reprint and get new signatures.
\reader{First}{First M. Last, PhD}{Professor of Computer Science}
\reader{Second}{First M. Last}{Professor of Computer Science}
\reader{Third}{First M. Last}{Assistant Professor of \ldots}
%% \todo[inline]{double check honerifics}

% The Major Professor is the same as the first reader, but must be
% specified again for the abstract page.
\numadvisors=1
\majorprof{First M. Last, PhD}{{Professor of Computer Science}}

%%%%%%%%%%%%%%%%%%%%%%%%%%%%%%%%%%%%%%%%%%%%%%%%%%%%%%%%%%%%%%%%  

%                       PRELIMINARY PAGES
% According to the BU guide the preliminary pages consist of:
% title, copyright (optional), approval,  acknowledgments (opt.),
% abstract, preface (opt.), Table of contents, List of tables (if
% any), List of illustrations (if any). The \tableofcontents,
% \listoffigures, and \listoftables commands can be used in the
% appropriate places. For other things like preface, do it manually
% with something like \newpage\section*{Preface}.

% This is an additional page to print a boxed-in title, author name and
% degree statement so that they are visible through the opening in BU
% covers used for reports. This makes a nicely bound copy. Uncomment only
% if you are printing a hardcopy for such covers. Leave commented out
% when producing PDF for library submission.
%\buecethesistitleboxpage

% Make the titlepage based on the above information.  If you need
% something special and can't use the standard form, you can specify
% the exact text of the titlepage yourself.  Put it in a titlepage
% environment and leave blank lines where you want vertical space.
% The spaces will be adjusted to fill the entire page.
\maketitle
\cleardoublepage

% The copyright page is blank except for the notice at the bottom. You
% must provide your name in capitals.
\copyrightpage
\cleardoublepage

% Now include the approval page based on the readers information
% Once the approval page is approved by the Mugar Library staff, please
% comment out the "\approvalpagewithcomment" line and uncomment "\approvalpage"
\approvalpagewithcomment
%\approvalpage
\cleardoublepage

% Here goes your favorite quote. This page is optional.
\newpage
%\thispagestyle{empty}
\phantom{.}
\vspace{4in}

\begin{singlespace}
\begin{quote}
  What I mean is that if you really want to understand something, the best way is to try and explain it to someone else. That forces you to sort it out in your own mind. And the more slow and dim-witted your pupil, the more you have to break things down into more and more simple ideas. And that’s really the essence of programming. By the time you’ve sorted out a complicated idea into little steps that even a stupid machine can deal with, you’ve certainly learned something about it yourself. The teacher usually learns more than the pupil. Isn’t that true?

  \hfill{Douglas Adams, Dirk Gently's Holistic Detective Agency}
\end{quote}
\end{singlespace}

% \vspace{0.7in}
%
% \noindent
% [The descent to Avernus is easy; the gate of Pluto stands open night
% and day; but to retrace one's steps and return to the upper air, that
% is the toil, that the difficulty.]

\cleardoublepage

% The acknowledgment page should go here. Use something like
% \newpage\section*{Acknowledgments} followed by your text.
\newpage
\section*{\centerline{Acknowledgments}}
First and foremost my fiancé Stephanie Savir, I could not have finished this without her support. %% And she fixed some typeos
% 2021 was a rough year, and her positivity
 
Boston University is an excellent school to learn Computer Science.
It is so full of intelligent, passionate, and kind people that any list of people will feel incomplete.
Given that I especially want to thank Dr.(!) Tomislav Petrovic his wife Ana and their adorably strong willed children for the longest friendship at BU;
Qiancheng Fu, Cheng Zhang, and William Blair for their discussions, collaborations and friendship;
and my advisor Dr. Hongwei Xi who has been more supportive than I would have thought possible.
Also he is a genius.
Thanks to Malavika Vishwanath of the Writing Assistance center for reviewing drafts of this thesis.
Finally to the administrators (especially Kori MacDonald) who have always managed to get paperwork where it is needed to go in spite of me.

My family has also supported, encouraged and tolerated this thesis process.
Matt Lemay and his wife Alex and their dog Lilly were always ready to provide a needed destraction and encouragement\footnote{``Cs get degrees bro''}.
My parents Bob and Carol Lemay also need to be thanked for their support.

Our roommates and friends Eric Gibbs and Jess Noble (and pets Padfoot, Crookshanks, Ms. Noris) for keeping us sane(?) during the pandemic lockdown.
There are too many other friends to list here, but special thanks to Ramsay Hoguet for proofreading a draft of this thesis.

\todo[inline]{Anonoumous people from discord}
\todo[inline]{Steph's Fam?}
\todo[inline]{Alley S for chairing if not recorded elsewhere}
\todo[inline]{other people?}
\todo[inline]{sign name?}
% \vskip 1in

% \noindent
% Mark Lemay\\
% % Professor\\
% % ECE Department
\cleardoublepage

% The abstractpage environment sets up everything on the page except
% the text itself.  The title and other header material are put at the
% top of the page, and the supervisors are listed at the bottom.  A
% new page is begun both before and after.  Of course, an abstract may
% be more than one page itself.  If you need more control over the
% format of the page, you can use the abstract environment, which puts
% the word "Abstract" at the beginning and single spaces its text.

\begin{abstractpage}
Dependent types offer a uniform foundation for both proof systems and programming languages.
While the proof systems built with dependent types have become relatively popular, dependently typed programming languages are far from mainstream.

One key issue with existing dependently typed languages is the overly conservative definitional equality that programmers are forced to use.
When combined with a traditional typing workflow, these systems can be quite challenging and require a large amount of expertise to master.

This thesis explores an alternative workflow and a more liberal handling of equality.
Programmers are given warnings that contain the same information as the type errors that would be given by an existing system.
Programmers can run these programs optimistically, and they will behave appropriately unless a direct contradiction confirming the warning is found.

This is achieved by localizing equality constraints using a new form of elaboration based on bidirectional type inference.
These local checks, or casts, are given a runtime behavior (similar to those of contract and monitors).
The elaborated terms have a weakened form of type soundness, they will not get stuck without an explicit counter example.

The language explored in this thesis will be a calculus of constructions like language with recursion, \tit, data types with dependent indexing and pattern matching.

Several meta-theoretic results will be presented.
The key result is that the core language, called the \textbf{cast system}, ``will not get stuck without a counter example''; a result called \textbf{cast soundness}.
A proof of cast soundness is fully worked out for the fragment of the system without user defined data, and a Coq proof is available. Several other properties based on the gradual guarantees of gradual typing are also presented.
In the presence of user defined data and pattern matching these properties are conjectured to hold.

A prototype implementation of this work is available.

\todo[inline]{review of not new stuff}
\end{abstractpage}
\cleardoublepage

% Now you can include a preface. Again, use something like
% \newpage\section*{Preface} followed by your text

% Table of contents comes after preface
\tableofcontents
\cleardoublepage

% If you do not have tables, comment out the following lines
% \newpage
% \listoftables
% \cleardoublepage

% If you have figures, uncomment the following line
\newpage
\listoffigures
\cleardoublepage

% List of Abbrevs is NOT optional (Martha Wellman likes all abbrevs listed)
\chapter*{List of Abbreviations}

{\bf The list below must be in alphabetical order as per BU library instructions or it will be returned to you for re-ordering.}

% \begin{center}
%   \begin{tabular}{lll}
%     \hspace*{2em} & \hspace*{1in} & \hspace*{4.5in} \\
%     CAD  & \dotfill & Computer-Aided Design \\
%     CO   & \dotfill & Cytochrome Oxidase \\
%     DOG  & \dotfill & Difference Of Gaussian (distributions) \\
%     FWHM & \dotfill & Full-Width at Half Maximum \\
%     LGN  & \dotfill & Lateral Geniculate Nucleus \\
%     ODC  & \dotfill & Ocular Dominance Column \\
%     PDF  & \dotfill & Probability Distribution Function \\
%     $\mathbb{R}^{2}$  & \dotfill & the Real plane \\
%   \end{tabular}
% \end{center}
\cleardoublepage

\newpage
\endofprelim
