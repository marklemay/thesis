\addcontentsline{toc}{chapter}{Curriculum Vitae}
% Both doctoral and master’s degree candidates must include a vita in their work. The
% vita is a short description of your professional life up to the point of being awarded
% your degree. The vita may be written in CV, résumé, or narrative format. Include the
% following basic information: your full name, and a contact address (e.g., your
% department) where you expect you can be reached for at least the next 1–2 years. A vita
% may also include (but is not limited to) prior education, degrees, awards or honors,
% professional positions held, and publications. Please try to limit the vita to three or four
% pages.

\begin{center}
{\LARGE {\bf CURRICULUM VITAE}}\\
\vspace{0.5in}
{\large {\bf Mark Lemay}}
\end{center}
PhD candidate in Boston University's Computer Science Department

\todo[inline]{this is really silly.  It will be out of date the minute it is printed}

\section*{Areas of Interest}
\begin{itemize}
  \item Primarily interested in type theory, formal methods, theorem proving technologies, and software engineering practices.
  \item Have some experience with computer graphics, machine learning, and web development.
\end{itemize}

% Basically, this needs to be worked out by each individual, however the same format, margins, typeface, and type size must be used as in the rest of the dissertation. 
\section*{Publications}
\todo[inline]{some of these citaitons are pulled from google scholar, wich frequently messes up info}
% \subsection*{First Author}
\begin{itemize}
  \item \textbf{Understanding Java usability by mining GitHub repositories}\cite{lemay2019understanding}
  \item \textbf{Automated Provenance Analytics: A Regular Grammar Based Approach with Applications in Security}\cite{lemay2017automated}
\end{itemize}
% Towards Scalable Cluster Auditing through Grammatical Inference over Provenance Graphs

\subsection*{Extended Abstracts}
\begin{itemize}
  \item \textbf{Gradual Correctness: a Dynamically Bidirectional Full-Spectrum Dependent Type Theory}\cite{gradualcorrectnessea}
  \item \textbf{Developing a Dependently Typed Language with Runtime Proof Search}\cite{extendedabstract}
\end{itemize}


\section*{Education}
\subsection*{Boston University}
PhD Candidate in Computer Science, September 2014 - Present
\begin{itemize}
  \item Developed a new dependently typed programming language with a unique equality system for my dissertation.
  \item Assisted teaching classes in programming languages, algorithms, python, and web development.
  \item Helped produce a Haskell based curriculum for the programming languages course.
\end{itemize}

\subsection*{Rochester Institute of Technology}
BS in Computational Mathematics, September 2007 - November 2010

\subsection*{Unaffiliated}
Co-founded \textbf{Math for People}\footnote{\url{https://www.meetup.com/Math-for-People/?_cookie-check=SbY9vQWXdNwWXB6y}}

\section*{Work}
\subsection*{Lincoln Laboratory}
Intern,	May 2016 - August 2016

Produced analytics for data provenance, resulting in \cite{lemay2017automated}.

\subsection*{DreamWorks Animation}
Research and Development Intern,	May 2015 - August 2015
 
Added features to DreamWorks’ animation tool, Premo.

\subsection*{Safari Books Online}
Software Engineer,	March 2012 - August 2014
\begin{itemize}
  \item Built a natural language processing system to parse academic citations.
  \item Helped to build De Gruyter Online, Loeb Classical Library and other sites running the PubFactory platform.
\end{itemize}
