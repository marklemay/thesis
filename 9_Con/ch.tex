\chapter{Conclusion}
\label{chapter:Conclusion}
\thispagestyle{myheadings}

This thesis has attempted to articulate and address a common hesitation
around dependent types. Programmers do not want do be interrupted.
Especially if the interruption is about a chance of an error. Addressing
this legitimate concern has lead to a new way of treating warnings
in a dependent type system. By creating a parallel system where checks
are made and given runtime behavior, programmers still get all of
the benefits, but fewer drawbacks of dependent type systems.

This turned out to be surpassingly more subtle then expected. As we
saw in Chapter 3, the programmer's intent needs to be inferred, so
that a reasonable check can be localized. This is possible through
an extension to bidirectional type checking. Runtime errors complicate
the semantics, this issue was sidestepped by applying a new relation
that extracts blame. Checks need their own runtime behavior, which
is strait forward in the pure functional setting.

Further, user defined data turned out to be far more complicated then
anticipated. Extending pattern matching to track equalities seems
like a clever idea, however the formalism in Chapter 5 is not as simple
as we might like. It is unclear if a simpler approach is possible.

Finally, in Chapter 6, there are several ways to improve the current
system and build towards future work.\todo{more}

The approach to warnings presented in this thesis may be more generally
applicable. Type systems can still be designed to harshly avoid errors,
but by creating a parallel system where checks are made and given
runtime behavior, the type system will be less imposing to new users.
For instance, many interesting linear type systems are currently being
explored, allowing warnings may make these systems more usable to
programmers who are not used to those restrictions.

Dependent types have seemed on the verge of mainstream use for decades.
While dependent types are not mainstream yet, they have the unique
potential to bridge the gap between those who program and those who
prove. Each community has built invaluable expertise that could benefit
the other. Once that connection is made solid, more robust software
is the least we should expect.

While this thesis has not single handedly made this connection, I
think it is a necessary piece of the puzzle.
