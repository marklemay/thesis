\section{\CSys{}}
 
% why cast soundness?
In a programming language, type soundness proves some undesirable behaviors are unreachable from a well typed term.
% Recall that type soundness is the primary property for a typed programming language to exhibit.
How should this apply to the \clang{}, where bad behaviors are intended to be reachable?
The \clang{} allows the program to be stuck in a bad state, but requires that when that state is reached we have a good explanation to give the programmer that can blame the original faulty type assumption in their source code.
Where the slogan for type soundness is ``well typed terms don't get stuck'', the slogan for cast soundness is ``well cast terms don't get stuck without blame''.

In \ch{2} we proved type soundness for a minimal dependently typed language, with a progress and preservation style proof given a suitable definition of term equivalence.
We can extend that proof to support cast soundness with only a few modifications.
 
\begin{figure}
\[
\frac{x:A\in \Gamma }{\Gamma \vdash x\,:\,A}
\rulename{cast-var}
\]
\[
\frac{\Gamma \vdash a:A\quad \Gamma \vdash A:\star\quad \Gamma \vdash B:\star}{\Gamma \vdash a::_{A,\ell ,o}B\::\:B}
\rulename{cast-::}
\]
 
\[
\frac{\,}{\Gamma \vdash\star:\,\star}
\rulename{cast-\star}
\]
 
\[
\frac{\Gamma \vdash A:\star\quad \Gamma ,x:A\vdash B:\star}{\Gamma \vdash\left(x:A\right)\rightarrow B\,:\,\star}
\rulename{cast-\mathsf{fun}-ty}
\]
 
\[
\frac{\Gamma ,f:\left(x:A\right)\rightarrow B,x:A\vdash b:B}{\Gamma \vdash\mathsf{fun}\,f\,x\Rightarrow b\,:\,\left(x:A\right)\rightarrow B}
\rulename{cast-\mathsf{fun}}
\]
 
\[
\frac{\Gamma \vdash b:\left(x:A\right)\rightarrow B\quad \Gamma \vdash a:A}{\Gamma \vdash b\,a\,:\,B\left[x\coloneqq a\right]}
\rulename{cast-fun-app}
\]
 
\[
\frac{\Gamma \vdash a:A\quad A\equiv A'}{\Gamma \vdash a:A'}
\rulename{cast-conv}
\]
 
\todo[inline]{review regularity stuff}
 
\caption{\CLang{} Type Assignment Rules}
\label{fig:cast-tas-rules}
\end{figure}
 
The \clang{} supports its own type assignment system, defined in \Fref{cast-tas-rules}\ccnote{you do not need the type constraint on the conversion rule! it turned out easier to do in coq without it}.
This system ensures that computations will not get stuck without enough information for good runtime error messages.
Specifically computations will not get stuck without a source location and a witness of inequality.
The only rule that works differently than the \slang{} is the \rulename{cast-::} rule that allows runtime type assertions.
 
As before we need a suitable reduction relation to generate our equivalence relation.
\Fref{cast-reduction} shows that system of reductions.
The full rule for function reduction is given in \rulename{\Rrightarrow-\mathsf{fun}-::-red} which makes the behavior from the examples explicit: argument types are swapped as a term is applied under a cast.
Casts from a type universe to a type universe are allowed by the \rulename{\Rrightarrow-::-red} rule.
Since observations embed expressions, they must also be given parallel reductions.
 
\begin{figure}
\[
\frac{b\Rrightarrow b'\quad a\Rrightarrow a'}{\left(\mathsf{fun}\,f\,x\Rightarrow b\right)a\Rrightarrow b'\left[f\coloneqq\mathsf{fun}\,f\,x\Rightarrow b',x\coloneqq a'\right]}
\rulename{\Rrightarrow-\mathsf{fun}-app-red}
\]
 
\[
\frac{b\Rrightarrow b'\quad a\Rrightarrow a'\quad A_{1}\Rrightarrow A_{1}'\quad A_{2}\Rrightarrow A_{2}'\quad B_{1}\Rrightarrow B_{1}'\quad B_{2}\Rrightarrow B_{2}'\quad o\Rrightarrow o'}{\begin{array}{c}
\left(b::_{\left(x:A_{1}\right)\rightarrow B_{1},\ell ,o}\left(x:A_{2}\right)\rightarrow B_{2}\right)a\Rrightarrow\\
\left(b'\,a'::_{A_{2}',\ell,o.Arg}A_{1}'\right)::_{B_{1}'\left[x\coloneqq a'::_{A_{2}',\ell,o'.Arg}A_{1}'\right],\ell ,o'.Bod_{a'} }B_{2}'\left[x\coloneqq a'\right]
\end{array}}
\rulename{\Rrightarrow-fun-::-red}
\]
 
\[
\frac{a\Rrightarrow a'}{a::_{\star,\ell,o}\star\Rrightarrow a'}
\rulename{\Rrightarrow-::-red}
\]

\[
\frac{a\Rrightarrow a'\quad A_{1}\Rrightarrow A_{1}'\quad A_{2}\Rrightarrow A_{2}'\quad o\Rrightarrow o'}{a::_{A_{1},\ell ,o}A_{2}\Rrightarrow a'::_{A_{1}',\ell ,o'}A_{2}'}
\rulename{\Rrightarrow-::}
\]

\begin{align*}
\frac{\,}{x\Rrightarrow x}
\rulename{\Rrightarrow-var} \quad \quad \quad & \quad \quad \quad \frac{\,}{\star\Rrightarrow\star}
\rulename{\Rrightarrow-\star}\\
\end{align*}

\begin{align*}
\frac{A\Rrightarrow A'\quad B\Rrightarrow B'}{\left(x:A\right)\rightarrow B\Rrightarrow\left(x:A'\right)\rightarrow B'}
\rulename{\Rrightarrow-\mathsf{fun}-ty} \quad & \quad \frac{b\Rrightarrow b'}{\mathsf{fun}\,f\,x\Rightarrow b\,\Rrightarrow\,\mathsf{fun}\,f\,x\Rightarrow b'}
\rulename{\Rrightarrow-\mathsf{fun}}\\
\end{align*}

\[
\frac{b\Rrightarrow b'\quad a\Rrightarrow a'}{b\,a\Rrightarrow b'\,a'}
\rulename{\Rrightarrow-\mathsf{fun}-app}
\]
 
\begin{align*}
  \frac{\,}{.\Rrightarrow.}
\rulename{\Rrightarrow-obs-emp} &  \quad  \frac{o\Rrightarrow o'}{o.Arg\Rrightarrow o'.Arg}
\rulename{\Rrightarrow-obs-Arg} \quad  &  \frac{o\Rrightarrow o'\quad a\Rrightarrow a'}{o.Bod_a \Rrightarrow o'.Bod_{a'}}
\rulename{\Rrightarrow-obs-Bod}  \\
\end{align*}

% \[
% \frac{\,}{a\Rrightarrow_{\ast}a}
% \rulename{\Rrightarrow_{\ast}-refl}
% \]
% \[
% \frac{a\Rrightarrow_{\ast}a'\quad a'\Rrightarrow a''}{a\Rrightarrow_{\ast}a''}
% \rulename{\Rrightarrow_{\ast}-trans}
% \]

\[
\frac{a\Rrightarrow_{\ast}a''\quad a'\Rrightarrow_{\ast}a''}{a\equiv a'}
\rulename{\equiv-def}
\]
\todo[inline]{better layout}
\caption{\CLang{} Parallel Reductions}
\label{fig:cast-reduction}
\end{figure}
 
\subsection{Definitional Equality}
 
As in \ch{2}, we will define a suitable notion of definitional equality to derive the other properties of the system.
While it may seem counterintuitive to define a definitional equality in a system that is intended to avoid definitional equality, this is fine since programmers will never interact directly with the \csys.
Programmers will only interact with elaboration, and elaboration will only result in well cast terms.
The \csys{} only exists to give theoretical assurances.
 
As before $\Rrightarrow_{*}$ can be shown to be confluent, and used to generate the equality relation.
The proofs follow the same structure as \ch{2}, but since observations can contain terms, $\Rrightarrow$ and $\textbf{max}$ must be extended to observations.
Proofs must be extended to mutually induct on observations, since they can contain expressions that could also reduce.
 
The explicit new rules for $\textbf{max}$ are given in \Fref{cast-sys-max} with the structural rules omitted since they are the same as \ch{2}.
 
\begin{sidewaysfigure}
\begin{tabular}{cccll}
$\textbf{max}($ & $\left(\mathsf{fun}\,f\,x\Rightarrow b\right)\,a$ & $)=$ & $\textbf{max}\left(b\right)\left[f\coloneqq\mathsf{fun}\,f\,x\Rightarrow \textbf{max}\left(b\right),x\coloneqq \textbf{max}\left(a\right)\right]$ \tabularnewline
$\textbf{max}($ &
$\left(b::_{\left(x:A_{1}\right)\rightarrow B_{1},\ell ,o}\left(x:A_{2}\right)\rightarrow B_{2}\right)a$ & $)=$ &
\makecell[l]{
 $\left(\textbf{max}\left(b\right)\,\left(\textbf{max}\left(a\right)::_{\textbf{max}\left(A_{2}\right),\ell,\textbf{max}\left(o\right).Arg}\textbf{max}\left(A_{1}\right)\right)\right)$ \\
 $::_{\textbf{max}\left(B_{2}\right)\left[x\coloneqq \textbf{max}\left(a\right)::_{\textbf{max}\left(A_{2}\right),\ell,\textbf{max}\left(o\right).Arg}\textbf{max}\left(A_{1}\right)\right],\ell ,\textbf{max}\left(o\right).Bod_{\textbf{max}\left(a\right)}}$ \\
 $\textbf{max}\left(B_{2}\right)\left[x\coloneqq \textbf{max}\left(a\right)\right]$
} & \tabularnewline

$\textbf{max}($ & $b::_{\star,\ell,o}\star$ & $)=$ & $\textbf{max}\left(b\right)$ \tabularnewline
 

& & & otherwise \tabularnewline
$\textbf{max}($ & $b::_{B_{1},\ell ,o}B_{2}$ & $)=$ & $\textbf{max}\left(b\right)::_{\textbf{max}\left(B_{1}\right),\ell ,\textbf{max}\left(o\right)}\textbf{max}\left(B_{2}\right)$ \tabularnewline
 
$\textbf{max}($ & ... & $)=$ & ... % corresponds to the definition in chapter 2
\tabularnewline
$\textbf{max}($ & $.$ & $)=$ & $.$ & \tabularnewline
$\textbf{max}($ & $o.Arg$ & $)=$ & $\textbf{max}\left(o\right).Arg$ \tabularnewline
$\textbf{max}($ & $o.Bod_a$ & $)=$ & $\textbf{max}\left(o\right).Bod_{\textbf{max}\left(a\right)}$ \tabularnewline
\end{tabular}
\todo[inline]{actually have all the rules but gray out the redundant ones}
\caption{The $\textbf{max}$ Function}
\label{fig:cast-sys-max}
\end{sidewaysfigure}
\ccnote{review the notes on this figure.}

The expected lemas hold.
 
\begin{lem} Triangle Properties of $\Rrightarrow$.
 
If $a\Rrightarrow a'$ then $a'\Rrightarrow \maxRed{} \left(a\right)$.

If $o\Rrightarrow o'$ then $o'\Rrightarrow \maxRed{} \left(o\right)$.
\end{lem}
\begin{proof}
By mutual induction on the derivations of $m\Rrightarrow\,m'$ and $o\Rrightarrow o'$.
\end{proof}
\todo{check this}

\begin{lem} Diamond Property of $\Rrightarrow$.
 
If $a\Rrightarrow a'$ and $a\Rrightarrow a''$ implies $a'\Rrightarrow\, \maxRed{} \left(a\right)$ and $a''\Rrightarrow\, \maxRed{} \left(a\right)$.
\end{lem}
\begin{proof}
  This follows directly from the triangle property.
\end{proof}
 
\begin{lem} Confluence of $\Rrightarrow_{\ast}$.
\end{lem}
\begin{proof}
By repeated application of the diamond property.
\end{proof}
   
\subsubsection{$\equiv$ is an equivalence}
As before, this allows us to prove $\equiv$ is transitive, and therefore $\equiv$ is an equivalence.

\begin{thm} $\equiv$ is transitive.
 
If $a\equiv a'$ and $a'\equiv a''$ implies $a\equiv a''$.
\end{thm}
\begin{proof}
Follows from the confluence of $\Rrightarrow_{\ast}$.
\end{proof}
 
\subsubsection{Stability}
Similar to \ch{2} we need to prove that equality is stable over type constructors.
\begin{lem} Stability of $\rightarrow$ over $\Rrightarrow_{\ast}$.
 
$\forall A,B,C.$ $\left(x:A\right)\rightarrow B\Rrightarrow_{\ast}C$ implies $\exists A',B'.C=\left(x:A'\right)\rightarrow B'\land A\Rrightarrow_{\ast}A'\land B\Rrightarrow_{\ast}B'$.
\end{lem}
\begin{proof}
By induction on $\Rrightarrow$, which implies the result for $\Rrightarrow_{\ast}$.
\end{proof}

\begin{cor} Stability of $\rightarrow$ over $\equiv$.
 
The following rule is admissible:
\[
\frac{\left(x:A\right)\rightarrow B\equiv\left(x:A'\right)\rightarrow B'}{A\equiv A'\quad B\equiv B'}
\]
\end{cor}
 
With these properties proving $\equiv$ is suitable as a definitional equivalence, we can now tackle the progress and preservation lemmas.
 
\subsection{Preservation}
 
As in \ch{2}, $\Rrightarrow$ preserves types.
The argument is similar to that of \ch{2} though more inversion lemmas are needed.
 
\subsubsection{Structural Properties}
We begin by proving the structural properties:
\begin{lem} Context Weakening.
 
The following rule is admissible:
 
\[
\frac{\Gamma \vdash a:A}{\Gamma ,\Gamma' \vdash a:A}
\]
\end{lem}
 
\begin{proof}
By induction on the derivations of the \csys{}.
\end{proof}

\begin{lem} Substitution preserves types.
 
The following rule is admissible:
 
\[
\frac{\Gamma \vdash c:C \quad \Gamma, x:C, \Gamma' \vdash a:A}{\Gamma, \Gamma' \left[x\coloneqq c\right]\vdash a\left[x\coloneqq c\right]:A\left[x\coloneqq c\right]}
\]
\end{lem}
\begin{proof}
By induction on the derivations of the \csys{}.
\end{proof}

As before the notion of definitional equality can be extended to cast contexts. % in \Fref{cast-Context-Equiv}.
 
% \begin{figure}
% \[
% \frac{\ }{\lozenge\equiv\lozenge}\,\rulename{\ensuremath{\equiv}-ctx-empty}
% \]
 
% \[
% \frac{\Gamma \equiv \Gamma' \quad A\equiv A'}{\Gamma, x:A \equiv \Gamma', x:A'}\,\rulename{\ensuremath{\equiv}-ctx-ext}
% \]
% \caption{Contextual Equivalence}
% \label{fig:cast-Context-Equiv}
% \end{figure}
 
\begin{lem}Contexts that are equivalent preserve types.
 
The following rule is admissible:
 
\[
\frac{\Gamma\vdash n:N\quad\Gamma\equiv\Gamma'}{\Gamma'\vdash n:N}
\]
\end{lem}
\begin{proof}
By induction over the derivations of the \csys{}.
\end{proof}

As before we show inversions on the term syntaxes, generalizing the induction hypothesis up to equality when needed.

\begin{lem} $\mathsf{fun}$-Inversion (generalized).
 
\[
\frac{\Gamma \vdash\mathsf{fun}\,f\,x\Rightarrow a\,:\,C\quad C\equiv\left(x:A\right)\rightarrow B}{\Gamma, f:\left(x:A\right)\rightarrow B,x:A\vdash b:B}
\]
\end{lem}
\begin{proof}
By induction on the derivations of the \csys{}.
\end{proof}
 
This allows us to conclude the corollary

\begin{cor} $\mathsf{fun}$-Inversion.
 
\[
\frac{\Gamma \vdash\mathsf{fun}\,f\,x\Rightarrow a\,:\,\left(x:A\right)\rightarrow B}{\Gamma, f:\left(x:A\right)\rightarrow B,x:A\vdash b:B}
\]
\end{cor}

Unlike \ch{2}, we also need an inversion for function types.

\begin{lem} $\rightarrow$-Inversion (generalized).
 
The following rule is admissible
\[
\frac{\Gamma \vdash\left(x:A\right)\rightarrow B\,:\,C\quad C\equiv\star}{\Gamma \vdash A:\star\quad \Gamma, x:A \vdash B:\star}
\]
\end{lem}
\begin{proof}
By induction on the typing derivations
\end{proof}

Which allows the expected corollary:

\begin{cor} $\rightarrow$-Inversion.
\[
\frac{\Gamma \vdash\left(x:A\right)\rightarrow B\,:\,\star}{\Gamma \vdash A:\star\quad \Gamma, x:A\vdash B:\star}
\]
\end{cor}
 
We also need a lemma that will invert the typing information out of the cast operator.
This can be proven directly without generalizing over definitional equality.

\begin{lem} $::$-Inversion.
 
The following rule is admissible:
\[
\frac{\Gamma \vdash a::_{A,\ell ,o}B\::\:C}{\Gamma \vdash a:A\quad \Gamma \vdash A:\star\quad \Gamma \vdash B:\star}
\]
\end{lem}
 
\todo{remove conditions for regularity?}
\begin{proof}
By induction on the typing derivations:
\begin{casenv}
 \item \rulename{cast-::} follows directly.
 \item \rulename{cast-conv} by induction.
 \item All other cases impossible!
\end{casenv}
\end{proof}
 
Note that the derivations of the conclusion of this theorem can always be made smaller than the derivation from the premise.
This allows other proofs to use induction on the output of this lemma while still being well founded.
 
\begin{thm} $\Rrightarrow$ Preserves types.
 
The following rule is admissible:
\[
\frac{a\Rrightarrow a'\quad \Gamma \vdash a:A}{\Gamma \vdash a':A}
\]
\end{thm}
 
\begin{proof}
By induction on the cast derivation $\Gamma \vdash m:M$, specializing on $m\Rrightarrow m'$:
 
\begin{casenv}
 \item If the term typed with \rulename{cast-::}:
 \begin{casenv}
   \item If the term reduced with \rulename{\Rrightarrow-::-red} then preservation follows by induction.
   \item If the term reduced with \rulename{\Rrightarrow-::} then preservation follows by induction and conversion.
 \end{casenv}
 \item If the term typed with \rulename{cast-fun-app}:
 \begin{casenv}
   \item If the term reduced with \rulename{\Rrightarrow-fun-::-red} then we have
   $\Gamma \vdash\left(b::_{\left(x:A_{1}\right)\rightarrow B_{1},\ell ,o}\left(x:A_{2}\right)\rightarrow B_{2}\right):\left(x:A_{2}\right)\rightarrow B_{2}$,
   $\Gamma \vdash a:A_{2}$,
   $b\Rrightarrow b'$, $a\Rrightarrow a'$, $A_{1}\Rrightarrow A_{1}'$,
   $A_{2}\Rrightarrow A_{2}'$, $B_{1}\Rrightarrow B_{1}'$,  $B_{2}\Rrightarrow B_{2}'$,
   and $o\Rrightarrow o'$.
   We must show $\Gamma \vdash\left(b'\ ac\right)::_{B_{1}'\left[x\coloneqq ac\right],\ell ,o'.Bod_{a'}}B_{2}'\left[x\coloneqq a'\right]$, where $ac=a'::_{A_{2}',\ell,o'.Arg}A_{1}'$.
  
   With cast-inversion we can show $\Gamma \vdash b:\left(x:A_{1}\right)\rightarrow B_{1}$, $\Gamma \vdash\left(x:A_{1}\right)\rightarrow B_{1}:\star$,
   $\Gamma \vdash\left(x:A_{2}\right)\rightarrow B_{2}:\star$.
   Since these derivations are structurally smaller, we can use induction on them.
 
   \begin{tabular}{llll}
     $\Gamma \vdash a':A_{2}'$ & $\rulename{cast-conv}$\tabularnewline
     $\left(x:A_{2}\right)\rightarrow B_{2}\Rrightarrow\left(x:A_{2}'\right)\rightarrow B_{2}'$ & by \rulename{\Rrightarrow-fun-ty}\tabularnewline
     $\Gamma \vdash\left(x:A_{2}'\right)\rightarrow B_{2}':\star$ & \makecell[l]{by induction with \\ $\Gamma \vdash\left(x:A_{2}\right)\rightarrow B_{2}:\star$} \tabularnewline
     $\Gamma \vdash A_{2}':\star$, $\Gamma, x:A_{2}'\vdash B_{2}':\star$ & fun-ty-inversion \tabularnewline
     $\left(x:A_{1}\right)\rightarrow B_{1}\Rrightarrow\left(x:A_{1}'\right)\rightarrow B_{1}'$ & \rulename{\Rrightarrow-fun-ty} \tabularnewline
     $\Gamma \vdash\left(x:A_{1}'\right)\rightarrow B_{1}':\star$ & \makecell[l]{by induction with \\ $\Gamma \vdash\left(x:A_{1}\right)\rightarrow B_{1}:\star$} \tabularnewline
     $\Gamma \vdash A_{1}':\star$, $\Gamma, x:A_{1}'\vdash B_{1}':\star$ & fun-ty-inversion\tabularnewline
     % let $ac=a'::_{A_{2}',\ell,o'.Arg}A_{1}'$ & \tabularnewline
     $\Gamma \vdash ac:A_{1}'$ & by $\rulename{cast-::}$\tabularnewline
     $\Gamma \vdash b':\left(x:A_{1}\right)\rightarrow B_{1}$ & \makecell[l]{by induction with \\ $\Gamma \vdash b:\left(x:A_{1}\right)\rightarrow B_{1}$} \tabularnewline
     $\Gamma \vdash b':\left(x:A_{1}'\right)\rightarrow B_{1}'$ & by \rulename{cast-conv}\tabularnewline
     $\Gamma \vdash b'\ ac:B_{1}'\left[x\coloneqq ac\right]$ & by \rulename{cast-fun-app}\tabularnewline
     $\Gamma \vdash B_{1}'\left[x\coloneqq ac\right]:\star$ & by substitution preservation\tabularnewline
     $\Gamma \vdash B_{2}'\left[x\coloneqq a'\right]:\star$ & by substitution preservation\tabularnewline
     % \makecell[l]{$\Gamma \vdash$ \\ $\left(b'\ ac\right)::_{B_{1}'\left[x\coloneqq ac\right],\ell ,o'.Bod_{a'}}B_{2}'\left[x\coloneqq a'\right]$} & by \rulename{cast-::} \tabularnewline
     \end{tabular}
     Which allows us to conclude $\Gamma \vdash\left(b'\ ac\right)::_{B_{1}'\left[x\coloneqq ac\right],\ell ,o'.Bod_{a'}}B_{2}'\left[x\coloneqq a'\right]$ by \rulename{cast-::}.
 
   \item All other reductions are similar to \ch{2}.
 \end{casenv}
 \item All other cases follow along the lines of \ch{2}.
\end{casenv}
\end{proof}
 
\todo{highlight the -fun-::-red case}
 
\subsubsection{Progress}
 
Preservation alone isn't sufficient for a cast sound language.
We also need to show that there is an evaluation that behaves appropriately in an empty typing context.
Again this will broadly follow the outline of the \slang{} proof in \ch{2}, with a few substantial changes.
 
Unlike the \slang{}, it is no longer practical to characterize values syntactically.
Values are specified by judgments in \Fref{cast-val}.
They are standard except for the \rulename{Val-::}, which states that a type ($\star$ or function type) under a cast is not a value.
\ccnote{deal with this after restructure}
 
\begin{figure}
\[
\frac{\,}{\star\,\textbf{Val}}
\rulename{Val-\star}
\]
\[
\frac{\,}{\left(x:A\right)\rightarrow B\,\textbf{Val}}
\rulename{Val-fun-ty}
\]
\[
\frac{\,}{\mathsf{fun}\,f\,x\Rightarrow b\:\textbf{Val}}
\rulename{Val-fun}
\]
\[
\frac{\begin{array}{c}
a\:\textbf{Val}\quad A\:\textbf{Val}\quad B\:\textbf{Val}\\
a\cancel{=}\star\\
a\cancel{=}\left(x:C\right)\rightarrow C'
\end{array}}{a::_{A,\ell, o}B\:\textbf{Val}}
\rulename{Val-::}
\]
\caption{\CLang{} Values}
\label{fig:cast-val}
\end{figure}
 
Small steps are listed in \Fref{cast-step}.\ccnote{deal with this after restructure}
They are standard for \cbv{} except that casts can distribute over application, and casts can reduce when both types are $\star$.
 
\begin{figure}
\[
\frac{a\,\textbf{Val}}{\left(\mathsf{fun}\,f\,x\Rightarrow b\right)a\rightsquigarrow b\left[f\coloneqq\mathsf{fun}\,f\,x\Rightarrow b,x\coloneqq a\right]}
\]
\[
\frac{b\,\textbf{Val}\quad a\,\textbf{Val}}{\begin{array}{c}
\left(b::_{\left(x:A_{1}\right)\rightarrow B_{1},\ell ,o}\left(x:A_{2}\right)\rightarrow B_{2}\right)a\rightsquigarrow\\
\left(b\,a::_{A_{2},\ell,o.Arg}A_{1}\right)::_{B_{1}\left[x\coloneqq a::_{A_{2},\ell,o.Arg}A_{1}\right],\ell ,o.Bod_a}B_{2}\left[x\coloneqq a\right]
\end{array}}
\]
\[
\frac{a\,\textbf{Val}}{a::_{\star,\ell ,o}\star\rightsquigarrow a}
\]
\[
\frac{a\rightsquigarrow a'}{a::_{A,\ell ,o}B\rightsquigarrow a'::_{A,\ell ,o}B}
\]
\[
\frac{a\,\textbf{Val}\quad A\rightsquigarrow A'}{a::_{A,\ell ,o}B\rightsquigarrow a::_{A',\ell ,o}B}
\]
\[
\frac{a\,\textbf{Val}\quad A\,\textbf{Val}\quad B\rightsquigarrow B'}{a::_{A,\ell ,o}B\rightsquigarrow a::_{A,\ell ,o}B'}
\]
\[
\frac{b\rightsquigarrow b'}{b\,a\rightsquigarrow b'\,a}
\]
\[
\frac{b\,\textbf{Val}\quad a\rightsquigarrow a'}{b\,a\rightsquigarrow b\,a'}
\]
 
\todo[inline]{name rules}
\caption{\CLang{} Small Step}
\label{fig:cast-step}
\end{figure}
 
%% walk through blame
In addition to small step and values we also specify blame judgments in \Fref{cast-blame}.
Blame tracks the information needed to create a good error message and is inspired by the many systems that use blame tracking \cite{10.1145/581478.581484,10.1007/978-3-642-00590-9_1,wadler:LIPIcs:2015:5033}.
Specifically the judgment \blame{a}{\ell}{o} means that $a$ witnesses a contradiction in the source code at location $\ell$ under the observations $o$.
With only dependent functions and universes, only inequalities of the form $*\,\cancel{=}\,A\rightarrow B$ can be witnessed.
The first two rules of the blame judgment witness these concrete type inequalities.
The rest of the blame rules recursively extract concrete witnesses from larger terms.
Limiting the observations to the form $*\,\cancel{=}\,A\rightarrow B$ makes the system in this Chapter simpler than the system in \ch{5} where more observations are possible because of the addition of data.
% "self evidently correct" since it is clear to only extract inequalities. like step and val? one of several possible contradiction extraction relations
 
\begin{figure}
\[
\frac{\,}{\blame{\left(a::_{\left(x:A\right)\rightarrow B,\ell ,o}\star\right)}{\ell}{o}}
\]
\[
\frac{\,}{\blame{\left(a::_{\star,\ell ,o}\left(x:A\right)\rightarrow B\right)}{\ell}{o}}
\]
\[
\frac{
 \blame{a}{\ell}{o}
}{
 \blame{\left(a::_{A,\ell',o'}B\right)}{\ell}{o}
}
\]
\[
\frac{
 \blame{a}{\ell}{o}
}{
 \blame{\left(a::_{A,\ell',o'}B\right)}{\ell}{o}
}
\]
\[
\frac{
 \blame{B}{\ell}{o}
}{
 \blame{\left(a::_{A,\ell',o'}B\right)}{\ell}{o}
}
\]
\[
\frac{
 \blame{b}{\ell}{o}
}{
 \blame{\left(b\,a\right)}{\ell}{o}
}
\]
\[
\frac{
 \blame{a}{\ell}{o}
}{
 \blame{\left(b\,a\right)}{\ell}{o}
}
\]
\todo[inline]{name rules}
\caption{\CLang{} Blame}
\label{fig:cast-blame}
\end{figure}
 
As before we have that $\rightsquigarrow$ preserves types.
 
\begin{fact} $\rightsquigarrow$ preserves types.
 
Since the following rule is admissible:
\[
\frac{m\rightsquigarrow m'}{m\Rrightarrow m'}
\]
\end{fact}
 
As in \ch{2} we will need technical lemmas that determine the syntax of a value of a given type in the empty context.
However, canonical function values look different because they must account for the possibility of blame arising from a stuck term.

\begin{lem} $\star$-Canonical forms (generalized).
 
If $\vdash a\,:\,A$ , $a\,\textbf{Val}$, and $A\equiv\star$ then either.

\textup{$a=\star$ ,}

\textup{or there exists $C$, $B$, such that $a=\left(x:C\right)\rightarrow B$.}
\end{lem}
\todo{alternatively produce the blame judgment}
\begin{proof}
By induction on the cast derivation:
\begin{casenv}
 \item \rulename{cast-\star} and \rulename{cast-fun-ty} follow directly.
 \item \rulename{ty-conv} follows by induction and that $\equiv$ is an equivalence.
 \item \rulename{cast-fun} is impossible since $\left(x:A\right)\rightarrow B\cancel{\equiv}\star$!
 \item \rulename{cast-::} is impossible!
   Inductively the underlying term must be $\star$, or $\left(x:C\right)\rightarrow B$.
   Which contradicts the side conditions of \rulename{Val-::}.
 \item Other rules are impossible, since they do not type values in an empty context!
\end{casenv}
\end{proof}

Leading to the corollary:

\begin{cor} $\star$-Canonical forms.
 
If $\vdash A:\star$, and $A\,\textbf{Val}$ then either

\textup{$A=\star$ , }

\textup{or there exists $C$, $B$, such that $A=\left(x:C\right)\rightarrow B$.}
\end{cor}
 
Likewise:

\begin{lem} $\rightarrow$-Canonical forms (generalized).
 
If $\vdash a\,:\,A$ , $a\,\textbf{Val}$, and $A\equiv\left(x:C\right)\rightarrow B$ then either
 
$a=\mathsf{fun}\,f\,x\Rightarrow b$
 
or $a=d::_{D,\ell ,o}\left(x:C'\right)\rightarrow B'$, $d\,\textbf{Val}$, $D\,\textbf{Val}$, $C'\equiv C$, $B'\equiv B$
\end{lem}
\todo{alternatively produce the blame judgment}
\begin{proof}
By induction on the cast derivation:
\begin{casenv}
 \item \rulename{cast-fun} follows directly.
 \item \rulename{cast-::} then it must be a value from \rulename{Val-::} satisfying the 2nd conclusion.
 \item \rulename{ty-conv} follows by induction and the transitivity of $\equiv$.
 
 \item \rulename{cast-\star} and \rulename{cast-fun-ty} are impossible by the stability of $\equiv$!
 \item Other rules are impossible, since they do not type values in an empty context!
\end{casenv}
\end{proof}

As a corollary:

\begin{cor} $\rightarrow$-Canonical forms.
 
If $\vdash a:\left(x:C\right)\rightarrow B$ , and $a\,\textbf{Val}$
 
$a=\mathsf{fun}\,f\,x\Rightarrow b$
 
or $a=d::_{D,\ell ,o}\left(x:C'\right)\rightarrow B'$, $d\,\textbf{Val}$, $D\,\textbf{Val}$, $C'\equiv C$, $B'\equiv B$.
\end{cor}
 
This further means if $\vdash a:\left(x:C\right)\rightarrow B$, and $a\,\textbf{Val}$ then $a$ is not a type, $a\cancel{=}\star,a\cancel{=}\left(x:C\right)\rightarrow C'$.
 
We can now prove the progress lemma.
 
\begin{thm} Progress.
 
If $\vdash a\,:\,A$ then either
 
$a\,\textbf{Val}$,
 
there exists $a'$ such that $a\rightsquigarrow a'$,
 
or there exists $\ell$, $o$ such that $\blame{a}{\ell}{o}$.
\end{thm}
\begin{proof}
As usual this follows form induction on the typing derivation:
\begin{casenv}
 \item \rulename{cast-\star}, \rulename{cast-fun-ty}, and \rulename{cast-fun} follow directly.
 \item \rulename{cast-var}, impossible in an empty context!
 \item \rulename{cast-conv}, by induction.
 \item \rulename{cast-::}, then the term is $a::_{A,\ell ,o}B\::\:B$.
 Each of $a$, $A$, and $B$ can be blamed, can step, or is a value (By induction):
 \begin{casenv}
   \item If any of $a$, $A$, or $B$, step then the entire term can step.
   \item If any of $a$, $A$, or $B$, can be blamed then the entire term can be blamed.
   \item If all of $a$, $A$, and $B$, are values then $A$ and $B$ are types of canonical form.
   \begin{casenv}
     \item If both $A$, and $B$, is $\star$ then step.
     \item If one of $A$, and $B$, is $\star$ and the other is $\left(x:C_{A}\right)\rightarrow D_{A}$ then blame.
    
     \item Otherwise the term is a value by the canonical forms.
   \end{casenv}
 \end{casenv}
 \item \rulename{cast-fun-app}, then the term is $(b\,a)$.
  Each of $a$, and $b$ can be blamed, can step, or is a value (By induction):
  \begin{casenv}
    \item If any of $b$, or $a$, step then the entire term can step
    \item If any of $b$, or $a$, can be blamed then the entire term can be blamed
    \item If $b\,\textbf{Val}$, $a\,\textbf{Val}$ by canonical forms:
    \begin{casenv}
      \item $b=\mathsf{fun}\,f\,x\Rightarrow c$ the term steps.
      \item $b=d_{b}::_{D_{b},\ell _{b}\,o_{b}}\left(x:A'\right)\rightarrow B'$ we have by canonical forms and induction that:
      \begin{casenv}
        \item $D_{b}=\star$ and blame can be generated.
        \item Otherwise $D_{b}=\left(x:A_{D_{b}}\right)\rightarrow B_{D_{b}}$ and a step is possible.
      \end{casenv}
    \end{casenv}
  \end{casenv}
\end{casenv}
\end{proof}
 
\subsection{Cast Soundness}
 
\todo{revise}
 
%% cast soundness
Cast soundness follows from progress and preservation as expected. \ccnote{address after restructure}
 
% For any $\lozenge\vdash c:C$, $c'$, $c\rightsquigarrow^{*}c'$, if $\textbf{Stuck}\,c'$ then $\textbf{Blame}\:\ell \,o\:c'$, where $\textbf{Stuck}\,c'$ means $c'$ is not a value and $c'$ does not step.
% This follows by iteration the progress and preservation lemmas.
 
\subsection{Discussion}
 
\todo{inferablility of inner cast}
 
\todo{regularity encoding, along with the cast}
 
Because of the conversion rule and non-termination, cast-checking is undecidable.
This is fine since the \csys{} only exists to ensure theoretical properties. \ccnote{I think this is cleared up in the next few sentences}
Programmers will only use the system through the elaboration procedure described in the next section.
Every term produced by elaboration will cast check, and the elaboration is decidable. % \footnote{As fomalized here, the assertimplemenation }.
 
As in the \slang{}\ac{TAS}, the \clang{} is logically unsound, by design.
 
Just as there are many different flavors of definitional equality that could have been used in \ch{2}, there are also many possible degrees to which runtime equality can be enforced.
The \Blame{} relation in \Fref{cast-blame} outlines a minimal checking strategy that supports cast soundness.
For instance\footnote{
  Assuming the data types of \ch{5}.
}, $\mathtt{head}\,\Bool{}\,1\,\left(\mathtt{rep}\,\Bool{}\,\True{}\,0\right)$ will result in blame since $1$ and $0$ have different head constructors.
But $\mathtt{head}\,\Bool{}\,1\,\left(\mathtt{rep}\,\Bool{}\,\True{}\,9\right)$ will not result in blame since $1$ and $9$ have the same head constructor and the computation can reduce to \True{}.
 
It is likely that more aggressive checking is preferable in practice, especially in the presence of data types.
That is why our implementation checks equalities up to binders. 
This corresponds better to the \cbv{} behavior of the implemented interpreter.
For this reason we call this strategy \textbf{check-by-value}.
 
This behavior is consistent with the conjectured partial correctness of logically unsound \cbv{} execution for dependent types in \cite{jia2010dependent}.
 
\todo{move?}
 
Unlike static type-checking, these runtime checks have runtime costs.
Since the language allows nontermination, checks can take forever to resolve at runtime.
We don't expect this to be a large issue in practice, at least any more than is usual in mainstream languages that allow many other sources of nontermination are possible.
The implementation optimizes away casts when it knows that blame is impossible.
Additionally, we could limit the number of steps allowed in cast normalization and blame slow code.