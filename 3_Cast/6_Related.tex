\section{Related Work}
 
% \subsection{Dependent types and equality}
 
% \todo{revise as bulleted list starting with ETT}
 
% Difficulties in dependently typed equality have motivated many research projects \cite{HoTTbook,sjoberg2015programming,cockx2021taming}.
% However, these impressive efforts currently require a high level of expertise from programmers.
% Further, since program equivalence is undecidable in general, no system will be able to statically verify every ``obvious'' equality for arbitrary user defined data types and functions.
% In the meantime systems should trust the programmer when they use an unverified equality, and use that advanced research to suppress warnings.
 
\subsection{\Bidir{} Placement of Casts}
\todo{better title}
 
This is not the first work to use \bidir{} type checking to place errors.
The Haskell compiler, GHC, supplements Hindley-Minler style type checking with bidirectionality to localize error messages.
This approach was extended in \cite{10.1145/2364527.2364554} which weakens the regular type checking to allow runtime casts\footnote{
 Available with the $\mathtt{-fdefer-type-errors}$ compiler flag.}.
The casts themselves are different from the ones described here since they do not optimistically compute, they will only give errors when reached.
Though more restrictive than our casts, that system enforces parametricity, which makes sense in the context of Haskell. 
% The only other work we are aware of

\todo{talk about the java paper it cites}
 
\todo[inline]{talk about dependent haskell?}

\subsection{Contract Systems}
 
Several of the tricks and notations in this Chapter find their basis in the large amount of work on higher order contracts and gradual types.
Higher order contracts were introduced in \cite{10.1145/581478.581484} as a way to dynamically enforce invariants of software interfaces, specifically higher order functions.
% contracts go back to the 70s apparently, but this seems a reasonable place to start the story, though Bigloo Scheme [28] is cited there.
The notion of blame dates at least that far back.
Swapping the type cast of the input argument of a function type is reminiscent of that paper's use of blame contravariance, though it is presented in a much different way.
% However the contract language of that paper was somewhat limited.
 
Contract semantics were revisited in \cite{10.1145/1925844.1926410,10.1007/978-3-642-28869-2_11} where a more specific correctness criteria based on blame is presented.

Contract systems still generally rely on users annotating their intentions explicitly.
Similar to how programmers might include $\mathtt{assert}$s in an imperative language.
In this thesis annotations are added automatically though elaboration based on type annotations.
 
While there are similarities between contract systems and the \csys{} outlined here, the \csys{} is designed to address only issues with definitional equality in a dependent type theory.
Since contract systems are generally used in untyped languages with contracts written in the host language, definitional equality simply isn't applicable in the vast majority of contract systems.

\todo[inline]{discuss why contracts are a bad fit for this problem}

\subsubsection{Gradual Types}
 
Types can be viewed as a very specific form of contracts that are usually enforced statically.
\textbf{Gradual type systems} allow for a mixing of the static type checking and dynamic type assertions.
Often type information can be inferred using standard techniques, allowing programmers to write fewer annotations.
\todo[inline]{cite first paper, DLS2008?}\todo[inline]{talk about type imprecision}
 
Gradual type systems usually achieve this by adding a $?$ meta character into the type language to denote imprecise typing information.
The first popular account of gradual type semantics appeared in \cite{siek_et_al:LIPIcs:2015:5031} with the alliterative ``gradual guarantee'' which has inspired some of the properties targeted in this Chapter.
 
% , which informally asserts that "runtime checks will not change the expected behavior", "runtime checks will not change the expected behavior" and "well typed code won't be blamed"
% and then like an endless back and forth over criteria
 
Additionally some of the formalism from this Chapter were inspired by the ``Abstracting gradual typing'' methodology \cite{10.1145/2837614.2837670}, where static evidence annotations become runtime checks.
% Unlike some impressive attempts to gradualize the polymorphic lambda calculus \cite{10.1145/3110283}, our system does not attempt to enforce any parametric properties of the base language. %example?
% It is unclear if such a restriction would be desirable for a dependently typed language in practice.
 
This thesis borrows some notational conventions from gradual typing such as the $a::A$ construct for type assertions.
 
A system for gradual dependent types has been proposed in \cite{10.1145/3341692}.
That paper is largely concerned with establishing a decidable type checking procedure via an approximate term normalization.
However, that system retains the conventional style of definitional equality, so that it is possible, in principle, to get $\Vect{}\,(1+x)\neq \Vect{}\,(x+1)$ as a runtime error.
Additionally it is unclear if adding the $?$ meta-symbol into an already very complicated type theory is easier or harder from the programmer's perspective.
 
The common motivation for gradual type systems is to gradually convert a code base from untyped to (usually simply) typed code.
% This chapter has a much tighter scope than the other work cited here, dealing only with equational assumptions.
However, anyone choosing to use a dependent type system has already bought into the usefulness of types in general and will probably not want fragments of completely untyped code.
Gradually converting untyped code to include dependent types is far less plausible than gradually converting untyped code to use simple types.
Especially considering that most real-life codebases will use effects, while adding effects into a simply typed programming language is straightforward, mixing dependent types and effects is a complicated area of ongoing research.

While the gradual typing goals of mixing static certainty with runtime checks are similar to our work here, the approach and details are different.
Instead of trying to strengthen untyped languages by adding types, we take a dependent type system and weaken it with a cast operator.
This leads to different trade-offs in the design space.
For instance, we cannot support completely unannotated code, but we do not need to complicate the type language with a $?$ meta-symbol for uncertainty.

One might characterize this work in this Chapter as gradualizing only the definitional equality relation with a degenerate notion of imprecision.
 
\subsubsection{Blame}
 
Blame is one of the key ideas explored in the contract type and gradual types literature\cite{10.1007/978-3-642-00590-9_1,wadler:LIPIcs:2015:5033,10.1145/3110283}.
Often the reasonableness of a system can be judged by the way blame is handled\cite{wadler:LIPIcs:2015:5033}.
% Blame is treated in \cite{wadler:LIPIcs:2015:5033} very similarly to the presentation in this chapter.
This Chapter goes beyond blaming a source location and also tracks a witnessing observation that can also be made.
 
\subsection{Refinement Style Approaches}
 
This thesis describes a \fullSp{} dependently typed language.
This means computation can appear uniformly in both term and type position.
An alternative approach to dependent types is found in \textbf{refinement type systems}.
 
Refinement type systems restrict type dependency, possibly to specific base types such as $\mathtt{int}$ or $\mathtt{bool}$.
Under this restriction, it is straightforward to check type level equalities and additional properties hold at runtime.
 
\todo[inline]{\cite{knowles2008compositional}}
One approach which explores this is \textbf{hybrid type checking} \cite{10.1145/1111037.1111059,10.1145/1667048.1667051} which performs ``static analysis where possible, ... dynamic checks where necessary''.
% This is a very similar goal to what has been proposed in this Chapter.
However, there are several differences in that work: they have a simply typed system, static warnings for programmers are not considered, and type checking can reject "clearly ill-typed programs".
For the system defined in this thesis there is no clear boundary between clearly ill-typed programs and subtly ill-typed programs, so we treat all potential inequalities uniformly with a static warning and a runtime check.

\todo{need to talk about soft typeing?}
% 'Hybrid type checking is inspired by prior work on soft typing [Fagan 1990;
% Wright and Catwright 1994; Aiken et al. 1994; Flanagan et al. 1996], but it
% extends soft typing by rejecting many ill-typed programs, in the spirit of static
% type checkers.'

Another notable example is \cite{10.1007/1-4020-8141-3_34} which describes a refinement system that limits predicates to base types.
Another example is \cite{10.1145/3093333.3009856}, a refinement system treated in a specifically gradual way.
A refinement type system with higher order features is gradualized in \cite{c4be73a0daf74c9aa4d13483a2c4dd0e}.
\todo{why is full spectrum better?}
 
\todo{cite my abstract}
 
% consider also citing https://www.youtube.com/watch?v=gIYMERs7AZQ https://www.youtube.com/watch?v=EGKeWg2ES0A