\section{Related Work}
 
% \subsection{Dependent types and equality}
 
% \todo{revise as bulleted list starting with ETT}
 
% Difficulties in dependently typed equality have motivated many research projects \cite{HoTTbook,sjoberg2015programming,cockx2021taming}.
% However, these impressive efforts currently require a high level of expertise from programmers.
% Further, since program equivalence is undecidable in general, no system will be able to statically verify every ``obvious'' equality for arbitrary user defined data types and functions.
% In the meantime systems should trust the programmer when they use an unverified equality, and use that advanced research to suppress warnings.
 
\subsection{Contract Systems}
 
Several of the tricks and notations in this Chapter find their basis in the large amount of work on higher order contracts and gradual types.
Higher order contracts were introduced in \cite{10.1145/581478.581484} as a way to dynamically enforce invariants of software interfaces, specifically higher order functions.
% contracts go back to the 70s apparently, but this seems a reasonable place to start the story, though Bigloo Scheme [28] is cited there.
The notion of blame dates at least that far back. Swapping the type cast of the input argument can be traced directly to that paper's use of blame contravariance, though it is presented in a much different way.
% However the contract language of that paper was somewhat limited.
 
Contract semantics were revisited in \cite{10.1145/1925844.1926410,10.1007/978-3-642-28869-2_11} where a more specific correctness criteria based on blame is presented.
 
Contract systems still generally rely on users annotating their invariants explicitly.
Similar to how programmers might include $\mathtt{assert}$s in an imperative language.
In this thesis annotations are added automatically though elaboration.
 
While there are similarities between contract systems and the cast system outlined here, the cast system is designed to address only issues with definitional equality in a dependent type theory.
Since contract systems are generally used in untyped languages with contracts written in the host language, definitional equality simply isn't relevant in the vast majority of contract systems.
 
\subsubsection{Gradual Types}
 
Types can be viewed as a very specific form of contracts.
Gradual type systems allow for a mixing of the static view of data and dynamic checking.
Often type information can be inferred using standard techniques, allowing programmers to write fewer contract annotations.
% TODO first paper, DLS2008
 
Gradual type systems usually achieve this by adding a $?$ meta character into the type language to denote imprecise typing information.
The first popular account of gradual type semantics appeared in \cite{siek_et_al:LIPIcs:2015:5031} with the alliterative ``gradual guarantee'' which has inspired some of the properties in this chapter.
 
% , which informally asserts that "runtime checks will not change the expected behavior", "runtime checks will not change the expected behavior" and "well typed code won't be blamed"
 
% and then like an endless back and forth over criteria
 
Additionally some of the formalism from this chapter was inspired by the ``Abstracting gradual typing'' methodology \cite{10.1145/2837614.2837670}, where static evidence annotations become runtime checks.
% Unlike some impressive attempts to gradualize the polymorphic lambda calculus \cite{10.1145/3110283}, our system does not attempt to enforce any parametric properties of the base language. %example?
% It is unclear if such a restriction would be desirable for a dependently typed language in practice.
 
This thesis borrows some notational conventions from gradual typing such as the $a::A$ construct for type assertions.
 
A system for gradual dependent types has been proposed in \cite{10.1145/3341692}.
That paper is largely concerned with establishing decidable type checking via an approximate term normalization.
However, that system retains the intentional style of definitional equality, so that it is possible, in principle, to get $vec\,(1+x)\neq vec\,(x+1)$ as a runtime error.
Additionally it is unclear if adding the $?$ meta-symbol into an already very complicated type theory is easier or harder from the programmer's perspective.
 
The common motivation for gradual type systems to gradually convert a code base from untyped to (usually simply) typed code.
This chapter has a much tighter scope than the other work cited here, dealing only with equational assumptions.
Anyone using a dependent type system has already bought into the usefulness of types in general and will probably not want fragments of completely untyped code.
Converting untyped code to dependent typed is far less plausible than gradually converting an untyped code base to use simple types\footnote{
 Especially considering that most real-life codebases will use effects, while dependent types and effects are a complicated area of ongoing research
 }.
 
While the gradual typing goals of mixing static certainty with runtime checks are similar to our work here, the approach and details are different.
Instead of trying to strengthen untyped languages by adding types, we take a dependent type system and weaken it with a cast operator.
This leads to different trade-offs in the design space.
For instance, we cannot support completely unannotated code, but we do not need to complicate the type language with wildcards for uncertainty.
However it might be reasonable to characterize this work in this chapter as gradualizing the definitional equality relation.
 
\subsubsection{Blame}
 
Blame is one of the key ideas explored in the contract type and gradual types literature\cite{10.1007/978-3-642-00590-9_1,wadler:LIPIcs:2015:5033,10.1145/3110283}.
Often the reasonableness of a system can be judged by the way blame is handled\cite{wadler:LIPIcs:2015:5033}.
% Blame is treated in \cite{wadler:LIPIcs:2015:5033} very similarly to the presentation in this chapter.
This chapter goes beyond blaming a source location and also tracks a witnessing observation thak can also be made.
 
\subsection{Refinement Style Approaches}
 
This thesis describes a full-spectrum dependently typed language.
This means computation can appear uniformly in both term and type position.
An alternative approach to dependent types is found in \textbf{refinement type systems}.
 
Refinement type systems restrict type dependency, possibly to specific base types such as $\mathtt{int}$ or $\mathtt{bool}$.
Under this restriction, it is straightforward to check these decidable equalities and additional properties hold at runtime.
One specific approach is called \textbf{hybrid type checking} \cite{10.1145/1111037.1111059}\todo{more}.
Another notable example is \cite{10.1007/1-4020-8141-3_34} which describes a refinement system that limits predicates to base types.
Another example is \cite{10.1145/3093333.3009856}, a refinement system treated in a specifically gradual way.
A refinement type system with higher order features is gradualized in \cite{c4be73a0daf74c9aa4d13483a2c4dd0e}.
\todo{why is full spectrum better?}
 
\todo{cite my abstract}
 
% consider also citing https://www.youtube.com/watch?v=gIYMERs7AZQ https://www.youtube.com/watch?v=EGKeWg2ES0A