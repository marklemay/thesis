\section{Elaboration}
 
% overview
Even though the cast language allows us to optimistically assert equalities, manually noting every cast would be unrealistically cumbersome.
This bureaucracy is solved with an elaboration procedure that translates (untyped) terms from the surface language into the cast language.
If the term is well typed in the surface language, elaboration will produce a term without blamable errors.
Terms with unproven equality in types are mapped to a cast with enough information to point out the original source when an inequality is witnessed.

Elaboration serves a similar role as the \bidir{} type system did in \ch{2}, and uses a similar methodology.
Instead of performaing a static equality check when the inference mode and the check mode meet, a runtime cast is inserted asserting the types are equal.

 
First we enrich the surface language with location information, $\ell$, at every position that could result in a type mismatch in \Fref{surface-pre-syntax-loc}.
Note that the location tags match exactly with the check annotations of the \bidir{} system.
For technical reasons the set of locations is nonempty, and a specific null location $.$\todo{- instead of . ?} is designated.
That null location can be used when we need to generate fresh terms, but have no sensible location information available.
All the meta theory from \ch{2} goes through assuming that all locations are equivalent and by generating null locations when needed\footnote{
 For instance, the parallel reduction relation will associate all locations,
 $\frac{M\Rrightarrow M'\quad N\Rrightarrow N'}{\left(x:M_{l}\right)\rightarrow N_{l'}\Rrightarrow\left(x:M'_{l''}\right)\rightarrow N'_{l'''}}\,\rulename{\Rrightarrow-fun-ty}$,
 so that the relation does not discriminate over syntaxes that come from different locations.
 While the $\textbf{max}$ function will map terms into the null location,
 $max\left(\left(x:M_{\ell}\right)\rightarrow N_{\ell'}\right)=\textbf{max}\left(\left(x:\textbf{max}\left(M\right)_{.}\right)\rightarrow \textbf{max}\left(N\right)_{.}\right)$
 }.
We will avoid writing these annotations when they are unneeded (explicitly in \Fref{surface-pre-syntax-loc-abrev}).
 
\begin{figure}
\begin{tabular}{lcll}
\multicolumn{4}{l}{source labels,}\tabularnewline
$\ell$ & $\Coloneqq$ & ... & \tabularnewline
& $|$ & $.$ & no source label\tabularnewline
\multicolumn{4}{l}{expressions,}\tabularnewline
$m,n,M,N$ & $\Coloneqq$ & $x$ & variable\tabularnewline
& $|$ & $m::_{\ell}M^{\ell'}$ & annotation\tabularnewline
& $|$ & $\star$ & type universe\tabularnewline
& $|$ & $\left(x:M_{\ell}\right)\rightarrow N_{\ell'}$ & function type\tabularnewline
& $|$ & $\mathsf{fun}\,f\,x\Rightarrow m$ & function\tabularnewline
& $|$ & $m_{\ell}\,n$ & application\tabularnewline
\end{tabular}\caption{Surface Language Syntax, With Locations}
\label{fig:surface-pre-syntax-loc}
\end{figure}
 
\begin{figure}
\begin{tabular}{lclll}
$m::_{\ell}M^{\ell'}$ & written & $m::_{\ell}M$ & when & $\ell$' is irrelevant\tabularnewline
$m::_{\ell}M$ & written & $m::M$ & when & $\ell$ is irrelevant\tabularnewline
$\left(x:M_{\ell}\right)\rightarrow N_{\ell'}$ & written & $\left(x:M\right)\rightarrow N$ & when & $\ell$, $\ell'$ are irrelevant\tabularnewline
$m_{\ell}\,n$ & written & $m_{\ensuremath{}}\,n$ & when & $\ell$ is irrelevant\tabularnewline
\end{tabular}
 
\caption{Surface Language Abbreviations}
\label{fig:surface-pre-syntax-loc-abrev}
\end{figure}
 
\subsection{Examples}
 
Funtions will elaborate the expected types to thier arguments when they are applied.
For example,
\begin{example}
$f:\mathbb{B}\rightarrow\mathbb{B}\vdash f_{\ell}7\ :\mathbb{B}$ elaborates to $f:\mathbb{B}\rightarrow\mathbb{B}\vdash f\left(7::_{\mathbb{N},\ell}\mathbb{B}\right)\ :\mathbb{B}$
\end{example}

As with \bidir{} type checking, variable types will be inferred from the typing environment.
For example,
\begin{example}
$\vdash(\lambda x\Rightarrow7)::_{\ell}\mathbb{B}\rightarrow\mathbb{B}$ elaborates to $\vdash(\lambda x\Rightarrow7::_{\mathbb{N}.\ell,.bod[x]}\mathbb{B})$
\end{example}
 
To keep teh theory simple, we allow vaous casts to be created,
\begin{example}
$f:\mathbb{N}\rightarrow\mathbb{B}\rightarrow\mathbb{B}\vdash f_{\ell}7_{\ell'}3\ :\mathbb{B}$ elaborates to $f:\mathbb{N}\rightarrow\mathbb{B}\rightarrow\mathbb{B}\vdash f\left(7::_{\mathbb{N}.\ell}\mathbb{N}\right)\left(3::_{\mathbb{N}.\ell'}\mathbb{B}\right)\ :\mathbb{B}$
\end{example}
 
\todo{dependent type example where cast muddles type}
 
Unlike in gradual typing, we cannot elaborate arbitrary untyped syntax.
The underlying type of a cast needs to be known so that a function type can swap its argument type at application.
For instance, $\lambda x\Rightarrow x$ will not elaborate since the intended type is not known.
Fortunately, our experimental testing suggests that a majority of randomly generated terms can be elaborated, while only a small minority of terms would type-check in the surface language.
The programmer can make any term elaborate if they annotate the intended type.
For instance, $\left(\lambda x\Rightarrow x\right)::*\rightarrow*$ will elaborate.
 
\subsection{Elaboration Procedure}
 

\begin{figure}
\[
\frac{
  x:A\in H
}{
  H\vdash x\,\textbf{Elab}\,x\overrightarrow{\,:\,}A
}
\rulename{\overrightarrow{\textbf{Elab}}-var}
\]
 
\[
\frac{\,}{H\vdash\star\,\textbf{Elab}\,\star\overrightarrow{\,:\,}\star}
\rulename{\overrightarrow{\textbf{Elab}}-\star}
\]
 
\[
\frac{H\vdash M\overleftarrow{\,:_{\ell,.}\,}\star\textbf{Elab}\ A\quad H,x:A\vdash N\overleftarrow{\,:_{\ell',.}\,}\star\textbf{Elab}\ B}{H\vdash\left(\left(x:M_{\ell}\right)\rightarrow N_{\ell'}\right)\textbf{Elab}\left(\left(x:A\right)\rightarrow B\right)\overrightarrow{\,:\,}\star}\rulename{\overrightarrow{\textbf{Elab}}-fun-ty}
\]
 
\[
\frac{H\vdash m\,\textbf{Elab}\ b\overrightarrow{\,:\,}\left(x:A\right)\rightarrow B\quad H\vdash n\overleftarrow{\,:_{\ell,.arg}\,}A\,\textbf{Elab}\,a}{H\vdash\left(m_{\ell}\,n\right)\textbf{Elab}\left(b\,a\right)\overrightarrow{\,:\,}B\left[x\coloneqq a\right]}\rulename{\overrightarrow{\textbf{Elab}}-fun-app}
\]
 
\[
\frac{H\vdash n\,\textbf{Elab}\ a\overrightarrow{\,:\,}\left(x:A\right)\rightarrow B\quad H\vdash n\overleftarrow{\,:_{\ell,.arg}\,}A\,\textbf{Elab}\ a}{H\vdash\left(m_{\ell}\,n\right)\textbf{Elab}\left(b\,a\right)\overrightarrow{\,:\,}B\left[x\coloneqq a\right]}\rulename{\overrightarrow{\textbf{Elab}}-fun-app}
\]
 
\[
\frac{H\vdash M\overleftarrow{\,:_{\ell',.}\,}\star\,\textbf{Elab}\ A\quad H\vdash m\overleftarrow{\,:_{\ell,.}\,}A\,\textbf{Elab}\ a}{H\vdash\left(m::_{\ell}M^{\ell'}\right)\textbf{Elab}\,a\overrightarrow{\,:\,}A}\rulename{\overrightarrow{\textbf{Elab}}-::}
\]
 
\[
\frac{H,f:\left(x:A\right)\rightarrow B,x:A\vdash m\overleftarrow{\,:_{\ell,o.bod[x]}\,}B\,\textbf{Elab}\ b}{H\vdash\left(\mathsf{fun}\,f\,x\Rightarrow m\right)\overleftarrow{\,:_{\ell,o}\,}\left(x:A\right)\rightarrow B\,\textbf{Elab}\left(\mathsf{fun}\,f\,x\Rightarrow b\right)}\rulename{\overleftarrow{\textbf{Elab}}-fun}
\]
 
\[
\frac{H\vdash m\,\textbf{Elab}\ a\overrightarrow{\,:\,}A}{H\vdash m\overleftarrow{\,:_{\ell,o}\,}B\ \textbf{Elab}\left(a::_{A,\ell,o}B\right)}\rulename{\overleftarrow{\textbf{Elab}}-cast}
\]
 
\[
\frac{H\vdash m\,\textbf{Elab}\ a\overrightarrow{\,:\,}\star}{H\vdash m\overleftarrow{\,:_{\ell,o}\,}\star\ \textbf{Elab}\,a}\rulename{\overleftarrow{\textbf{Elab}}-conv-\star}
\]
 
\todo[inline]{which syntax looks the best? on the left when input, or alway on the right like the typing judgment}
\todo[inline]{macro this stntax}
 
\caption{Elaboration}
\label{fig:elaboration}
\end{figure}
 
% rules
Like the \bidir{} rules, the rules for elaboration are broken into two judgments,
\begin{itemize}
\item $H\vdash m\overleftarrow{\,:_{\ell,o}\,}A\,\textbf{Elab}\ a$, that generates a cast term $a$ from a surface term $m$ given its expected type $A$ along with a location $\ell$ and observation $o$ that made that assertion
\item $H\vdash m\,\textbf{Elab}\ a\overrightarrow{\,:\,}A$, that generates a cast term $a$ and its type $A$ from a surface term $m$
\end{itemize}
The rules for elaboration are presented in \Fref{elaboration}.
Elaboration rules are written in a style of \bidir{} type checking.
However, unlike \bidir{} type checking, when checking an inference, elaboration adds a cast assertion that the two types are equal.
Thus any conversion checking can be suspended until runtime.
Additionally we will allow the mode to change at the type universe with the $\rulename{\overleftarrow{\textbf{Elab}}-conv-\star}$ rule, this is required to avoid stuck casts at $\star$.
As formulated here, the elaboration procedure is total.
 
There are several desirable properties of elaboration,
 
\begin{figure}
\begin{tabular}{ccc}
$|x|$ & = & $x$\tabularnewline
$|\star|$ & = & $\star$\tabularnewline
$|m::_{\ell}M|$ & = & $|m|$\tabularnewline
$|\left(x:M_{\ell}\right)\rightarrow N_{\ell'}|$ & = & $\left(x:|M|\right)\rightarrow|N|$\tabularnewline
$|m_{\ell}\,n|$ & = & $|m|\,|n|$\tabularnewline
$|\mathsf{fun}\,f\,x\Rightarrow m|$ & = & $\mathsf{fun}\,f\,x\Rightarrow|m|$\tabularnewline
$|\lozenge|$ & = & $\lozenge$\tabularnewline
$|\Gamma,x:A|$ & = & $|\Gamma|,x:|A|$\tabularnewline
$|a::_{A,\ell,o}B|$ & = & $|a|$\tabularnewline
$|\left(x:A\right)\rightarrow B|$ & = & $\left(x:|A|\right)\rightarrow|B|$\tabularnewline
$|\mathsf{fun}\,f\,x\Rightarrow b|$ & = & $\mathsf{fun}\,f\,x\Rightarrow|b|$\tabularnewline
$|b\,a|$ & = & $|b|\,|a|$\tabularnewline
$|H,x:M|$ & = & $|H|,x:|M|$\tabularnewline
\end{tabular}\caption{Erasure}
\label{fig:erasure}
\end{figure}
 
\begin{thm}
elaborated terms preserve the erasure (defined in \ref{fig:erasure})
 
if $H\vdash m\,\textbf{Elab}\ a\overrightarrow{\,:\,}A$ then $|m|=|a|$
 
if $H\vdash m\,a\overleftarrow{\,:_{\ell,o}\,}A\,\textbf{Elab}\,a$
then $|m|=|a|$
\end{thm}
 
\begin{proof}
by mutual induction on the $\textbf{Elab}$ derivations
 
\begin{tabular}{lll}
$\rulename{\overrightarrow{\textbf{Elab}}-var}$ & $|x|=x=|x|$  & \tabularnewline
$\rulename{\overrightarrow{\textbf{Elab}}-\star}$ & $|\star|=\star=|\star|$  & \tabularnewline
$\rulename{\overrightarrow{\textbf{Elab}}-fun-ty}$ & \multicolumn{2}{l}{$H\vdash M\overleftarrow{\,:_{\ell,.}\,}\star\textbf{Elab}\ A\quad H,x:A\vdash N\overleftarrow{\,:_{\ell',.}\,}\star\textbf{Elab}\ B$}\tabularnewline
& $|M|=|A|$ & by induction\tabularnewline
& $|N|=|B|$ & by induction\tabularnewline
& $|\left(x:M_{\ell}\right)\rightarrow N_{\ell'}|=\left(x:|M|\right)\rightarrow|N|=\left(x:|A|\right)\rightarrow|B|=|\left(x:A\right)\rightarrow B|$ & \tabularnewline
$\rulename{\overrightarrow{\textbf{Elab}}-fun-app}$ & \multicolumn{2}{l}{$H\vdash m\,\textbf{Elab}\ b\overrightarrow{\,:\,}C\quad C\Rrightarrow_{\ast}\left(x:A\right)\rightarrow B\quad H\vdash n\overleftarrow{\,:_{\ell,.arg}\,}A\,\textbf{Elab}\,a$}\tabularnewline
& $|m|=|b|$ & by induction\tabularnewline
& $|n|=|a|$ & by induction\tabularnewline
& $|m_{\ell}\,n|=|m|\,|n|=|b|\,|a|=|b\,a|$ & \tabularnewline
$\rulename{\overleftarrow{\textbf{Elab}}-fun}$ & \multicolumn{2}{l}{$H,f:\left(x:A\right)\rightarrow B,x:A\vdash m\overleftarrow{\,:_{\ell,o.bod[x]}\,}B\,\textbf{Elab}\ b$}\tabularnewline
& $|m|=|b|$ & by induction\tabularnewline
& $|\mathsf{fun}\,f\,x\Rightarrow m|=\mathsf{fun}\,f\,x\Rightarrow|m|=\mathsf{fun}\,f\,x\Rightarrow|b|=|\mathsf{fun}\,f\,x\Rightarrow b|$ & \tabularnewline
$\rulename{\overrightarrow{\textbf{Elab}}-::}$ & \multicolumn{2}{l}{$H\vdash M\overleftarrow{\,:_{\ell.}\,}\star\,\textbf{Elab}\ A\quad H\vdash m\overleftarrow{\,:_{\ell,.}\,}A\,\textbf{Elab}\ a$}\tabularnewline
& $|m|=|a|$ & by induction\tabularnewline
& $|m::_{\ell}M|=|m|=|a|$ & \tabularnewline
$\rulename{\overleftarrow{\textbf{Elab}}-conv}$ & $H\vdash m\,\textbf{Elab}\ a\overrightarrow{\,:\,}A$, ... & \tabularnewline
& $|m|=|a|$ & by induction\tabularnewline
& $|m|=|a|=|a::_{A,\ell,o}A'|$ & \tabularnewline
\end{tabular}
\end{proof}
It follows that whenever an elaborated cast term evaluates, the corresponding surface term evaluates consistently
\begin{thm}
Surface language and cast language have consistent evaluation
 
if $H\vdash m\,\textbf{Elab}\ a\overrightarrow{\,:\,}A$, and $a\rightsquigarrow_{*}\star$ then $m\rightsquigarrow_{*}\star$
 
if $H\vdash\textbf{Elab}\ m\,a\overrightarrow{\,:\,}A$, and $a\rightsquigarrow_{*}(x:A)\rightarrow B$ then there exists $N$ and $M$ such that $m\rightsquigarrow_{*}(x:N)\rightarrow M$
\end{thm}
 
\begin{proof}
Since $a\rightsquigarrow_{*}a'$ implies $|a|\rightsquigarrow_{*}|a'|$ and $m\rightsquigarrow_{*}m'$ implies $|m|\rightsquigarrow_{*}|m'|$.
\end{proof}
elaborated terms are well-cast in a well formed context \todo{need to define and state H ok}
\begin{thm}
elaborated terms are well-cast
 
for any $H\vdash a\,\textbf{Elab}\,m\overrightarrow{\,:\,}A$ then $H\vdash a:A$, $H\vdash A:\star$
 
for any $H\vdash$, $H\vdash A:\star$ , $H\vdash a\overleftarrow{\,:_{\ell,o}\,}A\,\textbf{Elab}\,m$ then $H\vdash a:A$
\end{thm}
 
\begin{proof}
by mutual induction on $\textbf{Elab}$ derivations
 
\todo{revise below, causing latex some issues}
 
\begin{tabular}{lll}
$\rulename{\overrightarrow{\textbf{Elab}}-var}$ & $x:A\in H$ & \tabularnewline
& $H\vdash A:\star$ & by $H\vdash$\tabularnewline
& $H\vdash x:A$ & $\rulename{cast-var}$\tabularnewline
$\rulename{\overrightarrow{\textbf{Elab}}-fun-ty}$ & \multicolumn{2}{l}{$H\vdash M\overleftarrow{\,:_{\ell,.}\,}\star\textbf{Elab}\ A\quad H,x:A\vdash N\overleftarrow{\,:_{\ell',.}\,}\star\textbf{Elab}\ B$}\tabularnewline
& $H\vdash A:\star$ & by induction\tabularnewline
& $H,x:A\vdash B:\star$ & by induction\tabularnewline
$\rulename{\overrightarrow{\textbf{Elab}}-fun-app}$ & \multicolumn{2}{l}{$H\vdash m\,\textbf{Elab}\ b\overrightarrow{\,:\,}C\quad C\Rrightarrow_{\ast}\left(x:A\right)\rightarrow B\quad H\vdash n\overleftarrow{\,:_{\ell,.arg}\,}A\,\textbf{Elab}\,a$}\tabularnewline
& $H\vdash b:C$, $H\vdash C:\star$ & by induction\tabularnewline
& $H\vdash\left(x:A\right)\rightarrow B:\star$ & by $\Rrightarrow_{\ast}$ preservation\tabularnewline
& $H\vdash A:\star$, $H,x:A\vdash B:\star$ & by fun-ty inversion\tabularnewline
& $H\vdash b:\left(x:A\right)\rightarrow B$ & $\rulename{cast-conv}$\tabularnewline
& $H\vdash a:A$ & by induction\tabularnewline
& $H\vdash B\left[x\coloneqq a\right]:\star$ & by subst preservation\tabularnewline
& $H\vdash b\,a\,:\,B\left[x\coloneqq a\right]$ & $\rulename{cast-fun-app}$\tabularnewline
$\rulename{\overleftarrow{\textbf{Elab}}-fun}$ & \multicolumn{2}{l}{$H,f:\left(x:A\right)\rightarrow B,x:A\vdash$, $H,f:\left(x:A\right)\rightarrow B,x:A\vdash B:\star$}\tabularnewline
& $H,f:\left(x:A\right)\rightarrow B,x:A\vdash m\overleftarrow{\,:_{\ell,o.bod[x]}\,}B\,\textbf{Elab}\ b$ & \tabularnewline
&  & TODO revise!\tabularnewline
& $H,f:\left(x:A\right)\rightarrow B,x:A\vdash A:\star$ & by reg\tabularnewline
& $H\vdash A:\star$ & by removing free vars\tabularnewline
& $H,x:A\vdash B:\star$ & similarly with fun-ty inversion\tabularnewline
& $H,f:\left(x:A\right)\rightarrow B,x:A\vdash b:B$ & by induction\tabularnewline
& $H\vdash\mathsf{fun}\,f\,x\Rightarrow b\,:\,\left(x:A\right)\rightarrow B$ & $\rulename{cast-fun}$\tabularnewline
$\rulename{\overrightarrow{\textbf{Elab}}-::}$ & \multicolumn{2}{l}{$H\vdash M\overleftarrow{\,:_{\ell.}\,}\star\,\textbf{Elab}\ A\quad H\vdash m\overleftarrow{\,:_{\ell,.}\,}A\,\textbf{Elab}\ a$}\tabularnewline
& $H\vdash A:\star$ & by induction\tabularnewline
& $H\vdash a:A$ & by induction\tabularnewline
$\rulename{\overleftarrow{\textbf{Elab}}-conv}$ & $H\vdash$, $H\vdash A':\star$ , $H\vdash m\,\textbf{Elab}\ a\overrightarrow{\,:\,}A$,
with some $\ell,o$ & \tabularnewline
& $H\vdash a:A$, $H\vdash A:\star$ & by induction\tabularnewline
& $H\vdash a::_{A,\ell,o}B\::\:B$ & $\rulename{cast-::}$\tabularnewline
\end{tabular}
 
Some additional properties are strongly conjectured
\end{proof}
\begin{conjecture}
Every term well typed in the \bidir{} surface language elaborates
 
if $\Gamma\vdash$, then there exists $H$ such that $\Gamma\,\textbf{Elab}\,H$
 
$\Gamma\vdash m\overrightarrow{\,:\,}M$ then there exists $H$, $a$ and $A$ such that $\Gamma\,\textbf{Elab}\,H$, $H\vdash m\,\textbf{Elab}\ a\overrightarrow{\,:\,}A$
 
$\Gamma\vdash m\overleftarrow{\,:\,}M$ and given $\ell$, $o$ then there exists $H$, $a$and $A$ such that $\Gamma\,\textbf{Elab}\,H$, $H\vdash\textbf{Elab}\,a\,m\overleftarrow{\,:_{\ell o}\,}A$
\end{conjecture}
 
Which if true would lead to the corollary
\begin{conjecture}
blame never points to something that checked in the \bidir{} system
 
if $\vdash m\overrightarrow{\,:\,}M$, and $\vdash\textbf{Elab}\ m\,a\overrightarrow{\,:\,}A$, then for no $a\rightsquigarrow_{*}a'$ will $\textbf{Blame}\:\ell\,o\:a'$ occur
 
\todo{revise precisely with labels}
\end{conjecture}
 
The last three guarantees are similar to the gradual guarantee \cite{siek_et_al:LIPIcs:2015:5031} for gradual typing.