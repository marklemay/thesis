\section{Elaboration}

\todo[inline]{resolve var for the typing context of this section: H? Gamma? some other greek?}

% overview
Even though the \clang{} allows us to optimistically assert equalities, manually noting every cast would be cumbersome.
This bureaucracy is solved with an elaboration procedure that translates (untyped) terms from the \slang{} into the \clang{}.
If the term is well typed in the \slang{}, elaboration will produce a term without blamable errors.
Terms with unproven equality in types are mapped to a cast with enough information to point out the original source when an inequality is witnessed.
 
Elaboration serves a similar role as the \bidir{} type system did in \ch{2}, and uses a similar methodology.
Instead of performing a static equality check when the inference mode and the check mode meet, a runtime cast is inserted asserting the types are equal.

In order to perform elaboration, the \slang{} needs to be enriched with location information, $\ell$, at every position that could result in a type mismatch.
This is done in \Fref{surface-pre-syntax-loc}.
Note that the location tags correspond with the check annotations of the \bidir{} system.
For technical reasons the set of locations is nonempty, and a specific null location ($.$)\todo{- instead of . ?} is designated.
That null location can be used when we need to generate fresh terms, but have no sensible location information available.
All the meta theory from \ch{2} goes through assuming that all locations are indistinguishable and by generating null locations when needed\footnote{
  For instance, the parallel reduction relation will associate all locations,
    $\frac{M\Rrightarrow M'\quad N\Rrightarrow N'}{\left(x:M_{l}\right)\rightarrow N_{l'}\Rrightarrow\left(x:M'_{l''}\right)\rightarrow N'_{l'''}}\,\rulename{\Rrightarrow-fun-ty}$,
    so that the relation does not discriminate over syntaxes that come from different locations.
  While the $\textbf{max}$ function will map terms into the null location,
    $\textbf{max}\left(\left(x:M_{\ell}\right)\rightarrow N_{\ell'}\right)=\textbf{max}\left(\left(x:\textbf{max}\left(M\right)_{.}\right)\rightarrow \textbf{max}\left(N\right)_{.}\right)$.
}.\todo{more explanation on this note? why set max up that way?}
We will avoid writing these annotations when they are unneeded (explicitly in \Fref{surface-pre-syntax-loc-abrev}).

\begin{figure}
\begin{tabular}{lcll}
\multicolumn{4}{l}{source labels,}\tabularnewline
$\ell$ & $\Coloneqq$ & ... & \tabularnewline
& $|$ & $.$ & no source label\tabularnewline
\multicolumn{4}{l}{expressions,}\tabularnewline
$m,n,M,N$ & $\Coloneqq$ & $x$ & variable\tabularnewline
& $|$ & $m::_{\ell}M^{\ell'}$ & annotation\tabularnewline
& $|$ & $\star$ & type universe\tabularnewline
& $|$ & $\left(x:M_{\ell}\right)\rightarrow N_{\ell'}$ & function type\tabularnewline
& $|$ & $\mathsf{fun}\,f\,x\Rightarrow m$ & function\tabularnewline
& $|$ & $m_{\ell}\,n$ & application\tabularnewline
\end{tabular}\caption{\SLang{} Syntax with Locations}
\label{fig:surface-pre-syntax-loc}
\end{figure}

\begin{figure}
\begin{tabular}{lclll}
$m::_{\ell}M^{\ell'}$ & written & $m::_{\ell}M$ & when & $\ell$' is irrelevant\tabularnewline
$m::_{\ell}M$ & written & $m::M$ & when & $\ell$ is irrelevant\tabularnewline
$\left(x:M_{\ell}\right)\rightarrow N_{\ell'}$ & written & $\left(x:M\right)\rightarrow N$ & when & $\ell$, $\ell'$ are irrelevant\tabularnewline
$m_{\ell}\,n$ & written & $m_{\ensuremath{}}\,n$ & when & $\ell$ is irrelevant\tabularnewline
\end{tabular}

\caption{\SLang{} Abbreviations}
\label{fig:surface-pre-syntax-loc-abrev}
\end{figure}

\subsection{Examples}

Functions will elaborate the expected types to their arguments when they are applied.
% For example,
\begin{example}
$f:\mathbb{B}_c \rightarrow\mathbb{B}_c \vdash f_{\ell}7_c \ :\mathbb{B}_c $ elaborates to $f:\mathbb{B}_c \rightarrow\mathbb{B}_c \vdash f\left(7_c ::_{\mathbb{N}_c ,\ell,.Arg}\mathbb{B}_c \right)\ :\mathbb{B}_c $.
\end{example}
 
\todo{arg is there because f may not be a function type}
 
As with \bidir{} type checking, variable types will be inferred from the typing environment.
% For example,
\begin{example}
$\vdash(\lambda x\Rightarrow 7_c)::_{\ell}\mathbb{B}_c \rightarrow\mathbb{B}_c$ elaborates to $\vdash(\lambda x\Rightarrow7::_{\mathbb{N}_c,\ell,Bod_x}\mathbb{B}_c )$.
\end{example}
\todo{these examples aer only precisly correct if the church terms infer their type}

To keep the theory simple, we allow vacuous casts to be created,
\begin{example}
$f:\mathbb{N}_c \rightarrow\mathbb{B}_c \rightarrow\mathbb{B}_c \vdash f_{\ell}7_{c\ell'}3_c \ :\mathbb{B}_c $ elaborates to $f:\mathbb{N}_c \rightarrow\mathbb{B}_c \rightarrow\mathbb{B}_c \vdash f\left(7_c ::_{\mathbb{N}_c,\ell,Arg}\mathbb{N}_c \right)\left(3_c ::_{\mathbb{N}_c,\ell',Arg}\mathbb{B}_c \right)\ :\mathbb{B}_c$.
\end{example}

\todo{dependent type example where cast muddles type}

Unlike in gradual typing, we cannot elaborate arbitrary untyped syntax.
The underlying type of a cast needs to be known so that a function type can swap its argument type at application.
For instance, $\lambda x\Rightarrow x$ will not elaborate since the intended type is not known.
Fortunately, our experimental testing suggests that a majority of randomly generated terms can be elaborated, compared to the \slang{} where only a small minority of terms would type check.
The programmer can make any term elaborate if they annotate the intended type.
For instance, $\left(\lambda x\Rightarrow x\right)::*\rightarrow*$ will elaborate.

\subsection{Elaboration Procedure}
\todo{review location placment in light of the implementation}

% rules
Like the \bidir{} rules, the rules for elaboration are broken into two judgments:
\begin{itemize}
\item $H\vdash m\overleftarrow{\,:_{\ell,o}\,}A\,\textbf{Elab}\ a$\todo{fix spacing to right of : under arrow}, that generates a cast term $a$ from a surface term $m$ given its expected type $A$ along with a location $\ell$ and observation $o$ that made that assertion.
\item $H\vdash m\,\textbf{Elab}\ a\overrightarrow{\,:\,}A$, that generates a cast term $a$ and its type $A$ from a surface term $m$.
\end{itemize}
The rules for elaboration are presented in \Fref{elaboration}.
Elaboration rules are written in a style of \bidir{} type checking, with arrows pointing in the direction information flows.
However, unlike \bidir{} type checking, when checking an inference in \rulename{\overleftarrow{\textbf{Elab}}-cast}, elaboration adds a cast assertion that the two types are equal.
Thus any conversion checking can be suspended until runtime.
Additionally we will allow the mode to change at the type universe with the \rulename{\overleftarrow{\textbf{Elab}}-conv-\star} rule, to avoid unneeded checks on the type universe.\todo{i belive this rule is no longer needed? but it is convient}
% Without the \rulename{\overleftarrow{\textbf{Elab}}-conv-\star} rule, the \slang{} term $1_c :: \mathbb{N}_c$ could not elaborate becouse runtime checks required to cherck teh well formedness of $\mathbb{N}_c$ would result in a term $\mathbb{N}_c :: \star$.
As formulated here, the elaboration procedure is terminating.

\begin{figure}
\[
\frac{
  x:A\in H
}{
  H\vdash x\,\textbf{Elab}\,x\overrightarrow{\,:\,}A
}
\rulename{\overrightarrow{\textbf{Elab}}-var}
\]

\[
\frac{\,}{H\vdash\star\,\textbf{Elab}\,\star\overrightarrow{\,:\,}\star}
\rulename{\overrightarrow{\textbf{Elab}}-\star}
\]

\[
\frac{
  H\vdash M\overleftarrow{\,:_{\ell,.}\,}\star\textbf{Elab}\ A\quad H,x:A\vdash N\overleftarrow{\,:_{\ell',.}\,}\star\textbf{Elab}\ B
}{
  H\vdash\left(\left(x:M_{\ell}\right)\rightarrow N_{\ell'}\right)\textbf{Elab}\left(\left(x:A\right)\rightarrow B\right)\overrightarrow{\,:\,}\star
}
\rulename{\overrightarrow{\textbf{Elab}}-fun-ty}
\]

\[
\frac{H\vdash m\,\textbf{Elab}\ b\overrightarrow{\,:\,}\left(x:A\right)\rightarrow B\quad H\vdash n\overleftarrow{\,:_{\ell,Arg}\,}A\,\textbf{Elab}\,a}{H\vdash\left(m_{\ell}\,n\right)\textbf{Elab}\left(b\,a\right)\overrightarrow{\,:\,}B\left[x\coloneqq a\right]}\rulename{\overrightarrow{\textbf{Elab}}-fun-app}
\]

\[
\frac{H\vdash M\overleftarrow{\,:_{\ell',.}\,}\star\,\textbf{Elab}\ A\quad H\vdash m\overleftarrow{\,:_{\ell,.}\,}A\,\textbf{Elab}\ a}{H\vdash\left(m::_{\ell}M^{\ell'}\right)\textbf{Elab}\,a\overrightarrow{\,:\,}A}
\rulename{\overrightarrow{\textbf{Elab}}-::}
\]

\[
\frac{H,f:\left(x:A\right)\rightarrow B,x:A\vdash m\overleftarrow{\,:_{\ell,o.Bod_x}\,}B\,\textbf{Elab}\ b}{H\vdash\left(\mathsf{fun}\,f\,x\Rightarrow m\right)\overleftarrow{\,:_{\ell,o}\,}\left(x:A\right)\rightarrow B\,\textbf{Elab}\left(\mathsf{fun}\,f\,x\Rightarrow b\right)}\rulename{\overleftarrow{\textbf{Elab}}-fun}
\]

\[
\frac{H\vdash m\,\textbf{Elab}\ a\overrightarrow{\,:\,}A}{H\vdash m\overleftarrow{\,:_{\ell,o}\,}B\ \textbf{Elab}\left(a::_{A,\ell,o}B\right)}\rulename{\overleftarrow{\textbf{Elab}}-cast}
\]

\[
\frac{H\vdash m\,\textbf{Elab}\ a\overrightarrow{\,:\,}\star}{H\vdash m\overleftarrow{\,:_{\ell,o}\,}\star\ \textbf{Elab}\,a}
\rulename{\overleftarrow{\textbf{Elab}}-conv-\star}
\]

\todo[inline]{which syntax looks the best? on the left when input, or alway on the right like the typing judgment}
\todo[inline]{macro this stntax}

\caption{Elaboration}
\label{fig:elaboration}
\end{figure}



There are several desirable properties of elaboration that can be shown with the help of an erasure function (defined in \ref{fig:erasure}).
Erasure is defined over all syntactic forms, removing annotations, locations, and casts.

\begin{figure}
\begin{tabular}{ccc}
$|x|$ & = & $x$\tabularnewline
$|\star|$ & = & $\star$\tabularnewline
$|m::_{\ell}M|$ & = & $|m|$\tabularnewline
$|\left(x:M_{\ell}\right)\rightarrow N_{\ell'}|$ & = & $\left(x:|M|\right)\rightarrow|N|$\tabularnewline
$|m_{\ell}\,n|$ & = & $|m|\,|n|$\tabularnewline
$|\mathsf{fun}\,f\,x\Rightarrow m|$ & = & $\mathsf{fun}\,f\,x\Rightarrow|m|$\tabularnewline
$|\lozenge|$ & = & $\lozenge$\tabularnewline
$|\Gamma,x:A|$ & = & $|\Gamma|,x:|A|$\tabularnewline
$|a::_{A,\ell,o}B|$ & = & $|a|$\tabularnewline
$|\left(x:A\right)\rightarrow B|$ & = & $\left(x:|A|\right)\rightarrow|B|$\tabularnewline
$|\mathsf{fun}\,f\,x\Rightarrow b|$ & = & $\mathsf{fun}\,f\,x\Rightarrow|b|$\tabularnewline
$|b\,a|$ & = & $|b|\,|a|$\tabularnewline
$|H,x:M|$ & = & $|H|,x:|M|$\tabularnewline
\end{tabular}
\caption{Erasure}
\label{fig:erasure}
\end{figure}

\begin{thm} Elaborated terms preserve erasure.
 
If $H\vdash m\,\textbf{Elab}\ a\overrightarrow{\,:\,}A$ then $|m|=|a|$.
 
If $H\vdash m\,a\overleftarrow{\,:_{\ell,o}\,}A\,\textbf{Elab}\,a$ then $|m|=|a|$.
\end{thm}
\begin{proof}
By mutual induction on the $\textbf{Elab}$ derivations.
\end{proof}

It follows that whenever an elaborated cast term evaluates, the corresponding surface term evaluates consistently.
Explicitly,
\begin{thm} \Slang{} and \clang{} have consistent evaluation.
 
If $H\vdash m\,\textbf{Elab}\ a\overrightarrow{\,:\,}A$, and $a\rightsquigarrow_{*}\star$ then $m\rightsquigarrow_{*}\star$.
 
If $H\vdash m\overleftarrow{\,:_{\ell,o}\,}A\,\textbf{Elab}\ a$, and $a\rightsquigarrow_{*}(x:A)\rightarrow B$ then there exists $N$ and $M$ such that $m\rightsquigarrow_{*}(x:N)\rightarrow M$.
\end{thm}

\begin{proof}
Since $a\rightsquigarrow_{*}a'$ implies $|a|\rightsquigarrow_{*}|a'|$ and $m\rightsquigarrow_{*}m'$ implies $|m|\rightsquigarrow_{*}|m'|$.
\end{proof}

Elaborated terms are well-cast in a well formed context.
We will use $H\ \textbf{ok}$ to mean for all $x$, $x : A \in H$ then $H \vdash A : \star$.

\begin{thm} Elaborated terms are well-cast.
 
For any $H\,\textbf{ok}$, $H\vdash a\,\textbf{Elab}\,m\overrightarrow{\,:\,}A$ then $H\vdash a:A$, $H\vdash A:\star$.
 
For any $H\,\textbf{ok}$, $H\vdash A:\star$, $H\vdash m\overleftarrow{\,:_{\ell,o}\,}A\,\textbf{Elab}\,a$ then $H\vdash a:A$.

For any $H\,\textbf{ok}$, $H\vdash M\overleftarrow{\,:_{\ell,o}\,}\star\,\textbf{Elab}\,A$ then $H\vdash A:\star$.
\end{thm}
\begin{proof}
By mutual induction on $\textbf{Elab}$ derivations.
\todo{double check}
\end{proof}

Some additional properties are conjectured to hold, though they have not yet been proven.

\begin{conjecture}
Every term well typed in the \bidir{} \slang{} elaborates.
 
If $\Gamma\vdash$, then there exists $H$ such that $\Gamma\,\textbf{Elab}\,H$
 
$\Gamma\vdash m\overrightarrow{\,:\,}M$ then there exists $H$, $a$ and $A$ such that $\Gamma\,\textbf{Elab}\,H$, $H\vdash m\,\textbf{Elab}\ a\overrightarrow{\,:\,}A$
 
$\Gamma\vdash m\overleftarrow{\,:\,}M$ and given $\ell$, $o$ then there exists $H$, $a$and $A$ such that $\Gamma\,\textbf{Elab}\,H$, $H\vdash\textbf{Elab}\,a\,m\overleftarrow{\,:_{\ell o}\,}A$
\end{conjecture}
 
Which if true would lead to the corollary
\begin{conjecture}
Blame never points to something that checked in the \bidir{} system.
 
If $\vdash m\overrightarrow{\,:\,}M$, and $\vdash\textbf{Elab}\ m\,a\overrightarrow{\,:\,}A$, then for no $a\rightsquigarrow_{*}a'$ will \blame{a'}{\ell}{o} occur.
 
\todo{revise precisely with labels}
\end{conjecture}
 
These properties are inspired by the gradual guarantee\cite{siek_et_al:LIPIcs:2015:5031} for gradual typing.