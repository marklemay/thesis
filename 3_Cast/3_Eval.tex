\section{\CLang{} Evaluation and Blame}

As in \ch{2} we can equip the \slang{} with a \cbv{} reduction system.

Unlike the \slang{}, it is no longer practical to characterize values syntactically.
Values are specified by judgments in \Fref{cast-val}.
They are standard except for the \rulename{Val-::}, which states that a type ($\star$ or function type) under a cast is not a value.
 
\begin{figure}
\[
\frac{\,}{\star\,\textbf{Val}}
\rulename{Val-\star}
\]
\[
\frac{\,}{\left(x:A\right)\rightarrow B\,\textbf{Val}}
\rulename{Val-fun-ty}
\]
\[
\frac{\,}{\mathsf{fun}\,f\,x\Rightarrow b\:\textbf{Val}}
\rulename{Val-fun}
\]
\[
\frac{\begin{array}{c}
a\:\textbf{Val}\quad A\:\textbf{Val}\quad B\:\textbf{Val}\\
a\cancel{=}\star\\
a\cancel{=}\left(x:C\right)\rightarrow C'
\end{array}}{a::_{A,\ell, o}B\:\textbf{Val}}
\rulename{Val-::}
\]
\caption{\CLang{} Values}
\label{fig:cast-val}
\end{figure}

For example, $(\lambda x\Rightarrow a) ::\star$ is a value while $\star ::\star$ is not a value.
Values are characterized this way to match reduction, since  $\star ::\star\ \rightsquigarrow\ \star$.
If the underling term of a cast is a type, such as $A ::_{B}B'$ then the cast will eventually reduce to $A ::_{B}B'\ \rightsquigarrow_{\ast}\ A ::_{\star}\star\ \rightsquigarrow\ A$ if there is no blame.
If there is blame then the reduction will get stuck around that cast.

\Cbv{} reductions are listed in \Fref{cast-step}.
They are standard for \cbv{} except that casts can distribute over application in \rulename{\rightsquigarrow-::-app-red}, and casts can reduce when both types are $\star$ in \rulename{\rightsquigarrow-::-\star-red}.
 
\begin{figure}
\[
\frac{a\,\textbf{Val}}{\left(\mathsf{fun}\,f\,x\Rightarrow b\right)a\rightsquigarrow b\left[f\coloneqq\mathsf{fun}\,f\,x\Rightarrow b,x\coloneqq a\right]}
\rulename{\rightsquigarrow-\mathsf{fun}-app-red}
\]
\[
\frac{b\,\textbf{Val}\quad a\,\textbf{Val}}{\begin{array}{c}
\left(b::_{\left(x:A_{1}\right)\rightarrow B_{1},\ell ,o}\left(x:A_{2}\right)\rightarrow B_{2}\right)a\rightsquigarrow\\
\left(b\,\left(a::_{A_{2},\ell,o.Arg}A_{1}\right)\right)::_{B_{1}\left[x\coloneqq a::_{A_{2},\ell,o.Arg}A_{1}\right],\ell ,o.Bod_a}B_{2}\left[x\coloneqq a\right]
\end{array}}
\rulename{\rightsquigarrow-::-app-red}
\]
\[
\frac{A\,\textbf{Val}}{A::_{\star,\ell ,o}\star\rightsquigarrow A}
\rulename{\rightsquigarrow-::-\star-red}
\]
\[
\frac{a\rightsquigarrow a'}{a::_{A,\ell ,o}B\rightsquigarrow a'::_{A,\ell ,o}B}
\rulename{\rightsquigarrow-::-1}
\]
\[
\frac{a\,\textbf{Val}\quad A\rightsquigarrow A'}{a::_{A,\ell ,o}B\rightsquigarrow a::_{A',\ell ,o}B}
\rulename{\rightsquigarrow-::-2}
\]
\[
\frac{a\,\textbf{Val}\quad A\,\textbf{Val}\quad B\rightsquigarrow B'}{a::_{A,\ell ,o}B\rightsquigarrow a::_{A,\ell ,o}B'}
\rulename{\rightsquigarrow-::-3}
\]
\[
\frac{b\rightsquigarrow b'}{b\,a\rightsquigarrow b'\,a}
\rulename{\rightsquigarrow-app-1}
\]
\[
\frac{b\,\textbf{Val}\quad a\rightsquigarrow a'}{b\,a\rightsquigarrow b\,a'}
\rulename{\rightsquigarrow-app-2}
\]
 
\caption{\CLang{} \CbV{} Reductions}
\label{fig:cast-step}
\end{figure}

As hinted at in the examples, the \rulename{\rightsquigarrow-::-app-red} rule allows an argument to be pushed under a cast between two function types.
For instance, if $\left(b::_{\left(x:A_{1}\right)\rightarrow B_{1}}\left(x:A_{2}\right)\rightarrow B_{2}\right)a$ is well cast then $b\ :\ \left(x:A_{1}\right)\rightarrow B_{1}$ and $a\ :\ A_{2}$.
We cannot move $a$ directly under the cast since it may not have the correct type. 
However, the cast asserts $\left(x:A_{1}\right)\rightarrow B_{1}\ =\ \left(x:A_{2}\right)\rightarrow B_{2}$, so we can assume $A_{1}\ =\ A_{2}$ and use it to construct a new cast for $a::_{A_{2}}A_{1}$.
Similarly we can perform the substitution $B_{1}[x\coloneqq a::_{A_{2}}A_{1}]$, since $B_{1}$ is expecting a type of $A_{1}$, while the substitution $A_{2}[x\coloneqq a]$ can be performed directly.
Finally, the location and observation data is accounted for.

The \rulename{\rightsquigarrow-::-\star-red} is the only way to remove a cast in this reduction system.
This rule is sufficient to keep reductions from getting stuck when there is no blame.
Since types without blame will have their casts reduced, leaving $\star$ or $\left(x:A\right)\rightarrow B$.
Which is exactly what is needed so that \rulename{\rightsquigarrow-::-app-red} and \rulename{\rightsquigarrow-::-\star-red} are not blocked.

The definition of $\textbf{Stuck}$ from \ch{2} applies equally well to the \clang{}: $m\ \textbf{Stuck}$ if $m$ is not a value and there does not exist $m'$ such that $m \rightsquigarrow m'$.

%% walk through blame
In addition to reductions and values we also specify blame judgments in \Fref{cast-blame}.
Blame tracks the information needed to create a good error message and is inspired by the many systems that use blame tracking \cite{10.1145/581478.581484,10.1007/978-3-642-00590-9_1,wadler:LIPIcs:2015:5033}.
Specifically the judgment \blame{a}{\ell}{o} means that $a$ witnesses a contradiction in the source code at location $\ell$ under the observations $o$.
With only dependent functions and universes, only inequalities of the form $*\,\cancel{=}\,A\rightarrow B$ can be witnessed.
The first two rules of the blame judgment witness these concrete type inequalities.
The rest of the blame rules recursively extract concrete witnesses from larger terms.
Limiting the observations to the form $*\,\cancel{=}\,A\rightarrow B$ which makes the system in this Chapter simpler than the system in \ch{5} where more observations are possible because of the addition of data.
% "self evidently correct" since it is clear to only extract inequalities. like step and val? one of several possible contradiction extraction relations
 
\begin{figure}
\[
\frac{\,}{\blame{\left(a::_{\left(x:A\right)\rightarrow B,\ell ,o}\star\right)}{\ell}{o}}
\]
\[
\frac{\,}{\blame{\left(a::_{\star,\ell ,o}\left(x:A\right)\rightarrow B\right)}{\ell}{o}}
\]
\[
\frac{
 \blame{a}{\ell}{o}
}{
 \blame{\left(a::_{A,\ell',o'}B\right)}{\ell}{o}
}
\]
\[
\frac{
 \blame{a}{\ell}{o}
}{
 \blame{\left(a::_{A,\ell',o'}B\right)}{\ell}{o}
}
\]
\[
\frac{
 \blame{B}{\ell}{o}
}{
 \blame{\left(a::_{A,\ell',o'}B\right)}{\ell}{o}
}
\]
\[
\frac{
 \blame{b}{\ell}{o}
}{
 \blame{\left(b\,a\right)}{\ell}{o}
}
\]
\[
\frac{
 \blame{a}{\ell}{o}
}{
 \blame{\left(b\,a\right)}{\ell}{o}
}
\]
\todo[inline]{name rules}
\caption{\CLang{} Blame}
\label{fig:cast-blame}
\end{figure}
 