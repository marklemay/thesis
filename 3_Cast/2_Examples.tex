\section{Examples}
% \ch{2} has presented enough 
We can re-examine some of the example terms from \ch{2}, but this time using casts that contain non standard equality assumptions.
\todo{more intro?}
 
\subsection{Higher Order Functions}
 
%% walkthrough
Higher order functions are dealt with by distributing casts around applications.
If a cast of function type is applied, the argument and body casts are separated and the arguments are swapped.
For instance:
\begin{align*}
\, & \left(\left(\lambda x\Rightarrow x\,\&\,x\right)::_{\mathbb{B}_c \rightarrow\mathbb{B}_c , \ell}\mathbb{N}_c \rightarrow\mathbb{N}_c \right) 7_c\\
\rightsquigarrow & \left(\left(\lambda x\Rightarrow x\,\&\,x\right)\left(7::_{\mathbb{N}_c , \ell, Arg}\mathbb{B}_c \right)\right)::_{\mathbb{B}_c , \ell, Bod_{7_c}}\mathbb{N}_c \\
\rightsquigarrow & \left(\left(7_c ::_{\mathbb{N}_c , \ell, Arg}\mathbb{B}_c \right)\,\&\,\left(7_c ::_{\mathbb{N}_c , \ell, Arg}\mathbb{B}_c \right)\right) ::_{\mathbb{B}_c , \ell, Bod_{7_c}}\mathbb{N}_c 
\end{align*}
\todo{use church encodings? or use data from chapter 6}
If evaluation gets stuck on $\&$ we can blame the argument of the cast for equating $\mathbb{N}_c$ and $\mathbb{B}_c$.
The $Bod$y observation records the argument the function is called with.
For instance, in the $Bod_{7_c}$ observation.
In a dependently typed function the exact argument may be important to give a good error.

\subsection{Type Universes}
 
Because casts can be embedded inside of casts, types themselves need to normalize and casts need to simplify.
Since our system has one universe of types, type casts only need to simplify themselves when a term of type $\star$ is cast to $\star$.
For instance:
\todo[inline]{better example}
\begin{align*}
\, & \left(\left(\lambda x\Rightarrow x\right)::_{\left(\mathbb{B}_c \rightarrow\mathbb{B}_c \right)::_{\star,\ell,Arg}\star, \ell'}\mathbb{N}_c \rightarrow\mathbb{N}_c \right) 7_c\\
\rightsquigarrow & \left(\left(\lambda x\Rightarrow x\right)::_{\mathbb{B}_c \rightarrow\mathbb{B}_c, \ell'}\mathbb{N}_c \rightarrow\mathbb{N}_c \right) 7_c\\
% \rightsquigarrow & \left(\left(\lambda x\Rightarrow x\right)\left(7::_{\mathbb{N}_c ,\ell,.arg}\mathbb{B}\right)\right)::_{\mathbb{B},\ell,.bod[7]}\mathbb{N}_c \\
% \rightsquigarrow & \left(\left(7::_{\mathbb{N}_c ,\ell,.arg}\mathbb{B}\right)\right)::_{\mathbb{B},\ell,.bod[7]}\mathbb{N}_c 
\end{align*}

\subsection{Pretending $true=false$}
 
Recall that we proved $\lnot true_{c}\doteq_{\mathbb{B}_{c}}false_{c}$ in \ch{2}.
What happens if it is assumed anyway?
Every type equality assumption needs an underlying term, here we can choose $refl_{true_{c}:\mathbb{B}_{c}}:true_{c}\doteq_{\mathbb{B}_{c}}true_{c}$, and cast that term to $true_{c}\doteq_{\mathbb{B}_{c}}false_{c}$ resulting in $refl_{true_{c}:\mathbb{B}_{c}}::_{true_{c}\doteq_{\mathbb{B}_{c}}true_{c}}true_{c}\doteq_{\mathbb{B}_{c}}false_{c}$.
Recall that $\lnot true_{c}\doteq_{\mathbb{B}_{c}}false_{c}$ is a shorthand for $true_{c}\doteq_{\mathbb{B}_{c}}false_{c}\rightarrow\perp_{c}$.
What if we try to use our term of type $true_{c}\doteq_{\mathbb{B}_{c}}false_{c}$ to get a term of type $\perp_{c}$?

There is enough static information to generate a warning like:

$\left(C:\left(\mathbb{B}_{c}\rightarrow\star\right)\right)\rightarrow C\,true_{c}\rightarrow\underline{C\,true_{c}}
\overset{?}{=}
\left(C:\left(\mathbb{B}_{c}\rightarrow\star\right)\right) \rightarrow C\,true_{c}\rightarrow\underline{C\,false_{c}}$

To let the programmer know they are not doing something safe.
But the program can still be run, the reductions are presented in \Fref{cast-ex-tf}.

\begin{sidewaysfigure}
\begin{tabular}{ll}
& $\left(\lambda pr\Rightarrow pr\,toLogic\ tt_{c}\right)\left(refl_{true_{c}:\mathbb{B}_{c}}::_{true_{c}\doteq_{\mathbb{B}_{c}}true_{c}}true_{c}\doteq_{\mathbb{B}_{c}}false_{c}\right)$\tabularnewline
$\rightsquigarrow$ & $\left(refl_{true_{c}:\mathbb{B}_{c}}::_{true_{c}\doteq_{\mathbb{B}_{c}}true_{c}}true_{c}\doteq_{\mathbb{B}_{c}}false_{c}\right)\,toLogic\ tt_{c}$\tabularnewline
\makecell[l]{$true_{c}\doteq_{\mathbb{B}_{c}}false_{c}$ expands to\\
  $\ \ \left(C:\left(\mathbb{B}_{c}\rightarrow\star\right)\right)\rightarrow C\,true_{c}\rightarrow C\,false_{c}$} & $\left(refl_{true_{c}:\mathbb{B}_{c}}::_{true_{c}\doteq_{\mathbb{B}_{c}}true_{c}}\left(C:\left(\mathbb{B}_{c}\rightarrow\star\right)\right)\rightarrow C\,true_{c}\rightarrow C\,false_{c}\right)\,toLogic\ tt_{c}$\tabularnewline
\makecell[l]{$true_{c}\doteq_{\mathbb{B}_{c}}true_{c}$ expands to\\
  $\ \ \left(C:\left(\mathbb{B}_{c}\rightarrow\star\right)\right)\rightarrow C\,true_{c}\rightarrow C\,true_{c}$} & 
  \makecell[l]{$\left(refl_{true_{c}:\mathbb{B}_{c}}::_{\left(C:\left(\mathbb{B}_{c}\rightarrow\star\right)\right)\rightarrow C\,true_{c}\rightarrow C\,true_{c}}\left(C:\left(\mathbb{B}_{c}\rightarrow\star\right)\right)\rightarrow C\,true_{c}\rightarrow C\,false_{c}\right)$\\
  $\ \ toLogic\ tt_{c}$}\tabularnewline

$\rightsquigarrow\rightsquigarrow_{=}$ & \makecell[l]{$(\left(refl_{true_{c}:\mathbb{B}_{c}} toLogic\right)$\\$\ \ ::_{toLogic\ true_{c}\rightarrow toLogic\ true_{c}}toLogic\ true_{c}\rightarrow toLogic\ false_{c})\ tt_{c}$}\tabularnewline

$\rightsquigarrow\rightsquigarrow_{=}$ & $\left(refl_{true_{c}:\mathbb{B}_{c}}\ toLogic\ tt_{c}\right)::_{toLogic\ true_{c}}toLogic\ false_{c}$
\tabularnewline

$refl_{true_{c}:\mathbb{B}_{c}}$ expands to $\lambda-\,cx\Rightarrow cx$ & 
  $\left( \left(\lambda-\,cx\Rightarrow cx \right) \ toLogic\ tt_{c}\right)::_{toLogic\ true_{c}}toLogic\ false_{c}$\tabularnewline
$\rightsquigarrow\rightsquigarrow$ & $tt_{c}::_{toLogic\ true_{c}}toLogic\ false_{c}$\tabularnewline

$toLogic\coloneqq\lambda b\Rightarrow b\,\star\,Unit_{c}\,\perp_{c}$ & 
  $tt_{c}::_{\left(\lambda b\Rightarrow b\,\star\,Unit_{c}\,\perp_{c}\right)\ true_{c}}\left(\lambda b\Rightarrow b\,\star\,Unit_{c}\,\perp_{c}\right)\ false_{c}$
\tabularnewline
$\rightsquigarrow\rightsquigarrow$ & $tt_{c}::_{true_{c}\,\star\,Unit_{c}\,\perp_{c}}false_{c}\,\star\,Unit_{c}\,\perp_{c}$\tabularnewline

$true_{c}\coloneqq\lambda-\,x\,-\Rightarrow x$ & $tt_{c}::_{\left(\lambda-\,x\,-\Rightarrow x\right)\,\star\,Unit_{c}\,\perp_{c}}false_{c}\,\star\,Unit_{c}\,\perp_{c}$\tabularnewline

$\rightsquigarrow\rightsquigarrow\rightsquigarrow$ & $tt_{c}::_{Unit_{c}}false_{c}\,\star\,Unit_{c}\,\perp_{c}$\tabularnewline

$\rightsquigarrow\rightsquigarrow\rightsquigarrow$ & $tt_{c}::_{Unit_{c}}false_{c}\,\star\,Unit_{c}\,\perp_{c}$\tabularnewline

$false_{c}\coloneqq\lambda-\,-\,y\Rightarrow y$ & $tt_{c}::_{Unit_{c}}\left(\lambda-\,-\,y \Rightarrow y\right)\,\star\,Unit_{c}\,\perp_{c}$\tabularnewline

$\rightsquigarrow\rightsquigarrow\rightsquigarrow$ & $tt_{c}::_{Unit_{c}}\perp_{c}$\tabularnewline
\end{tabular}
Multiple reductions are sometimes grouped together with multiple $\rightsquigarrow\rightsquigarrow...$, vacuous casts are ignored with $\rightsquigarrow_{=}$.
\todo[inline]{review, also offset left of table?}
\caption{true=false}
\label{fig:cast-ex-tf}
\end{sidewaysfigure}

The term reduces to $tt_{c}::_{Unit_{c}}\perp_{c}$, but has not yet ``gotten stuck''.
Applying the term to any input will uncover the error, so we can inspect the term with $\star$.
These reductions are listed in \Fref{cast-ex-tf-2}.

\begin{figure}
\begin{tabular}{ll}
& $\left(tt_{c}::_{Unit_{c}}\perp_{c}\right)\star$\tabularnewline
$\perp_{c}\coloneqq\left(X:\star\right)\rightarrow X$ & $\left( tt_{c}::_{Unit_{c}}\left(X:\star\right)\rightarrow X\right)  \star$\tabularnewline

$Unit_{c}\coloneqq\left(X:\star\right)\rightarrow X\rightarrow X$ & $\left( tt_{c}::_{\left(X:\star\right)\rightarrow X\rightarrow X}\left(\left(X:\star\right)\rightarrow X\right)\right)  \star$\tabularnewline

$\rightsquigarrow_{=}$ & $\left(tt_{c}\star\right)::_{\star\rightarrow\star}\star$\tabularnewline
Blame! & $\left(tt_{c}\star\right)::_{\star\underline{\rightarrow}\star}\underline{\star}$\tabularnewline
\end{tabular}
\caption{true=false cont.}
\label{fig:cast-ex-tf-2}
\end{figure}

If we explicitly tracked the location and observation information an error message could be generated:
 
$\left(C:\left(\mathbb{B}_{c}\rightarrow\star\right)\right)\rightarrow C\,true_{c}\rightarrow\underline{C\,true_{c}}
\neq
\left(C:\left(\mathbb{B}_{c}\rightarrow\star\right)\right) \rightarrow C\,true_{c}\rightarrow\underline{C\,false_{c}}$
 
when
 
$C\coloneqq\lambda b \Rightarrow b\,\star\,\star\,\perp_c$
 
$C\, true_c = \perp_c \neq \star = C\, false_c$
 
Reminding the programmer not to confuse true with false.