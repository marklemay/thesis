
\section{Examples}
\todo{lead in}
% We can reexamine some of the examples terms from Chapter 2, but this time with equality assumptions

\subsection{Higher Order Functions}

%% walkthrough
Higher order functions are dealt with by distributing function casts around applications.
If an application happens to a cast of function type, the argument and body cast is separated and the argument cast is swapped.
For instance, in 
\begin{align*}
\, & \left(\left(\lambda x\Rightarrow x\&\&x\right)::_{\mathbb{B}\rightarrow\mathbb{B},\ell,.}\mathbb{N}\rightarrow\mathbb{N}\right)7\\
\rightsquigarrow & \left(\left(\lambda x\Rightarrow x\&\&x\right)\left(7::_{\mathbb{N},\ell,.arg}\mathbb{B}\right)\right)::_{\mathbb{B},\ell,.bod[7]}\mathbb{N}\\
\rightsquigarrow & \left(\left(7::_{\mathbb{N},\ell,.arg}\mathbb{B}\right)\&\&\left(7::_{\mathbb{N},\ell,.arg}\mathbb{B}\right)\right) ::_{\mathbb{B},\ell,.bod[7]}\mathbb{N}
\end{align*}
if evaluation gets stuck on $\&\&$ we can blame the argument of the cast for equating $\mathbb{N}$ and $\mathbb{B}$.
The body observation records the argument the function is called with.
For instance, in the $.bod[7]$ observation.
In a dependently typed function the exact argument may be important to give a good error.
Because casts can be embedded inside of casts, types themselves need to normalize and casts need to simplify.
Since our system has one universe of types, type casts only need to simplify themselves when a term of type $\star$ is cast to $\star$.
For instance, 
\begin{align*}
\, & \left(\left(\lambda x\Rightarrow x\right)::_{\left(\mathbb{B}\rightarrow\mathbb{B}\right)::_{\star,\ell,.arg}\star,\ell,.}\mathbb{N}\rightarrow\mathbb{N}\right)7\\
\rightsquigarrow & \left(\left(\lambda x\Rightarrow x\right)::_{\mathbb{B}\rightarrow\mathbb{B}}\mathbb{N}\rightarrow\mathbb{N}\right)7\\
\rightsquigarrow & \left(\left(\lambda x\Rightarrow x\right)\left(7::_{\mathbb{N},\ell,.arg}\mathbb{B}\right)\right)::_{\mathbb{B},\ell,.bod[7]}\mathbb{N}\\
\rightsquigarrow & \left(\left(7::_{\mathbb{N},\ell,.arg}\mathbb{B}\right)\right)::_{\mathbb{B},\ell,.bod[7]}\mathbb{N}
\end{align*}

\subsection{Pretending $true=false$}

Recall that we proved $\lnot true_{c}\doteq_{\mathbb{B}_{c}}false_{c}$ in Chapter 2.
What happens if it is assumed anyway?
Every type equality assumption needs an underlying term, here we can choose $refl_{true_{c}:\mathbb{B}_{c}}:true_{c}\doteq_{\mathbb{B}_{c}}true_{c}$, and cast that term to $true_{c}\doteq_{\mathbb{B}_{c}}false_{c}$ resulting in $refl_{true_{c}:\mathbb{B}_{c}}::_{true_{c}\doteq_{\mathbb{B}_{c}}true_{c}}true_{c}\doteq_{\mathbb{B}_{c}}false_{c}$.
Recall that $\lnot true_{c}\doteq_{\mathbb{B}_{c}}false_{c}$ is a short hand for $true_{c}\doteq_{\mathbb{B}_{c}}false_{c}\rightarrow\perp$.
What if we try to use our term of type $true_{c}\doteq_{\mathbb{B}_{c}}false_{c}$ to get a term of type $\perp$?

\begin{sidewaysfigure}
\begin{tabular}{lr}
 & $\left(\lambda pr\Rightarrow pr\,toLogic\,tt_{c}\right)\left(refl_{true_{c}:\mathbb{B}_{c}}::_{true_{c}\doteq_{\mathbb{B}_{c}}true_{c}}true_{c}\doteq_{\mathbb{B}_{c}}false_{c}\right)$\tabularnewline
$\rightsquigarrow$ & $\left(refl_{true_{c}:\mathbb{B}_{c}}::_{true_{c}\doteq_{\mathbb{B}_{c}}true_{c}}true_{c}\doteq_{\mathbb{B}_{c}}false_{c}\right)\,toLogic\,tt_{c}$\tabularnewline
$true_{c}\doteq_{\mathbb{B}_{c}}false_{c}\coloneqq\left(C:\left(\mathbb{B}_{c}\rightarrow\star\right)\right)\rightarrow C\,true_{c}\rightarrow C\,false_{c}$ & $\left(refl_{true_{c}:\mathbb{B}_{c}}::_{true_{c}\doteq_{\mathbb{B}_{c}}true_{c}}\left(C:\left(\mathbb{B}_{c}\rightarrow\star\right)\right)\rightarrow C\,true_{c}\rightarrow C\,false_{c}\right)\,toLogic\,tt_{c}$\tabularnewline
$true_{c}\doteq_{\mathbb{B}_{c}}true_{c}\coloneqq\left(C:\left(\mathbb{B}_{c}\rightarrow\star\right)\right)\rightarrow C\,true_{c}\rightarrow C\,true_{c}$ & $\left(refl_{true_{c}:\mathbb{B}_{c}}::_{\left(C:\left(\mathbb{B}_{c}\rightarrow\star\right)\right)\rightarrow C\,true_{c}\rightarrow C\,true_{c}}\left(C:\left(\mathbb{B}_{c}\rightarrow\star\right)\right)\rightarrow C\,true_{c}\rightarrow C\,false_{c}\right)\,toLogic\,tt_{c}$\tabularnewline
$\rightsquigarrow_{=}$ & $\left(refl_{true_{c}:\mathbb{B}_{c}}\,toLogic::toLogic\,true_{c}\rightarrow toLogic\,true_{c}\right)::\left(toLogic\,true_{c}\rightarrow toLogic\,false_{c}\,tt_{c}\right)$\tabularnewline
$refl_{true_{c}:\mathbb{B}_{c}}\coloneqq\lambda-\,cx\Rightarrow cx$ & $\left(\left(\lambda-\,cx\Rightarrow cx\right)\,toLogic::toLogic\,true_{c}\rightarrow toLogic\,true_{c}\right)::\left(toLogic\,true_{c}\rightarrow toLogic\,false_{c}\,tt_{c}\right)$\tabularnewline
$\rightsquigarrow$ & $\left(\left(\lambda cx\Rightarrow cx\right)::toLogic\,true_{c}\rightarrow toLogic\,true_{c}\right)::\left(toLogic\,true_{c}\rightarrow toLogic\,false_{c}\,tt_{c}\right)$\tabularnewline
$\rightsquigarrow_{=}$ & $\left(\left(\lambda cx\Rightarrow cx\right)tt_{c}::toLogic\,true_{c}\right)::toLogic\,false_{c}$\tabularnewline
$\rightsquigarrow$ & $\left(tt_{c}::toLogic\,true_{c}\right)::toLogic\,false_{c}$\tabularnewline
$toLogic\coloneqq\lambda b\Rightarrow b\,\star\,Unit_{c}\,\perp_{c}$ & $\left(tt_{c}::\left(\lambda b\Rightarrow b\,\star\,Unit_{c}\,\perp_{c}\right)\,true_{c}\right)::toLogic\,false_{c}$\tabularnewline
$\rightsquigarrow$ & $\left(tt_{c}::\left(true_{c}\,\star\,Unit_{c}\,\perp_{c}\right)\right)::toLogic\,false_{c}$\tabularnewline
$true_{c}\coloneqq\lambda-\,x\,-\Rightarrow x$ & $\left(tt_{c}::\left(\left(\lambda-\,x\,-\Rightarrow x\right)\,\star\,Unit_{c}\,\perp_{c}\right)\right)::toLogic\,false_{c}$\tabularnewline
$\rightsquigarrow$ & $\left(tt_{c}::\left(\left(\lambda x\,-\Rightarrow x\right)\,Unit_{c}\,\perp_{c}\right)\right)::toLogic\,false_{c}$\tabularnewline
$\rightsquigarrow$ & $\left(tt_{c}::\left(\left(\lambda-\Rightarrow Unit_{c}\right)\,\perp_{c}\right)\right)::toLogic\,false_{c}$\tabularnewline
$\rightsquigarrow$ & $\left(tt_{c}::Unit_{c}\right)::toLogic\,false_{c}$\tabularnewline
$toLogic\coloneqq\lambda b\Rightarrow b\,\star\,Unit_{c}\,\perp_{c}$ & $\left(tt_{c}::Unit_{c}\right)::\left(\lambda b\Rightarrow b\,\star\,Unit_{c}\,\perp_{c}\right)\,false_{c}$\tabularnewline
$\rightsquigarrow$ & $\left(tt_{c}::Unit_{c}\right)::\left(false_{c}\,\star\,Unit_{c}\,\perp_{c}\right)$\tabularnewline
$false_{c}\coloneqq\lambda-\,-\,y\Rightarrow y$ & $\left(tt_{c}::Unit_{c}\right)::\left(\left(\lambda-\,-\,y\Rightarrow y\right)\,\star\,Unit_{c}\,\perp_{c}\right)$\tabularnewline
$\rightsquigarrow_{*}$ & $\left(tt_{c}::Unit_{c}\right)::\perp_{c}$\tabularnewline
\end{tabular}
\todo[inline]{need to clean this up}
\caption{true=false}
\label{fig:cast-ex-tf}
\end{sidewaysfigure}

The example is worked in \Fref{cast-ex-tf}.
As in the prevous examples, the term $\left(tt_{c}::Unit_{c}\right)::\perp_{c}$ has not yet ``gotten stuck''.
Applying any type will uncover the error.

\begin{tabular}{cc}
 & $\left(\left(tt_{c}::Unit_{c}\right)::\perp_{c}\right)\star$\tabularnewline
$\perp_{c}\coloneqq\left(X:\star\right)\rightarrow X$ & $\left(\left(tt_{c}::Unit_{c}\right)::\left(X:\star\right)\rightarrow X\right)\star$\tabularnewline
$\rightsquigarrow_{=}$ & $\left(\left(tt_{c}::Unit_{c}\right)\star\right)::\star$\tabularnewline
$Unit_{c}\coloneqq\left(X:\star\right)\rightarrow X\rightarrow X$ & $\left(\left(tt_{c}::\left(\left(X:\star\right)\rightarrow X\rightarrow X\right)\right)\star\right)::\star$\tabularnewline
$\rightsquigarrow_{=}$ & $\left(tt_{c}\star\right)::\left(\star\rightarrow\star\right)::\star$\tabularnewline
$tt_{c}\coloneqq\lambda-\,x\Rightarrow x$ & $\left(\left(\lambda-\,x\Rightarrow x\right)\star\right)::\left(\star\rightarrow\star\right)::\star$\tabularnewline
$\rightsquigarrow$ & $\left(\lambda x\Rightarrow x\right)::\left(\star\rightarrow\star\right)::\star$\tabularnewline
Blame! & $\left(\lambda x\Rightarrow x\right)::\left(\star\underline{\rightarrow}\star\right)::\underline{\star}$\tabularnewline
 & \tabularnewline
\end{tabular}

The location and observation information that was left out of the computation could generate an error message like

$\left(C:\left(\mathbb{B}_{c}\rightarrow\star\right)\right).C\,true_{c}\rightarrow\underline{C\,true_{c}}\neq\left(C:\left(\mathbb{B}_{c}\rightarrow\star\right)\right).C\,true_{c}\rightarrow\underline{C\,false_{c}}$

when 

$C\coloneqq\lambda b:\mathbb{B}_{c}.b\,\star\,\star\,\perp$

$C\,true_{c}=\perp\neq\star=C\,false_{c}$

Reminding the programmer that they should not confuse true with false.