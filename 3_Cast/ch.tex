\chapter{a Dependent Cast System}
\label{chapter:Cast}
\thispagestyle{myheadings}

\section{Introduction}

We can now tackle the most fundamental problem with dependent type
systems: Definitional equalities are pervasive and unintuitive.

%% example of dependent types
The motivating example form the introduction can now be stated with
the surface language. Recall, dependent types can prevent an out-of-bounds
error when indexing into a length indexed list. 

\begin{align*}
\mathtt{Vec} & :*\rightarrow\mathbb{N}_{c}\rightarrow*,\\
\mathtt{rep} & :\left(X:*\right)\rightarrow X\rightarrow\left(y:\mathbb{N}_{c}\right)\rightarrow\mathtt{Vec\,}X\,y,\\
\mathtt{head} & :\left(X:*\right)\rightarrow\left(y:\mathbb{N}_{c}\right)\rightarrow\mathtt{Vec}\,X\,\left(1_{c}+_{c}y\right)\rightarrow X
\end{align*}
\[
\vdash\lambda x\Rightarrow\mathtt{head}\,\mathbb{B}_{c}\,x\,\left(\mathtt{rep}\,\mathbb{B}_{c}\,true_{c}\,\left(1_{c}+_{c}x\right)\right)\,:\,\mathbb{N}_{c}\rightarrow\mathbb{B}_{c}
\]

$\mathtt{head}$ is a function that expects a list of length $1_{c}+_{c}y$,
making it impossible for $\mathtt{head}$ to inspect an empty list.

%% example of problem
Unfortunately, the following will not type check in the surface language,

\[
\cancel{\vdash}\lambda x\Rightarrow\mathtt{head}\,\mathbb{B}_{c}\,x\,\left(\mathtt{rep}\,\mathbb{B}_{c}\,true_{c}\,\left(x+_{c}1_{c}\right)\right)\,:\,\mathbb{N}_{c}\rightarrow\mathbb{B}_{c}
\]

%% explanation of example
\sloppy While ``obviously'' $1+x=x+1$, in the surface language
definitional equality does not associate these two terms, $1_{c}+_{c}x\cancel{\equiv}x+_{c}1_{c}$. 

In this chapter we will show how to side step definitional equality
and deal with issues at runtime in a principled way. This will be
done with the \textbf{cast system} described in this chapter. The
cast system is comprised of
\begin{itemize}
\item The \textbf{cast language}, a dependently typed language with embedded
runtime checks, that have evaluation behavior.
\item The \textbf{elaboration procedure} that transforms appropriate untyped
surface syntax into checked cast language terms\todo{as a picture}
\end{itemize}
We show that a novel form of type soundness holds, that we call \textbf{cast
soundness}. Instead of ``well typed terms don't get stuck'', we
prove ``well cast terms don't get stuck without blame''. 

Additionally, by construction, blame (in the sense of contracts and
monitors) is reasonably handled. Several desirable properties hold
for the system overall.

\section{Cast Language}

\subsection{Syntax}

\begin{figure}
\begin{tabular}{lcll}
\multicolumn{4}{l}{source location information,}\tabularnewline
$\ensuremath{\ell}$ &  &  & \tabularnewline
\multicolumn{4}{l}{variable contexts,}\tabularnewline
$H$ & $\Coloneqq$ & $\lozenge$ | $H,x:A$ & \tabularnewline
\multicolumn{4}{l}{expressions,}\tabularnewline
$a,b$,$A,B$ & $\Coloneqq$ & $x$ & \tabularnewline
 & | & $a::_{A,\ensuremath{\ell},o}B$ & cast\tabularnewline
 & | & $\star$ & \tabularnewline
 & | & $\left(x:A\right)\rightarrow B$ & \tabularnewline
 & | & $\mathsf{fun}\,f\,x\Rightarrow b$ & \tabularnewline
 & | & $b\,a$ & \tabularnewline
\multicolumn{4}{l}{observations,}\tabularnewline
o & $\Coloneqq$ & . & current location\tabularnewline
 & | & $o.arg$ & function type-arg\tabularnewline
 & | & $o.bod[a]$ & function type-body\tabularnewline
\end{tabular}

\todo[inline]{would $a_{A}:?:_{\ensuremath{\ell},o}B$ be clearer syntax then $a::_{A,\ensuremath{\ell},o}B$
?}

\todo[inline]{would it be clearer to add the $\ell$ to the observation?}

\caption{Cast Language Syntax}
\label{fig:cast-pre-syntax}
\end{figure}

%% pre-syntax
The syntax for the cast language can be seen in figure \ref{fig:cast-pre-syntax}.
By design the cast language is almost identical to the surface language
except that the cast construct has been added and annotations have
been removed.

The cast language can assume arbitrary equalities over types, $A=B$,
with a cast, $a::_{A,\ensuremath{\ell},o}B$ given
\begin{itemize}
\item an underlying term $a$
\item a source location $\ell$ where it was asserted, 
\item a concrete observation $o$ that would witness inequality, 
\item the type of the underlying $a$ term $A$, 
\item and the expected type of the term $B$. 
\end{itemize}
Every time there is a definitional mismatch between the type inferred
from a term and the type expected from the usage, the elaboration
procedure will produce a cast.

Observations allow indexing into terms to pinpoint errors. For instance
if we want to highlight the $C$ sub expression in $\left(x:A\right)\rightarrow\left(y:\left(x:B\right)\rightarrow\underline{C}\right)\rightarrow D$
we can use the observation $..Bod[x].arg.Bod[y]$. In general, the
$C$ may specifically depend on $x$ and $y$ so they are tracked
as part of the observation. For instance, given the type $\left(X:\star\right)\rightarrow X$
we might want to point out $A$ when $X=A\rightarrow B$ resulting
in the type $\left(X:\star\right)\rightarrow\left(\underline{A}\rightarrow B\right)$
, the observation would then read $..Bod[A\rightarrow B].arg$ recording
the specific type argument that allows an argument to be inspected.

Locations and observations will be used to form blame and produce
the runtime error message users might see.

In addition to the abbreviations from Chapter 2, some new abbreviations
for the cast language are listed in Figure \ref{fig:cast-pre-syntax-abrev}.
\begin{figure}
\begin{tabular}{lclll}
$a::_{A,\ensuremath{\ell},o}B$ & written & $a::_{A,\ensuremath{\ell}}B$ & when & the observation is not relevant\tabularnewline
$a::_{A,\ensuremath{\ell}}B$ & written & $a::_{A}B$ & when & the location is not relevant\tabularnewline
$a::_{A}B$ & written & $a::B$ & when & the the type of $a$is clear\tabularnewline
\end{tabular}

\caption{Surface Language Abbreviations}
\label{fig:cast-pre-syntax-abrev}
\end{figure}


\subsection{How Should Casts Reduce?}

How does the cast construct interact with the existing constructs?
These are all the interactions that could cause a term to be stuck
in evaluation or block type checking

\begin{tabular}{cc}
$\star::B$ & universe under cast\tabularnewline
$\left(\left(x:A\right)\rightarrow B\right)::C$ & function type under cast\tabularnewline
$\left(b::C\right)a$ & application blocked by cast\tabularnewline
\end{tabular}

We can account for these by realizing obvious casts should evaporate,
freeing up the underling term. The most interesting case is when a
cast confirms that the applied term is a function, but with potentially
different input and output types. Then we use the function cast to
determine a reasonable cast over the argument, and maintain the appropriate
cast over the resulting computation. This operation is similar to
the way higher order contracts invert the polarity of blame for the
arguments of higher order functions \cite{10.1145/581478.581484}
\footnote{and also found in gradual type systems, such as \cite{10.1007/978-3-642-00590-9_1}}.

\begin{tabular}{ccccc}
$\star::\star$ & $\rightsquigarrow$ & $\star$ &  & \tabularnewline
$\star::B$ & $\rightsquigarrow$ & $\star::B'$ & when  & $B\rightsquigarrow B'$\tabularnewline
$\star::_{\ell,o}B$ & blame & $\ell,o$ & when  & $B$ cannot be $\star$ \tabularnewline
$\left(\left(x:A\right)\rightarrow B\right)::\star$ & $\rightsquigarrow$ & $\star$ &  & \tabularnewline
$\left(\left(x:A\right)\rightarrow B\right)::C$ & $\rightsquigarrow$ & $\left(\left(x:A\right)\rightarrow B\right)::C'$ & when  & $C\rightsquigarrow C'$\tabularnewline
$\left(\left(x:A\right)\rightarrow B\right)::_{\ell,o}C$ & blame & $\ell,o$ & when  & $C$ cannot be $\star$ \tabularnewline
$\left(b::_{\left(x:A'\right)\rightarrow B'}\left(x:A\right)\rightarrow B\right)a$ & $\rightsquigarrow$ & $\left(b\,\left(a::A'\right)\right)::_{B'\left[x\coloneqq a::A'\right]}B\left[x\coloneqq A\right]$ &  & \tabularnewline
$\left(b::C\right)a$ & $\rightsquigarrow$ & $\left(b::C'\right)a$ & when  & $C\rightsquigarrow C'$\tabularnewline
$\left(b::_{\ell,o}C\right)a$ & blame & $\ell,o$ & when  & $C$ cannot be $\left(x:A\right)\rightarrow B$ \tabularnewline
\end{tabular}

The fully formal reduction rules are listed later, but their complete
detail can be distracting. Type universes live in the type universe,
so any cast that contradicts this should be blamed. Similarly for
function types. Terms that take input must be functions, so any cast
that contradicts this should blame the source location.

Note that the rules outlined here are not deterministic since there
are cases when we might blame or continue reducing the argument. One
of the subtle innovations of the system described in this chapter
is to completely separate blame from reduction. This sidesteps many
of the complexities of having a reduction relevant $\mathtt{abort}$
term in a dependent type theory \cite{sjoberg2012irrelevance,pedrot2018failure}.
As far as reduction is concerned, bad terms simply ``get stuck''
as it might on a variable assumed in the typing context. Otherwise
the reduction behavior is well behaved. Terms my be blamable by the
rules outlined in this chapter, but it is easy to imagine more sophisticated
ways to extract blame without interfering with reduction. 

This outlines the minimum requirements for cast reductions, there
are plausibly many additional reductions that could be considered.
Some tempting reductions are

\begin{tabular}{ccccc}
$a::_{C}C$ & $\rightsquigarrow_{=}$ & $a$ &  & \tabularnewline
$a::_{C'}C$ & $\rightsquigarrow_{\equiv}$ & $a$ & when & $C'\equiv C$\tabularnewline
\end{tabular}

However these rules preclude extracting blame that may be embedded
within the casts themselves. These rules also seem to complicate the
type theory. Despite this we will use these reductions in examples
to keep the book keeping to a minimum.

\subsection{Examples }

% We can reexamine some of the examples terms from Chapter 2, but this time with equality assumptions

\subsubsection{Higher Order Functions}

%% walkthrough
 Higher order functions are dealt with by distributing function casts
around applications. If an application happens to a cast of function
type, the argument and body cast is separated and the argument cast
is swapped. For instance in 
\begin{align*}
\, & \left(\left(\lambda x\Rightarrow x\&\&x\right)::_{\mathbb{B}\rightarrow\mathbb{B},\ell,.}\mathbb{N}\rightarrow\mathbb{N}\right)7\\
\rightsquigarrow & \left(\left(\lambda x\Rightarrow x\&\&x\right)\left(7::_{\mathbb{N},\ell,.arg}\mathbb{B}\right)\right)::_{\mathbb{B},\ell,.bod[7]}\mathbb{N}\\
\rightsquigarrow & \left(\left(7::_{\mathbb{N},\ell,.arg}\mathbb{B}\right)\&\&\left(7::_{\mathbb{N},\ell,.arg}\mathbb{B}\right)\right)\\
\, & \ ::_{\mathbb{B},\ell,.bod[7]}\mathbb{N}
\end{align*}
if evaluation gets stuck on $\&\&$ and we can blame the argument
of the cast for equating $\mathbb{N}$ and $\mathbb{B}$. The body
observation records the argument the function is called with. For
instance in the $.bod[7]$ observation. In a dependently typed function
the exact argument may be important to give a good error. Because
casts can be embedded inside of casts, types themselves need to normalize
and casts need to simplify. Since our system has one universe of types,
type casts only need to simplify themselves when a term of type $\star$
is cast to $\star$. For instance, 
\begin{align*}
\, & \left(\left(\lambda x\Rightarrow x\right)::_{\left(\mathbb{B}\rightarrow\mathbb{B}\right)::_{\star,\ell,.arg}\star,\ell,.}\mathbb{N}\rightarrow\mathbb{N}\right)7\\
\rightsquigarrow & \left(\left(\lambda x\Rightarrow x\right)::_{\mathbb{B}\rightarrow\mathbb{B}}\mathbb{N}\rightarrow\mathbb{N}\right)7\\
\rightsquigarrow & \left(\left(\lambda x\Rightarrow x\right)\left(7::_{\mathbb{N},\ell,.arg}\mathbb{B}\right)\right)::_{\mathbb{B},\ell,.bod[7]}\mathbb{N}\\
\rightsquigarrow & \left(\left(7::_{\mathbb{N},\ell,.arg}\mathbb{B}\right)\right)::_{\mathbb{B},\ell,.bod[7]}\mathbb{N}
\end{align*}


\subsubsection{Pretending $true=false$}

Recall that we proved $\lnot true_{c}\doteq_{\mathbb{B}_{c}}false_{c}$
in Chapter 2. What happens if it is assumed anyway? Every type equality
assumption needs an underlying term, here we can choose $refl_{true_{c}:\mathbb{B}_{c}}:true_{c}\doteq_{\mathbb{B}_{c}}true_{c}$,
and cast that term to $true_{c}\doteq_{\mathbb{B}_{c}}false_{c}$
resulting in $refl_{true_{c}:\mathbb{B}_{c}}::_{true_{c}\doteq_{\mathbb{B}_{c}}true_{c}}true_{c}\doteq_{\mathbb{B}_{c}}false_{c}$.
Recall that $\lnot true_{c}\doteq_{\mathbb{B}_{c}}false_{c}$ is a
short hand for $true_{c}\doteq_{\mathbb{B}_{c}}false_{c}\rightarrow\perp$.
What if we try to use our term of type $true_{c}\doteq_{\mathbb{B}_{c}}false_{c}$
to get a term of type $\perp$?

\todo[inline]{break out figure}

\begin{tabular}{cc}
 & $\left(\lambda pr\Rightarrow pr\,toLogic\,tt_{c}\right)\left(refl_{true_{c}:\mathbb{B}_{c}}::_{true_{c}\doteq_{\mathbb{B}_{c}}true_{c}}true_{c}\doteq_{\mathbb{B}_{c}}false_{c}\right)$\tabularnewline
$\rightsquigarrow$ & $\left(refl_{true_{c}:\mathbb{B}_{c}}::_{true_{c}\doteq_{\mathbb{B}_{c}}true_{c}}true_{c}\doteq_{\mathbb{B}_{c}}false_{c}\right)\,toLogic\,tt_{c}$\tabularnewline
$true_{c}\doteq_{\mathbb{B}_{c}}false_{c}\coloneqq\left(C:\left(\mathbb{B}_{c}\rightarrow\star\right)\right)\rightarrow C\,true_{c}\rightarrow C\,false_{c}$ & $\left(refl_{true_{c}:\mathbb{B}_{c}}::_{true_{c}\doteq_{\mathbb{B}_{c}}true_{c}}\left(C:\left(\mathbb{B}_{c}\rightarrow\star\right)\right)\rightarrow C\,true_{c}\rightarrow C\,false_{c}\right)\,toLogic\,tt_{c}$\tabularnewline
$true_{c}\doteq_{\mathbb{B}_{c}}true_{c}\coloneqq\left(C:\left(\mathbb{B}_{c}\rightarrow\star\right)\right)\rightarrow C\,true_{c}\rightarrow C\,true_{c}$ & $\left(refl_{true_{c}:\mathbb{B}_{c}}::_{\left(C:\left(\mathbb{B}_{c}\rightarrow\star\right)\right)\rightarrow C\,true_{c}\rightarrow C\,true_{c}}\left(C:\left(\mathbb{B}_{c}\rightarrow\star\right)\right)\rightarrow C\,true_{c}\rightarrow C\,false_{c}\right)\,toLogic\,tt_{c}$\tabularnewline
$\rightsquigarrow_{=}$ & $\left(refl_{true_{c}:\mathbb{B}_{c}}\,toLogic::toLogic\,true_{c}\rightarrow toLogic\,true_{c}\right)::\left(toLogic\,true_{c}\rightarrow toLogic\,false_{c}\,tt_{c}\right)$\tabularnewline
$refl_{true_{c}:\mathbb{B}_{c}}\coloneqq\lambda-\,cx\Rightarrow cx$ & $\left(\left(\lambda-\,cx\Rightarrow cx\right)\,toLogic::toLogic\,true_{c}\rightarrow toLogic\,true_{c}\right)::\left(toLogic\,true_{c}\rightarrow toLogic\,false_{c}\,tt_{c}\right)$\tabularnewline
$\rightsquigarrow$ & $\left(\left(\lambda cx\Rightarrow cx\right)::toLogic\,true_{c}\rightarrow toLogic\,true_{c}\right)::\left(toLogic\,true_{c}\rightarrow toLogic\,false_{c}\,tt_{c}\right)$\tabularnewline
$\rightsquigarrow_{=}$ & $\left(\left(\lambda cx\Rightarrow cx\right)tt_{c}::toLogic\,true_{c}\right)::toLogic\,false_{c}$\tabularnewline
$\rightsquigarrow$ & $\left(tt_{c}::toLogic\,true_{c}\right)::toLogic\,false_{c}$\tabularnewline
$toLogic\coloneqq\lambda b\Rightarrow b\,\star\,Unit_{c}\,\perp_{c}$ & $\left(tt_{c}::\left(\lambda b\Rightarrow b\,\star\,Unit_{c}\,\perp_{c}\right)\,true_{c}\right)::toLogic\,false_{c}$\tabularnewline
$\rightsquigarrow$ & $\left(tt_{c}::\left(true_{c}\,\star\,Unit_{c}\,\perp_{c}\right)\right)::toLogic\,false_{c}$\tabularnewline
$true_{c}\coloneqq\lambda-\,x\,-\Rightarrow x$ & $\left(tt_{c}::\left(\left(\lambda-\,x\,-\Rightarrow x\right)\,\star\,Unit_{c}\,\perp_{c}\right)\right)::toLogic\,false_{c}$\tabularnewline
$\rightsquigarrow$ & $\left(tt_{c}::\left(\left(\lambda x\,-\Rightarrow x\right)\,Unit_{c}\,\perp_{c}\right)\right)::toLogic\,false_{c}$\tabularnewline
$\rightsquigarrow$ & $\left(tt_{c}::\left(\left(\lambda-\Rightarrow Unit_{c}\right)\,\perp_{c}\right)\right)::toLogic\,false_{c}$\tabularnewline
$\rightsquigarrow$ & $\left(tt_{c}::Unit_{c}\right)::toLogic\,false_{c}$\tabularnewline
$toLogic\coloneqq\lambda b\Rightarrow b\,\star\,Unit_{c}\,\perp_{c}$ & $\left(tt_{c}::Unit_{c}\right)::\left(\lambda b\Rightarrow b\,\star\,Unit_{c}\,\perp_{c}\right)\,false_{c}$\tabularnewline
$\rightsquigarrow$ & $\left(tt_{c}::Unit_{c}\right)::\left(false_{c}\,\star\,Unit_{c}\,\perp_{c}\right)$\tabularnewline
$false_{c}\coloneqq\lambda-\,-\,y\Rightarrow y$ & $\left(tt_{c}::Unit_{c}\right)::\left(\left(\lambda-\,-\,y\Rightarrow y\right)\,\star\,Unit_{c}\,\perp_{c}\right)$\tabularnewline
$\rightsquigarrow_{*}$ & $\left(tt_{c}::Unit_{c}\right)::\perp_{c}$\tabularnewline
\end{tabular}

As in the above the example, the term $\left(tt_{c}::Unit_{c}\right)::\perp_{c}$
has not yet ``gotten stuck''. Applying any type will uncover the
error.

\begin{tabular}{cc}
 & $\left(\left(tt_{c}::Unit_{c}\right)::\perp_{c}\right)\star$\tabularnewline
$\perp_{c}\coloneqq\left(X:\star\right)\rightarrow X$ & $\left(\left(tt_{c}::Unit_{c}\right)::\left(X:\star\right)\rightarrow X\right)\star$\tabularnewline
$\rightsquigarrow_{=}$ & $\left(\left(tt_{c}::Unit_{c}\right)\star\right)::\star$\tabularnewline
$Unit_{c}\coloneqq\left(X:\star\right)\rightarrow X\rightarrow X$ & $\left(\left(tt_{c}::\left(\left(X:\star\right)\rightarrow X\rightarrow X\right)\right)\star\right)::\star$\tabularnewline
$\rightsquigarrow_{=}$ & $\left(tt_{c}\star\right)::\left(\star\rightarrow\star\right)::\star$\tabularnewline
$tt_{c}\coloneqq\lambda-\,x\Rightarrow x$ & $\left(\left(\lambda-\,x\Rightarrow x\right)\star\right)::\left(\star\rightarrow\star\right)::\star$\tabularnewline
$\rightsquigarrow$ & $\left(\lambda x\Rightarrow x\right)::\left(\star\rightarrow\star\right)::\star$\tabularnewline
Blame! & $\left(\lambda x\Rightarrow x\right)::\left(\star\underline{\rightarrow}\star\right)::\underline{\star}$\tabularnewline
 & \tabularnewline
\end{tabular}

The location and observation information that was left out of the
computation could generate an error message like

$\left(C:\left(\mathbb{B}_{c}\rightarrow\star\right)\right).C\,true_{c}\rightarrow\underline{C\,true_{c}}\neq\left(C:\left(\mathbb{B}_{c}\rightarrow\star\right)\right).C\,true_{c}\rightarrow\underline{C\,false_{c}}$

when 

$C\coloneqq\lambda b:\mathbb{B}_{c}.b\,\star\,\star\,\perp$

$C\,true_{c}=\perp\neq\star=C\,false_{c}$

Reminding the programmer that they should not confuse true with false.

\subsection{Cast Language Type Assignment System}

% why cast soundness?

Recall that type soundness is the primary property for a typed programming
language to exhibit. In a programming language, type soundness proves
some undesirable behaviors are unreachable from a well typed term.
How should this apply to the cast language, where bad behaviors are
intended to be reachable? We allow bad behavior, but require that
when a bad state is reached we blame the original faulty type annotations.
Where the slogan for type soundness is ``well typed terms don't get
stuck'', the slogan for cast soundness is ``well cast terms don't
get stuck without blame''.

In Chapter 2 we proved type soundness for a minimal dependently typed
language, with a progress and preservation style proof given a suitable
definition of term equivalence. We can extend that proof to support
cast soundness with only a few modifications.

\begin{figure}
\[
\frac{x:A\in H}{H\vdash x\,:\,A}\operatorname{cast-var}
\]
\[
\frac{H\vdash a:A\quad H\vdash A:\star\quad H\vdash B:\star}{H\vdash a::_{A,\ensuremath{\ell},o}B\::\:B}\operatorname{cast-::}
\]

\[
\frac{\,}{H\vdash\star:\,\star}\operatorname{cast-\star}
\]

\[
\frac{H\vdash A:\star\quad H,x:A\vdash B:\star}{H\vdash\left(x:A\right)\rightarrow B\,:\,\star}\operatorname{cast-fun-ty}
\]

\[
\frac{H,f:\left(x:A\right)\rightarrow B,x:A\vdash b:B}{H\vdash\mathsf{fun}\,f\,x\Rightarrow b\,:\,\left(x:A\right)\rightarrow B}\operatorname{cast-fun}
\]

\[
\frac{H\vdash b:\left(x:A\right)\rightarrow B\quad H\vdash a:A}{H\vdash b\,a\,:\,B\left[x\coloneqq a\right]}\operatorname{cast-fun-app}
\]

\[
\frac{H\vdash a:A\quad A\equiv A'}{H\vdash a:A'}\operatorname{cast-conv}
\]

\todo[inline]{review regularity stuff}

\caption{Cast Language Type Assignment Rules}
\label{fig:cast-tas-rules}
\end{figure}

The cast language supports its own type assignment system, Figure
\ref{fig:cast-tas-rules}. This system ensures that computations will
not get stuck without enough information for good runtime error messages.
Specifically computations will not get stuck without a source location
and a witness of inequality. The only rule that types differently
than the surface language is the $\operatorname{cast-::}$ rule that
allows any type to appear like another type. 

As before we need a suitable reduction relation to generate our equivalence
relation. Figure \ref{fig:cast-reduction} shows that system of reductions.
The full rule for function reduction is given in $\textrm{\ensuremath{\Rrightarrow}-fun-::-red}$
which makes the intuition listed above explicit. Casts from a type
univer to a type universe are allowed by the $\textrm{\ensuremath{\Rrightarrow}-::-red}$
rule. Since observations embed expressions, they must also be given
parallel reductions.

\begin{figure}
\[
\frac{b\Rrightarrow b'\quad a\Rrightarrow a'}{\left(\mathsf{fun}\,f\,x\Rightarrow b\right)a\Rrightarrow b'\left[f\coloneqq\mathsf{fun}\,f\,x\Rightarrow b',x\coloneqq a'\right]}\,\textrm{\ensuremath{\Rrightarrow}-fun-app-red}
\]

\[
\frac{b\Rrightarrow b'\quad a\Rrightarrow a'\quad A_{1}\Rrightarrow A_{1}'\quad A_{2}\Rrightarrow A_{2}'\quad B_{1}\Rrightarrow B_{1}'\quad B_{2}\Rrightarrow B_{2}'\quad o\Rrightarrow o'}{\begin{array}{c}
\left(b::_{\left(x:A_{1}\right)\rightarrow B_{1},\ensuremath{\ell},o}\left(x:A_{2}\right)\rightarrow B_{2}\right)a\Rrightarrow\\
\left(b'\,a'::_{A_{2}',\ell,o.arg}A_{1}'\right)::_{B_{1}'\left[x\coloneqq a'::_{A_{2}',\ell,o.arg}A_{1}'\right],\ensuremath{\ell},o'.bod[a']}B_{2}'\left[x\coloneqq a'\right]
\end{array}}\,\textrm{\ensuremath{\Rrightarrow}-fun-::-red}
\]

\[
\frac{a\Rrightarrow a'}{a::_{\star,\ensuremath{\ell},o}\star\Rrightarrow a'}\,\textrm{\ensuremath{\Rrightarrow}-::-red}
\]

\[
\frac{\,}{x\Rrightarrow x}\,\textrm{\ensuremath{\Rrightarrow}-var}
\]
\[
\frac{a\Rrightarrow a'\quad A_{1}\Rrightarrow A_{1}'\quad A_{2}\Rrightarrow A_{2}'\quad o\Rrightarrow o'}{a::_{A_{1},\ensuremath{\ell},o}A_{2}\Rrightarrow a'::_{A_{1}',\ensuremath{\ell},o'}A_{2}'}\,\textrm{\ensuremath{\Rrightarrow}-::}
\]

\[
\frac{\,}{\star\Rrightarrow\star}\,\textrm{\ensuremath{\Rrightarrow}-}\star
\]

\[
\frac{A\Rrightarrow A'\quad B\Rrightarrow B'}{\left(x:A\right)\rightarrow B\Rrightarrow\left(x:A'\right)\rightarrow B'}\,\textrm{\ensuremath{\Rrightarrow}-fun-ty}
\]

\[
\frac{b\Rrightarrow b'}{\mathsf{fun}\,f\,x\Rightarrow b\,\Rrightarrow\,\mathsf{fun}\,f\,x\Rightarrow b'}\,\textrm{\ensuremath{\Rrightarrow}-fun}
\]

\[
\frac{b\Rrightarrow b'\quad a\Rrightarrow a'}{b\,a\Rrightarrow b'\,a'}\,\textrm{\ensuremath{\Rrightarrow}-fun-app}
\]

\[
\frac{\,}{.\Rrightarrow.}\,\textrm{\ensuremath{\Rrightarrow}-obs-emp}
\]

\[
\frac{o\Rrightarrow o'}{o.arg\Rrightarrow o'.arg}\,\textrm{\ensuremath{\Rrightarrow}-obs-arg}
\]

\[
\frac{o\Rrightarrow o'\quad a\Rrightarrow a'}{o.bod[a]\Rrightarrow o'.bod[a']}\,\textrm{\ensuremath{\Rrightarrow}-obs-bod}
\]

\[
\frac{\,}{a\Rrightarrow_{\ast}a}\,\textrm{\ensuremath{\Rrightarrow_{\ast}}-refl}
\]
\[
\frac{a\Rrightarrow_{\ast}a'\quad a'\Rrightarrow a''}{a\Rrightarrow_{\ast}a''}\,\textrm{\ensuremath{\Rrightarrow_{\ast}}-trans}
\]
\[
\frac{a\Rrightarrow_{\ast}a''\quad a'\Rrightarrow_{\ast}a''}{a\equiv a'}\,\textrm{\ensuremath{\equiv}-def}
\]

\todo[inline]{font stuff}\caption{Cast Language Parallel Reductions}
\label{fig:cast-reduction}
\end{figure}


\subsubsection{Definitional Equality}

\todo{Address the irony of using a def eq to avoid the issues of a def eq}

As in Chapter 2, $\Rrightarrow_{*}$ can be shown to be confluent.
The proofs follow the same structure, but since observations can contain
terms, $\Rrightarrow$ and $max$ must be extended to observations.
Proofs must be extended to mutually induct on observations, since
they can contain expressions that could also reduce.

\begin{tabular}{ccccc}
$max($ & $\left(\mathsf{fun}\,f\,x\Rightarrow b\right)\,a$ & $)=$ & $max\left(b\right)\left[f\coloneqq\mathsf{fun}\,f\,x\Rightarrow max\left(b\right),x\coloneqq max\left(a\right)\right]$ & otherwise\tabularnewline
\multirow{2}{*}{$max($} & \multirow{2}{*}{$\left(b::_{\left(x:A_{1}\right)\rightarrow B_{1},\ensuremath{\ell},o}\left(x:A_{2}\right)\rightarrow B_{2}\right)a$} & \multirow{2}{*}{$)=$} & $\left(max\left(b\right)\,\left(max\left(a\right)::_{max\left(A_{2}\right),\ell,max\left(o\right).arg}max\left(A_{1}\right)\right)\right)$ & \multirow{2}{*}{otherwise}\tabularnewline
 &  &  & $::_{max\left(B_{2}\right)\left[x\coloneqq max\left(a\right)::_{max\left(A_{2}\right),\ell,max\left(o\right).arg}max\left(A_{1}\right)\right],\ensuremath{\ell},max\left(o\right).bod[max\left(a\right)]}max\left(B_{2}\right)\left[x\coloneqq max\left(a\right)\right]$ & \tabularnewline
$max($ & $b::_{B_{1},\ensuremath{\ell},o}B_{2}$ & $)=$ & $max\left(b\right)::_{max\left(B_{1}\right),\ensuremath{\ell},max\left(o\right)}max\left(B_{2}\right)$ & \tabularnewline
$max($ & ... & $)=$ & ... & corresponds to the definition in chapter 2\tabularnewline
$max($ & $.$ & $)=$ & $.$ & \tabularnewline
$max($ & $o.arg$ & $)=$ & $max\left(o\right).arg$ & \tabularnewline
$max($ & $o.bod[a]$ & $)=$ & $max\left(o\right).bod[max\left(a\right)]$ & \tabularnewline
\end{tabular}
\begin{lem}
Triangle properties

If $a\Rrightarrow a'$ then $a'\Rrightarrow max\left(a\right)$ 

If $o\Rrightarrow o'$ then $o'\Rrightarrow max\left(o\right)$
\end{lem}

\begin{proof}
by mutual induction on the derivations of $m\Rrightarrow\,m'$ and
$o\Rrightarrow o'$.
\end{proof}
\todo{check this}
\begin{lem}
Diamond property 

If $a\Rrightarrow a'$, $a\Rrightarrow a''$, implies $a'\Rrightarrow\,max\left(a\right)$
,$a''\Rrightarrow\,max\left(a\right)$ 
\end{lem}

\begin{proof}
The follows directly from triangle property
\end{proof}
\begin{thm}
$\equiv$ is transitive

If $a\equiv a'$, $a'\equiv a''$, implies $a\equiv a''$ 
\end{thm}

\begin{proof}
The diamond property implies the confluence of $\Rrightarrow_{\ast}$
.
\end{proof}
\begin{lem}
Stability (generalized)

$\forall A,B,C.\left(x:A\right)\rightarrow B\Rrightarrow_{\ast}C\:\Rightarrow\:\exists A',B'.C=\left(x:A'\right)\rightarrow B'\land A\Rrightarrow_{\ast}A'\land B\Rrightarrow_{\ast}B'$
\end{lem}

\begin{proof}
by induction on $\Rrightarrow_{\ast}$
\end{proof}
\begin{cor}
Stability

The following rule is admissible
\[
\frac{\left(x:A\right)\rightarrow B\equiv\left(x:A'\right)\rightarrow B'}{A\equiv A'\quad B\equiv B'}
\]
\end{cor}


\subsubsection{Preservation}
\begin{lem}
Context Weakening

The following rule is admissible

\[
\frac{H\vdash a:A}{H,H'\vdash a:A}
\]
\end{lem}

\begin{proof}
by induction on typing derivations
\end{proof}
\begin{lem}
Substitution Preservation

The following rule is admissible

\[
\frac{H\vdash c:C\quad H,x:C,H'\vdash a:A}{H,H'\left[x\coloneqq c\right]\vdash a\left[x\coloneqq c\right]:A\left[x\coloneqq c\right]}
\]
\end{lem}

\begin{proof}
by induction over typing derivations

\begin{tabular}{lll}
$\operatorname{cast-::}$ & \multicolumn{2}{l}{$H,x:C,H'\vdash a:A$, $H,x:C,H'\vdash A:\star$, $H,x:C,H'\vdash B:\star$,
wellformed $\ensuremath{\ell},o$}\tabularnewline
 & $H,H'\left[x\coloneqq c\right]\vdash a\left[x\coloneqq c\right]:A\left[x\coloneqq c\right]$ & by induction\tabularnewline
 & $H,H'\left[x\coloneqq c\right]\vdash A\left[x\coloneqq c\right]:\star$ & by induction\tabularnewline
 & $H,H'\left[x\coloneqq c\right]\vdash B\left[x\coloneqq c\right]:\star$ & by induction\tabularnewline
 & $H,H'\left[x\coloneqq c\right]\vdash a\left[x\coloneqq c\right]::_{A\left[x\coloneqq c\right],\ensuremath{\ell},o\left[x\coloneqq c\right]}B\left[x\coloneqq c\right]\::\:B\left[x\coloneqq c\right]$ & $\operatorname{cast-::}$\tabularnewline
other rules & ... & correspond to the inductive cases in Chapter 2\tabularnewline
\end{tabular}
\end{proof}
As before the notion of definitional equality can be extended to cast
contexts in \ref{fig:surface-Context-Equiv}.

\begin{figure}
\[
\frac{\ }{\lozenge\equiv\lozenge}\,\textrm{\ensuremath{\equiv}-ctx-empty}
\]

\[
\frac{H\equiv H'\quad A\equiv A'}{H,x:A\equiv H',x:A'}\,\textrm{\ensuremath{\equiv}-ctx-ext}
\]

\caption{Contextual Equivalence}
\label{fig:surface-Context-Equiv}
\end{figure}

\begin{lem}
Context Preservation

the following rule is admissible

\[
\frac{\Gamma\vdash n:N\quad\Gamma\equiv\Gamma'}{\Gamma'\vdash n:N}
\]
\end{lem}

\begin{proof}
by induction over typing derivations

\begin{tabular}{lll}
$\operatorname{cast-::}$ & \multicolumn{2}{l}{$H\vdash a:A$, $H\vdash A:\star$, $H\vdash B:\star$, wellformed
$\ensuremath{\ell},o$}\tabularnewline
 & $H'\vdash a:A$ & by induction\tabularnewline
 & $H'\vdash A:\star$ & by induction\tabularnewline
 & $H'\vdash B:\star$ & by induction\tabularnewline
 & $H'\vdash a::_{A,\ensuremath{\ell},o}B\::\:B$ & $\operatorname{cast-::}$\tabularnewline
other rules & ... & correspond to the inductive cases in Chapter 2\tabularnewline
\end{tabular}
\end{proof}
As before we show inversions on the term syntaxes, generalizing the
induction hypothesis up to equality, when needed.
\begin{lem}
$\mathsf{fun}$-Inversion (generalized)

\[
\frac{H\vdash\mathsf{fun}\,f\,x\Rightarrow a\,:\,C\quad C\equiv\left(x:A\right)\rightarrow B}{H,f:\left(x:A\right)\rightarrow B,x:A\vdash b:B}
\]
\end{lem}

\begin{proof}
by induction on the cast derivation

\begin{tabular}{lll}
$\operatorname{cast-fun}$ & \multicolumn{2}{l}{$H,f:\left(x:A'\right)\rightarrow B',x:A'\vdash b\,:\,B'$, $\left(x:A'\right)\rightarrow B'\equiv\left(x:A\right)\rightarrow B$}\tabularnewline
 & $A'\equiv A,\quad B'\equiv B$ & by stability of fun-ty\tabularnewline
 & $H,f:\left(x:A'\right)\rightarrow B',x:A'\equiv H,f:\left(x:A\right)\rightarrow B,x:A$ & by reflexivity of $\equiv$ , extended with previous equalities\tabularnewline
 & $H,f:\left(x:A\right)\rightarrow B,x:A\vdash b\,:\,B'$ & by context preservation\tabularnewline
 & $H,f:\left(x:A\right)\rightarrow B,x:A\vdash b\,:\,B$ & $\textrm{ty-conv}$\tabularnewline
$\operatorname{cast-conv}$ & \multicolumn{2}{l}{$H\vdash\mathsf{fun}\,f\,x\Rightarrow b\,:\,C'$, $C'\equiv C$, $C\equiv\left(x:A\right)\rightarrow B$}\tabularnewline
 & $C'\equiv\left(x:A\right)\rightarrow B$ & by transitivity\tabularnewline
 & ... & by induction\tabularnewline
other rules & impossible & the term position has the form $\mathsf{fun}\,f\,x\Rightarrow m$\tabularnewline
\end{tabular}
\end{proof}
This allows us to conclude the corollary 
\begin{cor}
$\mathsf{fun}$-Inversion 

\[
\frac{H\vdash\mathsf{fun}\,f\,x\Rightarrow a\,:\,\left(x:A\right)\rightarrow B}{H,f:\left(x:A\right)\rightarrow B,x:A\vdash b:B}
\]
\end{cor}

\begin{lem}
$\rightarrow$-Inversion (generalized)

The following rule is admissible
\[
\frac{H\vdash\left(x:A\right)\rightarrow B\,:\,C\quad C\equiv\star}{H\vdash A:\star\quad H,x:A\vdash B:\star}
\]
\end{lem}

\begin{proof}
By induction on the typing derivations

\begin{tabular}{lll}
$\operatorname{cast-fun-ty}$ & \multicolumn{2}{l}{$H\vdash\left(x:A\right)\rightarrow B\,:\,C$}\tabularnewline
 &  & follows directly\tabularnewline
$\operatorname{cast-conv}$ & \multicolumn{2}{l}{$H\vdash\left(x:A\right)\rightarrow B\::\:C'$, $C'\equiv C$}\tabularnewline
 & $C'\equiv\star$ & by transitivity\tabularnewline
 & ... & by induction\tabularnewline
other rules & impossible & since the term position has the form $\left(x:A\right)\rightarrow B$\tabularnewline
\end{tabular}
\end{proof}
leading to the corollary
\begin{cor}
$\rightarrow$-Inversion
\[
\frac{H\vdash\left(x:A\right)\rightarrow B\,:\,\star}{H\vdash A:\star\quad H,x:A\vdash B:\star}
\]
\end{cor}

\begin{lem}
$::$-Inversion

The following rule is admissible
\end{lem}

\[
\frac{H\vdash a::_{A,\ensuremath{\ell},o}B\::\:C}{H\vdash a:A\quad H\vdash A:\star\quad H\vdash B:\star}
\]

\todo{remove conditions for regularity?}
\begin{proof}
By induction on the typing derivations

\begin{tabular}{lll}
$\operatorname{cast-::}$ & \multicolumn{2}{l}{$H\vdash a:A\quad H\vdash A:\star\quad H\vdash B:\star$}\tabularnewline
 &  & follows directly\tabularnewline
$\operatorname{cast-conv}$ & \multicolumn{2}{l}{$H\vdash a::_{A,\ensuremath{\ell},o}B\::\:C'$, $C'\equiv C$}\tabularnewline
 & ... & by induction\tabularnewline
other rules & impossible & the term position has the form $a::_{A,\ensuremath{\ell},o}B$\tabularnewline
\end{tabular}
\end{proof}
Note that each of the output judgments are smaller then the input
judgments, this allows other proofs to use induction on the output
of this lemma.
\begin{thm}
$\Rrightarrow$-Preservation 

The following rule is admissible

\[
\frac{a\Rrightarrow a'\quad H\vdash a:A}{H\vdash a':A}
\]
\end{thm}

\begin{proof}
by induction

\begin{tabular}{llll}
$\operatorname{cast-\star}$ & $\textrm{\ensuremath{\Rrightarrow}-}\star$ & $H\vdash\star\,:\,\star$, $\star\Rrightarrow\star$ & follows directly\tabularnewline
$\operatorname{cast-var}$ & $\textrm{\ensuremath{\Rrightarrow}-var}$ & $H\vdash x\,:\,A$, $x\Rrightarrow x$ & follows directly\tabularnewline
$\textrm{ty-conv}$ &  & $H\vdash a\,:\,A$, $A\equiv A'$ & \tabularnewline
 & all $\Rrightarrow$ & $a\Rrightarrow a'$ & \tabularnewline
 &  & $H\vdash a'\,:\,A$ & by induction\tabularnewline
 &  & $H\vdash a'\,:\,A'$ & $\operatorname{cast-conv}$\tabularnewline
$\operatorname{cast-::}$ &  & well formed $\ell,o$ & \tabularnewline
 & $\textrm{\ensuremath{\Rrightarrow}-::-red}$ & $H\vdash a:\star$, $a\Rrightarrow a'$ & \tabularnewline
 &  & $H\vdash a'\,:\,\star$ & directly by induction\tabularnewline
 & $\textrm{\ensuremath{\Rrightarrow}-::}$ & \multicolumn{2}{l}{$H\vdash a:A_{1}$, $H\vdash A_{2}:\star$, $a\Rrightarrow a'$, $A_{1}\Rrightarrow A_{1}'$,
$A_{2}\Rrightarrow A_{2}'$, $o\Rrightarrow o'$}\tabularnewline
 &  & $H\vdash a'\,:\,A_{1}$ & by induction\tabularnewline
 &  & $H\vdash a'\,:\,A_{1}'$ & $\operatorname{cast-conv}$\tabularnewline
 &  & $H\vdash A_{2}':\star$ & by induction\tabularnewline
 &  & $H\vdash a'::_{A_{1}',\ell,o'}A_{2}'\,:\,A_{2}'$ & $\operatorname{cast-::}$\tabularnewline
 &  & $H\vdash a'::_{A_{1}',\ell,o'}A_{2}'\,:\,A_{2}$ & $\operatorname{cast-conv}$\tabularnewline
$\operatorname{cast-fun-ty}$ &  & $H\vdash A\,:\,\star$, $H,x:A\vdash B\,:\,\star$ & \tabularnewline
 & $\textrm{\ensuremath{\Rrightarrow}-fun-ty}$ & $A\Rrightarrow A'$, $B\Rrightarrow B'$ & \tabularnewline
 &  & $H\vdash A'\,:\,\star$ & by induction\tabularnewline
 &  & $H,x:A\vdash B':\star$ & by induction\tabularnewline
 &  & $H,x:A\equiv H,x:A'$ & by reflexivity of $\equiv$, extended with $A\equiv A'$\tabularnewline
 &  & $H,x:A'\vdash B'\,:\,\star$ & by preservation of contexts\tabularnewline
 &  & $H\vdash\left(x:A'\right)\rightarrow B'\,:\,\star$ & $\operatorname{cast-fun-ty}$\tabularnewline
$\operatorname{cast-fun}$ &  & $H,f:\left(x:A\right)\rightarrow B,x:A\vdash a\,:\,A$ & \tabularnewline
 & $\textrm{\ensuremath{\Rrightarrow}-fun}$ & $a\Rrightarrow a'$ & \tabularnewline
 &  & $H,f:\left(x:A\right)\rightarrow B,x:A\vdash a':A$ & by induction\tabularnewline
 &  & $H\vdash\mathsf{fun}\,f\,x\Rightarrow a'\,:\,\left(x:A\right)\rightarrow B$ & $\operatorname{cast-fun}$\tabularnewline
$\operatorname{cast-fun-app}$ &  &  & \tabularnewline
 & $\textrm{\ensuremath{\Rrightarrow}-fun-app-red}$ & \multicolumn{2}{l}{$H\vdash\mathsf{fun}\,f\,x\Rightarrow b\,:\,\left(x:A\right)\rightarrow B$,
$H\vdash a:A$, $a\Rrightarrow a'$, $b\Rrightarrow b'$}\tabularnewline
 &  & $\mathsf{fun}\,f\,x\Rightarrow b\Rrightarrow\mathsf{fun}\,f\,x\Rightarrow b'$ & $\textrm{\ensuremath{\Rrightarrow}-fun}$\tabularnewline
 &  & $H\vdash\mathsf{fun}\,f\,x\Rightarrow b'\,:\,\left(x:A\right)\rightarrow B$ & by induction\tabularnewline
 &  & $H,f:\left(x:A\right)\rightarrow B,x:A\vdash a'$ & by fun-inversion\tabularnewline
 &  & $H\vdash a'\,:\,A$ & by induction\tabularnewline
 &  & $H\vdash b'\left[f\coloneqq\mathsf{fun}\,f\,x\Rightarrow b',x\coloneqq a'\right]:B\left[x\coloneqq a'\right]$ & by typed substitutions\tabularnewline
 &  & $B\left[x\coloneqq a'\right]\equiv M\left[x\coloneqq a\right]$ & by substitution by steps, $\equiv$ symmetry\tabularnewline
 &  & $H\vdash b'\left[f\coloneqq\mathsf{fun}\,f\,x\Rightarrow b',x\coloneqq a'\right]:B\left[x\coloneqq A\right]$ & $\operatorname{cast-conv}$\tabularnewline
 & $\textrm{\ensuremath{\Rrightarrow}-fun-app}$ & $H\vdash a\,:\,\left(x:A\right)\rightarrow B$, $H\vdash a:A$, $a\Rrightarrow a'$,
$b\Rrightarrow b'$ & \tabularnewline
 &  & $H\vdash b'\,:\,\left(x:B\right)\rightarrow A$ & by induction\tabularnewline
 &  & $H\vdash A':A$ & by induction\tabularnewline
 &  & $H\vdash b'\,a':\,B\left[x\coloneqq a'\right]$ & $\operatorname{cast-fun-app}$\tabularnewline
 &  & $B\left[x\coloneqq a'\right]\equiv B\left[x\coloneqq a\right]$ & by substitution by steps, $\equiv$ symmetry\tabularnewline
 &  & $H\vdash b'\left[f\coloneqq\mathsf{fun}\,f\,x\Rightarrow b',x\coloneqq a'\right]:B\left[x\coloneqq a\right]$ & $\operatorname{cast-conv}$\tabularnewline
 & $\textrm{\ensuremath{\Rrightarrow}-fun-::-red}$ & \multicolumn{2}{l}{$H\vdash\left(b::_{\left(x:A_{1}\right)\rightarrow B_{1},\ensuremath{\ell},o}\left(x:A_{2}\right)\rightarrow B_{2}\right):\left(x:A_{2}\right)\rightarrow B_{2}$,
$H\vdash a:A_{2}$, }\tabularnewline
 &  & \multicolumn{2}{l}{$b\Rrightarrow b'$, $a\Rrightarrow a'$, $A_{1}\Rrightarrow A_{1}'$,
$A_{2}\Rrightarrow A_{2}'$, $B_{1}\Rrightarrow B_{1}'$, $B_{2}\Rrightarrow B_{2}'$,
$o\Rrightarrow o'$, }\tabularnewline
 &  & $H\vdash a':A_{2}$ & by induction\tabularnewline
 &  & $H\vdash a':A_{2}'$ & $\operatorname{cast-conv}$\tabularnewline
 &  & $H\vdash b:\left(x:A_{1}\right)\rightarrow B_{1}$, $H\vdash\left(x:A_{1}\right)\rightarrow B_{1}:\star$,
$H\vdash\left(x:A_{2}\right)\rightarrow B_{2}:\star$ & cast-inversion \tabularnewline
 &  & $\left(x:A_{2}\right)\rightarrow B_{2}\Rrightarrow\left(x:A_{2}'\right)\rightarrow B_{2}'$ & $\textrm{\ensuremath{\Rrightarrow}-fun-ty}$\tabularnewline
 &  & $H\vdash\left(x:A_{2}'\right)\rightarrow B_{2}':\star$ & by induction with $H\vdash\left(x:A_{2}\right)\rightarrow B_{2}:\star$,
\footnote{well founded since cast-inversion produces smaller judgments}\tabularnewline
 &  & $H\vdash A_{2}':\star$, $H,x:A_{2}'\vdash B_{2}':\star$ & fun-ty-inversion\tabularnewline
 &  & $\left(x:A_{1}\right)\rightarrow B_{1}\Rrightarrow\left(x:A_{1}'\right)\rightarrow B_{1}'$ & $\textrm{\ensuremath{\Rrightarrow}-fun-ty}$\tabularnewline
 &  & $H\vdash\left(x:A_{1}'\right)\rightarrow B_{1}':\star$ & by induction with $H\vdash\left(x:A_{1}\right)\rightarrow B_{1}:\star$,
\footnote{well founded since cast-inversion produces smaller judgments}\tabularnewline
 &  & $H\vdash A_{1}':\star$, $H,x:A_{1}'\vdash B_{1}':\star$ & fun-ty-inversion\tabularnewline
 &  & let $ac=a'::_{A_{2}',\ell,o'.arg}A_{1}'$ & \tabularnewline
 &  & $H\vdash ac:A_{1}'$ & $\operatorname{cast-::}$\tabularnewline
 &  & $H\vdash b':\left(x:A_{1}\right)\rightarrow B_{1}$ & by induction with $H\vdash b:\left(x:A_{1}\right)\rightarrow B_{1}$,
\footnote{well founded since cast-inversion produces smaller judgments}\tabularnewline
 &  & $H\vdash b':\left(x:A_{1}'\right)\rightarrow B_{1}'$ & $\operatorname{cast-conv}$\tabularnewline
 &  & $H\vdash b'\ ac:B_{1}'\left[x\coloneqq ac\right]$ & $\operatorname{cast-fun-app}$\tabularnewline
 &  & $H\vdash B_{1}'\left[x\coloneqq ac\right]:\star$ & by substitution preservation\tabularnewline
 &  & $H\vdash B_{2}'\left[x\coloneqq a'\right]:\star$ & by substitution preservation\tabularnewline
 &  & $H\vdash\left(b'\ ac\right)::_{B_{1}'\left[x\coloneqq ac\right],\ensuremath{\ell},o'.bod[a']}B_{2}'\left[x\coloneqq a'\right]$ & $\operatorname{cast-::}$\tabularnewline
\end{tabular}
\end{proof}
\todo{both vertically and horz too big}

\todo{highlight the -fun-::-red case}

\subsubsection{Progress}

\begin{figure}
\[
\frac{\,}{\star\,\textbf{Val}}\operatorname{Val-\star}
\]
\[
\frac{\,}{\left(x:A\right)\rightarrow B\,\textbf{Val}}\operatorname{Val-fun-ty}
\]
\[
\frac{\,}{\mathsf{fun}\,f\,x\Rightarrow b\:\textbf{Val}}\operatorname{Val-fun}
\]
\[
\frac{\begin{array}{c}
a\:\textbf{Val}\quad A\:\textbf{Val}\quad B\:\textbf{Val}\\
a\cancel{=}\star\\
a\cancel{=}\left(x:C\right)\rightarrow C'
\end{array}}{a::_{A,\ensuremath{\ell}o}B\:\textbf{Val}}\operatorname{Val-::}
\]
\caption{Cast Language Values}
\label{fig:cast-val}
\end{figure}

Unlike the surface language, it is not longer practical to characterize
values syntactically. Values are specified by judgments in Figure
\ref{fig:cast-val}. They are standard except for the $\operatorname{Val-::}$,
which states that a type ($\star$ or function type) under a cast
is not a value.

\begin{figure}
\[
\frac{a\,\textbf{Val}}{\left(\mathsf{fun}\,f\,x\Rightarrow b\right)a\rightsquigarrow b\left[f\coloneqq\mathsf{fun}\,f\,x\Rightarrow b,x\coloneqq a\right]}
\]
\[
\frac{b\,\textbf{Val}\quad a\,\textbf{Val}}{\begin{array}{c}
\left(b::_{\left(x:A_{1}\right)\rightarrow B_{1},\ensuremath{\ell},o}\left(x:A_{2}\right)\rightarrow B_{2}\right)a\rightsquigarrow\\
\left(b\,a::_{A_{2},\ell,o.arg}A_{1}\right)::_{B_{1}\left[x\coloneqq a::_{A_{2},\ell,o.arg}A_{1}\right],\ensuremath{\ell},o.bod[a]}B_{2}\left[x\coloneqq a\right]
\end{array}}
\]
\[
\frac{a\,\textbf{Val}}{a::_{\star,\ensuremath{\ell},o}\star\rightsquigarrow a}
\]
\[
\frac{a\rightsquigarrow a'}{a::_{A,\ensuremath{\ell},o}B\rightsquigarrow a'::_{A,\ensuremath{\ell},o}B}
\]
\[
\frac{a\,\textbf{Val}\quad A\rightsquigarrow A'}{a::_{A,\ensuremath{\ell},o}B\rightsquigarrow a::_{A',\ensuremath{\ell},o}B}
\]
\[
\frac{a\,\textbf{Val}\quad A\,\textbf{Val}\quad B\rightsquigarrow B'}{a::_{A,\ensuremath{\ell},o}B\rightsquigarrow a::_{A,\ensuremath{\ell},o}B'}
\]
\[
\frac{b\rightsquigarrow b'}{b\,a\rightsquigarrow b'\,a}
\]
\[
\frac{b\,\textbf{Val}\quad a\rightsquigarrow a'}{b\,a\rightsquigarrow b\,a'}
\]

\todo[inline]{name rules}\caption{Cast Language Small Step}
\label{fig:cast-step}
\end{figure}

Small steps are listed in Figure \ref{fig:cast-step}. They are standard
for call-by-value except that casts can distribute over application,
and casts can reduce when both types are $\star$.

\begin{figure}
\[
\frac{\,}{\textbf{Blame}\:\ensuremath{\ell}\,o\,\left(a::_{\left(x:A\right)\rightarrow B,\ensuremath{\ell},o}\star\right)}
\]
\[
\frac{\,}{\textbf{Blame}\:\ensuremath{\ell}\,o\,\left(a::_{\star,\ensuremath{\ell},o}\left(x:A\right)\rightarrow B\right)}
\]
\[
\frac{\textbf{Blame}\:\ensuremath{\ell}\,o\,a}{\textbf{Blame}\:\ensuremath{\ell}\,o\,\left(a::_{A,\ensuremath{\ell'},o'}B\right)}
\]
\[
\frac{\textbf{Blame}\:\ensuremath{\ell}\,o\,A}{\textbf{Blame}\:\ensuremath{\ell}\,o\,\left(a::_{A,\ensuremath{\ell'},o'}B\right)}
\]
\[
\frac{\textbf{Blame}\:\ensuremath{\ell}\,o\,B}{\textbf{Blame}\:\ensuremath{\ell}\,o\,\left(a::_{A,\ensuremath{\ell'},o'}B\right)}
\]
\[
\frac{\textbf{Blame}\:\ensuremath{\ell}\,o\,b}{\textbf{Blame}\:\ensuremath{\ell}\,o\,\left(b\,a\right)}
\]
\[
\frac{\textbf{Blame}\:\ensuremath{\ell}\,o\,a}{\textbf{Blame}\:\ensuremath{\ell}\,o\,\left(b\,a\right)}
\]
\todo[inline]{name rules}\caption{Cast Language Blame}
\label{fig:cast-blame}
\end{figure}

%% walk through blame
In addition to small step and values we also specify blame judgments
in figure \ref{fig:cast-blame}. Blame tracks the information needed
to create a good error message and is inspired by the many systems
that use blame tracking \cite{10.1145/581478.581484,10.1007/978-3-642-00590-9_1,wadler:LIPIcs:2015:5033}.
Specifically the judgment $\textbf{Blame}\:\ensuremath{\ell}\,o\,a$
means that $a$witnesses an a contradiction in the source code at
location $\ensuremath{\ell}$ under the observations $o$. With only
dependent functions and universes, only inequalities of the form $*\,\cancel{=}\,A\rightarrow B$
can be witnessed. The first 2 rules of the blame judgment witness
these concrete type inequalities. The rest of the blame rules will
recursively extract concrete witnesses from larger terms. The limited
amount of observation is the main reason why the theory in this chapter
is tractable. %% "self evidently correct" sinc it is clear to only extract inequalities. like step and val? one of several possible contrediction extraction relations
\begin{fact}
$\rightsquigarrow$ preserves types 

since the following rule is admissible
\end{fact}

\[
\frac{m\rightsquigarrow m'}{m\Rrightarrow m'}
\]

As in Chapter 2 we will need technical lemmas that determine the form
of a value of a type in the empty context. However, canonical function
values look different because they must account for the possibility
of blame arising from a stuck term.
\begin{lem}
$\star$-Canonical forms (generalized)

If $\vdash a\,:\,A$ , $a\,\textbf{Val}$, and $A\equiv\star$ then
either

\textup{$a=\star$ ,}

\textup{or there exists $C$, $B$, such that $a=\left(x:C\right)\rightarrow B$}
\end{lem}

\todo{alternatively produce the blame judgment}
\begin{proof}
by induction on the cast derivation

\begin{tabular}{lll}
$\operatorname{cast-\star}$ & $\vdash\star:\star$ & follows since $a=\star$\tabularnewline
$\operatorname{cast-fun-ty}$ & $\vdash\left(x:A\right)\rightarrow B\,:\,\star$ & follows since $a=\left(x:A\right)\rightarrow B$\tabularnewline
$\textrm{ty-conv}$ & $\vdash a:A$, $a\,\textbf{Val}$, $A\equiv A'$, $A'\equiv\star$ & which concluded $\vdash a\,:\,A'$\tabularnewline
 & $A'\equiv\star$ & by transitivity, symmetry\tabularnewline
 & $a=\star$ or there exists $C$, $B$, such that $a=\left(x:C\right)\rightarrow B$  & by induction\tabularnewline
$\operatorname{cast-::}$ & $\vdash a:A_{1}$, $a::_{A_{1},\ensuremath{\ell},o}A_{2}\:\textbf{Val}$,$\vdash A_{1}:\star$,
$\vdash A_{2}:\star$, $A_{2}\equiv\star$ & \tabularnewline
 & $a\cancel{=}\star,a\cancel{=}\left(x:C\right)\rightarrow B$, $a\:\textbf{Val}$ & since it must have been a value by $\operatorname{Val-::}$ \tabularnewline
 & but $a=\star$ or there exists $C$, $B$, such that $a=\left(x:C\right)\rightarrow B$  & by induction \tabularnewline
 & ! & so $\operatorname{cast-::}$ case was impossible\tabularnewline
$\operatorname{cast-fun}$ & $f:\left(x:A\right)\rightarrow B,x:A\vdash b:B$, $\left(x:A\right)\rightarrow B\equiv\star$ & \tabularnewline
 & $\left(x:A\right)\rightarrow B\cancel{\equiv}\star$ & by the stability of $\equiv$\tabularnewline
 & ! & so $\operatorname{cast-fun}$ case was impossible\tabularnewline
other rules & impossible & since they do not type values in an empty ctx\tabularnewline
\end{tabular}
\end{proof}
Leading to the corollary ,
\begin{cor}
$\star$-Canonical forms

If $\vdash A:\star$, and $A\,\textbf{Val}$ then either 

\textup{$A=\star$ , }

\textup{or there exists $C$, $B$, such that $A=\left(x:C\right)\rightarrow B$}
\end{cor}

\begin{proof}
by reflexivity of $\equiv$
\end{proof}
Likewise
\begin{lem}
$\rightarrow$-Canonical forms (generalized)

If $\vdash a\,:\,A$ , $a\,\textbf{Val}$, and $A\equiv\left(x:C\right)\rightarrow B$
then either

$a=\mathsf{fun}\,f\,x\Rightarrow b$ 

or $a=d::_{D,\ensuremath{\ell},o}\left(x:C'\right)\rightarrow B'$,
$d\,\textbf{Val}$, $D\,\textbf{Val}$, $C'\equiv C$, $B'\equiv B$ 
\end{lem}

\todo{alternatively produce the blame judgment}
\begin{proof}
by induction on the cast derivation

\begin{tabular}{lll}
$\operatorname{cast-\star}$ & $\star\equiv\left(x:C\right)\rightarrow B$ & \tabularnewline
 & $\star\cancel{\equiv}\left(x:C\right)\rightarrow B$ ! & by the stability of $\equiv$\tabularnewline
$\operatorname{cast-fun-ty}$ & $\star\equiv\left(x:C\right)\rightarrow B$ & \tabularnewline
 & $\star\cancel{\equiv}\left(x:C\right)\rightarrow B$ ! & by the stability of $\equiv$\tabularnewline
$\operatorname{cast-fun}$ & \multicolumn{2}{l}{$f:\left(x:C'\right)\rightarrow B',x:C'\vdash b:B'$, $\left(x:C'\right)\rightarrow B'\equiv\left(x:C\right)\rightarrow B$}\tabularnewline
 &  & follows directly\tabularnewline
$\operatorname{cast-::}$ & $\vdash a:A_{1}$, $\vdash A_{1}:\star$, $\vdash A_{2}:\star$, $A_{2}\equiv\left(x:C\right)\rightarrow B$ & \tabularnewline
 & $a\cancel{=}\star,a\cancel{=}\left(x:C\right)\rightarrow B$, $a\:\textbf{Val}$,
$A_{1}\:\textbf{Val}$, $A_{2}\:\textbf{Val}$ & since it must have been a value by $\operatorname{Val-::}$\tabularnewline
 & $A_{2}=\left(x:C'\right)\rightarrow B'$ & by the stability of $\equiv$, $A_{2}\:\textbf{Val}$\footnote{TODO: make an explicit lemma?}\tabularnewline
 & $C'\equiv C$, $B'\equiv B$ & by the stability of $\equiv$\tabularnewline
$\textrm{ty-conv}$ & $\vdash a:A$, $A\equiv A'$, $A'\equiv\left(x:C\right)\rightarrow B$ & \tabularnewline
 & $A\equiv\left(x:C\right)\rightarrow B$ & by transitivity\tabularnewline
 & ... & by induction\tabularnewline
other rules & impossible & since they do not type values of $\star$\tabularnewline
\end{tabular}
\end{proof}
As a corollary
\begin{cor}
$\rightarrow$-Canonical forms

If $\vdash a:\left(x:C\right)\rightarrow B$ , and $a\,\textbf{Val}$

$a=\mathsf{fun}\,f\,x\Rightarrow b$ 

or $a=d::_{D,\ensuremath{\ell},o}\left(x:C'\right)\rightarrow B'$,
$d\,\textbf{Val}$, $D\,\textbf{Val}$, $C'\equiv C$, $B'\equiv B$ 
\end{cor}

\begin{proof}
by reflexivity of $\equiv$
\end{proof}
\begin{cor}
$\rightarrow$-Canonical terms

If $\vdash a:\left(x:C\right)\rightarrow B$ , and $a\,\textbf{Val}$
then $a$is not a type, $a\cancel{=}\star,a\cancel{=}\left(x:C\right)\rightarrow C'$
\end{cor}

\begin{thm}
Progress

If $\vdash a\,:\,A$ then either 

$a\,\textbf{Val}$

there exists $a'$ such that $a\rightsquigarrow a'$

or there exists $\ell$, $o$ such that $\textbf{Blame}\:\ensuremath{\ell}\,o\,a$
\end{thm}

\begin{proof}
As usual this follows form induction on the typing derivation

\begin{tabular}{lllll}
$\operatorname{cast-var}$ & $\vdash\star\,:\,\star$ &  &  & \tabularnewline
 & $\star\,\textbf{Val}$ &  &  & $\operatorname{Val-\star}$\tabularnewline
$\operatorname{cast-var}$ & $\vdash x:A$ &  &  & impossible in an empty context\tabularnewline
$\operatorname{cast-conv}$ & $\vdash a:A'$, $A'\equiv A$ &  &  & \tabularnewline
 & \multicolumn{3}{l}{$a\,\textbf{Val}$, there exists $a'$ such that $a\rightsquigarrow a'$,
or there exists $\ell$, $o$ such that $\textbf{Blame}\:\ensuremath{\ell}\,o\,a$} & by induction on $\vdash a:A'$\tabularnewline
$\operatorname{cast-::}$ & $\vdash a:\,A$, $\vdash A:\star$, $\vdash B:\star$ &  &  & \tabularnewline
 & \multicolumn{3}{l}{$a\,\textbf{Val}$, there exists $a'$ such that $a\rightsquigarrow a'$,
or there exists $\ell_{a}$, $o_{a}$ such that $\textbf{Blame}\:\ensuremath{\ell}_{a}\,o_{a}\,a$} & by induction\tabularnewline
 & \multicolumn{3}{l}{$A\,\textbf{Val}$, there exists $A'$ such that $A\rightsquigarrow A'$,
or there exists $\ell_{A}$, $o_{A}$ such that $\textbf{Blame}\:\ensuremath{\ell}_{A}\,o_{A}\,A$} & by induction\tabularnewline
 & \multicolumn{3}{l}{$B\,\textbf{Val}$, there exists $B'$ such that $B\rightsquigarrow B'$,
or there exists $\ell_{B}$, $o_{B}$ such that $\textbf{Blame}\:\ensuremath{\ell}_{B}\,o_{B}\,B$} & by induction\tabularnewline
 & if $a\rightsquigarrow a'$,  & $a::_{A,\ensuremath{\ell},o}B\rightsquigarrow a'::_{A,\ensuremath{\ell},o}B$ &  & \tabularnewline
 & if $a\,\textbf{Val}$, $A\rightsquigarrow A'$, & $a::_{A,\ensuremath{\ell},o}B\rightsquigarrow a::_{A',\ensuremath{\ell},o}B$ &  & \tabularnewline
 & if $a\,\textbf{Val}$, $A\,\textbf{Val}$, $B\rightsquigarrow B'$ & $a::_{A,\ensuremath{\ell},o}B\rightsquigarrow a::_{A,\ensuremath{\ell},o}B'$ &  & \tabularnewline
 & if $a\,\textbf{Val}$, $A\,\textbf{Val}$, $B\,\textbf{Val}$, &  &  & \tabularnewline
 &  & $A=\star$ or $A=\left(x:C_{A}\right)\rightarrow D_{A}$ &  & by canonical forms of $\star$\tabularnewline
 &  & $B=\star$ or $B=\left(x:C_{B}\right)\rightarrow D_{B}$ &  & by canonical forms of $\star$\tabularnewline
 &  & if $A=\star$, $B=\star$  & $a::_{\star,\ensuremath{\ell},o}\star\rightsquigarrow a$ & \tabularnewline
 &  & if $A=\left(x:C_{A}\right)\rightarrow D_{A}$, $B=\left(x:C_{B}\right)\rightarrow D_{B}$ &  & \tabularnewline
 &  &  & $a\cancel{=}\star,a\cancel{=}\left(x:C\right)\rightarrow C'$ & by canonical forms of fun \tabularnewline
 &  &  & $a::_{A,\ensuremath{\ell}o}B\:\textbf{Val}$ & \tabularnewline
 &  & if $A=\star$, $B=\left(x:C_{B}\right)\rightarrow D_{B}$  & $\textbf{Blame}\:\ensuremath{\ell}\,o\,\left(a::_{\star,\ensuremath{\ell},o}\left(x:C_{B}\right)\rightarrow D_{B}\right)$ & \tabularnewline
 &  & if $A=\left(x:C_{A}\right)\rightarrow D_{A}$, $B=\star$ & $\textbf{Blame}\:\ensuremath{\ell}\,o\,\left(a::_{\left(x:C_{A}\right)\rightarrow B_{A},\ensuremath{\ell},o}\star\right)$ & \tabularnewline
 & \multicolumn{2}{l}{if there exists $\ell_{a}$, $o_{a}$ such that $\textbf{Blame}\:\ensuremath{\ell}_{a}\,o_{a}\,a$} & $\textbf{Blame}\:\ensuremath{\ell}_{a}\,o_{a}\,\left(a::_{A,\ensuremath{\ell},o}B\right)$ & \tabularnewline
 & \multicolumn{2}{l}{if there exists $\ell_{A}$, $o_{A}$ such that $\textbf{Blame}\:\ensuremath{\ell}_{A}\,o_{A}\,A$} & $\textbf{Blame}\:\ensuremath{\ell}_{A}\,o_{A}\,\left(a::_{A,\ensuremath{\ell},o}B\right)$ & \tabularnewline
 & \multicolumn{2}{l}{if there exists $\ell_{B}$, $o_{B}$ such that $\textbf{Blame}\:\ensuremath{\ell}_{B}\,o_{B}\,B$} & $\textbf{Blame}\:\ensuremath{\ell}_{B}\,o_{B}\,\left(a::_{A,\ensuremath{\ell},o}B\right)$ & \tabularnewline
$\operatorname{cast-fun-ty}$ & $\vdash\left(x:A\right)\rightarrow B\,:\,\star$ &  &  & \tabularnewline
 & $\left(x:A\right)\rightarrow B\,\textbf{Val}$ &  &  & $\operatorname{Val-fun-ty}$\tabularnewline
$\operatorname{cast-fun-app}$ & $\vdash\mathsf{fun}\,f\,x\Rightarrow b\,:\,\left(x:A\right)\rightarrow B$ &  &  & \tabularnewline
 & $\mathsf{fun}\,f\,x\Rightarrow b\:\textbf{Val}$ &  &  & $\operatorname{Val-fun}$\tabularnewline
$\operatorname{cast-fun-app}$ & \multicolumn{2}{l}{$\vdash b\,:\,\left(x:A\right)\rightarrow B$, $\Gamma\vdash a\,:\,A$} & \tabularnewline
 & \multicolumn{3}{l}{$a\,\textbf{Val}$, there exists $a'$ such that $a\rightsquigarrow a'$,
or there exists $\ell_{a}$, $o_{a}$ such that $\textbf{Blame}\:\ensuremath{\ell}_{a}\,o_{a}\,a$} & by induction\tabularnewline
 & \multicolumn{3}{l}{$b\,\textbf{Val}$, there exists $b'$ such that $b\rightsquigarrow b'$,
or there exists $\ell_{b}$, $o_{b}$ such that $\textbf{Blame}\:\ensuremath{\ell}_{b}\,o_{b}\,b$} & by induction\tabularnewline
 & if $b\rightsquigarrow b'$, & $b\,a\rightsquigarrow b'\,a$ &  & \tabularnewline
 & if $b\,\textbf{Val}$, $a\rightsquigarrow a'$, & $b\,a\rightsquigarrow b\,a'$ &  & \tabularnewline
 & if $b\,\textbf{Val}$, $a\,\textbf{Val}$, & \multicolumn{2}{l}{$b=\mathsf{fun}\,f\,x\Rightarrow c$ or $b=d_{b}::_{D_{b},\ensuremath{\ell}_{b}\,o_{b}}\left(x:A'\right)\rightarrow B'$,
$d_{b}\,\textbf{Val}$, $D_{b}\,\textbf{Val}$, $A'\equiv A$, $B'\equiv B$\footnote{todo clean up extra conclusions}} & by canonical forms of functions\tabularnewline
 &  & if $b=\mathsf{fun}\,f\,x\Rightarrow c$ & $\left(\mathsf{fun}\,f\,x\Rightarrow c\right)a\rightsquigarrow c\left[f\coloneqq\mathsf{fun}\,f\,x\Rightarrow c,x\coloneqq a\right]$ & \tabularnewline
 &  & \multicolumn{2}{l}{$b=d_{b}::_{D_{b},\ensuremath{\ell}_{b},o_{b}}\left(x:A'\right)\rightarrow B'$,
$d_{b}\,\textbf{Val}$, $D_{b}\,\textbf{Val}$, $A'\equiv A$, $B'\equiv B$} & \tabularnewline
 &  &  & $\vdash D_{b}:\star$ & \tabularnewline
 &  &  & $D_{b}=\star$ or $D_{b}=\left(x:A_{D_{b}}\right)\rightarrow B_{D_{b}}$ & by canonical forms of $\star$\tabularnewline
 &  &  & if $D_{b}=\star$,  & \tabularnewline
 &  &  & $\textbf{Blame}\:\ensuremath{\ell}_{b}\,o_{b}\,\left(b::_{\star,\ensuremath{\ell}_{b},o_{b}}\left(x:A'\right)\rightarrow B'\right)$ & \tabularnewline
 &  &  & if $D_{b}=D_{b}=\left(x:A_{D_{b}}\right)\rightarrow B_{D_{b}}$, & \tabularnewline
 &  &  & let $a_{c}=a::_{A',\ensuremath{\ell}_{b},o_{b}.arg}A_{D_{b}}$ & \tabularnewline
 &  &  & $\left(d_{b}::_{\left(x:A_{D_{b}}\right)\rightarrow B_{D_{b}},\ensuremath{\ell}_{b},o_{b}}\left(x:A'\right)\rightarrow B'\right)a\rightsquigarrow$ & \tabularnewline
 &  &  & $\left(d_{b}\,a_{c}\right)::_{B_{D_{b}}\left[x\coloneqq a_{c}\right],\ensuremath{\ell},o.bod[a]}B'\left[x\coloneqq a\right]$ & \tabularnewline
 & \multicolumn{2}{l}{if there exists $\ell_{b}$, $o_{b}$ such that $\textbf{Blame}\:\ensuremath{\ell}_{b}\,o_{b}\,b$} & $\textbf{Blame}\:\ensuremath{\ell}_{b}\,o_{b}\,\left(b\ a\right)$ & \tabularnewline
 & \multicolumn{2}{l}{if there exists $\ell_{a}$, $o_{a}$ such that $\textbf{Blame}\:\ensuremath{\ell}_{a}\,o_{a}\,a$} & $\textbf{Blame}\:\ensuremath{\ell}_{a}\,o_{a}\,\left(b\ a\right)$ & \tabularnewline
\end{tabular}
\end{proof}

\subsection{Cast Soundness}

\todo{revise}

%% cast soundness
We can now prove cast soundness.

For any $\lozenge\vdash c:C$, $c'$, $c\rightsquigarrow^{*}c'$,
if $\textbf{Stuck}\,c'$ then $\textbf{Blame}\:\ensuremath{\ell}\,o\:c'$,
where $\textbf{Stuck}\,c'$ means $c'$ is not a value and $c'$ does
not step. This follows by iteration the progress and preservation
lemmas.

\subsection{Discussion}

\todo{inferablility of inner cast}

\todo{regularity encoding, along with the cast}

Because of the conversion rule and non-termination, cast-checking
is undecidable. All previous arguments from Chapter 2 apply to why
this may be acceptable, but the most relevant is that the system should
only be used through the Elaboration procedure described in the next
section, which avoids the need to cast check.

\todo{example}

As in the surface languages, the cast language is logically unsound,
by design.

Just as there are many different flavors of definitional equality
that could have been used in Chapter 2, there are also many possible
degrees that runtime equality can be enforced. The \textbf{Blame}
relation in \ref{fig:cast-blame} outlines the minimal possible checking
to support cast soundness. For instance\footnote{assuming the data types of chapter 5},
$\mathtt{head}\,\mathbb{B}\,1\,\left(\mathtt{rep}\,\mathbb{B}\,true\,0\right)$
will result in blame since $1$ and $0$ have different head constructors.
But $\mathtt{head}\,\mathbb{B}\,1\,\left(\mathtt{rep}\,\mathbb{B}\,true\,9\right)$
will not result in blame since $1$ and $9$ have the same head constructor
and the computation can reduce to $true$. 

It is likely that more aggressive checking is preferable in practice,
especially in the presence of data types. That is why in our implementation
we check equalities up to call-by-value. 

This behavior is consistent with the conjectured partial correctness
of logically unsound Call-by-Value execution for dependent types in
\cite{jia2010dependent}. 

\todo{move?}

Unlike static type-checking, these runtime checks have runtime costs.
Since the language allows nontermination, checks can take forever
to resolve. We don't expect this to be an issue in practice, since
we could limit the number of steps allowed. Additionally, the implementation
avoids casts when it knows that blame is impossible. 

\section{Elaboration}

% overview
Even though the cast language allows us to optimistically assert equalities,
manually noting every cast would be unrealistically cumbersome. This
bureaucracy is solved with an elaboration procedure that translates
(untyped) terms from the surface language into the cast language.
If the term is well typed in the surface language elaboration will
produce a term without blamable errors. Terms with unproven equality
in types are mapped to a cast with enough information to point out
the original source when an inequality is witnessed. Elaboration serves
a similar role as the bidirectional type system did in Chapter 2,
and uses a similar methodology

First we enrich the surface language with location information, $\ell$,
at every position that could result in a type mismatch in Figure \ref{fig:surface-pre-syntax-loc}.
Note that the location tags match almost exactly\todo{or will after rewritten}
with the check annotations of the bidirectional system. For technical
reasons the set of locations is nonempty, and a specific null location
$.$ \todo{- instead of . ?}is designated. That null location can
be used when we need to generate fresh terms, but have no sensible
location information available. All the meta theory from Chapter 2
goes through assuming that all locations are equivalent and by generating
null locations when needed\footnote{For instance, the parallel reduction relation will associate all locations,
$\frac{M\Rrightarrow M'\quad N\Rrightarrow N'}{\left(x:M_{l}\right)\rightarrow N_{l'}\Rrightarrow\left(x:M'_{l''}\right)\rightarrow N'_{l'''}}\,\textrm{\ensuremath{\Rrightarrow}-fun-ty}$,
so that the relation does not discriminate over syntaxes that come
from different locations. While the $max$ function will map terms
into the null location, $max\left(\left(x:M_{\ensuremath{\ell}}\right)\rightarrow N_{\ensuremath{\ell'}}\right)=max\left(\left(x:max\left(M\right)_{.}\right)\rightarrow max\left(N\right)_{.}\right)$
so that the output is unique.}\todo{fn a little awk}. We will avoid writing these annotations when
they are unneeded (explicitly in Figure \ref{fig:surface-pre-syntax-abrev}).

\begin{figure}
\begin{tabular}{lcll}
\multicolumn{4}{l}{source labels,}\tabularnewline
$\ensuremath{\ell}$ & $\Coloneqq$ & ... & \tabularnewline
 & | & $.$ & no source label\tabularnewline
\multicolumn{4}{l}{expressions,}\tabularnewline
$m,n,M,N$ & $\Coloneqq$ & $x$ & variable\tabularnewline
 & | & $m::_{\ensuremath{\ell}}M^{\ensuremath{\ell}'}$ & annotation\tabularnewline
 & | & $\star$ & type universe\tabularnewline
 & | & $\left(x:M_{\ensuremath{\ell}}\right)\rightarrow N_{\ensuremath{\ell'}}$ & function type\tabularnewline
 & | & $\mathsf{fun}\,f\,x\Rightarrow m$ & function\tabularnewline
 & | & $m_{\ensuremath{\ell}}\,n$ & application\tabularnewline
\end{tabular}\caption{Surface Language Syntax, With Locations}
\label{fig:surface-pre-syntax-loc}
\end{figure}

\begin{figure}
\begin{tabular}{lclll}
$m::_{\ensuremath{\ell}}M^{\ensuremath{\ell}'}$ & written & $m::_{\ensuremath{\ell}}M$ & when & $\ensuremath{\ell}$' is irrelevant\tabularnewline
$m::_{\ensuremath{\ell}}M$ & written & $m::M$ & when & $\ensuremath{\ell}$ is irrelevant\tabularnewline
$\left(x:M_{\ensuremath{\ell}}\right)\rightarrow N_{\ensuremath{\ell'}}$ & written & $\left(x:M\right)\rightarrow N$ & when & $\ensuremath{\ell}$, $\ensuremath{\ell}'$ are irrelevant\tabularnewline
$m_{\ensuremath{\ell}}\,n$ & written & $m_{\ensuremath{}}\,n$ & when & $\ensuremath{\ell}$ is irrelevant\tabularnewline
\end{tabular}

\caption{Surface Language Abbreviations}
\label{fig:surface-pre-syntax-abrev}
\end{figure}


\subsection{Examples}

As with bidirectional type checking variable types will be inferred
by casting the outer type. For example, 
\begin{example}
$\vdash(\lambda x\Rightarrow7)::_{\ell}\mathbb{B}\rightarrow\mathbb{B}$
elaborates to $\vdash(\lambda x\Rightarrow7::_{\mathbb{N}.\ell,.bod[x]}\mathbb{B})$
\end{example}

Arguments will receive expected types when they are applied. For example, 
\begin{example}
$f:\mathbb{B}\rightarrow\mathbb{B}\vdash f_{\ell}7\ :\mathbb{B}$
elaborates to $f:\mathbb{B}\rightarrow\mathbb{B}\vdash f\left(7::_{\mathbb{N}.\ell,.arg}\mathbb{B}\right)\ :\mathbb{B}$
\end{example}

Vacuous cast can be created,
\begin{example}
$f:\mathbb{N}\rightarrow\mathbb{B}\rightarrow\mathbb{B}\vdash f_{\ell}7_{\ell'}3\ :\mathbb{B}$
elaborates to $f:\mathbb{N}\rightarrow\mathbb{B}\rightarrow\mathbb{B}\vdash f\left(7::_{\mathbb{N}.\ell,.arg}\mathbb{N}\right)\left(3::_{\mathbb{N}.\ell',.arg}\mathbb{B}\right)\ :\mathbb{B}$
\end{example}

\todo{dependent type example where cast muddles type}

Unlike in gradual typing, we cannot elaborate arbitrary untyped syntax.
The underlying type of a cast needs to be known so that a function
type can swap its argument type at application. For instance, $\lambda x\Rightarrow x$
will not elaborate since the intended type is not known. Fortunately,
our experimental testing suggests that a majority of randomly generated
terms can be elaborated, while only a small minority of terms would
type-check in the surface language. The programmer can make any term
elaborate if they annotate the intended type. For instance, $\left(\lambda x\Rightarrow x\right)::*\rightarrow*$
will elaborate.

\subsection{Elaboration Procedure}

\begin{figure}
\[
\frac{x:A\in H}{H\vdash x\,\textbf{Elab}\,x\overrightarrow{\,:\,}A}\operatorname{\overrightarrow{\textbf{Elab}}-var}
\]

\[
\frac{\,}{H\vdash\star\,\textbf{Elab}\,\star\overrightarrow{\,:\,}\star}\operatorname{\overrightarrow{\textbf{Elab}}-\star}
\]

\[
\frac{H\vdash M\overleftarrow{\,:_{\ell,.}\,}\star\textbf{Elab}\ A\quad H,x:A\vdash N\overleftarrow{\,:_{\ell',.}\,}\star\textbf{Elab}\ B}{H\vdash\left(\left(x:M_{\ensuremath{\ell}}\right)\rightarrow N_{\ensuremath{\ell'}}\right)\textbf{Elab}\left(\left(x:A\right)\rightarrow B\right)\overrightarrow{\,:\,}\star}\operatorname{\overrightarrow{\textbf{Elab}}-fun-ty}
\]

\[
\frac{H\vdash m\,\textbf{Elab}\ b\overrightarrow{\,:\,}\left(x:A\right)\rightarrow B\quad H\vdash n\overleftarrow{\,:_{\ell,.arg}\,}A\,\textbf{Elab}\,a}{H\vdash\left(m_{\ell}\,n\right)\textbf{Elab}\left(b\,a\right)\overrightarrow{\,:\,}B\left[x\coloneqq a\right]}\operatorname{\overrightarrow{\textbf{Elab}}-fun-app}
\]

\[
\frac{H\vdash n\,\textbf{Elab}\ a\overrightarrow{\,:\,}\left(x:A\right)\rightarrow B\quad H\vdash n\overleftarrow{\,:_{\ell,.arg}\,}A\,\textbf{Elab}\ a}{H\vdash\left(m_{\ell}\,n\right)\textbf{Elab}\left(b\,a\right)\overrightarrow{\,:\,}B\left[x\coloneqq a\right]}\operatorname{\overrightarrow{\textbf{Elab}}-fun-app}
\]

\[
\frac{H\vdash M\overleftarrow{\,:_{\ell',.}\,}\star\,\textbf{Elab}\ A\quad H\vdash m\overleftarrow{\,:_{\ell,.}\,}A\,\textbf{Elab}\ a}{H\vdash\left(m::_{\ell}M^{\ell'}\right)\textbf{Elab}\,a\overrightarrow{\,:\,}A}\operatorname{\overrightarrow{\textbf{Elab}}-::}
\]

\[
\frac{H,f:\left(x:A\right)\rightarrow B,x:A\vdash m\overleftarrow{\,:_{\ell,o.bod[x]}\,}B\,\textbf{Elab}\ b}{H\vdash\left(\mathsf{fun}\,f\,x\Rightarrow m\right)\overleftarrow{\,:_{\ell,o}\,}\left(x:A\right)\rightarrow B\,\textbf{Elab}\left(\mathsf{fun}\,f\,x\Rightarrow b\right)}\operatorname{\overleftarrow{\textbf{Elab}}-fun}
\]

\[
\frac{H\vdash m\,\textbf{Elab}\ a\overrightarrow{\,:\,}A}{H\vdash m\overleftarrow{\,:_{\ell,o}\,}B\ \textbf{Elab}\left(a::_{A,\ensuremath{\ell},o}B\right)}\operatorname{\overleftarrow{\textbf{Elab}}-cast}
\]

\[
\frac{H\vdash m\,\textbf{Elab}\ a\overrightarrow{\,:\,}\star}{H\vdash m\overleftarrow{\,:_{\ell,o}\,}\star\ \textbf{Elab}\,a}\operatorname{\overleftarrow{\textbf{Elab}}-conv-\star}
\]

\todo[inline]{which syntax looks the best? on the left when input, or alway on the
right like the typing judgment}

\caption{Elaboration}
\label{fig:elaboration}
\end{figure}

% rules
Like the bidirectional rules, the rules for elaboration are broken
into two judgments, 
\begin{itemize}
\item $H\vdash m\overleftarrow{\,:_{\ell,o}\,}A\,\textbf{Elab}\ a$, that
generates a cast term $a$ from a surface term $m$ given its expected
type $A$ along with a location $\ell$ and observation $o$ that
made that assertion
\item $H\vdash m\,\textbf{Elab}\ a\overrightarrow{\,:\,}A$, that generates
a cast term $a$ and its type $A$ from a surface term $m$
\end{itemize}
The rules for elaboration are presented in Figure \ref{fig:elaboration}.
Elaboration rules are written in a style of bidirectional type checking.
However, unlike bidirectional type checking, when checking an inference
elaboration adds a cast assertion that the two types are equal. Thus
any conversion checking can be suspended until runtime. Additionally
we will allow the mode to change at the type universe with the $\operatorname{\overleftarrow{\textbf{Elab}}-conv-\star}$
rule, this is required to avoid stuck casts at $\star$. As formulated
here, the elaboration procedure is total. 

There are several desirable properties of elaboration, 

\begin{figure}
\begin{tabular}{ccc}
$|x|$ & = & $x$\tabularnewline
$|\star|$ & = & $\star$\tabularnewline
$|m::_{\ell}M|$ & = & $|m|$\tabularnewline
$|\left(x:M_{\ensuremath{\ell}}\right)\rightarrow N_{\ensuremath{\ell'}}|$ & = & $\left(x:|M|\right)\rightarrow|N|$\tabularnewline
$|m_{\ensuremath{\ell}}\,n|$ & = & $|m|\,|n|$\tabularnewline
$|\mathsf{fun}\,f\,x\Rightarrow m|$ & = & $\mathsf{fun}\,f\,x\Rightarrow|m|$\tabularnewline
$|\lozenge|$ & = & $\lozenge$\tabularnewline
$|\Gamma,x:A|$ & = & $|\Gamma|,x:|A|$\tabularnewline
$|a::_{A,\ensuremath{\ell},o}B|$ & = & $|a|$\tabularnewline
$|\left(x:A\right)\rightarrow B|$ & = & $\left(x:|A|\right)\rightarrow|B|$\tabularnewline
$|\mathsf{fun}\,f\,x\Rightarrow b|$ & = & $\mathsf{fun}\,f\,x\Rightarrow|b|$\tabularnewline
$|b\,a|$ & = & $|b|\,|a|$\tabularnewline
$|H,x:M|$ & = & $|H|,x:|M|$\tabularnewline
\end{tabular}\caption{Erasure}
\label{fig:erasure}
\end{figure}

\begin{thm}
elaborated terms preserve the erasure (defined in \ref{fig:erasure})

if $H\vdash m\,\textbf{Elab}\ a\overrightarrow{\,:\,}A$ then $|m|=|a|$ 

if $H\vdash m\,a\overleftarrow{\,:_{\ensuremath{\ell},o}\,}A\,\textbf{Elab}\,a$
then $|m|=|a|$ 
\end{thm}

\begin{proof}
by mutual induction on the $\textbf{Elab}$ derivations

\begin{tabular}{lll}
$\operatorname{\overrightarrow{\textbf{Elab}}-var}$ & $|x|=x=|x|$  & \tabularnewline
$\operatorname{\overrightarrow{\textbf{Elab}}-\star}$ & $|\star|=\star=|\star|$  & \tabularnewline
$\operatorname{\overrightarrow{\textbf{Elab}}-fun-ty}$ & \multicolumn{2}{l}{$H\vdash M\overleftarrow{\,:_{\ell,.}\,}\star\textbf{Elab}\ A\quad H,x:A\vdash N\overleftarrow{\,:_{\ell',.}\,}\star\textbf{Elab}\ B$}\tabularnewline
 & $|M|=|A|$ & by induction\tabularnewline
 & $|N|=|B|$ & by induction\tabularnewline
 & $|\left(x:M_{\ensuremath{\ell}}\right)\rightarrow N_{\ensuremath{\ell'}}|=\left(x:|M|\right)\rightarrow|N|=\left(x:|A|\right)\rightarrow|B|=|\left(x:A\right)\rightarrow B|$ & \tabularnewline
$\operatorname{\overrightarrow{\textbf{Elab}}-fun-app}$ & \multicolumn{2}{l}{$H\vdash m\,\textbf{Elab}\ b\overrightarrow{\,:\,}C\quad C\Rrightarrow_{\ast}\left(x:A\right)\rightarrow B\quad H\vdash n\overleftarrow{\,:_{\ell,.arg}\,}A\,\textbf{Elab}\,a$}\tabularnewline
 & $|m|=|b|$ & by induction\tabularnewline
 & $|n|=|a|$ & by induction\tabularnewline
 & $|m_{\ensuremath{\ell}}\,n|=|m|\,|n|=|b|\,|a|=|b\,a|$ & \tabularnewline
$\operatorname{\overleftarrow{\textbf{Elab}}-fun}$ & \multicolumn{2}{l}{$H,f:\left(x:A\right)\rightarrow B,x:A\vdash m\overleftarrow{\,:_{\ell,o.bod[x]}\,}B\,\textbf{Elab}\ b$}\tabularnewline
 & $|m|=|b|$ & by induction\tabularnewline
 & $|\mathsf{fun}\,f\,x\Rightarrow m|=\mathsf{fun}\,f\,x\Rightarrow|m|=\mathsf{fun}\,f\,x\Rightarrow|b|=|\mathsf{fun}\,f\,x\Rightarrow b|$ & \tabularnewline
$\operatorname{\overrightarrow{\textbf{Elab}}-::}$ & \multicolumn{2}{l}{$H\vdash M\overleftarrow{\,:_{\ell.}\,}\star\,\textbf{Elab}\ A\quad H\vdash m\overleftarrow{\,:_{\ell,.}\,}A\,\textbf{Elab}\ a$}\tabularnewline
 & $|m|=|a|$ & by induction\tabularnewline
 & $|m::_{\ell}M|=|m|=|a|$ & \tabularnewline
$\operatorname{\overleftarrow{\textbf{Elab}}-conv}$ & $H\vdash m\,\textbf{Elab}\ a\overrightarrow{\,:\,}A$, ... & \tabularnewline
 & $|m|=|a|$ & by induction\tabularnewline
 & $|m|=|a|=|a::_{A,\ensuremath{\ell},o}A'|$ & \tabularnewline
\end{tabular}
\end{proof}
It follows that whenever an elaborated cast term evaluates, the corresponding
surface term evaluates consistently 
\begin{thm}
Surface language and cast language have consistent evaluation

if $H\vdash m\,\textbf{Elab}\ a\overrightarrow{\,:\,}A$, and $a\rightsquigarrow_{*}\star$
then $m\rightsquigarrow_{*}\star$ 

if $H\vdash\textbf{Elab}\ m\,a\overrightarrow{\,:\,}A$, and $a\rightsquigarrow_{*}(x:A)\rightarrow B$
then there exists $N$ and $M$ such that $m\rightsquigarrow_{*}(x:N)\rightarrow M$ 
\end{thm}

\begin{proof}
Since $a\rightsquigarrow_{*}a'$ implies $|a|\rightsquigarrow_{*}|a'|$
and $m\rightsquigarrow_{*}m'$ implies $|m|\rightsquigarrow_{*}|m'|$.
\end{proof}
elaborated terms are well-cast in a well formed context \todo{need to define and state H ok}
\begin{thm}
elaborated terms are well-cast

for any $H\vdash a\,\textbf{Elab}\,m\overrightarrow{\,:\,}A$ then
$H\vdash a:A$, $H\vdash A:\star$

for any $H\vdash$, $H\vdash A:\star$ , $H\vdash a\overleftarrow{\,:_{\ell,o}\,}A\,\textbf{Elab}\,m$
then $H\vdash a:A$ 
\end{thm}

\begin{proof}
by mutual induction on $\textbf{Elab}$ derivations

\todo{revise below, causing latex some issues}

\begin{tabular}{lll}
$\operatorname{\overrightarrow{\textbf{Elab}}-var}$ & $x:A\in H$ & \tabularnewline
 & $H\vdash A:\star$ & by $H\vdash$\tabularnewline
 & $H\vdash x:A$ & $\operatorname{cast-var}$\tabularnewline
$\operatorname{\overrightarrow{\textbf{Elab}}-fun-ty}$ & \multicolumn{2}{l}{$H\vdash M\overleftarrow{\,:_{\ell,.}\,}\star\textbf{Elab}\ A\quad H,x:A\vdash N\overleftarrow{\,:_{\ell',.}\,}\star\textbf{Elab}\ B$}\tabularnewline
 & $H\vdash A:\star$ & by induction\tabularnewline
 & $H,x:A\vdash B:\star$ & by induction\tabularnewline
$\operatorname{\overrightarrow{\textbf{Elab}}-fun-app}$ & \multicolumn{2}{l}{$H\vdash m\,\textbf{Elab}\ b\overrightarrow{\,:\,}C\quad C\Rrightarrow_{\ast}\left(x:A\right)\rightarrow B\quad H\vdash n\overleftarrow{\,:_{\ell,.arg}\,}A\,\textbf{Elab}\,a$}\tabularnewline
 & $H\vdash b:C$, $H\vdash C:\star$ & by induction\tabularnewline
 & $H\vdash\left(x:A\right)\rightarrow B:\star$ & by $\Rrightarrow_{\ast}$ preservation\tabularnewline
 & $H\vdash A:\star$, $H,x:A\vdash B:\star$ & by fun-ty inversion\tabularnewline
 & $H\vdash b:\left(x:A\right)\rightarrow B$ & $\operatorname{cast-conv}$\tabularnewline
 & $H\vdash a:A$ & by induction\tabularnewline
 & $H\vdash B\left[x\coloneqq a\right]:\star$ & by subst preservation\tabularnewline
 & $H\vdash b\,a\,:\,B\left[x\coloneqq a\right]$ & $\operatorname{cast-fun-app}$\tabularnewline
$\operatorname{\overleftarrow{\textbf{Elab}}-fun}$ & \multicolumn{2}{l}{$H,f:\left(x:A\right)\rightarrow B,x:A\vdash$, $H,f:\left(x:A\right)\rightarrow B,x:A\vdash B:\star$}\tabularnewline
 & $H,f:\left(x:A\right)\rightarrow B,x:A\vdash m\overleftarrow{\,:_{\ell,o.bod[x]}\,}B\,\textbf{Elab}\ b$ & \tabularnewline
 &  & TODO revise!\tabularnewline
 & $H,f:\left(x:A\right)\rightarrow B,x:A\vdash A:\star$ & by reg\tabularnewline
 & $H\vdash A:\star$ & by removing free vars\tabularnewline
 & $H,x:A\vdash B:\star$ & similarly with fun-ty inversion\tabularnewline
 & $H,f:\left(x:A\right)\rightarrow B,x:A\vdash b:B$ & by induction\tabularnewline
 & $H\vdash\mathsf{fun}\,f\,x\Rightarrow b\,:\,\left(x:A\right)\rightarrow B$ & $\operatorname{cast-fun}$\tabularnewline
$\operatorname{\overrightarrow{\textbf{Elab}}-::}$ & \multicolumn{2}{l}{$H\vdash M\overleftarrow{\,:_{\ell.}\,}\star\,\textbf{Elab}\ A\quad H\vdash m\overleftarrow{\,:_{\ell,.}\,}A\,\textbf{Elab}\ a$}\tabularnewline
 & $H\vdash A:\star$ & by induction\tabularnewline
 & $H\vdash a:A$ & by induction\tabularnewline
$\operatorname{\overleftarrow{\textbf{Elab}}-conv}$ & $H\vdash$, $H\vdash A':\star$ , $H\vdash m\,\textbf{Elab}\ a\overrightarrow{\,:\,}A$,
with some $\ell,o$ & \tabularnewline
 & $H\vdash a:A$, $H\vdash A:\star$ & by induction\tabularnewline
 & $H\vdash a::_{A,\ensuremath{\ell},o}B\::\:B$ & $\operatorname{cast-::}$\tabularnewline
\end{tabular}

Some additional properties are strongly conjectured
\end{proof}
\begin{conjecture}
Every term well typed in the bidirectional surface language elaborates 

if $\Gamma\vdash$, then there exists $H$ such that $\Gamma\,\textbf{Elab}\,H$ 

$\Gamma\vdash m\overrightarrow{\,:\,}M$ then there exists $H$, $a$
and $A$ such that $\Gamma\,\textbf{Elab}\,H$, $H\vdash m\,\textbf{Elab}\ a\overrightarrow{\,:\,}A$ 

$\Gamma\vdash m\overleftarrow{\,:\,}M$ and given $\ell$, $o$then
there exists $H$, $a$and $A$ such that $\Gamma\,\textbf{Elab}\,H$,
$H\vdash\textbf{Elab}\,a\,m\overleftarrow{\,:_{\ell o}\,}A$ 
\end{conjecture}

Which if true would lead to the corollary
\begin{conjecture}
blame never points to something that checked in the bidirectional
system 

if $\vdash m\overrightarrow{\,:\,}M$, and $\vdash\textbf{Elab}\ m\,a\overrightarrow{\,:\,}A$
, then for no $a\rightsquigarrow_{*}a'$ will $\textbf{Blame}\:\ensuremath{\ell}\,o\:a'$
occur 

\todo{revise precisely with lables}
\end{conjecture}

The last three guarantees are similar to the gradual guarantee \cite{siek_et_al:LIPIcs:2015:5031}
for gradual typing. 

\section{Suitable Warnings}

As presented here, not every cast corresponds to a reasonable warning.
For instance, $\left(\lambda x\Rightarrow x\right)::_{\star\rightarrow\star}\star\rightarrow\star$
is a possible output from elaboration. By the rules given the cast
will not reduce without input, it is however inert, and will not cause
blame. In fact since the user only interacts with the surface language,
any cast $a::_{A}B$ where $|A|\equiv|B|$ will not produce an understandable
warning, and further should not be a direct source of runtime errors.
A reasonable first attempt would be to simply remove the casts of
the form $a::_{A}A$, but this ignores the possibility that casts
themselves may contain casts\todo{example}. Currently the implementation
leaves most casts intact and filters our equivalent casts from the
warnings shown to the user

\todo{as fig}

\begin{tabular}{llllll}
$Warns($ & $a::_{A,\ensuremath{\ell},o}B$ & $)=$ & $\left\{ (a,\ensuremath{\ell},o,B)\right\} \cup Warns(a)\cup Warns(A)\cup Warns(B)$ & if & $|A|\cancel{\equiv}|B|$\tabularnewline
$Warns($ & $a::_{A,\ensuremath{\ell},o}B$ & $)=$ & $Warns(a)\cup Warns(A)\cup Warns(B)$ & if & $|A|\equiv|B|$\tabularnewline
$Warns($ & $\star$ & $)=$ & $\emptyset$ &  & \tabularnewline
$Warns($ & $x$ & $)=$ & $\emptyset$ &  & \tabularnewline
$Warns($ & $\left(x:A\right)\rightarrow B$ & $)=$ & $Warns(A)\cup Warns(B)$ &  & \tabularnewline
$Warns($ & $\mathsf{fun}\,f\,x\Rightarrow b$ & $)=$ & $Warns(b)$ &  & \tabularnewline
$Warns($ & $b\,a$ & $)=$ & $Warns(a)\cup Warns(b)$ &  & \tabularnewline
\end{tabular}

Since the $\equiv$ relation is undecidable an approximation can be
used in practice. Removing impossible casts should be considered like
a compiler optimization. In Chapter 6 casts will need to be extended,
and it will be more clear when a cast can through its own error or
merely exists to hold other casts.

\section{Related Work}

\subsection{Dependent types and equality}

\todo{revise as bulleted list starting with ETT}

Difficulties in dependently typed equality have motivated many research
projects \cite{HoTTbook,sjoberg2015programming,cockx2021taming}.
However, these impressive efforts currently require a high level of
expertise from programmers. Further, since program equivalence is
undecidable in general, no system will be able to statically verify
every ``obvious'' equality for arbitrary user defined data types
and functions. In the meantime systems should trust the programmer
when they use an unverified equality, and use that advanced research
to suppress warnings. 

\subsection{Contract Systems}

Several of the tricks and notations in this chapter find their basis
in the large amount of work on higher order contracts and gradual
types. Higher order contracts were introduced in \cite{10.1145/581478.581484}
as a way to dynamically enforce invariants of software interfaces,
specifically higher order functions% cotracts go back to the 70s aparently, but this seems a reasonable place to start the story, though Bigloo Scheme [28] is cited there.
The notion of blame dates at least that far back. Swapping the type
cast of the input argument can be traced directly to that paper's
use of blame contravarience, though it is presented in a much different
way. % However the contract language of that paper was somewhat limited.

Contract semantics were revisited in \cite{10.1145/1925844.1926410,10.1007/978-3-642-28869-2_11}
where a more specific correctness criteria based in blame is presented.

Contract systems still generally rely on users annotating their invariants
explicitly. Similar to how programmers might include $\mathtt{assert}$s
in an imperative language. In this thesis annotations are added automatically
though elaboration.

While there are similarities between contract systems and the cast
system outlined here, the cast system is designed to address only
issues with definitional equality in a dependent type theory. Since
contract systems are generally used in untyped languages with contracts
written in the host language, definitional equality simply isn't relevant
in the vast majority of contract systems.

\subsubsection{Gradual Types}

Types can be viewed as a very specific form of contracts, gradual
type systems allow for a mixing of the static view of data and dynamic
checking. Often type information can be inferred using standard techniques,
allowing programmers to write fewer contract annotations. % TODO first paper, DLS2008

Gradual type systems usually achieve this by adding in a $?$ meta
character to denote imprecise typing information. The first popular
account of gradual typed semantic's appeared in \cite{siek_et_al:LIPIcs:2015:5031}
with the alliterative ``gradual guarantee'' which has inspired some
of the properties in this chapter.

% , which informally asserts that "runtime checks will not change the expected behavour", "runtime checks will not change the expected behavour" and "well typed code won't be blamed"

% and then like an endless back and forth over criteria

Additionally some of the formalism from this chapter was inspired
by the ``Abstracting gradual typing'' methodology \cite{10.1145/2837614.2837670},
where static evidence annotations become runtime checks. % Unlike some impressive attempts to gradualize the polymorphic lambda calculus \cite{10.1145/3110283}, our system does not attempt to enforce any parametric properties of the base language. %example?
% It is unclear if such a restriction would be desirable for a dependently typed language in practice.

This thesis borrows some notational conventions from gradual typing
such as the $a::A$ construct for type assertions.

A system for gradual dependent types has been preposed in \cite{10.1145/3341692}.
That paper is largely concerned with establishing decidable type checking
via an approximate term normalization. However, that system retains
the intentional style of definitional equality, so that it is possible,
in principle, to get $vec\,(1+x)\neq vec\,(x+1)$ as a runtime error.
Additionally it is unclear if adding the $?$ meta-symbol into an
already very complicated type theory, is easier or harder from the
programmers perspective.

The common motivation for gradual type systems to gradually convert
a code base from untyped to (usually simply) typed code. This chapter
has a much tighter scope then the other work cited here, dealing only
with equational assumptions. Anyone using a dependent type system
has already bought into the usefulness of types in general and will
probably not want fragments of completely untyped code. Converting
untyped code to dependent typed is far less plausible than gradually
converting an untyped code base to use simple types\footnote{Especially considering that most real-life codebases will use effects,
while dependent types and effects are a complicated area of ongoing
research}.

While the gradual typing goals of mixing static certainty with runtime
checks are similar to our work here, the approach and details are
different. Instead of trying to strengthen untyped languages by adding
types, we take a dependent type system and weaken it with a cast operator.
This leads to different trade-offs in the design space. For instance,
we cannot support completely unannotated code, but we do not need
to complicate the type language with wildcards for uncertainty. However
it might be reasonable to characterize this work in this chapter as
gradualizing the definitional equality relation.

\subsubsection{Blame}

Blame is one of the key ideas explored in the contract type and gradual
types literature\cite{10.1007/978-3-642-00590-9_1,wadler:LIPIcs:2015:5033,10.1145/3110283}.
Often the reasonableness of a system can be judged by the way blame
is handled\cite{wadler:LIPIcs:2015:5033}. Blame is treated in \cite{wadler:LIPIcs:2015:5033}
very similarly to the presentation in this chapter. Though in addition
to merely blaming a source locations we ensure that a witnessing observation
can also be made.

\subsection{Refinement Style Approaches}

In this thesis I describe a full-spectrum dependently typed language.
This means computation can appear uniformly in both term and type
position. An alternative approach to dependent types is found in refinement
type systems. Refinement type systems restrict type dependency, possibly
to specific base types such as $\mathtt{int}$ or $\mathtt{bool}$.
Under this restriction, it is straightforward to check these decidable
equalities and some additional properties hold at runtime. One specific
approach is called \textbf{Hybrid Type Checking} \cite{10.1145/1111037.1111059}\todo{more}.
Another notable example is \cite{10.1007/1-4020-8141-3_34} which
describes a refinement system that limits predicates to base types.
Another example is \cite{10.1145/3093333.3009856}, a refinement system
treated in a specifically gradual way. A refinement type system with
higher order features is gradualized in \cite{c4be73a0daf74c9aa4d13483a2c4dd0e}.
\todo{why is full spectrum better?}

% consider also citing https://www.youtube.com/watch?v=gIYMERs7AZQ https://www.youtube.com/watch?v=EGKeWg2ES0A
