\chapter{The Dependent Cast System}
\label{chapter:Cast}
\thispagestyle{myheadings}
 
We can now tackle the most fundamental problem with dependent type systems: Definitional equalities are pervasive and unintuitive.
 
%% example of dependent types
The motivating example from the introduction can be stated more precisely in terms of the surface language.
Recall, dependent types can prevent an out-of-bounds error when indexing into a length indexed list.
 
\begin{align*}
\mathtt{Vec} & :*\rightarrow\mathbb{N}_{c}\rightarrow*,\\
\mathtt{rep} & :\left(X:*\right)\rightarrow X\rightarrow\left(y:\mathbb{N}_{c}\right)\rightarrow\mathtt{Vec\,}X\,y,\\
\mathtt{head} & :\left(X:*\right)\rightarrow\left(y:\mathbb{N}_{c}\right)\rightarrow\mathtt{Vec}\,X\,\left(1_{c}+_{c}y\right)\rightarrow X
\end{align*}
\[
\vdash\lambda x\Rightarrow\mathtt{head}\,\mathbb{B}_{c}\,x\,\left(\mathtt{rep}\,\mathbb{B}_{c}\,true_{c}\,\left(1_{c}+_{c}x\right)\right)\,:\,\mathbb{N}_{c}\rightarrow\mathbb{B}_{c}
\]
 
$\mathtt{head}$ is a function that expects a list of length $1_{c}+_{c}y$, making it impossible for $\mathtt{head}$ to inspect an empty list.
 
%% example of problem
Unfortunately, the following will not type check in the surface language,
 
\[
\cancel{\vdash}\lambda x\Rightarrow\mathtt{head}\,\mathbb{B}_{c}\,x\,\left(\mathtt{rep}\,\mathbb{B}_{c}\,true_{c}\,\left(x+_{c}1_{c}\right)\right)\,:\,\mathbb{N}_{c}\rightarrow\mathbb{B}_{c}
\]
 
%% explanation of example
\sloppy While ``obviously'' $1+x=x+1$, in the surface language, definitional equality does not associate these two terms, $1_{c}+_{c}x\cancel{\equiv}x+_{c}1_{c}$.
 
In this chapter we will show how to side step definitional equality and deal with incorrect equalities at runtime in a principled way.
This will be done with the \textbf{cast system} described in this chapter.
The cast system is comprised of
\begin{itemize}
\item The \textbf{cast language}, a dependently typed language with embedded runtime checks that have evaluation behavior.
\item The \textbf{elaboration procedure} that transforms appropriate untyped surface syntax into checked cast language terms.
\todo{as a picture}
\end{itemize}
 
The presentation in this chapter mirrors the system in chapter 2.
The cast language plays the role of the type assignment system, while the elaboration procedure corresponds with the \bidir system.
 
We show that a novel form of type soundness holds, that we call \textbf{cast soundness}.
Instead of ``well typed terms don't get stuck'', we prove ``well cast terms don't get stuck without blame''.
 
Additionally, by construction, blame (in the sense of contracts and monitors) is reasonably handled.
Several desirable properties hold for the system overall. \todo{which?}
 
\section{Cast Language}
 
\begin{figure}
\begin{tabular}{lcll}
\multicolumn{4}{l}{source locations,}\tabularnewline
$\ensuremath{\ell}$ &  &  & \tabularnewline
\multicolumn{4}{l}{variable contexts,}\tabularnewline
$H$ & $\Coloneqq$ & $\lozenge$ $|$ $H,x:A$ & \tabularnewline
\multicolumn{4}{l}{expressions,}\tabularnewline
$a,b$,$A,B$ & $\Coloneqq$ & $x$ & \tabularnewline
& $|$ & $a::_{A,\ensuremath{\ell},o}B$ & cast\tabularnewline
& $|$ & $\star$ & \tabularnewline
& $|$ & $\left(x:A\right)\rightarrow B$ & \tabularnewline
& $|$ & $\mathsf{fun}\,f\,x\Rightarrow b$ & \tabularnewline
& $|$ & $b\,a$ & \tabularnewline
\multicolumn{4}{l}{observations,}\tabularnewline
o & $\Coloneqq$ & . & current location\tabularnewline
& $|$ & $o.arg$ & function type-arg\tabularnewline
& $|$ & $o.bod[a]$ & function type-body\tabularnewline
\end{tabular}
 
\todo[inline]{would $a_{A}:?:_{\ensuremath{\ell},o}B$ be clearer syntax then $a::_{A,\ensuremath{\ell},o}B$?}
 
\todo[inline]{would it be clearer to add the $\ell$ to the observation?}
 
\caption{Cast Language Syntax}
\label{fig:cast-pre-syntax}
\end{figure}
 
%% pre-syntax
The syntax for the cast language can be found in \Fref{cast-pre-syntax}.
By design the cast language is almost identical to the surface language except that the cast construct has been added and annotations have been removed.
 
The cast language can assume arbitrary equalities over types, $A=B$, with a cast, $a::_{A,\ensuremath{\ell},o}B$ given
\begin{itemize}
\item an underlying term $a$,
\item a source location $\ell$ where it was asserted,
\item a concrete observation $o$ that would witness inequality,
\item the type of the underlying $a$ term $A$,
\item and the expected type of the term $B$.
\end{itemize}
Every time there is a definitional mismatch between the type inferred from a term and the type expected from the usage, the elaboration procedure will produce a cast.
 
Observations allow indexing into terms to pinpoint errors.
For instance, if we want to highlight the $C$ sub expression in $\left(x:A\right)\rightarrow\left(y:\left(x:B\right)\rightarrow\underline{C}\right)\rightarrow D$ we can use the observation $..Bod[x].arg.Bod[y]$.
In general, the $C$ may specifically depend on $x$ and $y$ so they are tracked as part of the observation.
For instance, given the type $\left(X:\star\right)\rightarrow X$ we might want to point out $A$ when $X=A\rightarrow B$ resulting in the type $\left(X:\star\right)\rightarrow\left(\underline{A}\rightarrow B\right)$.
The observation would then read $..Bod[A\rightarrow B].arg$ recording the specific type argument that allows an argument to be inspected.
 
Locations and observations will be used to form blame and produce the runtime error message users might see.
 
In addition to the abbreviations from Chapter 2, some new abbreviations for the cast language are listed in Figure \ref{fig:cast-pre-syntax-abrev}.
\begin{figure}
\begin{tabular}{lclll}
$a::_{A,\ensuremath{\ell},o}B$ & written & $a::_{A,\ensuremath{\ell}}B$ & when & the observation is not relevant\tabularnewline
$a::_{A,\ensuremath{\ell}}B$ & written & $a::_{A}B$ & when & the location is not relevant\tabularnewline
$a::_{A}B$ & written & $a::B$ & when & the the type of $a$is clear\tabularnewline
\end{tabular}
 
\caption{Surface Language Abbreviations}
\label{fig:cast-pre-syntax-abrev}
\end{figure}
 
\subsection{How Should Casts Reduce?}
 
How does the cast construct interact with the existing constructs?
There are three interactions that could cause a term to be stuck in evaluation or block type checking:
 
\begin{tabular}{ll}
$\star::B$ & universe under cast\tabularnewline
$\left(\left(x:A\right)\rightarrow B\right)::C$ & function type under cast\tabularnewline
$\left(b::C\right)a$ & application blocked by cast\tabularnewline
\end{tabular}
 
We can account for these by realizing obvious casts should evaporate, freeing up the underlying term.
The most interesting case is when a cast confirms that the applied term is a function, but with potentially different input and output types.\todo{awk}
Then we use the function type syntax to determine a reasonable cast over the argument, and maintain the appropriate cast over the resulting computation.
This operation is similar to the way higher order contracts invert the polarity of blame for the arguments of higher order functions \cite{10.1145/581478.581484} and also found in gradual type systems, such as \cite{10.1007/978-3-642-00590-9_1}.
 
\begin{sidewaysfigure}
\begin{tabular}{ccccc}
$\star::\star$ & $\rightsquigarrow$ & $\star$ &  & \tabularnewline
$\star::B$ & $\rightsquigarrow$ & $\star::B'$ & when  & $B\rightsquigarrow B'$\tabularnewline
$\star::_{\ell,o}B$ & blame & $\ell,o$ & when  & $B$ cannot be $\star$ \tabularnewline
$\left(\left(x:A\right)\rightarrow B\right)::\star$ & $\rightsquigarrow$ & $\left(\left(x:A\right)\rightarrow B\right)::\star$ & $\rightsquigarrow$ & \tabularnewline
$\left(\left(x:A\right)\rightarrow B\right)::C$ & $\rightsquigarrow$ & $\left(\left(x:A\right)\rightarrow B\right)::C'$ & when  & $C\rightsquigarrow C'$\tabularnewline
$\left(\left(x:A\right)\rightarrow B\right)::_{\ell,o}C$ & blame & $\ell,o$ & when  & $C$ cannot be $\star$ \tabularnewline
$\left(b::_{\left(x:A'\right)\rightarrow B'}\left(x:A\right)\rightarrow B\right)a$ & $\rightsquigarrow$ & $\left(b\,\left(a::A'\right)\right)::_{B'\left[x\coloneqq a::A'\right]}B\left[x\coloneqq A\right]$ &  & \tabularnewline
$\left(b::C\right)a$ & $\rightsquigarrow$ & $\left(b::C'\right)a$ & when  & $C\rightsquigarrow C'$\tabularnewline
$\left(b::_{\ell,o}C\right)a$ & blame & $\ell,o$ & when  & $C$ cannot be $\left(x:A\right)\rightarrow B$ \tabularnewline
\end{tabular}
 
\caption{Approximate Cast Language Reductions}
\label{fig:cast-aprox-red}
\end{sidewaysfigure}
 
\Fref{cast-aprox-red} shows approximately how these reductions should be carried out.
The fully formal reduction rules are listed later, but their complete detail can be distracting.
Type universes live in the type universe, so any cast that contradicts this should be blamed.
Similarly for function types.
Terms that take input must be functions, so any cast that contradicts this should blame the source location.
\todo{awk}
 
Note that the rules outlined here are not deterministic since there are cases when we might blame or continue reducing the argument.
One of the subtle innovations of the system described in this chapter is to completely separate blame from reduction.
This sidesteps many of the complexities of having a reduction relevant $\mathtt{abort}$ term in a dependent type theory \cite{sjoberg2012irrelevance,pedrot2018failure}.
As far as reduction is concerned, bad terms simply ``get stuck'' as it might on a variable from a nonempty typing context.
Otherwise the reduction behavior is well behaved.
Terms may be blamable by the rules outlined in this chapter, but it is easy to imagine more sophisticated ways to extract blame from terms.
 
This outlines the minimum requirements for cast reductions, there are plausibly many additional reductions that could be considered.
Some tempting reductions are
 
\begin{tabular}{ccccc}
$a::_{C}C$ & $\rightsquigarrow_{=}$ & $a$ &  & \tabularnewline
$a::_{C'}C$ & $\rightsquigarrow_{\equiv}$ & $a$ & when & $C'\equiv C$\tabularnewline
\end{tabular}
 
However these rules preclude extracting blame that may be embedded within the casts themselves.
These rules also seem to complicate the meta theory.
Despite this we will use these reductions in examples to keep the book keeping to a minimum.
 
 
\section{Examples}
\todo{lead in}
% We can re-examine some of the examples terms from Chapter 2, but this time with equality assumptions
 
\subsection{Higher Order Functions}
 
%% walkthrough
Higher order functions are dealt with by distributing function casts around applications.
If an application happens to a cast of function type, the argument and body cast is separated and the argument cast is swapped.
For instance, in
\begin{align*}
\, & \left(\left(\lambda x\Rightarrow x\&\&x\right)::_{\mathbb{B}\rightarrow\mathbb{B},\ell,.}\mathbb{N}\rightarrow\mathbb{N}\right)7\\
\rightsquigarrow & \left(\left(\lambda x\Rightarrow x\&\&x\right)\left(7::_{\mathbb{N},\ell,.arg}\mathbb{B}\right)\right)::_{\mathbb{B},\ell,.bod[7]}\mathbb{N}\\
\rightsquigarrow & \left(\left(7::_{\mathbb{N},\ell,.arg}\mathbb{B}\right)\&\&\left(7::_{\mathbb{N},\ell,.arg}\mathbb{B}\right)\right) ::_{\mathbb{B},\ell,.bod[7]}\mathbb{N}
\end{align*}
If evaluation gets stuck on $\&\&$ we can blame the argument of the cast for equating $\mathbb{N}$ and $\mathbb{B}$.
The body observation records the argument the function is called with.
For instance, in the $.bod[7]$ observation.
In a dependently typed function the exact argument may be important to give a good error.
Because casts can be embedded inside of casts, types themselves need to normalize and casts need to simplify.
Since our system has one universe of types, type casts only need to simplify themselves when a term of type $\star$ is cast to $\star$.
For instance,
\begin{align*}
\, & \left(\left(\lambda x\Rightarrow x\right)::_{\left(\mathbb{B}\rightarrow\mathbb{B}\right)::_{\star,\ell,.arg}\star,\ell,.}\mathbb{N}\rightarrow\mathbb{N}\right)7\\
\rightsquigarrow & \left(\left(\lambda x\Rightarrow x\right)::_{\mathbb{B}\rightarrow\mathbb{B}}\mathbb{N}\rightarrow\mathbb{N}\right)7\\
\rightsquigarrow & \left(\left(\lambda x\Rightarrow x\right)\left(7::_{\mathbb{N},\ell,.arg}\mathbb{B}\right)\right)::_{\mathbb{B},\ell,.bod[7]}\mathbb{N}\\
\rightsquigarrow & \left(\left(7::_{\mathbb{N},\ell,.arg}\mathbb{B}\right)\right)::_{\mathbb{B},\ell,.bod[7]}\mathbb{N}
\end{align*}
 
\subsection{Pretending $true=false$}
 
Recall that we proved $\lnot true_{c}\doteq_{\mathbb{B}_{c}}false_{c}$ in \ch{2}.
What happens if it is assumed anyway?
Every type equality assumption needs an underlying term, here we can choose $refl_{true_{c}:\mathbb{B}_{c}}:true_{c}\doteq_{\mathbb{B}_{c}}true_{c}$, and cast that term to $true_{c}\doteq_{\mathbb{B}_{c}}false_{c}$ resulting in $refl_{true_{c}:\mathbb{B}_{c}}::_{true_{c}\doteq_{\mathbb{B}_{c}}true_{c}}true_{c}\doteq_{\mathbb{B}_{c}}false_{c}$.
Recall that $\lnot true_{c}\doteq_{\mathbb{B}_{c}}false_{c}$ is a shorthand for $true_{c}\doteq_{\mathbb{B}_{c}}false_{c}\rightarrow\perp_{c}$.
What if we try to use our term of type $true_{c}\doteq_{\mathbb{B}_{c}}false_{c}$ to get a term of type $\perp_{c}$?
 
\begin{sidewaysfigure}
\begin{tabular}{ll}
  % \makecell[l]{Void, ``empty'' type,\\ logical false}
& $\left(\lambda pr\Rightarrow pr\,toLogic\ tt_{c}\right)\left(refl_{true_{c}:\mathbb{B}_{c}}::_{true_{c}\doteq_{\mathbb{B}_{c}}true_{c}}true_{c}\doteq_{\mathbb{B}_{c}}false_{c}\right)$\tabularnewline
$\rightsquigarrow$ & $\left(refl_{true_{c}:\mathbb{B}_{c}}::_{true_{c}\doteq_{\mathbb{B}_{c}}true_{c}}true_{c}\doteq_{\mathbb{B}_{c}}false_{c}\right)\,toLogic\ tt_{c}$\tabularnewline
\makecell[l]{$true_{c}\doteq_{\mathbb{B}_{c}}false_{c}\coloneqq$\\
  $\ \ \left(C:\left(\mathbb{B}_{c}\rightarrow\star\right)\right)\rightarrow C\,true_{c}\rightarrow C\,false_{c}$} & $\left(refl_{true_{c}:\mathbb{B}_{c}}::_{true_{c}\doteq_{\mathbb{B}_{c}}true_{c}}\left(C:\left(\mathbb{B}_{c}\rightarrow\star\right)\right)\rightarrow C\,true_{c}\rightarrow C\,false_{c}\right)\,toLogic\ tt_{c}$\tabularnewline
\makecell[l]{$true_{c}\doteq_{\mathbb{B}_{c}}true_{c}\coloneqq$\\
  $\ \ \left(C:\left(\mathbb{B}_{c}\rightarrow\star\right)\right)\rightarrow C\,true_{c}\rightarrow C\,true_{c}$} & 
  \makecell[l]{$\left(refl_{true_{c}:\mathbb{B}_{c}}::_{\left(C:\left(\mathbb{B}_{c}\rightarrow\star\right)\right)\rightarrow C\,true_{c}\rightarrow C\,true_{c}}\left(C:\left(\mathbb{B}_{c}\rightarrow\star\right)\right)\rightarrow C\,true_{c}\rightarrow C\,false_{c}\right)$\\
  $\ \ toLogic\ tt_{c}$}\tabularnewline

$\rightsquigarrow\rightsquigarrow_{=}$ & \makecell[l]{$\left(\left(refl_{true_{c}:\mathbb{B}_{c}} toLogic\right)::_{toLogic\ true_{c}\rightarrow toLogic\ true_{c}}toLogic\ true_{c}\rightarrow toLogic\ false_{c}\right)$\\
$\ \ tt_{c}$}\tabularnewline

$\rightsquigarrow\rightsquigarrow_{=}$ & $\left(refl_{true_{c}:\mathbb{B}_{c}}\ toLogic\ tt_{c}\right)::_{toLogic\ true_{c}}toLogic\ false_{c}$
\tabularnewline

$refl_{true_{c}:\mathbb{B}_{c}}\coloneqq\lambda-\,cx\Rightarrow cx$ & 
  $\left( \left(\lambda-\,cx\Rightarrow cx \right) \ toLogic\ tt_{c}\right)::_{toLogic\ true_{c}}toLogic\ false_{c}$\tabularnewline
$\rightsquigarrow\rightsquigarrow$ & $tt_{c}::_{toLogic\ true_{c}}toLogic\ false_{c}$\tabularnewline

$toLogic\coloneqq\lambda b\Rightarrow b\,\star\,Unit_{c}\,\perp_{c}$ & 
  $tt_{c}::_{\left(\lambda b\Rightarrow b\,\star\,Unit_{c}\,\perp_{c}\right)\ true_{c}}\left(\lambda b\Rightarrow b\,\star\,Unit_{c}\,\perp_{c}\right)\ false_{c}$
\tabularnewline
$\rightsquigarrow\rightsquigarrow$ & $tt_{c}::_{true_{c}\,\star\,Unit_{c}\,\perp_{c}}false_{c}\,\star\,Unit_{c}\,\perp_{c}$\tabularnewline

$true_{c}\coloneqq\lambda-\,x\,-\Rightarrow x$ & $tt_{c}::_{\left(\lambda-\,x\,-\Rightarrow x\right)\,\star\,Unit_{c}\,\perp_{c}}false_{c}\,\star\,Unit_{c}\,\perp_{c}$\tabularnewline

$\rightsquigarrow\rightsquigarrow\rightsquigarrow$ & $tt_{c}::_{Unit_{c}}false_{c}\,\star\,Unit_{c}\,\perp_{c}$\tabularnewline

$\rightsquigarrow\rightsquigarrow\rightsquigarrow$ & $tt_{c}::_{Unit_{c}}false_{c}\,\star\,Unit_{c}\,\perp_{c}$\tabularnewline

$false_{c}\coloneqq\lambda-\,-\,y\Rightarrow y$ & $tt_{c}::_{Unit_{c}}\left(\lambda-\,-\,y\right)\,\star\,Unit_{c}\,\perp_{c}$\tabularnewline

$\rightsquigarrow\rightsquigarrow\rightsquigarrow$ & $tt_{c}::_{Unit_{c}}\perp_{c}$\tabularnewline

\multicolumn{2}{l}{using this term will discover the error}\tabularnewline

& $\left(tt_{c}::_{Unit_{c}}\perp_{c}\right)\star$\tabularnewline
$\perp_{c}\coloneqq\left(X:\star\right)\rightarrow X$ & $\left( tt_{c}::_{Unit_{c}}\left(X:\star\right)\rightarrow X\right)  \star$\tabularnewline

$Unit_{c}\coloneqq\left(X:\star\right)\rightarrow X\rightarrow X$ & $\left( tt_{c}::_{\left(X:\star\right)\rightarrow X\rightarrow X}\left(\left(X:\star\right)\rightarrow X\right)\right)  \star$\tabularnewline

$\rightsquigarrow_{=}$ & $\left(tt_{c}\star\right)::_{\star\rightarrow\star}\star$\tabularnewline
Blame! & $\left(tt_{c}\star\right)::_{\star\underline{\rightarrow}\star}\underline{\star}$\tabularnewline

\end{tabular}
\caption{true=false}
\label{fig:cast-ex-tf}
\end{sidewaysfigure}
 
The example is worked in \Fref{cast-ex-tf}.
As in the previous examples, the term $tt_{c}::_{Unit_{c}}\perp_{c}$ has not yet ``gotten stuck''.
Applying any type will uncover the error.

If \Fref{cast-ex-tf} explicitly tracked the location and observation information ane error message could be generated:
 
$\left(C:\left(\mathbb{B}_{c}\rightarrow\star\right)\right)\rightarrow C\,true_{c}\rightarrow\underline{C\,true_{c}}\neq\left(C:\left(\mathbb{B}_{c}\rightarrow\star\right)\right) \rightarrow C\,true_{c}\rightarrow\underline{C\,false_{c}}$
 
when
 
$C\coloneqq\lambda b:\mathbb{B}_{c}.b\,\star\,\star\,\perp$
 
$C\,true_{c}=\perp\neq\star=C\,false_{c}$

\todo{statically enough infor for a wrning}
Reminding the programmer that they should not confuse true with false.
\section{\CSys{}}
 
% why cast soundness?
In a programming language, type soundness proves some undesirable behaviors are unreachable from a well typed term.
% Recall that type soundness is the primary property for a typed programming language to exhibit.
How should this apply to the \clang{}, where bad behaviors are intended to be reachable?
The \clang{} allows the program to be stuck in a bad state, but requires that when that state is reached we have a good explanation to give the programmer that can blame the original faulty type assumption in their source code.
Where the slogan for type soundness is ``well typed terms don't get stuck'', the slogan for cast soundness is ``well cast terms don't get stuck without blame''.

In \ch{2} we proved type soundness for a minimal dependently typed language, with a progress and preservation style proof given a suitable definition of term equivalence.
We can extend that proof to support cast soundness with only a few modifications.
 
\begin{figure}
\[
\frac{x:A\in \Gamma }{\Gamma \vdash x\,:\,A}
\rulename{cast-var}
\]
\[
\frac{\Gamma \vdash a:A\quad \Gamma \vdash A:\star\quad \Gamma \vdash B:\star}{\Gamma \vdash a::_{A,\ell ,o}B\::\:B}
\rulename{cast-::}
\]
 
\[
\frac{\,}{\Gamma \vdash\star:\,\star}
\rulename{cast-\star}
\]
 
\[
\frac{\Gamma \vdash A:\star\quad \Gamma ,x:A\vdash B:\star}{\Gamma \vdash\left(x:A\right)\rightarrow B\,:\,\star}
\rulename{cast-\mathsf{fun}-ty}
\]
 
\[
\frac{\Gamma ,f:\left(x:A\right)\rightarrow B,x:A\vdash b:B}{\Gamma \vdash\mathsf{fun}\,f\,x\Rightarrow b\,:\,\left(x:A\right)\rightarrow B}
\rulename{cast-\mathsf{fun}}
\]
 
\[
\frac{\Gamma \vdash b:\left(x:A\right)\rightarrow B\quad \Gamma \vdash a:A}{\Gamma \vdash b\,a\,:\,B\left[x\coloneqq a\right]}
\rulename{cast-fun-app}
\]
 
\[
\frac{\Gamma \vdash a:A\quad A\equiv A'}{\Gamma \vdash a:A'}
\rulename{cast-conv}
\]
 
\todo[inline]{review regularity stuff}
 
\caption{\CLang{} Type Assignment Rules}
\label{fig:cast-tas-rules}
\end{figure}
 
The \clang{} supports its own type assignment system, defined in \Fref{cast-tas-rules}.
This system ensures that computations will not get stuck without enough information for good runtime error messages.
Specifically computations will not get stuck without a source location and a witness of inequality.
The only rule that works differently than the \slang{} is the \rulename{cast-::} rule that allows runtime type assertions.
 
As before we need a suitable reduction relation to generate our equivalence relation.
\Fref{cast-reduction} shows that system of reductions.
The full rule for function reduction is given in \rulename{\Rrightarrow-\mathsf{fun}-::-red} which makes the behavior from the examples explicit: argument types are swapped as a term is applied under a cast.
Casts from a type universe to a type universe are allowed by the \rulename{\Rrightarrow-::-red} rule.
Since observations embed expressions, they must also be given parallel reductions.
 
\begin{figure}
\[
\frac{b\Rrightarrow b'\quad a\Rrightarrow a'}{\left(\mathsf{fun}\,f\,x\Rightarrow b\right)a\Rrightarrow b'\left[f\coloneqq\mathsf{fun}\,f\,x\Rightarrow b',x\coloneqq a'\right]}
\rulename{\Rrightarrow-\mathsf{fun}-app-red}
\]
 
\[
\frac{b\Rrightarrow b'\quad a\Rrightarrow a'\quad A_{1}\Rrightarrow A_{1}'\quad A_{2}\Rrightarrow A_{2}'\quad B_{1}\Rrightarrow B_{1}'\quad B_{2}\Rrightarrow B_{2}'\quad o\Rrightarrow o'}{\begin{array}{c}
\left(b::_{\left(x:A_{1}\right)\rightarrow B_{1},\ell ,o}\left(x:A_{2}\right)\rightarrow B_{2}\right)a\Rrightarrow\\
\left(b'\,a'::_{A_{2}',\ell,o.Arg}A_{1}'\right)::_{B_{1}'\left[x\coloneqq a'::_{A_{2}',\ell,o'.Arg}A_{1}'\right],\ell ,o'.Bod_{a'} }B_{2}'\left[x\coloneqq a'\right]
\end{array}}
\rulename{\Rrightarrow-fun-::-red}
\]
 
\[
\frac{a\Rrightarrow a'}{a::_{\star,\ell,o}\star\Rrightarrow a'}
\rulename{\Rrightarrow-::-red}
\]

\[
\frac{a\Rrightarrow a'\quad A_{1}\Rrightarrow A_{1}'\quad A_{2}\Rrightarrow A_{2}'\quad o\Rrightarrow o'}{a::_{A_{1},\ell ,o}A_{2}\Rrightarrow a'::_{A_{1}',\ell ,o'}A_{2}'}
\rulename{\Rrightarrow-::}
\]

\begin{align*}
\frac{\,}{x\Rrightarrow x}
\rulename{\Rrightarrow-var} \quad \quad \quad & \quad \quad \quad \frac{\,}{\star\Rrightarrow\star}
\rulename{\Rrightarrow-\star}\\
\end{align*}

\begin{align*}
\frac{A\Rrightarrow A'\quad B\Rrightarrow B'}{\left(x:A\right)\rightarrow B\Rrightarrow\left(x:A'\right)\rightarrow B'}
\rulename{\Rrightarrow-\mathsf{fun}-ty} \quad & \quad \frac{b\Rrightarrow b'}{\mathsf{fun}\,f\,x\Rightarrow b\,\Rrightarrow\,\mathsf{fun}\,f\,x\Rightarrow b'}
\rulename{\Rrightarrow-\mathsf{fun}}\\
\end{align*}

\[
\frac{b\Rrightarrow b'\quad a\Rrightarrow a'}{b\,a\Rrightarrow b'\,a'}
\rulename{\Rrightarrow-\mathsf{fun}-app}
\]
 
\begin{align*}
  \frac{\,}{.\Rrightarrow.}
\rulename{\Rrightarrow-obs-emp} &  \quad  \frac{o\Rrightarrow o'}{o.Arg\Rrightarrow o'.Arg}
\rulename{\Rrightarrow-obs-Arg} \quad  &  \frac{o\Rrightarrow o'\quad a\Rrightarrow a'}{o.Bod_a \Rrightarrow o'.Bod_{a'}}
\rulename{\Rrightarrow-obs-Bod}  \\
\end{align*}

% \[
% \frac{\,}{a\Rrightarrow_{\ast}a}
% \rulename{\Rrightarrow_{\ast}-refl}
% \]
% \[
% \frac{a\Rrightarrow_{\ast}a'\quad a'\Rrightarrow a''}{a\Rrightarrow_{\ast}a''}
% \rulename{\Rrightarrow_{\ast}-trans}
% \]

\[
\frac{a\Rrightarrow_{\ast}a''\quad a'\Rrightarrow_{\ast}a''}{a\equiv a'}
\rulename{\equiv-def}
\]
\todo[inline]{better layout}
\caption{\CLang{} Parallel Reductions}
\label{fig:cast-reduction}
\end{figure}
 
\subsection{Definitional Equality}
 
As in \ch{2}, we will define a suitable notion of definitional equality to derive the other properties of the system.
While it may seem counterintuitive to define a definitional equality in a system that is intended to avoid definitional equality, this is fine since programmers will never interact directly with cast checking.
Programmers will only interact with elaboration, and elaboration will only result in well cast terms.
The \csys{} only exists to give theoretical assurances.
 
As before $\Rrightarrow_{*}$ can be shown to be confluent, and used to generate the equality relation.
The proofs follow the same structure as \ch{2}, but since observations can contain terms, $\Rrightarrow$ and $\textbf{max}$ must be extended to observations.
Proofs must be extended to mutually induct on observations, since they can contain expressions that could also reduce.
 
The explicit new rules for $\textbf{max}$ are given in \Fref{cast-sys-max} with the structural rules omitted since they are the same as \ch{2}.
 
\begin{sidewaysfigure}
\begin{tabular}{cccll}
$\textbf{max}($ & $\left(\mathsf{fun}\,f\,x\Rightarrow b\right)\,a$ & $)=$ & $\textbf{max}\left(b\right)\left[f\coloneqq\mathsf{fun}\,f\,x\Rightarrow \textbf{max}\left(b\right),x\coloneqq \textbf{max}\left(a\right)\right]$ \tabularnewline
$\textbf{max}($ &
$\left(b::_{\left(x:A_{1}\right)\rightarrow B_{1},\ell ,o}\left(x:A_{2}\right)\rightarrow B_{2}\right)a$ & $)=$ &
\makecell[l]{
 $\left(\textbf{max}\left(b\right)\,\left(\textbf{max}\left(a\right)::_{\textbf{max}\left(A_{2}\right),\ell,\textbf{max}\left(o\right).Arg}\textbf{max}\left(A_{1}\right)\right)\right)$ \\
 $::_{\textbf{max}\left(B_{2}\right)\left[x\coloneqq \textbf{max}\left(a\right)::_{\textbf{max}\left(A_{2}\right),\ell,\textbf{max}\left(o\right).Arg}\textbf{max}\left(A_{1}\right)\right],\ell ,\textbf{max}\left(o\right).Bod_{\textbf{max}\left(a\right)}}$ \\
 $\textbf{max}\left(B_{2}\right)\left[x\coloneqq \textbf{max}\left(a\right)\right]$
} & \tabularnewline

$\textbf{max}($ & $b::_{\star,\ell,o}\star$ & $)=$ & $\textbf{max}\left(b\right)$ \tabularnewline
 

& & & otherwise \tabularnewline
$\textbf{max}($ & $b::_{B_{1},\ell ,o}B_{2}$ & $)=$ & $\textbf{max}\left(b\right)::_{\textbf{max}\left(B_{1}\right),\ell ,\textbf{max}\left(o\right)}\textbf{max}\left(B_{2}\right)$ \tabularnewline
 
$\textbf{max}($ & ... & $)=$ & ... % corresponds to the definition in chapter 2
\tabularnewline
$\textbf{max}($ & $.$ & $)=$ & $.$ & \tabularnewline
$\textbf{max}($ & $o.Arg$ & $)=$ & $\textbf{max}\left(o\right).Arg$ \tabularnewline
$\textbf{max}($ & $o.Bod_a$ & $)=$ & $\textbf{max}\left(o\right).Bod_{\textbf{max}\left(a\right)}$ \tabularnewline
\end{tabular}
\todo[inline]{actually have all the rules but gray out the redundant ones}
\caption{$\textbf{max}$}
\label{fig:cast-sys-max}
\end{sidewaysfigure}

The expected lemas hold.
 
\begin{lem} Triangle Properties of $\Rrightarrow$.
 
If $a\Rrightarrow a'$ then $a'\Rrightarrow \maxRed{} \left(a\right)$.

If $o\Rrightarrow o'$ then $o'\Rrightarrow \maxRed{} \left(o\right)$.
\end{lem}
\begin{proof}
By mutual induction on the derivations of $m\Rrightarrow\,m'$ and $o\Rrightarrow o'$.
\end{proof}
\todo{check this}

\begin{lem} Diamond Property of $\Rrightarrow$.
 
If $a\Rrightarrow a'$ and $a\Rrightarrow a''$ implies $a'\Rrightarrow\, \maxRed{} \left(a\right)$ and $a''\Rrightarrow\, \maxRed{} \left(a\right)$.
\end{lem}
\begin{proof}
  This follows directly from the triangle property.
\end{proof}
 
\begin{lem} Confluence of $\Rrightarrow_{\ast}$.
\end{lem}
\begin{proof}
By repeated application of the diamond property.
\end{proof}
   
\subsubsection{$\equiv$ is an equivalence}
As before, this allows us to prove $\equiv$ is transitive, and therefore $\equiv$ is an equivalence.

\begin{thm} $\equiv$ is transitive.
 
If $a\equiv a'$ and $a'\equiv a''$ implies $a\equiv a''$.
\end{thm}
\begin{proof}
The diamond property implies the confluence of $\Rrightarrow_{\ast}$.
\end{proof}
 
\subsubsection{Stability}
Similar to \ch{2} we need to prove that equality is stable over type constructors.
\begin{lem} Stability of $\rightarrow$ over $\Rrightarrow_{\ast}$.
 
$\forall A,B,C.$ $\left(x:A\right)\rightarrow B\Rrightarrow_{\ast}C$ implies $\exists A',B'.C=\left(x:A'\right)\rightarrow B'\land A\Rrightarrow_{\ast}A'\land B\Rrightarrow_{\ast}B'$.
\end{lem}
\begin{proof}
By induction on $\Rrightarrow_{\ast}$.
\end{proof}

\begin{cor} Stability of $\rightarrow$ over $\equiv$.
 
The following rule is admissible:
\[
\frac{\left(x:A\right)\rightarrow B\equiv\left(x:A'\right)\rightarrow B'}{A\equiv A'\quad B\equiv B'}
\]
\end{cor}
 
With these properties proving $\equiv$ is suitable as a definitional equivalence, we can now tackle the progress and preservation lemmas.
 
\subsection{Preservation}
 
As in \ch{2}, $\Rrightarrow$ preserves types.
The argument is similar to that of \ch{2} though more inversion lemmas are needed.
 
\subsubsection{Structural Properties}
We begin by proving the structural properties:
\begin{lem} Context Weakening.
 
The following rule is admissible:
 
\[
\frac{\Gamma \vdash a:A}{\Gamma ,\Gamma' \vdash a:A}
\]
\end{lem}
 
\begin{proof}
By induction on cast derivations.
\end{proof}

\begin{lem} Substitution preserves types.
 
The following rule is admissible:
 
\[
\frac{\Gamma \vdash c:C \quad \Gamma, x:C, \Gamma' \vdash a:A}{\Gamma, \Gamma' \left[x\coloneqq c\right]\vdash a\left[x\coloneqq c\right]:A\left[x\coloneqq c\right]}
\]
\end{lem}
\begin{proof}
By induction on cast derivations.
\end{proof}

As before the notion of definitional equality can be extended to cast contexts. % in \Fref{cast-Context-Equiv}.
 
% \begin{figure}
% \[
% \frac{\ }{\lozenge\equiv\lozenge}\,\rulename{\ensuremath{\equiv}-ctx-empty}
% \]
 
% \[
% \frac{\Gamma \equiv \Gamma' \quad A\equiv A'}{\Gamma, x:A \equiv \Gamma', x:A'}\,\rulename{\ensuremath{\equiv}-ctx-ext}
% \]
% \caption{Contextual Equivalence}
% \label{fig:cast-Context-Equiv}
% \end{figure}
 
\begin{lem}Contexts that are equivalent preserve types.
 
The following rule is admissible:
 
\[
\frac{\Gamma\vdash n:N\quad\Gamma\equiv\Gamma'}{\Gamma'\vdash n:N}
\]
\end{lem}
\begin{proof}
By induction over cast derivations.
\end{proof}

As before we show inversions on the term syntaxes, generalizing the induction hypothesis up to equality when needed.

\begin{lem} $\mathsf{fun}$-Inversion (generalized).
 
\[
\frac{\Gamma \vdash\mathsf{fun}\,f\,x\Rightarrow a\,:\,C\quad C\equiv\left(x:A\right)\rightarrow B}{\Gamma, f:\left(x:A\right)\rightarrow B,x:A\vdash b:B}
\]
\end{lem}
\begin{proof}
by induction on the cast derivations
\end{proof}
 
This allows us to conclude the corollary

\begin{cor} $\mathsf{fun}$-Inversion.
 
\[
\frac{\Gamma \vdash\mathsf{fun}\,f\,x\Rightarrow a\,:\,\left(x:A\right)\rightarrow B}{\Gamma, f:\left(x:A\right)\rightarrow B,x:A\vdash b:B}
\]
\end{cor}

Unlike \ch{2}, we also need an inversion for function types.

\begin{lem} $\rightarrow$-Inversion (generalized).
 
The following rule is admissible
\[
\frac{\Gamma \vdash\left(x:A\right)\rightarrow B\,:\,C\quad C\equiv\star}{\Gamma \vdash A:\star\quad \Gamma, x:A \vdash B:\star}
\]
\end{lem}
\begin{proof}
By induction on the typing derivations
\end{proof}

Which allows the expected corollary:

\begin{cor} $\rightarrow$-Inversion.
\[
\frac{\Gamma \vdash\left(x:A\right)\rightarrow B\,:\,\star}{\Gamma \vdash A:\star\quad \Gamma, x:A\vdash B:\star}
\]
\end{cor}
 
We also need a lemma that will invert the typing information out of the cast operator.
This can be proven directly without generalizing over definitional equality.
 
\todo{fun font after}
\begin{lem} $::$-Inversion.
 
The following rule is admissible:
\[
\frac{\Gamma \vdash a::_{A,\ell ,o}B\::\:C}{\Gamma \vdash a:A\quad \Gamma \vdash A:\star\quad \Gamma \vdash B:\star}
\]
\end{lem}
 
\todo{remove conditions for regularity?}
\begin{proof}
By induction on the typing derivations:
\begin{casenv}
 \item \rulename{cast-::} follows directly.
 \item \rulename{cast-conv} by induction.
 \item All other cases impossible!
\end{casenv}
\end{proof}
 
Note that the derivations of the conclusion of this theorem can alway be made smaller than the derivation from the premise.
This allows other proofs to use induction on the output of this lemma while still being well founded.
 
\begin{thm} $\Rrightarrow$ Preserves types.
 
The following rule is admissible:
\[
\frac{a\Rrightarrow a'\quad \Gamma \vdash a:A}{\Gamma \vdash a':A}
\]
\end{thm}
 
\begin{proof}
By induction on the cast derivation $\Gamma \vdash m:M$, specializing on $m\Rrightarrow m'$:
 
\begin{casenv}
 \item If the term typed with \rulename{cast-::}:
 \begin{casenv}
   \item If the term reduced with \rulename{\Rrightarrow-::-red} then preservation follows by induction.
   \item If the term reduced with \rulename{\Rrightarrow-::} then preservation follows by induction and conversion.
 \end{casenv}
 \item If the term typed with \rulename{cast-fun-app}:
 \begin{casenv}
   \item If the term reduced with \rulename{\Rrightarrow-fun-::-red} then we have
   $\Gamma \vdash\left(b::_{\left(x:A_{1}\right)\rightarrow B_{1},\ell ,o}\left(x:A_{2}\right)\rightarrow B_{2}\right):\left(x:A_{2}\right)\rightarrow B_{2}$,
   $\Gamma \vdash a:A_{2}$,
   $b\Rrightarrow b'$, $a\Rrightarrow a'$, $A_{1}\Rrightarrow A_{1}'$,
   $A_{2}\Rrightarrow A_{2}'$, $B_{1}\Rrightarrow B_{1}'$,  $B_{2}\Rrightarrow B_{2}'$,
   and $o\Rrightarrow o'$.
   We must show $\Gamma \vdash\left(b'\ ac\right)::_{B_{1}'\left[x\coloneqq ac\right],\ell ,o'.Bod_{a'}}B_{2}'\left[x\coloneqq a'\right]$, where $ac=a'::_{A_{2}',\ell,o'.Arg}A_{1}'$.
  
   With cast-inversion we can show $\Gamma \vdash b:\left(x:A_{1}\right)\rightarrow B_{1}$, $\Gamma \vdash\left(x:A_{1}\right)\rightarrow B_{1}:\star$,
   $\Gamma \vdash\left(x:A_{2}\right)\rightarrow B_{2}:\star$.
   Since these derivations are structurally smaller, we can use induction on them.
 
   \begin{tabular}{llll}
     $\Gamma \vdash a':A_{2}'$ & $\rulename{cast-conv}$\tabularnewline
     $\left(x:A_{2}\right)\rightarrow B_{2}\Rrightarrow\left(x:A_{2}'\right)\rightarrow B_{2}'$ & by \rulename{\Rrightarrow-fun-ty}\tabularnewline
     $\Gamma \vdash\left(x:A_{2}'\right)\rightarrow B_{2}':\star$ & \makecell[l]{by induction with \\ $\Gamma \vdash\left(x:A_{2}\right)\rightarrow B_{2}:\star$} \tabularnewline
     $\Gamma \vdash A_{2}':\star$, $\Gamma, x:A_{2}'\vdash B_{2}':\star$ & fun-ty-inversion \tabularnewline
     $\left(x:A_{1}\right)\rightarrow B_{1}\Rrightarrow\left(x:A_{1}'\right)\rightarrow B_{1}'$ & \rulename{\Rrightarrow-fun-ty} \tabularnewline
     $\Gamma \vdash\left(x:A_{1}'\right)\rightarrow B_{1}':\star$ & \makecell[l]{by induction with \\ $\Gamma \vdash\left(x:A_{1}\right)\rightarrow B_{1}:\star$} \tabularnewline
     $\Gamma \vdash A_{1}':\star$, $\Gamma, x:A_{1}'\vdash B_{1}':\star$ & fun-ty-inversion\tabularnewline
     % let $ac=a'::_{A_{2}',\ell,o'.Arg}A_{1}'$ & \tabularnewline
     $\Gamma \vdash ac:A_{1}'$ & by $\rulename{cast-::}$\tabularnewline
     $\Gamma \vdash b':\left(x:A_{1}\right)\rightarrow B_{1}$ & \makecell[l]{by induction with \\ $\Gamma \vdash b:\left(x:A_{1}\right)\rightarrow B_{1}$} \tabularnewline
     $\Gamma \vdash b':\left(x:A_{1}'\right)\rightarrow B_{1}'$ & by \rulename{cast-conv}\tabularnewline
     $\Gamma \vdash b'\ ac:B_{1}'\left[x\coloneqq ac\right]$ & by \rulename{cast-fun-app}\tabularnewline
     $\Gamma \vdash B_{1}'\left[x\coloneqq ac\right]:\star$ & by substitution preservation\tabularnewline
     $\Gamma \vdash B_{2}'\left[x\coloneqq a'\right]:\star$ & by substitution preservation\tabularnewline
     % \makecell[l]{$\Gamma \vdash$ \\ $\left(b'\ ac\right)::_{B_{1}'\left[x\coloneqq ac\right],\ell ,o'.Bod_{a'}}B_{2}'\left[x\coloneqq a'\right]$} & by \rulename{cast-::} \tabularnewline
     \end{tabular}
     Which allows us to conclude $\Gamma \vdash\left(b'\ ac\right)::_{B_{1}'\left[x\coloneqq ac\right],\ell ,o'.Bod_{a'}}B_{2}'\left[x\coloneqq a'\right]$ by \rulename{cast-::}.
 
   \item All other reductions are similar to \ch{2}.
 \end{casenv}
 \item All other cases follow along the lines of \ch{2}.
\end{casenv}
\end{proof}
 
\todo{highlight the -fun-::-red case}
 
\subsubsection{Progress}
 
Preservation alone isn't sufficient for a cast sound language.
We also need to show that there is an evaluation that behaves appropriately in an empty typing context.
Again this will broadly follow the outline of the \slang{} proof in \ch{2}, with a few substantial changes.
 
Unlike the \slang{}, it is no longer practical to characterize values syntactically.
Values are specified by judgments in \Fref{cast-val}.
They are standard except for the \rulename{Val-::}, which states that a type ($\star$ or function type) under a cast is not a value.
 
\begin{figure}
\[
\frac{\,}{\star\,\textbf{Val}}
\rulename{Val-\star}
\]
\[
\frac{\,}{\left(x:A\right)\rightarrow B\,\textbf{Val}}
\rulename{Val-fun-ty}
\]
\[
\frac{\,}{\mathsf{fun}\,f\,x\Rightarrow b\:\textbf{Val}}
\rulename{Val-fun}
\]
\[
\frac{\begin{array}{c}
a\:\textbf{Val}\quad A\:\textbf{Val}\quad B\:\textbf{Val}\\
a\cancel{=}\star\\
a\cancel{=}\left(x:C\right)\rightarrow C'
\end{array}}{a::_{A,\ell, o}B\:\textbf{Val}}
\rulename{Val-::}
\]
\caption{\CLang{} Values}
\label{fig:cast-val}
\end{figure}
 
Small steps are listed in \Fref{cast-step}.
They are standard for \cbv{} except that casts can distribute over application, and casts can reduce when both types are $\star$.
 
\begin{figure}
\[
\frac{a\,\textbf{Val}}{\left(\mathsf{fun}\,f\,x\Rightarrow b\right)a\rightsquigarrow b\left[f\coloneqq\mathsf{fun}\,f\,x\Rightarrow b,x\coloneqq a\right]}
\]
\[
\frac{b\,\textbf{Val}\quad a\,\textbf{Val}}{\begin{array}{c}
\left(b::_{\left(x:A_{1}\right)\rightarrow B_{1},\ell ,o}\left(x:A_{2}\right)\rightarrow B_{2}\right)a\rightsquigarrow\\
\left(b\,a::_{A_{2},\ell,o.Arg}A_{1}\right)::_{B_{1}\left[x\coloneqq a::_{A_{2},\ell,o.Arg}A_{1}\right],\ell ,o.Bod_a}B_{2}\left[x\coloneqq a\right]
\end{array}}
\]
\[
\frac{a\,\textbf{Val}}{a::_{\star,\ell ,o}\star\rightsquigarrow a}
\]
\[
\frac{a\rightsquigarrow a'}{a::_{A,\ell ,o}B\rightsquigarrow a'::_{A,\ell ,o}B}
\]
\[
\frac{a\,\textbf{Val}\quad A\rightsquigarrow A'}{a::_{A,\ell ,o}B\rightsquigarrow a::_{A',\ell ,o}B}
\]
\[
\frac{a\,\textbf{Val}\quad A\,\textbf{Val}\quad B\rightsquigarrow B'}{a::_{A,\ell ,o}B\rightsquigarrow a::_{A,\ell ,o}B'}
\]
\[
\frac{b\rightsquigarrow b'}{b\,a\rightsquigarrow b'\,a}
\]
\[
\frac{b\,\textbf{Val}\quad a\rightsquigarrow a'}{b\,a\rightsquigarrow b\,a'}
\]
 
\todo[inline]{name rules}
\caption{\CLang{} Small Step}
\label{fig:cast-step}
\end{figure}
 
%% walk through blame
In addition to small step and values we also specify blame judgments in \Fref{cast-blame}.
Blame tracks the information needed to create a good error message and is inspired by the many systems that use blame tracking \cite{10.1145/581478.581484,10.1007/978-3-642-00590-9_1,wadler:LIPIcs:2015:5033}.
Specifically the judgment \blame{a}{\ell}{o} means that $a$ witnesses a contradiction in the source code at location $\ell$ under the observations $o$.
With only dependent functions and universes, only inequalities of the form $*\,\cancel{=}\,A\rightarrow B$ can be witnessed.
The first 2 rules of the blame judgment witness these concrete type inequalities.
The rest of the blame rules will recursively extract concrete witnesses from larger terms.
Limiting the observations to the form $*\,\cancel{=}\,A\rightarrow B$ makes the system in this Chapter simpler than the system in \ch{5} where more observations are possible because of the addition of data.
% "self evidently correct" since it is clear to only extract inequalities. like step and val? one of several possible contradiction extraction relations
 
\begin{figure}
\[
\frac{\,}{\blame{\left(a::_{\left(x:A\right)\rightarrow B,\ell ,o}\star\right)}{\ell}{o}}
\]
\[
\frac{\,}{\blame{\left(a::_{\star,\ell ,o}\left(x:A\right)\rightarrow B\right)}{\ell}{o}}
\]
\[
\frac{
 \blame{a}{\ell}{o}
}{
 \blame{\left(a::_{A,\ell',o'}B\right)}{\ell}{o}
}
\]
\[
\frac{
 \blame{a}{\ell}{o}
}{
 \blame{\left(a::_{A,\ell',o'}B\right)}{\ell}{o}
}
\]
\[
\frac{
 \blame{B}{\ell}{o}
}{
 \blame{\left(a::_{A,\ell',o'}B\right)}{\ell}{o}
}
\]
\[
\frac{
 \blame{b}{\ell}{o}
}{
 \blame{\left(b\,a\right)}{\ell}{o}
}
\]
\[
\frac{
 \blame{a}{\ell}{o}
}{
 \blame{\left(b\,a\right)}{\ell}{o}
}
\]
\todo[inline]{name rules}
\caption{\CLang{} Blame}
\label{fig:cast-blame}
\end{figure}
 
As before we have that $\rightsquigarrow$ preserves types.
 
\begin{fact} $\rightsquigarrow$ preserves types.
 
Since the following rule is admissible:
\[
\frac{m\rightsquigarrow m'}{m\Rrightarrow m'}
\]
\end{fact}
 
As in \ch{2} we will need technical lemmas that determine the syntax of a value of a given type in the empty context.
However, canonical function values look different because they must account for the possibility of blame arising from a stuck term.

\begin{lem} $\star$-Canonical forms (generalized).
 
If $\vdash a\,:\,A$ , $a\,\textbf{Val}$, and $A\equiv\star$ then either

\textup{$a=\star$ ,}

\textup{or there exists $C$, $B$, such that $a=\left(x:C\right)\rightarrow B$.}
\end{lem}
\todo{alternatively produce the blame judgment}
\begin{proof}
By induction on the cast derivation:
\begin{casenv}
 \item \rulename{cast-\star} and \rulename{cast-fun-ty} follow directly.
 \item \rulename{ty-conv} follows by induction and that $\equiv$ is an equivalence.
 \item \rulename{cast-fun} is impossible since $\left(x:A\right)\rightarrow B\cancel{\equiv}\star$!
 \item \rulename{cast-::} is impossible!
   Inductively the underlying term must be $\star$, or $\left(x:C\right)\rightarrow B$.
   Which contradicts the side conditions of \rulename{Val-::}.
 \item Other rules are impossible, since they do not type values in an empty context!
\end{casenv}
\end{proof}

Leading to the corollary:

\begin{cor} $\star$-Canonical forms.
 
If $\vdash A:\star$, and $A\,\textbf{Val}$ then either

\textup{$A=\star$ , }

\textup{or there exists $C$, $B$, such that $A=\left(x:C\right)\rightarrow B$.}
\end{cor}
 
Likewise:

\begin{lem} $\rightarrow$-Canonical forms (generalized).
 
If $\vdash a\,:\,A$ , $a\,\textbf{Val}$, and $A\equiv\left(x:C\right)\rightarrow B$ then either
 
$a=\mathsf{fun}\,f\,x\Rightarrow b$
 
or $a=d::_{D,\ell ,o}\left(x:C'\right)\rightarrow B'$, $d\,\textbf{Val}$, $D\,\textbf{Val}$, $C'\equiv C$, $B'\equiv B$
\end{lem}
\todo{alternatively produce the blame judgment}
\begin{proof}
By induction on the cast derivation:
\begin{casenv}
 \item \rulename{cast-fun} follows directly.
 \item \rulename{cast-::} then it must be a value from \rulename{Val-::} satisfying the 2nd conclusion.
 \item \rulename{ty-conv} follows by induction and the transitivity of $\equiv$.
 
 \item \rulename{cast-\star} and \rulename{cast-fun-ty} are impossible by the stability of $\equiv$!
 \item Other rules are impossible, since they do not type values in an empty context!
\end{casenv}
\end{proof}

As a corollary:

\begin{cor} $\rightarrow$-Canonical forms.
 
If $\vdash a:\left(x:C\right)\rightarrow B$ , and $a\,\textbf{Val}$
 
$a=\mathsf{fun}\,f\,x\Rightarrow b$
 
or $a=d::_{D,\ell ,o}\left(x:C'\right)\rightarrow B'$, $d\,\textbf{Val}$, $D\,\textbf{Val}$, $C'\equiv C$, $B'\equiv B$
\end{cor}
 
This further means if $\vdash a:\left(x:C\right)\rightarrow B$, and $a\,\textbf{Val}$ then $a$ is not a type, $a\cancel{=}\star,a\cancel{=}\left(x:C\right)\rightarrow C'$.
 
We can now prove the progress lemma.
 
\begin{thm} Progress.
 
If $\vdash a\,:\,A$ then either
 
$a\,\textbf{Val}$
 
there exists $a'$ such that $a\rightsquigarrow a'$
 
or there exists $\ell$, $o$ such that $\blame{a}{\ell}{o}$
\end{thm}
\begin{proof}
As usual this follows form induction on the typing derivation:
\begin{casenv}
 \item \rulename{cast-\star}, \rulename{cast-fun-ty}, and \rulename{cast-fun} follow directly.
 \item \rulename{cast-var}, impossible in an empty context!
 \item \rulename{cast-conv}, by induction.
 \item \rulename{cast-::}, then the term is $a::_{A,\ell ,o}B\::\:B$.
 Each of $a$, $A$, and $B$ can be blamed, can step, or is a value (By induction):
 \begin{casenv}
   \item If any of $a$, $A$, or $B$, step then the entire term can step.
   \item If any of $a$, $A$, or $B$, can be blamed then the entire term can be blamed.
   \item If all of $a$, $A$, and $B$, are values then $A$ and $B$ are types of canonical form.
   \begin{casenv}
     \item If both $A$, and $B$, is $\star$ then step.
     \item If one of $A$, and $B$, is $\star$ and the other is $\left(x:C_{A}\right)\rightarrow D_{A}$ then blame.
    
     \item Otherwise the term is a value by the canonical forms.
   \end{casenv}
 \end{casenv}
 \item \rulename{cast-fun-app}, then the term is $(b\,a)$.
  Each of $a$, and $b$ can be blamed, can step, or is a value (By induction):
  \begin{casenv}
    \item If any of $b$, or $a$, step then the entire term can step
    \item If any of $b$, or $a$, can be blamed then the entire term can be blamed
    \item If $b\,\textbf{Val}$, $a\,\textbf{Val}$ by canonical forms:
    \begin{casenv}
      \item $b=\mathsf{fun}\,f\,x\Rightarrow c$ the term steps.
      \item $b=d_{b}::_{D_{b},\ell _{b}\,o_{b}}\left(x:A'\right)\rightarrow B'$ we have by canonical forms and induction that:
      \begin{casenv}
        \item $D_{b}=\star$ and blame can be generated.
        \item Otherwise $D_{b}=\left(x:A_{D_{b}}\right)\rightarrow B_{D_{b}}$ and a step is possible.
      \end{casenv}
    \end{casenv}
  \end{casenv}
\end{casenv}
\end{proof}
 
\subsection{Cast Soundness}
 
\todo{revise}
 
%% cast soundness
Cast soundness follows from progress and preservation as expected.
 
% For any $\lozenge\vdash c:C$, $c'$, $c\rightsquigarrow^{*}c'$, if $\textbf{Stuck}\,c'$ then $\textbf{Blame}\:\ell \,o\:c'$, where $\textbf{Stuck}\,c'$ means $c'$ is not a value and $c'$ does not step.
% This follows by iteration the progress and preservation lemmas.
 
\subsection{Discussion}
 
\todo{inferablility of inner cast}
 
\todo{regularity encoding, along with the cast}
 
Because of the conversion rule and non-termination, cast-checking is undecidable.
This is fine since the \csys{} only exists to ensure theoretical properties. 
Programmers will only use the system through the elaboration procedure described in the next section.
Every term produced by elaboration will cast check, and the elaboration is decidable. % \footnote{As fomalized here, the assertimplemenation }.
 
As in the \slang{}, the \clang{} is logically unsound, by design.
 
Just as there are many different flavors of definitional equality that could have been used in \ch{2}, there are also many possible degrees that runtime equality can be enforced.
The \Blame{} relation in \Fref{cast-blame} outlines a minimal checking strategy that supports cast soundness.
For instance\footnote{
  Assuming the data types of \ch{5}.
}, $\mathtt{head}\,\Bool{}\,1\,\left(\mathtt{rep}\,\Bool{}\,\True{}\,0\right)$ will result in blame since $1$ and $0$ have different head constructors.
But $\mathtt{head}\,\Bool{}\,1\,\left(\mathtt{rep}\,\Bool{}\,\True{}\,9\right)$ will not result in blame since $1$ and $9$ have the same head constructor and the computation can reduce to \True{}.
 
It is likely that more aggressive checking is preferable in practice, especially in the presence of data types.
That is why our implementation checks equalities up to binders. 
This corresponds better to the \cbv{} behavior of the implemented interpreter.
For this reason we call this strategy \textbf{check-by-value}.
 
This behavior is consistent with the conjectured partial correctness of logically unsound \cbv{} execution for dependent types in \cite{jia2010dependent}.
 
\todo{move?}
 
Unlike static type-checking, these runtime checks have runtime costs.
Since the language allows nontermination, checks can take forever to resolve at runtime.
We don't expect this to be a large issue in practice, at least any more than is usual in mainstream languages that allow many other sources of nontermination are possible.
The implementation avoids casts when it knows that blame is impossible.
Additionally, we could limit the number of steps allowed in cast normalization and blame slow code and nonterminating.
\section{Elaboration}

\todo[inline]{resolve var for the typing context of this section: H? Gamma? some other greek?}

% overview
Even though the \clang{} allows us to optimistically assert equalities, manually noting every cast would be cumbersome.
This bureaucracy is solved with an elaboration procedure that translates (untyped) terms from the \slang{} into the \clang{}.
If the term is well typed in the \slang{}, elaboration will produce a term without blamable errors.
Terms with unproven equality in types are mapped to a cast with enough information to point out the original source when an inequality is witnessed.
 
Elaboration serves a similar role as the \bidir{} type system did in \ch{2}, and uses a similar methodology.
Instead of performing a static equality check when the inference mode and the check mode meet, a runtime cast is inserted asserting the types are equal.

In order to perform elaboration, the \slang{} needs to be enriched with location information, $\ell$, at every position that could result in a type mismatch.
This is done in \Fref{surface-pre-syntax-loc}.
Note that the location tags correspond with the check annotations of the \bidir{} system.
For technical reasons the set of locations is nonempty, and a specific null location ($.$)\todo{- instead of . ?} is designated.
That null location can be used when we need to generate fresh terms, but have no sensible location information available.
All the meta theory from \ch{2} goes through assuming that all locations are indistinguishable and by generating null locations when needed\footnote{
  For instance, the parallel reduction relation will associate all locations,
    $\frac{M\Rrightarrow M'\quad N\Rrightarrow N'}{\left(x:M_{l}\right)\rightarrow N_{l'}\Rrightarrow\left(x:M'_{l''}\right)\rightarrow N'_{l'''}}\,\rulename{\Rrightarrow-fun-ty}$,
    so that the relation does not discriminate over syntaxes that come from different locations.
  While the $\textbf{max}$ function will map terms into the null location,
    $\textbf{max}\left(\left(x:M_{\ell}\right)\rightarrow N_{\ell'}\right)=\textbf{max}\left(\left(x:\textbf{max}\left(M\right)_{.}\right)\rightarrow \textbf{max}\left(N\right)_{.}\right)$.
}.\todo{more explanation on this note? why set max up that way?}
We will avoid writing these annotations when they are unneeded (explicitly in \Fref{surface-pre-syntax-loc-abrev}).

\begin{figure}
\begin{tabular}{lcll}
\multicolumn{4}{l}{source labels,}\tabularnewline
$\ell$ & $\Coloneqq$ & ... & \tabularnewline
& $|$ & $.$ & no source label\tabularnewline
\multicolumn{4}{l}{expressions,}\tabularnewline
$m,n,M,N$ & $\Coloneqq$ & $x$ & variable\tabularnewline
& $|$ & $m::_{\ell}M^{\ell'}$ & annotation\tabularnewline
& $|$ & $\star$ & type universe\tabularnewline
& $|$ & $\left(x:M_{\ell}\right)\rightarrow N_{\ell'}$ & function type\tabularnewline
& $|$ & $\mathsf{fun}\,f\,x\Rightarrow m$ & function\tabularnewline
& $|$ & $m_{\ell}\,n$ & application\tabularnewline
\end{tabular}\caption{\SLang{} Syntax with Locations}
\label{fig:surface-pre-syntax-loc}
\end{figure}

\begin{figure}
\begin{tabular}{lclll}
$m::_{\ell}M^{\ell'}$ & written & $m::_{\ell}M$ & when & $\ell$' is irrelevant\tabularnewline
$m::_{\ell}M$ & written & $m::M$ & when & $\ell$ is irrelevant\tabularnewline
$\left(x:M_{\ell}\right)\rightarrow N_{\ell'}$ & written & $\left(x:M\right)\rightarrow N$ & when & $\ell$, $\ell'$ are irrelevant\tabularnewline
$m_{\ell}\,n$ & written & $m_{\ensuremath{}}\,n$ & when & $\ell$ is irrelevant\tabularnewline
\end{tabular}

\caption{\SLang{} Abbreviations}
\label{fig:surface-pre-syntax-loc-abrev}
\end{figure}

\subsection{Examples}

Functions will elaborate the expected types to their arguments when they are applied.
% For example,
\begin{example}
$f:\mathbb{B}_c \rightarrow\mathbb{B}_c \vdash f_{\ell}7_c \ :\mathbb{B}_c $ elaborates to $f:\mathbb{B}_c \rightarrow\mathbb{B}_c \vdash f\left(7_c ::_{\mathbb{N}_c ,\ell,.Arg}\mathbb{B}_c \right)\ :\mathbb{B}_c $.
\end{example}
 
\todo{arg is there because f may not be a function type}
 
As with \bidir{} type checking, variable types will be inferred from the typing environment.
% For example,
\begin{example}
$\vdash(\lambda x\Rightarrow 7_c)::_{\ell}\mathbb{B}_c \rightarrow\mathbb{B}_c$ elaborates to $\vdash(\lambda x\Rightarrow7::_{\mathbb{N}_c,\ell,Bod_x}\mathbb{B}_c )$.
\end{example}
\todo{these examples aer only precisly correct if the church terms infer their type}

To keep the theory simple, we allow vacuous casts to be created,
\begin{example}
$f:\mathbb{N}_c \rightarrow\mathbb{B}_c \rightarrow\mathbb{B}_c \vdash f_{\ell}7_{c\ell'}3_c \ :\mathbb{B}_c $ elaborates to $f:\mathbb{N}_c \rightarrow\mathbb{B}_c \rightarrow\mathbb{B}_c \vdash f\left(7_c ::_{\mathbb{N}_c,\ell,Arg}\mathbb{N}_c \right)\left(3_c ::_{\mathbb{N}_c,\ell',Arg}\mathbb{B}_c \right)\ :\mathbb{B}_c$.
\end{example}

\todo{dependent type example where cast muddles type}

Unlike in gradual typing, we cannot elaborate arbitrary untyped syntax.
The underlying type of a cast needs to be known so that a function type can swap its argument type at application.
For instance, $\lambda x\Rightarrow x$ will not elaborate since the intended type is not known.
Fortunately, our experimental testing suggests that a majority of randomly generated terms can be elaborated, compared to the \slang{} where only a small minority of terms would type check.
The programmer can make any term elaborate if they annotate the intended type.
For instance, $\left(\lambda x\Rightarrow x\right)::*\rightarrow*$ will elaborate.

\subsection{Elaboration Procedure}
\todo{review location placment in light of the implementation}

% rules
Like the \bidir{} rules, the rules for elaboration are broken into two judgments:
\begin{itemize}
\item $H\vdash m\overleftarrow{\,:_{\ell,o}\,}A\,\textbf{Elab}\ a$\todo{fix spacing to right of : under arrow}, that generates a cast term $a$ from a surface term $m$ given its expected type $A$ along with a location $\ell$ and observation $o$ that made that assertion.
\item $H\vdash m\,\textbf{Elab}\ a\overrightarrow{\,:\,}A$, that generates a cast term $a$ and its type $A$ from a surface term $m$.
\end{itemize}
The rules for elaboration are presented in \Fref{elaboration}.
Elaboration rules are written in a style of \bidir{} type checking, with arrows pointing in the direction information flows.
However, unlike \bidir{} type checking, when checking an inference in \rulename{\overleftarrow{\textbf{Elab}}-cast}, elaboration adds a cast assertion that the two types are equal.
Thus any conversion checking can be suspended until runtime.
Additionally we will allow the mode to change at the type universe with the \rulename{\overleftarrow{\textbf{Elab}}-conv-\star} rule, to avoid unneeded checks on the type universe.\todo{i belive this rule is no longer needed? but it is convient}
% Without the \rulename{\overleftarrow{\textbf{Elab}}-conv-\star} rule, the \slang{} term $1_c :: \mathbb{N}_c$ could not elaborate becouse runtime checks required to cherck teh well formedness of $\mathbb{N}_c$ would result in a term $\mathbb{N}_c :: \star$.
As formulated here, the elaboration procedure is terminating.

\begin{figure}
\[
\frac{
  x:A\in H
}{
  H\vdash x\,\textbf{Elab}\,x\overrightarrow{\,:\,}A
}
\rulename{\overrightarrow{\textbf{Elab}}-var}
\]

\[
\frac{\,}{H\vdash\star\,\textbf{Elab}\,\star\overrightarrow{\,:\,}\star}
\rulename{\overrightarrow{\textbf{Elab}}-\star}
\]

\[
\frac{
  H\vdash M\overleftarrow{\,:_{\ell,.}\,}\star\textbf{Elab}\ A\quad H,x:A\vdash N\overleftarrow{\,:_{\ell',.}\,}\star\textbf{Elab}\ B
}{
  H\vdash\left(\left(x:M_{\ell}\right)\rightarrow N_{\ell'}\right)\textbf{Elab}\left(\left(x:A\right)\rightarrow B\right)\overrightarrow{\,:\,}\star
}
\rulename{\overrightarrow{\textbf{Elab}}-fun-ty}
\]

\[
\frac{H\vdash m\,\textbf{Elab}\ b\overrightarrow{\,:\,}\left(x:A\right)\rightarrow B\quad H\vdash n\overleftarrow{\,:_{\ell,Arg}\,}A\,\textbf{Elab}\,a}{H\vdash\left(m_{\ell}\,n\right)\textbf{Elab}\left(b\,a\right)\overrightarrow{\,:\,}B\left[x\coloneqq a\right]}\rulename{\overrightarrow{\textbf{Elab}}-fun-app}
\]

\[
\frac{H\vdash M\overleftarrow{\,:_{\ell',.}\,}\star\,\textbf{Elab}\ A\quad H\vdash m\overleftarrow{\,:_{\ell,.}\,}A\,\textbf{Elab}\ a}{H\vdash\left(m::_{\ell}M^{\ell'}\right)\textbf{Elab}\,a\overrightarrow{\,:\,}A}
\rulename{\overrightarrow{\textbf{Elab}}-::}
\]

\[
\frac{H,f:\left(x:A\right)\rightarrow B,x:A\vdash m\overleftarrow{\,:_{\ell,o.Bod_x}\,}B\,\textbf{Elab}\ b}{H\vdash\left(\mathsf{fun}\,f\,x\Rightarrow m\right)\overleftarrow{\,:_{\ell,o}\,}\left(x:A\right)\rightarrow B\,\textbf{Elab}\left(\mathsf{fun}\,f\,x\Rightarrow b\right)}\rulename{\overleftarrow{\textbf{Elab}}-fun}
\]

\[
\frac{H\vdash m\,\textbf{Elab}\ a\overrightarrow{\,:\,}A}{H\vdash m\overleftarrow{\,:_{\ell,o}\,}B\ \textbf{Elab}\left(a::_{A,\ell,o}B\right)}\rulename{\overleftarrow{\textbf{Elab}}-cast}
\]

\[
\frac{H\vdash m\,\textbf{Elab}\ a\overrightarrow{\,:\,}\star}{H\vdash m\overleftarrow{\,:_{\ell,o}\,}\star\ \textbf{Elab}\,a}
\rulename{\overleftarrow{\textbf{Elab}}-conv-\star}
\]

\todo[inline]{which syntax looks the best? on the left when input, or alway on the right like the typing judgment}
\todo[inline]{macro this stntax}

\caption{Elaboration}
\label{fig:elaboration}
\end{figure}



There are several desirable properties of elaboration that can be shown with the help of an erasure function (defined in \ref{fig:erasure}).
Erasure is defined over all syntactic forms, removing annotations, locations, and casts.

\begin{figure}
\begin{tabular}{ccc}
$|x|$ & = & $x$\tabularnewline
$|\star|$ & = & $\star$\tabularnewline
$|m::_{\ell}M|$ & = & $|m|$\tabularnewline
$|\left(x:M_{\ell}\right)\rightarrow N_{\ell'}|$ & = & $\left(x:|M|\right)\rightarrow|N|$\tabularnewline
$|m_{\ell}\,n|$ & = & $|m|\,|n|$\tabularnewline
$|\mathsf{fun}\,f\,x\Rightarrow m|$ & = & $\mathsf{fun}\,f\,x\Rightarrow|m|$\tabularnewline
$|\lozenge|$ & = & $\lozenge$\tabularnewline
$|\Gamma,x:A|$ & = & $|\Gamma|,x:|A|$\tabularnewline
$|a::_{A,\ell,o}B|$ & = & $|a|$\tabularnewline
$|\left(x:A\right)\rightarrow B|$ & = & $\left(x:|A|\right)\rightarrow|B|$\tabularnewline
$|\mathsf{fun}\,f\,x\Rightarrow b|$ & = & $\mathsf{fun}\,f\,x\Rightarrow|b|$\tabularnewline
$|b\,a|$ & = & $|b|\,|a|$\tabularnewline
$|H,x:M|$ & = & $|H|,x:|M|$\tabularnewline
\end{tabular}
\caption{Erasure}
\label{fig:erasure}
\end{figure}

\begin{thm} Elaborated terms preserve erasure.
 
If $H\vdash m\,\textbf{Elab}\ a\overrightarrow{\,:\,}A$ then $|m|=|a|$.
 
If $H\vdash m\,a\overleftarrow{\,:_{\ell,o}\,}A\,\textbf{Elab}\,a$ then $|m|=|a|$.
\end{thm}
\begin{proof}
By mutual induction on the $\textbf{Elab}$ derivations.
\end{proof}

It follows that whenever an elaborated cast term evaluates, the corresponding surface term evaluates consistently.
Explicitly,
\begin{thm} \Slang{} and \clang{} have consistent evaluation.
 
If $H\vdash m\,\textbf{Elab}\ a\overrightarrow{\,:\,}A$, and $a\rightsquigarrow_{*}\star$ then $m\rightsquigarrow_{*}\star$.
 
If $H\vdash m\overleftarrow{\,:_{\ell,o}\,}A\,\textbf{Elab}\ a$, and $a\rightsquigarrow_{*}(x:A)\rightarrow B$ then there exists $N$ and $M$ such that $m\rightsquigarrow_{*}(x:N)\rightarrow M$.
\end{thm}

\begin{proof}
Since $a\rightsquigarrow_{*}a'$ implies $|a|\rightsquigarrow_{*}|a'|$ and $m\rightsquigarrow_{*}m'$ implies $|m|\rightsquigarrow_{*}|m'|$.
\end{proof}

Elaborated terms are well-cast in a well formed context.
We will use $H\ \textbf{ok}$ to mean for all $x$, $x : A \in H$ then $H \vdash A : \star$.

\begin{thm} Elaborated terms are well-cast.
 
For any $H\,\textbf{ok}$, $H\vdash a\,\textbf{Elab}\,m\overrightarrow{\,:\,}A$ then $H\vdash a:A$, $H\vdash A:\star$.
 
For any $H\,\textbf{ok}$, $H\vdash A:\star$, $H\vdash m\overleftarrow{\,:_{\ell,o}\,}A\,\textbf{Elab}\,a$ then $H\vdash a:A$.

For any $H\,\textbf{ok}$, $H\vdash M\overleftarrow{\,:_{\ell,o}\,}\star\,\textbf{Elab}\,A$ then $H\vdash A:\star$.
\end{thm}
\begin{proof}
By mutual induction on $\textbf{Elab}$ derivations.
\todo{double check}
\end{proof}

Some additional properties are conjectured to hold, though they have not yet been proven.

\begin{conjecture}
Every term well typed in the \bidir{} \slang{} elaborates.
 
If $\Gamma\vdash$, then there exists $H$ such that $\Gamma\,\textbf{Elab}\,H$
 
$\Gamma\vdash m\overrightarrow{\,:\,}M$ then there exists $H$, $a$ and $A$ such that $\Gamma\,\textbf{Elab}\,H$, $H\vdash m\,\textbf{Elab}\ a\overrightarrow{\,:\,}A$
 
$\Gamma\vdash m\overleftarrow{\,:\,}M$ and given $\ell$, $o$ then there exists $H$, $a$and $A$ such that $\Gamma\,\textbf{Elab}\,H$, $H\vdash\textbf{Elab}\,a\,m\overleftarrow{\,:_{\ell o}\,}A$
\end{conjecture}
 
Which if true would lead to the corollary
\begin{conjecture}
Blame never points to something that checked in the \bidir{} system.
 
If $\vdash m\overrightarrow{\,:\,}M$, and $\vdash\textbf{Elab}\ m\,a\overrightarrow{\,:\,}A$, then for no $a\rightsquigarrow_{*}a'$ will \blame{a'}{\ell}{o} occur.
 
\todo{revise precisely with labels}
\end{conjecture}
 
These properties are inspired by the gradual guarantee\cite{siek_et_al:LIPIcs:2015:5031} for gradual typing.
\section{Suitable Warnings}
 
As presented here, not every cast corresponds to a reasonable warning.
For instance, $\left(\lambda x\Rightarrow x\right)::_{\star\rightarrow\star}\star\rightarrow\star$ is a possible output from elaboration.
By the rules given the cast will not reduce without input, it will never cause blame.
In fact since the user only interacts with the \slang{}, any cast $a::_{A}B$ where $|A|\equiv|B|$ will not produce an understandable warning.
% A reasonable first attempt would be to simply remove the casts of the form $a::_{A}A$, but this ignores the possibility that casts themselves may contain casts\todo{example}.
% Currently the implementation leaves most casts intact and filters our equivalent casts from the warnings shown to the user
 
% \todo{as fig}
 
% \begin{tabular}{llllll}
% $Warns($ & $a::_{A,\ensuremath{\ell},o}B$ & $)=$ & $\left\{ (a,\ensuremath{\ell},o,B)\right\} \cup Warns(a)\cup Warns(A)\cup Warns(B)$ & if & $|A|\cancel{\equiv}|B|$\tabularnewline
% $Warns($ & $a::_{A,\ensuremath{\ell},o}B$ & $)=$ & $Warns(a)\cup Warns(A)\cup Warns(B)$ & if & $|A|\equiv|B|$\tabularnewline
% $Warns($ & $\star$ & $)=$ & $\emptyset$ &  & \tabularnewline
% $Warns($ & $x$ & $)=$ & $\emptyset$ &  & \tabularnewline
% $Warns($ & $\left(x:A\right)\rightarrow B$ & $)=$ & $Warns(A)\cup Warns(B)$ &  & \tabularnewline
% $Warns($ & $\mathsf{fun}\,f\,x\Rightarrow b$ & $)=$ & $Warns(b)$ &  & \tabularnewline
% $Warns($ & $b\,a$ & $)=$ & $Warns(a)\cup Warns(b)$ &  & \tabularnewline
% \end{tabular}
 
% Since the $\equiv$ relation is undecidable an approximation can be used in practice.
% Removing impossible casts should be considered like a compiler optimization.
In \ch{5} casts will be separated from the assertions that they contain, and it will be more clear how to extract warnings.

\section{Related Work}
 
% \subsection{Dependent types and equality}
 
% \todo{revise as bulleted list starting with ETT}
 
% Difficulties in dependently typed equality have motivated many research projects \cite{HoTTbook,sjoberg2015programming,cockx2021taming}.
% However, these impressive efforts currently require a high level of expertise from programmers.
% Further, since program equivalence is undecidable in general, no system will be able to statically verify every ``obvious'' equality for arbitrary user defined data types and functions.
% In the meantime systems should trust the programmer when they use an unverified equality, and use that advanced research to suppress warnings.
 
\subsection{\Bidir{} Placement of Casts}
\todo{better title}
 
This is not the first work to use \bidir{} type checking to place errors.
The Haskell compiler, GHC, supplements Hindley-Minler style type checking with bidirectionality to localize error messages.
This approach was extended in \cite{10.1145/2364527.2364554} which weakens the regular type checking to allow runtime casts\footnote{
 Available with the $\mathtt{-fdefer-type-errors}$ compiler flag.}.
The casts themselves are different from the ones described here since they do not optimistically compute, they will only give errors when reached.
Though more restrictive than our casts, that system enforces parametricity, which makes sense in the context of Haskell. 
% The only other work we are aware of

\todo{talk about the java paper it cites}
 
\todo[inline]{talk about dependent haskell?}

\subsection{Contract Systems}
 
Several of the tricks and notations in this Chapter find their basis in the large amount of work on higher order contracts and gradual types.
Higher order contracts were introduced in \cite{10.1145/581478.581484} as a way to dynamically enforce invariants of software interfaces, specifically higher order functions.
% contracts go back to the 70s apparently, but this seems a reasonable place to start the story, though Bigloo Scheme [28] is cited there.
The notion of blame dates at least that far back.
Swapping the type cast of the input argument of a function type is reminiscent of that paper's use of blame contravariance, though it is presented in a much different way.
% However the contract language of that paper was somewhat limited.
 
Contract semantics were revisited in \cite{10.1145/1925844.1926410,10.1007/978-3-642-28869-2_11} where a more specific correctness criteria based on blame is presented.

Contract systems still generally rely on users annotating their intentions explicitly.
Similar to how programmers might include $\mathtt{assert}$s in an imperative language.
In this thesis annotations are added automatically though elaboration based on type annotations.
 
While there are similarities between contract systems and the \csys{} outlined here, the \csys{} is designed to address only issues with definitional equality in a dependent type theory.
Since contract systems are generally used in untyped languages with contracts written in the host language, definitional equality simply isn't applicable in the vast majority of contract systems.

\todo[inline]{discuss why contracts are a bad fit for this problem}

\subsubsection{Gradual Types}
 
Types can be viewed as a very specific form of contracts that are usually enforced statically.
\textbf{Gradual type systems} allow for a mixing of the static type checking and dynamic type assertions.
Often type information can be inferred using standard techniques, allowing programmers to write fewer annotations.
\todo[inline]{cite first paper, DLS2008?}\todo[inline]{talk about type imprecision}
 
Gradual type systems usually achieve this by adding a $?$ meta character into the type language to denote imprecise typing information.
The first popular account of gradual type semantics appeared in \cite{siek_et_al:LIPIcs:2015:5031} with the alliterative ``gradual guarantee'' which has inspired some of the properties targeted in this Chapter.
 
% , which informally asserts that "runtime checks will not change the expected behavior", "runtime checks will not change the expected behavior" and "well typed code won't be blamed"
% and then like an endless back and forth over criteria
 
Additionally some of the formalism from this Chapter were inspired by the ``Abstracting gradual typing'' methodology \cite{10.1145/2837614.2837670}, where static evidence annotations become runtime checks.
% Unlike some impressive attempts to gradualize the polymorphic lambda calculus \cite{10.1145/3110283}, our system does not attempt to enforce any parametric properties of the base language. %example?
% It is unclear if such a restriction would be desirable for a dependently typed language in practice.
 
This thesis borrows some notational conventions from gradual typing such as the $a::A$ construct for type assertions.
 
A system for gradual dependent types has been proposed in \cite{10.1145/3341692}.
That paper is largely concerned with establishing a decidable type checking procedure via an approximate term normalization.
However, that system retains the conventional style of definitional equality, so that it is possible, in principle, to get $\Vect{}\,(1+x)\neq \Vect{}\,(x+1)$ as a runtime error.
Additionally it is unclear if adding the $?$ meta-symbol into an already very complicated type theory is easier or harder from the programmer's perspective.
 
The common motivation for gradual type systems is to gradually convert a code base from untyped to (usually simply) typed code.
% This chapter has a much tighter scope than the other work cited here, dealing only with equational assumptions.
However, anyone choosing to use a dependent type system has already bought into the usefulness of types in general and will probably not want fragments of completely untyped code.
Gradually converting untyped code to include dependent types is far less plausible than gradually converting untyped code to use simple types.
Especially considering that most real-life codebases will use effects, while adding effects into a simply typed programming language is straightforward, mixing dependent types and effects is a complicated area of ongoing research.

While the gradual typing goals of mixing static certainty with runtime checks are similar to our work here, the approach and details are different.
Instead of trying to strengthen untyped languages by adding types, we take a dependent type system and weaken it with a cast operator.
This leads to different trade-offs in the design space.
For instance, we cannot support completely unannotated code, but we do not need to complicate the type language with a $?$ meta-symbol for uncertainty.

One might characterize this work in this Chapter as gradualizing only the definitional equality relation with a degenerate notion of imprecision.
 
\subsubsection{Blame}
 
Blame is one of the key ideas explored in the contract type and gradual types literature\cite{10.1007/978-3-642-00590-9_1,wadler:LIPIcs:2015:5033,10.1145/3110283}.
Often the reasonableness of a system can be judged by the way blame is handled\cite{wadler:LIPIcs:2015:5033}.
% Blame is treated in \cite{wadler:LIPIcs:2015:5033} very similarly to the presentation in this chapter.
This Chapter goes beyond blaming a source location and also tracks a witnessing observation that can also be made.
 
\subsection{Refinement Style Approaches}
 
This thesis describes a \fullSp{} dependently typed language.
This means computation can appear uniformly in both term and type position.
An alternative approach to dependent types is found in \textbf{refinement type systems}.
 
Refinement type systems restrict type dependency, possibly to specific base types such as $\mathtt{int}$ or $\mathtt{bool}$.
Under this restriction, it is straightforward to check type level equalities and additional properties hold at runtime.
 
One approach which explores this is \textbf{hybrid type checking} \cite{10.1145/1111037.1111059,10.1145/1667048.1667051} which performs ``static analysis where possible, ... dynamic checks where necessary''.
% This is a very similar goal to what has been proposed in this Chapter.
However, there are several differences in that work: they have a simply typed system, static warnings for programmers are not considered, and type checking can reject "clearly ill-typed programs".
For the system defined in this thesis there is no clear boundary between clearly ill-typed programs and subtly ill-typed programs, so we treat all potential inequalities uniformly with a static warning and a runtime check.

\todo{need to talk about soft typeing?}
% 'Hybrid type checking is inspired by prior work on soft typing [Fagan 1990;
% Wright and Catwright 1994; Aiken et al. 1994; Flanagan et al. 1996], but it
% extends soft typing by rejecting many ill-typed programs, in the spirit of static
% type checkers.'

Another notable example is \cite{10.1007/1-4020-8141-3_34} which describes a refinement system that limits predicates to base types.
Another example is \cite{10.1145/3093333.3009856}, a refinement system treated in a specifically gradual way.
A refinement type system with higher order features is gradualized in \cite{c4be73a0daf74c9aa4d13483a2c4dd0e}.
\todo{why is full spectrum better?}
 
\todo{cite my abstract}
 
% consider also citing https://www.youtube.com/watch?v=gIYMERs7AZQ https://www.youtube.com/watch?v=EGKeWg2ES0A