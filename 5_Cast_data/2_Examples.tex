\section{Examples}
\todo{support the new notation}

Consider some of the following examples of how surface langauge terms might elaborate.

\subsection{Head}

If the user case matches the head of $x$ where $x$ has type $\mathtt{Vec}\,A\,(\mathtt{S}n)$ in the surface language,

\begin{lstlisting}[basicstyle={\ttfamily\small}]
case x <_:Vec Bool (S n) => Bool> {
| Cons _ a _ _ => a
}
\end{lstlisting}

What can go wrong in the presence of casts?
\begin{itemize}
\item
A faulty cast assumption may have made $x$ appear to be a Vector even when it is not.
For instance, $\mathtt{True}::\mathtt{Vec}\,\mathtt{Bool}\,3$.
\item 
The vector may be empty but cast to look like it is inhabited.
For instance, $\mathtt{Nil}\,\mathtt{Bool}\,::\mathtt{Vec}\,\mathtt{Bool}\,5$.
\item 
The vector may have a type that is not $\mathtt{Bool}$.
For instance, $\mathtt{Cons}\,\mathtt{Nat}\,3\,...\,...::\mathtt{Vec}\,\mathtt{Bool}\,5$.
\end{itemize}

To handle these issues elaboration can use unification to generate the following cast langngae term,

\begin{lstlisting}[basicstyle={\ttfamily\small}]
case x {
| (Cons A a y _) :: p => a::InTCon0(p)
| (Nil A) :: p => !InTc1(p)
}
\end{lstlisting}

The elaborated case expression covers all possible constructors for the data type constructor $\mathtt{Vec}$. 

In the fist branch $A:\star$, $a:A$, $y:\mathtt{Nat}$, while $p$ is a path variable that contains evidence that the type of $\mathtt{Cons\,A\,a\,y\,-}$ is $\mathtt{Vec Bool (S n)}$.
We might say, $p:\mathtt{Vec}\ A\ (\mathtt{S}\,y)\approx\mathtt{Vec}\,\mathtt{Bool}\,(\mathtt{S}\,n)$.
$InT_{0}p$ extracts the 0th argument from the type constructor $p:\mathtt{Vec}\ \underline{A}\ (\mathtt{S}\,y)\approx\mathtt{Vec}\,\underline{\mathtt{Bool}}\,(\mathtt{S}\,n)$ resulting in the type $InT_{0}p:A\approx\mathtt{Bool}$.
The body of the branch casts $a$ along $InT_{0}\,p$ to $\mathtt{Bool}$.
Casts will need to generalized from \ch{3} to support this behavour.

In the the second branch, the pattern mactch gives $A:\star$, $p:\mathtt{Vec}\ A\ \mathtt{Z}\approx\mathtt{Vec}\,\mathtt{Bool}\,(\mathtt{S}\,y)$.
The body of the branch encodes the contradiction observed by noting $\mathtt{Z}\neq\mathtt{S}\,y$.

Since there is no assertion made in either branch, no warnings will be generated for this elaborated case term.
Any failure that arises will be redirected to the scrutinee, which must have made a blameable assumption.

Again consider the ways $x$ could go wrong,
\begin{itemize}
\item 
If the user tries to eliminate $\mathtt{True}::\mathtt{Vec}\,\mathtt{Bool}\,3$, the type constructor is not matched so the faulty assumption can be blamed automatically.
\item 
If the scrutinee is an empty vector, we will fall into the $\mathtt{Nil}$ branch, which will reflect the underling faulty assumption, via the explicit blame syntax.
\item
If the vector is inhabited by an incorrect type, such as $\mathtt{Cons}\,\mathtt{Nat}\,3\,...\,...::\mathtt{Vec}\,\mathtt{Bool}\,5$, the case will return $3::\mathtt{Bool}$ with a cast that rests on the faulty assumption of $\mathtt{Vec}\,\mathtt{Nat}\,5=\mathtt{Vec}\,\mathtt{Bool}\,\,5$.
When exactly this blame will surface depends on the evaulation and checking strategies.
In the implemenated langauge \cbv() and check-by-value are used at runtime and the blame will surface imediatly.
Using a \whnf{} strategy, the blame will be embedded in the resulting term and discovered whenever that term is eliminated.
\end{itemize}

\subsection{Sum}

Consider the more involved example that sums the 2 numbers in a vector of legnth 2,

\begin{lstlisting}[basicstyle={\ttfamily\small}]
case x <_:Vec Nat 2 => Nat> {
| Cons _ i _ (Cons _ j _ _) => i+j
}
\end{lstlisting}

The elaboration procedure will produce

\begin{lstlisting}[basicstyle={\ttfamily\small}]
case x {
| (Cons Nat' i n' (Cons Nat'' j n'' rest):: p1):: p2 => 
  i::InTc0(p2) + j::(InTc0(p1).InTc0(p2))
| (Nil Nat') :: p => 
  !InTc1(p)
| (Cons Nat' i n' (Nil Nat''):: p1):: p2 =>
  !InTc1(p1).InTc1(p2) 
}
\end{lstlisting}

\begin{itemize}
\item
In the first branch we have the variables in scope, $p1:\mathtt{Vec}\ Nat''\ (\mathtt{S}\ n'')\approx\mathtt{Vec}\ Nat'\ n'$, $p2:\mathtt{Vec}\ Nat''\ (\mathtt{S}\ n')\approx\mathtt{Vec}\ \mathtt{Nat}\ 2$, $i:Nat'$, $j:Nat''$.
\begin{itemize}
\item 
This means we can deduce $InTc_{0}(p2):Nat'\approx\mathtt{Nat}$, and $InTc_{0}(p1)InTc_{0}(p2):Nat''\approx\mathtt{Nat}$.
\todo[inline]{how should paths be concatenated? ;}
\item 
We know from unification what the type and value of every term is supposed to be, so casts can be injected using evidence from the pattern.
\end{itemize}
\item 
In the 2nd branch we have , $p:\mathtt{Vec}\ Nat''\ 0\approx\mathtt{Vec}\ Nat'\ 2$.
\begin{itemize}
\item 
Which is contredictory, by $InTc_{1}(p):0\approx2$.
\end{itemize}
\item 
\todo{explanation is a little wrong, need to congruence this under a the S contructor}
In the 3rd branch, $p1:\mathtt{Vec}\ Nat''\ 0\approx\mathtt{Vec}\ Nat'\ n'$, $p2:\mathtt{Vec}\ Nat'\ (\mathtt{S}\ n')\approx\mathtt{Vec}\ \mathtt{Nat}\ 2$. 
\begin{itemize}
\item
Which is unsatisfiable by $InTc_{1}(p1)InTc_{1}(p2):0\approx2$.
We don't need to know which sub path is problematic beforehand, only that the combination causes trouble.
If this branch is reached in an empty context, we can observe a problem in either branch or both.
\end{itemize}
\end{itemize}

\subsection{Variables Not In Branch}
There are more complicated possibilities that need to be addressed.
For instance, sometimes congruence is needed even when variables are not used in the body of a branch.
Consider this partial pattern match where $\mathtt{rept: (x: Nat) -> Vec Bool x}$,

\begin{lstlisting}[basicstyle={\ttfamily\small}]
case x <x: Nat => Vec Bool x> {
| 2 => rept 2
}
\end{lstlisting}

will elaborate to 
\begin{lstlisting}[basicstyle={\ttfamily\small}]
case x  {
| (S (S (Z :: _) :: _) :: _) => rept 2
? (Z :: _)
? (S (Z :: _) :: _)
? (S (S (S _ :: _) :: _))
}
\end{lstlisting}

Where unification can confirm that the explicit branch has exatly the type of the motive and does not need a cast (since we will disregard any blame ariseing form the scrutinee).
Additionally the elaborator will form a covering of implicit patterns that handle any possible constructor.
Since the uinfier cannot find a contrediction for any of these cases, the user will be warned of possible runtime errors.

\subsection{Assignment at the type level}

Another more complicated possibility is this surface expression that takes in a propositional proof that $2=2$ and uses the named witness to generate a vector of length 2.
This will typecheck in the surface Langugae.

\todo[inline]{Note that an arbitrary choice was was made $InTC_{1}(p)$ $InTC_{2}(p)$}
\begin{lstlisting}[basicstyle={\ttfamily\small}]
case x <_:Id Nat 2 2 => Vec Bool 2> {
| refl _ a => rep a Bool True
}
\end{lstlisting}

This will elaborate to

\begin{lstlisting}[basicstyle={\ttfamily\small}]
case x <_:Id Nat 2 2 => Vec Bool 2> {
| (refl Nat' a)::p => 
  (rep (a::InTc0(p))):: Cong (x=> (A : *) -> A -> Vec Nat x)
                             (CastL (InTC0(p)) (InTC1 p))
}
\end{lstlisting}

Where $Nat':\star$, $a:Nat'$, and $p:\mathtt{Id}\ Nat'\ a\ a\approx\mathtt{Id}\ \mathtt{Nat}\ 2\ 2$.
Since we have $p:\mathtt{Id}\ Nat'\ a\ a\approx\mathtt{Id}\ \mathtt{Nat}\ 2\ 2$, we can derive $InTC_{0}(p):\ Nat'\ \approx\ \mathtt{Nat}$.
Which can be used in $a::InTC_{0}(p)$ to cast $a$ from $Nat'$ to $\mathtt{Nat}$.
But then we need evidence that $(A:\star)\rightarrow A\rightarrow\mathtt{Vec}\,\mathtt{Nat}\,(a::InTC_{0}(p))\approx(A:\star)\rightarrow A\rightarrow\mathtt{Vec}\,\mathtt{Nat}\,2$ to avoid a sperous assertion.
\todo[inline]{avoiding these sperous casts is important for the conjectured gradual correctenss, well typed surface language terms should not generate warnings}
Fist we need congruence to select the subterm that is of intrest, $(A:\star)\rightarrow A\rightarrow\mathtt{Vec}\,\mathtt{Nat}\,\underline{(a::InTC_{0}(p))}\approx(A:\star)\rightarrow A\rightarrow\mathtt{Vec}\,\mathtt{Nat}\,\underline{2}$.
The $Cong$ syntax eplixitly embedds a path into a larger expression, here $Cong_{x=> (A : *) -> A -> \mathtt{Vec} \mathtt{Nat} x} ...$ selects the relevent part of the type.

Ideally we would have a path from $a::InTC_{0}(p)\ \approx\ 2$, but we only have a path $InTC_{1}(p)\ :\ a\ \approx\ 2$.
We will use the $CastL$ to produce the path $CastL_{InTC_{0}(p)}\left(InTC_{1}p\right):(a::InTC_{0}p))\approx2$.
$CastL$ will be derivable within the extended cast system.


% \subsubsection{Need to remove a cast}
% \todo{better name}

% Consider this surface language expression that extracts the last element from a non-empty list.
% Assume the function $last:(n:Nat)\rightarrow Vec\,A\,(S\,N)\rightarrow A$ is in scope.
% \todo{or just define the recursive function}

% \begin{lstlisting}[basicstyle={\ttfamily\small}]
% case v <_: Vec A (S x) => A > {
% | Cons _ a (Z) _ => a
% | Cons _ _ (S n) rest => last n rest
% }
% \end{lstlisting}

% This will elaborate into 

% \begin{lstlisting}[basicstyle={\ttfamily\small}]
% case v <_: Vec A (S x) => A > {
% | (Cons A' a' (Z)::q rest) :: p => a' :: (inTC1(p))-1
% | (Cons A' a' (S n)::q rest)::p => last n (rest :: p')
% | (Nil A')::p => !TCon1(p)
% }
% \end{lstlisting}

% \todo[inline]{prettier expressions, rev to -1, in general it might be clearer if cast langauge is always in math mode}

% In the 2nd branch we have
%   $A':\star$, $a':A'$, $n:\mathtt{Nat}$,
%   $q:\mathtt{Nat}\approx \mathtt{Nat}$,
%   $rest:\mathtt{Vec}\,A'\,\left(\left(\mathtt{S}\,n\right)::q\right)$,
%   and $p:\mathtt{Vec}\,A'\,\left(\mathtt{S}\left(\left(\mathtt{S}\,n\right)::q\right)\right)\approx \mathtt{Vec}\,A\,\left(\mathtt{S}\,x\right)$.
% Elaboration cannot unify a solution unless we can remove casts, otherwise it becomes impossible to construct a path from $n \approx x$
%   from $\mathtt{Vec}\,A'\,\left(\left(\mathtt{S}\,n\right)\underline{::q}\right)\approx \mathtt{Vec}\,A\,\left(\mathtt{S}\,x\right)$, since the cast blocks
%   $\mathtt{S}\,n = \mathtt{S}\,x$ that would be derivable by unification in the surface language.
% We will need an operator that can remove casts from the endpoints of paths that arise from unification.
% We will call these operators $uncastL$ and $uncastR$ and they will be derivable in the cast language.
% With these operations we can match the process of surface level unification so that

% $p'\ =\ Cong_{x\Rightarrow \mathtt{Vec}\,A'\,x}\left(UncastR\left(refl\right)\right)\circ Cong_{x\Rightarrow \mathtt{Vec}\,x\,\left(\mathtt{S}\,n\right)}\left(TCon_{0}p\right):\ \mathtt{Vec}\,A'\,\left(\left(\mathtt{S}\,n\right)::q\right)\ \approx\ \mathtt{Vec}\,A\,\left(\mathtt{S}\,n\right)$

% Where 
% $UncastR\left(refl\right): \left(\mathtt{S}\,n\right)::q  \approx \mathtt{S}\,n $

% In the first branch we have, 
%   $A':\star$, $a':A'$,
%   $q:\mathtt{Nat}\approx \mathtt{Nat}$,
%   $rest:\mathtt{Vec}\,A'\,\left(\mathtt{Z}::q\right)$, 
%   and $p:\mathtt{Vec}\,A'\,\left(\mathtt{S}\left(\mathtt{Z}::q\right)\right)\approx \mathtt{Vec}\,A\,\left(\mathtt{S}\,x\right)$.
% Unification can proceed to derive $TCon_{1}\left(Con_{0}\left(p\right)\right)^{-1}:\ x\approx \mathtt{Z}::q$
% and $TCon_{1}\left(p\right)^{-1}:\ A\approx A'$.

% In the final branch we have $p:\mathtt{Vec}\,A'\,\mathtt{Z}\approx \mathtt{Vec}\,A'\,\left(\mathtt{S}\,x\right)$,
% which is contradicted by $TCon_{1}p:\mathtt{Z}\approx \mathtt{S}\,x$

\todo[inline]{Example: transitivity, translate out to motive}
