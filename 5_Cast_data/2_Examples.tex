\section{Examples}
 
Consider some of the following examples of how surface language pattern matches might elaborate.
 
\subsection{Head}
 
In the \slang{} the first element of $x$ can be extracted with,
 
\begin{lstlisting}[basicstyle={\ttfamily\small}]
case x <_:Vec Bool (S n) => Bool> {
| Cons _ a _ _ => a
}
\end{lstlisting}

Where $x$ has the apparent type $\Vect{}\,\Bool{}\,(\mathtt{S}\,n)$.

What can go wrong in the presence of casts?
\begin{itemize}
\item
A blamable cast may have made $x$ appear to be a \Vect{} even when it is not.
For instance, $\True{}::_{\ell}\Vect{}\,\Bool{}\,3$ (in \ch{3} notation).
\item
The vector may be empty but cast to look like it is inhabited.
For instance, $\mathtt{Nil}\,\mathtt{Bool}\,::_{\ell}\mathtt{Vec}\,\mathtt{Bool}\,5$.
\item
The vector may contain elements that are not $\mathtt{Bool}$.
For instance, $\mathtt{Cons}\,\mathtt{Nat}\,3\,...\,...::_{\ell}\mathtt{Vec}\,\mathtt{Bool}\,5$.
\end{itemize}
 
To handle these issues, elaboration can generate the following cast langngae term,
 
\begin{lstlisting}[basicstyle={\ttfamily\small}]
case x {
| (Cons A a y _) :: p => a::(TCon_0(p))
| (Nil A) :: p => !TCon_1(p)
}
\end{lstlisting}
\todo{subscripts in listings}
 
The elaborated case expression covers all possible constructors for the data type constructor $\mathtt{Vec}$, including patterns that did appear in the surface term.
Then unification solves the constraints to help elaborate the body.
 
In the first branch, the pattern captures typed variables, $A:\star$, $a:A$, $y:\mathtt{Nat}$, while $p$ is a path variable that contains evidence that the type of $\mathtt{Cons\,A\,a\,y\,-}$ is $\mathtt{Vec\,Bool\,(S\,n)}$.
So we will say, $p:\mathtt{Vec}\ A\ (\mathtt{S}\,y)\approx\mathtt{Vec}\,\mathtt{Bool}\,(\mathtt{S}\,n)$.
$TCon_{0}\,p$ extracts the 0th argument from the type constructor $p:\mathtt{Vec}\ \underline{A}\ (\mathtt{S}\,y)\approx\mathtt{Vec}\,\underline{\mathtt{Bool}}\,(\mathtt{S}\,n)$ resulting in the type $TCon_{0}p\ :\ A\approx\mathtt{Bool}$.
The body of the branch casts $a$ along $TCon_{0}\,p$ to $\mathtt{Bool}$.
Casts will need to be generalized from \ch{3} to contain evidence of equality.
 
In the the second branch, the pattern match gives $A:\star$, $p:\mathtt{Vec}\ A\ \mathtt{Z}\approx\mathtt{Vec}\,\mathtt{Bool}\,(\mathtt{S}\,y)$.
The body of that branch encodes the contradiction using explicit blame syntax by observing $\mathtt{Z}\neq\mathtt{S}\,y$ with $TCon_{1}p$.
Any match in that branch must be blameable.
 
Since there is no assertion made in either branch, no warnings will be reported for this elaborated case term.
Any failure that arises will be redirected to the scrutinee, which must have made a blameable assertion.
 
Again consider the ways $x$ could go wrong,
\begin{itemize}
\item
If the user tries to eliminate $x=\mathtt{True}::\mathtt{Vec}\,\mathtt{Bool}\,3$, the type constructor is not matched so the faulty assumption can be blamed automatically.
\item
If the scrutinee is an empty \Vect{}, we will fall into the $\mathtt{Nil}$ branch, which will reflect the underlying faulty assumption, via the explicit blame syntax.
\item
If the \Vect{} is inhabited by an incorrect type, such as $\mathtt{Cons}\,\mathtt{Nat}\,3\,...\,...::_{\ell}\mathtt{Vec}\,\mathtt{Bool}\,5$, the case will return $3::_{\ell,...}\mathtt{Bool}$\todo{fill in ..., typeing jusdment and undeline the other?} with a cast that rests on the blamable assertion of $\mathtt{Vec}\,\mathtt{Nat}\,5\approx\mathtt{Vec}\,\mathtt{Bool}\,\,5$.
When exactly this blame will surface depends on the evaluation and checking strategies.
In the implemented language \cbv{} and check-by-value are used at runtime and the blame will surface before the pattern match.
Using a \whnf{} strategy the blame will be embedded in the resulting term and discovered whenever that term is eliminated. % , as in the implemented elaborator, 
\end{itemize}
 
\subsection{Sum}
 
The body of a pattern match may need to make use of type level facts discovered from the pattern match.
For instance, in the \slang{} we can  sum the two numbers in a \Vect{} of length $2$ with
 
\begin{lstlisting}[basicstyle={\ttfamily\small}]
case x <_:Vec Nat 2 => Nat> {
| Cons _ i _ (Cons _ j _ _) => i+j
}
\end{lstlisting}
 
The elaboration procedure will produce
 
\begin{lstlisting}[basicstyle={\ttfamily\small}]
case x {
| (Cons Nat' i n' (Cons Nat'' j n'' rest):: p1):: p2 =>
  i::(TCon_0(p2)) + j::(TCon_0(p1) U TCon_0(p2))
| (Nil Nat') :: p =>
  !TCon_1(p)
| (Cons Nat' i n' (Nil Nat''):: p1):: p2 =>
  !(TCon_1(p1) U DCon_0(TCon_1(p2)))
}
\end{lstlisting}
 
\begin{itemize}
\item
In the first branch we have the variables in scope, $Nat':\star$, $Nat'':\star$, $i:Nat'$, $j:Nat''$, $p1:\mathtt{Vec}\ Nat''\ (\mathtt{S}\ n'')\approx\mathtt{Vec}\ Nat'\ n'$, and $p2:\mathtt{Vec}\ Nat''\ (\mathtt{S}\ n')\approx\mathtt{Vec}\ \mathtt{Nat}\ 2$.
\todo{rename Nat' -> N, Nat'' -> N' }
\begin{itemize}
\item
This means the elaborator can construct $TCon_0(p2):Nat'\approx\mathtt{Nat}$, and $TCon_0(p1):Nat''\approx Nat'$.
Thes facts can be combined to show $TCon_0(p1) \cup TCon_0(p2):Nat''\approx\mathtt{Nat}$.
\item
The elaborator knows what the type of every sub expression is supposed to be, so casts can be injected using evidence from the pattern.
\end{itemize}
\item
In the 2nd branch we have, $p:\mathtt{Vec}\ Nat''\ 0\approx\mathtt{Vec}\ Nat'\ 2$.
\begin{itemize}
\item
Which is contradictory, by $TCon_1(p):0\approx2$.
\end{itemize}
\item
\todo{note that Dcon needs to eat the S constructor}
In the 3rd branch, $p1:\mathtt{Vec}\ Nat''\ 0\approx\mathtt{Vec}\ Nat'\ n'$, $p2:\mathtt{Vec}\ Nat'\ (\mathtt{S}\ n')\approx\mathtt{Vec}\ \mathtt{Nat}\ 2$.
\begin{itemize}
\item
Which is unsatisfiable by $TCon_1(p1) \cup DCon_0(TCon_1(p2)):0\approx1$.
We don't need to know which sub path is problematic beforehand, only that the combination causes trouble.
If this branch is reached, we can observe a problem in at least one path.
\end{itemize}
\end{itemize}
 
\subsection{Missing Branches}
\todo{this example is a little meh, better with something that needs path evidence like ID}
What about unstated branches that cannot be excluded with type information?
% For instance, variables may not be used in the body of a branch.
Consider this partial pattern match where $\mathtt{rept}\ :\ (x: \Nat{}) \rightarrow \Vect{}\,\Bool{}\,x$,
 
\begin{lstlisting}[basicstyle={\ttfamily\small}]
case x <x: Nat => Vec Bool x> {
| 2 => rept 2
}
\end{lstlisting}
 
will elaborate to
\begin{lstlisting}[basicstyle={\ttfamily\small}]
case x  {
| S (S (Z :: _) :: _) :: _ => rept 2
? Z :: _
? S (Z :: _) :: _
? S (S (S _ :: _) :: _)
}
\end{lstlisting}
 
Substitution can confirm that the explicit branch has exactly the type of the motive and does not need a cast\footnote{
  While it is possible that blame was embedded in the $\mathtt{(S (S (Z :: -) :: -) :: -)}$ term, the cast system will allow $\mathtt{(S (S (Z :: -) :: -) :: -)}\equiv2$.}.
Additionally the elaborator will form a covering of implicit patterns that handle any possible constructor.
Since the unifier cannot find a contradiction for any of these cases, the user will be warned of possible runtime errors.
 
\subsection{Congruence (embedding equalities in terms)}
This surface expression that takes in a propositional proof that $2=2$ and uses the named witness to generate a vector of length 2, demonstrates some of the subtler possibilities that arise in dependently typed pattern matching.
% This will typecheck in the surface language.

\begin{lstlisting}[basicstyle={\ttfamily\small}]
case x <_:Id Nat 2 2 => Vec Bool 2> {
| refl _ a => rep Bool True a
}
\end{lstlisting}
 
This will elaborate to
 
\begin{lstlisting}[basicstyle={\ttfamily\small}]
case x {
| (refl N a)::p =>
 (rep Bool True (a :: (TCon_0(p)))) 
   :: Vec Bool (TCon_1(p))
}
\end{lstlisting}
\todo{standardize on the associativity of pat-match}

In the branch, $N:\star$, $a:N$, and $p:\mathtt{Id}\ N\ a\ a\approx\mathtt{Id}\ \mathtt{Nat}\ 2\ 2$.
Since we have $p:\mathtt{Id}\ N\ a\ a\approx\mathtt{Id}\ \mathtt{Nat}\ 2\ 2$, we can derive $TCon_0(p):\ N\ \approx\ \mathtt{Nat}$.
Which can be used in $a::(TCon_0(p))$ to cast $a$ from $N$ to \Nat{}.
But then we need evidence that $\Vect{}\, \Bool{}\, (a :: (TCon_0(p)))\ \approx\ \Vect{}\, \Bool{}\, 2$ to avoid a sperous assertion\todo{
  give an example of an assertion to remind people it is possible
}.
\todo[inline]{avoiding these sperous casts is important for the conjectured gradual correctness, well typed surface language terms should not generate warnings}
First, we need to select the subterm of interest, $\Vect{}\, \Bool{}\, \underline{(a :: (TCon_0(p)))}\ \approx\ Vect{}\, \Bool{}\, \underline{2}$.
Equality evidence is constructed specifically so that it can be embedded into terms.
If we have evidence, $q$, such that $q\ :\ (a ::(TCon_0(p)))\ \approx\ 2$ then $\Vect{}\, \Bool{}\, \underline{q}\ :\ \Vect{}\, \Bool{}\, \underline{(a :: (TCon_0(p)))}\ \approx\ \Vect{}\, \Bool{}\, \underline{2}$.
% The $Cong$ syntax explicitly embeds a path into a larger expression, here $Cong_{x=> (A : *) -> A -> \mathtt{Vec} \mathtt{Nat} x} ...$ selects the relevant part of the type.
 
The cast system will only require that terms are equated up to a definitional equality that disregards casts so instead of needing to show $a :: (TCon_0(p))\ \approx\ 2$, we only have to show $a\ \approx\ 2$.
Which we have in $TCon_1(p)\ :\ a\ \approx\ 2$ and $TCon_2(p)\ :\ a\ \approx\ 2$.
The elaborator can choose either to get a well cast term, and while the pattern will behave consistently on blameless terms, different behavior is possible when blame is discoverable.

For instance, given the elaboration above,
 
\begin{itemize}
  \item if $x$ is $\mathtt{refl}\,\Nat{}\,2\ ::\ \Id{}\,\Nat{}\,0\,2\ ::\ \Id{}\,\Nat{}\,2\,2$ then blame will be discoverable from the $TCon_1$ observation.
  \item if $x$ is $\mathtt{refl}\,\Nat{}\,2\ ::\ \Id{}\,\Nat{}\,2\,0\ ::\ \Id{}\,\Nat{}\,2\,2$ then blame will not be discoverable and a blameless \Vect{} is constructed.
\end{itemize}
In general there is no way around this, equality evidence may be constructable in subtle ways.
Not everything can be checked.

\todo[inline]{Example: translate out to motive}

% \subsection{Transitivity}

% \begin{figure}
%   \begin{lstlisting}[basicstyle=\ttfamily\small]
%   -- surface language term
%   trans : (A : *) -> (x : A) -> (y : A) -> (z : A)
%         -> (xy : Id A x y) -> (yz : Id A y z) -> Id A x z
%   trans A x y z xy yz =
%   case xy, yz < _ => _ => Id A x z > {
%     | (refl A' a') => (refl A'' a'') => (refl A' a')
%   } ;
  
%   -- elaborated cast language term
%   trans : (A : *) -> (x : A) -> (y : A) -> (z : A)
%         -> (xy : Id A x y) -> (yz : Id A y z) -> Id A x z
%   case xy, yz {
%     | (refl A' a')::v => (refl A'' a'')::w => 
%       (refl A' a'):: (Id Aeq aeq aeq)
%   } ;
% \end{lstlisting}

% \caption{Transitivity Example}
% \label{fig:cast-trans}
% \end{figure}

% \todo[inline]{may need to expand scruts if we are using flex vars}

% For another example consider the surface language function that validates the transitivity of the \Id{} type, and its cast language elaboration in \Fref{cast-trans}.
% % For the surface language term the constraints ($\text{Id A' a' a'}=\text{Id A x y}$,$\text{Id A'' a'' a''}\approx\text{Id A y z}$) will be solved and the branch type checks under the substitutions implied by those equalities ($A=A'=A''$, $a'=x=y=a''=z$).
% In the elaborated cast term the variables are not directly equated, instead assertion variables are added to scope to build evidence for these equalities ($v:\text{Id A' a' a'}\approx\text{Id A x y}$, $w:\text{Id A' a' a'}\approx\text{Id A x y}$).
% Elaboration based unification will generate terms that correspond to the equalities discovered by normal unification 
%   ($Aeq=\left(TCon_{0}\ v\right)\cup^{\star}\left(TCon_{0}\ w\right)\sqsupseteq A,A',A''$;
%   and $aeq=\left(TCon_{1}\ v\right)\cup^{\star}\left(TCon_{2}\ v\right)\cup^{\star}\left(TCon_{1}\ w\right)\cup^{\star}\left(TCon_{2}\ w\right)\sqsupseteq a',x,y,a'',z$).
% These assertions are embedded into the constructor allowing $\text{Id A' a' a'}=\text{Id A x z}$.
% Again, the $\mathtt{trans}$ function itself is blameless, any blame surfaces from the use of the term must come from a blamable input.


\subsection{Peeking}
\todo{move to the end?}

% Finally case expressions will be blamed if an incompatible constructor appears.
% For instance, if head is called with $\True{} ::(\Bool{} \ \sim_{\ell}^{\star} \Vect{} \ \Nat{}\ 1)$ then $\ell$ will be blamed immediately, since $\True{} $ does not match the correct type of the constructor $Cons$.
% Since the type constructor is known, it is possible to check the coverage of the patterns.
% If every constructor is accounted for, only blameable data remains.
% Quantifying over casts allows blame to be redirected, so if the program gets stuck in a pattern branch it can blame the malformed input.
% This extension seems to preserve cast soundness.

\begin{figure}
\begin{lstlisting}[basicstyle=\ttfamily\small]
peek : Id Nat 0 1 -> Nat
peek x =
case x <_: Id Nat 0 1 => Nat> {
  | (refl _ x :: w) => x :: (TCon_0 w)
}

-- under weak head evaluation
peek (refl 4 :: Id Nat 0 1) = 4
\end{lstlisting}
\caption{Cast Pattern Matching}
\label{fig:cast-peek}
\end{figure}

% As noted, the cast language will enforce a minimal amount of checking, 
Another example of a term that might potentially lead to unexpected behavior is the peek function defined in \Fref{cast-peek}.
$\mathtt{peek}$ will ignore several discrepancies in the index of the \Id{} type, if run in \whnf{}\footnote{
  The example can be extended to \cbv{} with functions, 
  $\mathtt{peek'} : \Id{} (\mathtt{Unit} \rightarrow \Nat{}) (\lambda - \Rightarrow 0) (\lambda - \Rightarrow 1)  \rightarrow  \mathtt{Unit} \rightarrow \Nat{}$.
}.
As in \ch{3}, our formalism uses a minimal amount of checking to maintain cast soundness, though more eager checking is implemented in the prototype.