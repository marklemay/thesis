\chapter{Data in the Cast Language}
\label{chapter:CastData}
\thispagestyle{myheadings}
 
%Surprisingly, the cast system can be extended with a pattern matching construct without unification.
\Ch{3} showed how to use the \ac{TAS} and the \bidir{} system as a guide to build a dependently typed language with runtime equality.
The \ac{TAS} inspired the cast system, while the \bidir{} system corresponds to elaboration.
% Specifically, the \bidir{} approach localizes the equality checks.
In this Chapter we will extend these systems for dependently indexed data and pattern matching.
This will turn out to be more complicated then the system in \ch{3}, since in \ch{3} equality was only testable at types which allowed for some syntactic and semantic shortcuts.
In the presence of data, equality needs to be testable at terms, which will not necessarily have the same type.
Additionally, the subtleties of pattern matching will need to be dealt with.
While the intuition built up in \ch{3} still holds, the cast language will need the appropriate revisions.
 
As before, we will take the (conjectured) surface language of \ch{4} and construct a cast language with corresponding features.
Though it is difficult to formalize a \ac{TAS} and corresponding \bidir{} system that has pattern matching, informally the unification of pattern matching belongs in the \bidir{} system.
Accordingly we will extend elaboration with a form of unification.
Because we will not need to deal with unification in the cast language can provide evidence that the cast system is \textbf{cast sound}. \todo{"show" if we ever get a proof}
While the lack of a formal \ac{TAS} and \bidir{} system in \ch{4} will make the other properties of \ch{3} difficult to show, we will design the system with an eye to preserving them.
 
% \todo{formulate case consistently}
% Elaboration should satisfy the additional correctness\todo{``gradual correctness''?} properties relative to a type assignment system and bidirectional system.
% In this case we will target the conjectured TAS from Chapter 4, with a first order unification style pattern matching.
 
Despite these caveats, there is an interesting interpretation of data and pattern matching when extended to the cast system.
 
Consider the normal forms of data terms, in a conventionally typed language.
In an empty context, data values must have a valid data constructor in head position (justifying the syntax of pattern matching).
In the cast language, the normal form of data in an empty context can have casts applied to an expression.
If the casts were blameless then the constructor in the head position will match the data type.
% Casts are wrapped around terms during the elaboration procedure, and will accumulate during evaluation.
If the pattern syntax is extended with a path variable that can represent evidence of equality, then that evidence can be extracted and used in the body of the branch.
 
As in conventional pattern matching, since the type constructor is known, it is possible to check the coverage of the patterns against its possible constructors.
If every constructor is accounted for, only blameable \scruts{} are possible.
Quantifying over evidence of equality allows blame to be redirected, so if the program gets stuck in a pattern branch it can blame the original faulty assumption.
 
To account for unreachable patterns, we can record the proof of inequality that makes the unification unsatisfiable with explicit syntax that will expose blame.
In the cast language it is possible for a case expression to reduce into such an ``unreachable'' branch.
If this happens blame will be reflected back onto a specific problematic assumption of the input.
This will require extending the cast language with terms to manipulate equality evidence directly.
 
During elaboration, after a pattern is unified, we will inject the proofs of equality where needed so that they cast check. 
% This will require that our notion of paths support congruence.

\section{Examples}
 
Consider some of the following examples of how \slang{} pattern matches might elaborate.
 
\subsection{Head}
 
In the \slang{} the first element of $x$ can be extracted with,
 
\begin{lstlisting}[basicstyle={\ttfamily\small}]
case x <_:Vec Bool (S n) => Bool> {
| Cons _ a _ _ => a
}
\end{lstlisting}

Where $x$ has the apparent type $\Vect{}\,\Bool{}\,(\Suc{}\,n)$.

What can go wrong in the presence of casts?
\begin{itemize}
\item
A blamable cast may have made $x$ appear to be a \Vect{} even when it is not.
For instance, $\True{}::_{\ell}\Vect{}\,\Bool{}\,3$ (in \ch{3} notation).
\item
The vector may be empty but cast to look like it is inhabited.
For instance, $\mathtt{Nil}\,\mathtt{Bool}\,::_{\ell}\mathtt{Vec}\,\mathtt{Bool}\,5$.
\item
The vector may contain elements that are not $\mathtt{Bool}$.
For instance, $\mathtt{Cons}\,\Nat{}\,3\,...\,...::_{\ell}\mathtt{Vec}\,\mathtt{Bool}\,5$.
\end{itemize}
 
To handle these issues, elaboration can generate the following cast langngae term,
 
\begin{lstlisting}[basicstyle={\ttfamily\small}]
case x {
| (Cons A a y _) :: p => a::(TCon_0(p))
| (Nil A) :: p => !TCon_1(p)
}
\end{lstlisting}
\todo{subscripts in listings}
 
The elaborated \case{} expression covers all possible constructors for the data type constructor $\mathtt{Vec}$, including patterns that did appear in the surface term.
Then unification solves the constraints to help elaborate the body.
 
In the first branch, the pattern captures typed variables, $A:\star$, $a:A$, $y:\Nat{}$, while $p$ is a path variable that contains evidence that the type of $\mathtt{Cons\,A\,a\,y\,-}$ is $\mathtt{Vec\,Bool\,(S\,n)}$.
So we will say, $p:\mathtt{Vec}\ A\ (\Suc{}\,y)\approx\mathtt{Vec}\,\mathtt{Bool}\,(\Suc{}\,n)$.
$TCon_{0}\,p$ extracts the 0th argument from the type constructor $p:\mathtt{Vec}\ \underline{A}\ (\Suc{}\,y)\approx\mathtt{Vec}\,\underline{\mathtt{Bool}}\,(\Suc{}\,n)$ resulting in the type $TCon_{0}p\ :\ A\approx\mathtt{Bool}$.
The body of the branch casts $a$ along $TCon_{0}\,p$ to $\mathtt{Bool}$.
Casts will need to be generalized from \ch{3} to contain evidence of equality.
 
In the the second branch, the pattern match gives $A:\star$, $p:\mathtt{Vec}\ A\ \mathtt{Z}\approx\mathtt{Vec}\,\mathtt{Bool}\,(\Suc{}\,y)$.
The body of that branch encodes the contradiction using explicit blame syntax by observing $\mathtt{Z}\neq\Suc{}\,y$ with $TCon_{1}p$.
Any match in that branch must be blameable.
 
Since there is no assertion made in either branch, no warnings will be reported for this elaborated \case{} term.
Any failure that arises will be redirected to the \scrut{}, which must have made a blameable assertion.
 
Again consider the ways $x$ could go wrong,
\begin{itemize}
\item
If the user tries to eliminate $x=\mathtt{True}::\mathtt{Vec}\,\mathtt{Bool}\,3$, the type constructor is not matched so the faulty assumption can be blamed automatically.
\item
If the \scrut{} is an empty \Vect{}, we will fall into the $\mathtt{Nil}$ branch, which will reflect the underlying faulty assumption, via the explicit blame syntax.
\item
If the \Vect{} is inhabited by an incorrect type, such as $\mathtt{Cons}\,\Nat{}\,3\,...\,...::_{\ell}\mathtt{Vec}\,\mathtt{Bool}\,5$, the \case{} will return $3::_{\ell,...}\mathtt{Bool}$\todo{fill in ..., typeing jusdment and undeline the other?} with a cast that rests on the blamable assertion of $\mathtt{Vec}\,\Nat{}\,5\approx\mathtt{Vec}\,\mathtt{Bool}\,\,5$.
When exactly this blame will surface depends on the evaluation and checking strategies.
In the implemented language \cbv{} and check-by-value are used at runtime and the blame will surface before the pattern match.
Using a \whnf{} strategy the blame will be embedded in the resulting term and discovered whenever that term is eliminated. % , as in the implemented elaborator, 
\end{itemize}
 
\subsection{Sum}
 
The body of a pattern match may need to make use of type level facts discovered from the pattern match.
For instance, in the \slang{} we can  sum the two numbers in a \Vect{} of length $2$ with
 
\begin{lstlisting}[basicstyle={\ttfamily\small}]
case x <_:Vec Nat 2 => Nat> {
| Cons _ i _ (Cons _ j _ _) => i+j
}
\end{lstlisting}
 
The elaboration procedure will produce
 
\begin{lstlisting}[basicstyle={\ttfamily\small}]
case x {
| (Cons Nat' i n' (Cons Nat'' j n'' rest):: p1):: p2 =>
  i::(TCon_0(p2)) + j::(TCon_0(p1) U TCon_0(p2))
| (Nil Nat') :: p =>
  !TCon_1(p)
| (Cons Nat' i n' (Nil Nat''):: p1):: p2 =>
  !(TCon_1(p1) U DCon_0(TCon_1(p2)))
}
\end{lstlisting}
 
\begin{itemize}
\item
In the first branch we have the variables in scope, $Nat':\star$, $Nat'':\star$, $i:Nat'$, $j:Nat''$, $p1:\mathtt{Vec}\ Nat''\ (\Suc{}\ n'')\approx\mathtt{Vec}\ Nat'\ n'$, and $p2:\mathtt{Vec}\ Nat''\ (\Suc{}\ n')\approx\mathtt{Vec}\ \Nat{}\ 2$.
\todo{rename Nat' -> N, Nat'' -> N' }
\begin{itemize}
\item
This means the elaborator can construct $TCon_0(p2):Nat'\approx\Nat{}$, and $TCon_0(p1):Nat''\approx Nat'$.
Thes facts can be combined to show $TCon_0(p1) \cup TCon_0(p2):Nat''\approx\Nat{}$.
\item
The elaborator knows what the type of every sub expression is supposed to be, so casts can be injected using evidence from the pattern.
\end{itemize}
\item
In the 2nd branch we have, $p:\mathtt{Vec}\ Nat''\ 0\approx\mathtt{Vec}\ Nat'\ 2$.
\begin{itemize}
\item
Which is contradictory, by $TCon_1(p):0\approx2$.
\end{itemize}
\item
\todo{note that Dcon needs to eat the S constructor}
In the 3rd branch, $p1:\mathtt{Vec}\ Nat''\ 0\approx\mathtt{Vec}\ Nat'\ n'$, $p2:\mathtt{Vec}\ Nat'\ (\Suc{}\ n')\approx\mathtt{Vec}\ \Nat{}\ 2$.
\begin{itemize}
\item
Which is unsatisfiable by $TCon_1(p1) \cup DCon_0(TCon_1(p2)):0\approx1$.
We don't need to know which sub path is problematic beforehand, only that the combination causes trouble.
If this branch is reached, we can observe a problem in at least one path.
\end{itemize}
\end{itemize}
 
\subsection{Missing Branches}
\todo{this example is a little meh, better with something that needs path evidence like ID}
What about unstated branches that cannot be excluded with type information?
% For instance, variables may not be used in the body of a branch.
Consider this partial pattern match where $\mathtt{rept}\ :\ (x: \Nat{}) \rightarrow \Vect{}\,\Bool{}\,x$,
 
\begin{lstlisting}[basicstyle={\ttfamily\small}]
case x <x: Nat => Vec Bool x> {
| 2 => rept 2
}
\end{lstlisting}
 
will elaborate to
\begin{lstlisting}[basicstyle={\ttfamily\small}]
case x  {
| S (S (Z :: _) :: _) :: _ => rept 2
? Z :: _
? S (Z :: _) :: _
? S (S (S _ :: _) :: _)
}
\end{lstlisting}
 
Substitution can confirm that the explicit branch has exactly the type of the motive and does not need a cast\footnote{
  While it is possible that blame was embedded in the $\mathtt{(S (S (Z :: -) :: -) :: -)}$ term, the \csys{} will allow $\mathtt{(S (S (Z :: -) :: -) :: -)}\equiv2$.
}.
Additionally the elaborator will form a covering of implicit patterns that handle any possible constructor.
Since the unifier cannot find a contradiction for any of these cases, the user will be warned of possible runtime errors.
 
\subsection{Congruence (embedding equalities in terms)}
This surface expression that takes in a propositional proof that $2=2$ and uses the named witness to generate a vector of length 2, demonstrates some of the subtler possibilities that arise in dependently typed pattern matching.
% This will typecheck in the \slang{}.

\begin{lstlisting}[basicstyle={\ttfamily\small}]
case x <_:Id Nat 2 2 => Vec Bool 2> {
| refl _ a => rep Bool True a
}
\end{lstlisting}
 
This will elaborate to
 
\begin{lstlisting}[basicstyle={\ttfamily\small}]
case x {
| (refl N a)::p =>
 (rep Bool True (a :: (TCon_0(p)))) 
   :: Vec Bool (TCon_1(p))
}
\end{lstlisting}
\todo{standardize on the associativity of pat-match}

In the branch, $N:\star$, $a:N$, and $p:\mathtt{Id}\ N\ a\ a\approx\mathtt{Id}\ \Nat{}\ 2\ 2$.
Since we have $p:\mathtt{Id}\ N\ a\ a\approx\mathtt{Id}\ \Nat{}\ 2\ 2$, we can derive $TCon_0(p):\ N\ \approx\ \mathtt{Nat}$.
Which can be used in $a::(TCon_0(p))$ to cast $a$ from $N$ to \Nat{}.
But then we need evidence that $\Vect{}\, \Bool{}\, (a :: (TCon_0(p)))\ \approx\ \Vect{}\, \Bool{}\, 2$ to avoid a sperous assertion\todo{
  give an example of an assertion to remind people it is possible
}.
\todo[inline]{avoiding these sperous casts is important for the conjectured gradual correctness, well typed \slang{} terms should not generate warnings}
First, we need to select the subterm of interest, $\Vect{}\, \Bool{}\, \underline{(a :: (TCon_0(p)))}\ \approx\ Vect{}\, \Bool{}\, \underline{2}$.
Equality evidence is constructed specifically so that it can be embedded into terms.
If we have evidence, $q$, such that $q\ :\ (a ::(TCon_0(p)))\ \approx\ 2$ then $\Vect{}\, \Bool{}\, \underline{q}\ :\ \Vect{}\, \Bool{}\, \underline{(a :: (TCon_0(p)))}\ \approx\ \Vect{}\, \Bool{}\, \underline{2}$.
% The $Cong$ syntax explicitly embeds a path into a larger expression, here $Cong_{x=> (A : *) -> A -> \mathtt{Vec} \mathtt{Nat} x} ...$ selects the relevant part of the type.
 
The \csys{} will only require that terms are equated up to a definitional equality that disregards casts so instead of needing to show $a :: (TCon_0(p))\ \approx\ 2$, we only have to show $a\ \approx\ 2$.
Which we have in $TCon_1(p)\ :\ a\ \approx\ 2$ and $TCon_2(p)\ :\ a\ \approx\ 2$.
The elaborator can choose either to get a well cast term, and while the pattern will behave consistently on blameless terms, different behavior is possible when blame is discoverable.

For instance, given the elaboration above,
 
\begin{itemize}
  \item if $x$ is $\mathtt{refl}\,\Nat{}\,2\ ::\ \Id{}\,\Nat{}\,0\,2\ ::\ \Id{}\,\Nat{}\,2\,2$ then blame will be discoverable from the $TCon_1$ observation.
  \item if $x$ is $\mathtt{refl}\,\Nat{}\,2\ ::\ \Id{}\,\Nat{}\,2\,0\ ::\ \Id{}\,\Nat{}\,2\,2$ then blame will not be discoverable and a blameless \Vect{} is constructed.
\end{itemize}
In general there is no way around this, equality evidence may be constructable in subtle ways.
Not everything can be checked.

\todo[inline]{Example: translate out to motive}

% \subsection{Transitivity}

% \begin{figure}
%   \begin{lstlisting}[basicstyle=\ttfamily\small]
%   -- surface language term
%   trans : (A : *) -> (x : A) -> (y : A) -> (z : A)
%         -> (xy : Id A x y) -> (yz : Id A y z) -> Id A x z
%   trans A x y z xy yz =
%   case xy, yz < _ => _ => Id A x z > {
%     | (refl A' a') => (refl A'' a'') => (refl A' a')
%   } ;
  
%   -- elaborated cast language term
%   trans : (A : *) -> (x : A) -> (y : A) -> (z : A)
%         -> (xy : Id A x y) -> (yz : Id A y z) -> Id A x z
%   case xy, yz {
%     | (refl A' a')::v => (refl A'' a'')::w => 
%       (refl A' a'):: (Id Aeq aeq aeq)
%   } ;
% \end{lstlisting}

% \caption{Transitivity Example}
% \label{fig:cast-trans}
% \end{figure}

% \todo[inline]{may need to expand scruts if we are using flex vars}

% For another example consider the \slang{} function that validates the transitivity of the \Id{} type, and its \clang{} elaboration in \Fref{cast-trans}.
% % For the \slang{} term the constraints ($\text{Id A' a' a'}=\text{Id A x y}$,$\text{Id A'' a'' a''}\approx\text{Id A y z}$) will be solved and the branch type checks under the substitutions implied by those equalities ($A=A'=A''$, $a'=x=y=a''=z$).
% In the elaborated cast term the variables are not directly equated, instead assertion variables are added to scope to build evidence for these equalities ($v:\text{Id A' a' a'}\approx\text{Id A x y}$, $w:\text{Id A' a' a'}\approx\text{Id A x y}$).
% Elaboration based unification will generate terms that correspond to the equalities discovered by normal unification 
%   ($Aeq=\left(TCon_{0}\ v\right)\cup^{\star}\left(TCon_{0}\ w\right)\sqsupseteq A,A',A''$;
%   and $aeq=\left(TCon_{1}\ v\right)\cup^{\star}\left(TCon_{2}\ v\right)\cup^{\star}\left(TCon_{1}\ w\right)\cup^{\star}\left(TCon_{2}\ w\right)\sqsupseteq a',x,y,a'',z$).
% These assertions are embedded into the constructor allowing $\text{Id A' a' a'}=\text{Id A x z}$.
% Again, the $\mathtt{trans}$ function itself is blameless, any blame surfaces from the use of the term must come from a blamable input.


\subsection{Peeking}
\todo{move to the end?}

% Finally case expressions will be blamed if an incompatible constructor appears.
% For instance, if head is called with $\True{} ::(\Bool{} \ \sim_{\ell}^{\star} \Vect{} \ \Nat{}\ 1)$ then $\ell$ will be blamed immediately, since $\True{} $ does not match the correct type of the constructor $Cons$.
% Since the type constructor is known, it is possible to check the coverage of the patterns.
% If every constructor is accounted for, only blameable data remains.
% Quantifying over casts allows blame to be redirected, so if the program gets stuck in a pattern branch it can blame the malformed input.
% This extension seems to preserve cast soundness.

\begin{figure}
\begin{lstlisting}[basicstyle=\ttfamily\small]
peek : Id Nat 0 1 -> Nat
peek x =
case x <_: Id Nat 0 1 => Nat> {
  | (refl _ x :: w) => x :: (TCon_0 w)
}

-- under weak head evaluation
peek (refl 4 :: Id Nat 0 1) = 4
\end{lstlisting}
\caption{Cast Pattern Matching}
\label{fig:cast-peek}
\end{figure}

% As noted, the \clang{} will enforce a minimal amount of checking, 
Another example of a term that might potentially lead to unexpected behavior is the peek function defined in \Fref{cast-peek}.
$\mathtt{peek}$ will ignore several discrepancies in the index of the \Id{} type, if run in \whnf{}\footnote{
  The example can be extended to \cbv{} with functions, 
  $\mathtt{peek'} : \Id{} (\mathtt{Unit} \rightarrow \Nat{}) (\lambda - \Rightarrow 0) (\lambda - \Rightarrow 1)  \rightarrow  \mathtt{Unit} \rightarrow \Nat{}$.
}.
As in \ch{3}, our formalism uses a minimal amount of checking to maintain cast soundness, though more eager checking is implemented in the prototype.

\section{OLD JUNK}


\subsubsection{Need to remove a cast}
\todo{better name}
 
Consider this surface language expression that extracts the last element from a non-empty list.
Assume the function $last:(n:Nat)\rightarrow Vec\,A\,(S\,N)\rightarrow A$ is in scope.
\todo{or just define the recursive function}
 
\begin{lstlisting}[basicstyle={\ttfamily\small}]
case v <_: Vec A (S x) => A > {
| Cons _ a (Z) _ => a
| Cons _ _ (S n) rest => last n rest
}
\end{lstlisting}
 
This will elaborate into
 
\begin{lstlisting}[basicstyle={\ttfamily\small}]
case v <_: Vec A (S x) => A > {
| (Cons A' a' (Z)::q rest) :: p => a' :: (inTC1(p))-1
| (Cons A' a' (S n)::q rest)::p => last n (rest :: p')
| (Nil A')::p => !TCon1(p)
}
\end{lstlisting}
 
\todo[inline]{prettier expressions, rev to -1, in general it might be clearer if cast language is always in math mode}
 
In the 2nd branch we have
  $A':\star$, $a':A'$, $n:\mathtt{Nat}$,
  $q:\mathtt{Nat}\approx \mathtt{Nat}$,
  $rest:\mathtt{Vec}\,A'\,\left(\left(\mathtt{S}\,n\right)::q\right)$,
  and $p:\mathtt{Vec}\,A'\,\left(\mathtt{S}\left(\left(\mathtt{S}\,n\right)::q\right)\right)\approx \mathtt{Vec}\,A\,\left(\mathtt{S}\,x\right)$.
Elaboration cannot unify a solution unless we can remove casts, otherwise it becomes impossible to construct a path from $n \approx x$
  from $\mathtt{Vec}\,A'\,\left(\left(\mathtt{S}\,n\right)\underline{::q}\right)\approx \mathtt{Vec}\,A\,\left(\mathtt{S}\,x\right)$, since the cast blocks
  $\mathtt{S}\,n = \mathtt{S}\,x$ that would be derivable by unification in the surface language.
We will need an operator that can remove casts from the endpoints of paths that arise from unification.
We will call these operators $uncastL$ and $uncastR$ and they will be derivable in the cast language.
With these operations we can match the process of surface level unification so that
 
$p'\ =\ Cong_{x\Rightarrow \mathtt{Vec}\,A'\,x}\left(UncastR\left(refl\right)\right)\circ Cong_{x\Rightarrow \mathtt{Vec}\,x\,\left(\mathtt{S}\,n\right)}\left(TCon_{0}p\right):\ \mathtt{Vec}\,A'\,\left(\left(\mathtt{S}\,n\right)::q\right)\ \approx\ \mathtt{Vec}\,A\,\left(\mathtt{S}\,n\right)$
 
Where
$UncastR\left(refl\right): \left(\mathtt{S}\,n\right)::q  \approx \mathtt{S}\,n $
 
In the first branch we have,
  $A':\star$, $a':A'$,
  $q:\mathtt{Nat}\approx \mathtt{Nat}$,
  $rest:\mathtt{Vec}\,A'\,\left(\mathtt{Z}::q\right)$,
  and $p:\mathtt{Vec}\,A'\,\left(\mathtt{S}\left(\mathtt{Z}::q\right)\right)\approx \mathtt{Vec}\,A\,\left(\mathtt{S}\,x\right)$.
Unification can proceed to derive $TCon_{1}\left(Con_{0}\left(p\right)\right)^{-1}:\ x\approx \mathtt{Z}::q$
and $TCon_{1}\left(p\right)^{-1}:\ A\approx A'$.
 
In the final branch we have $p:\mathtt{Vec}\,A'\,\mathtt{Z}\approx \mathtt{Vec}\,A'\,\left(\mathtt{S}\,x\right)$,
which is contradicted by $TCon_{1}p:\mathtt{Z}\approx \mathtt{S}\,x$

\todo[inline]{Substantail changes will be made made to the formulaitons bellow}


\todo[inline]{Example back to pattern matching?}


\begin{figure}

term reductions
\[
\frac{p\rightsquigarrow p'}{!_{p}\rightsquigarrow!_{p'}}
\]

\todo[inline]{could remove more then just assertions}

\[
\frac{\,}{\left\{ \star\sim_{k,o,\ell}\star\right\} \rightsquigarrow\star}
\]

\[
\frac{\,}{\left\{ \left(x:A\right)\rightarrow B\sim_{k,o,\ell}\left(x:A'\right)\rightarrow B'\right\} \rightsquigarrow\left(x:\left\{ A\sim_{k,o.arg,\ell}A'\right\} \right)\rightarrow\left\{ B\sim_{k,o.bod\left[x\right],\ell}B'\right\} }
\]

\[
\frac{\,}{\left\{ \mathsf{fun}\,f\,x\Rightarrow b\sim_{k,o,\ell}\mathsf{fun}\,f\,x\Rightarrow b'\right\} \rightsquigarrow\mathsf{fun}\,f\,x\Rightarrow\left\{ b\sim_{k,o.app\left[x\right],\ell}b'\right\} }
\]

\[
\frac{\,}{\left\{ d\overline{a}\sim_{k,o,\ell}d\overline{a'}\right\} \rightsquigarrow d\overline{\left\{ a_{i}\sim_{k,o.DCon[i],\ell}a'_{i}\right\} }}
\]

\[
\frac{\,}{\left\{ D\overline{a}\sim_{k,o,\ell}D\overline{a'}\right\} \rightsquigarrow D\overline{\left\{ a_{i}\sim_{k,o.TCon[i],\ell}a'_{i}\right\} }}
\]

\todo[inline]{double check}

\[
\frac{\overline{a}\ \mathbf{Match}\ \overline{patc}_{i}}{\mathsf{case}\,\overline{a,}\,\left\{ \overline{|\,\overline{patc\Rightarrow}b_{i}}\overline{|\,\overline{patc'\Rightarrow}!_{\ell}}\right\} \rightsquigarrow b_{i}\left[patc_{i}\coloneqq\overline{a}\right]}
\]

\caption{Summery of Cast Language Reductions}
\label{fig:cast-data-red}
\end{figure}

This extension to the syntax induces many more reduction rules. We
include a summery of selected reduction rules in \ref{fig:cast-data-red}.
We do not show the value restrictions to avoid clutter\footnote{there are also multiple ways to lay them out. For instance we could
evaluate paths left to right or right to left.}. 


\begin{figure}
\todo[inline]{swap, pattern on the left?}

\todo[inline]{record substitutions}

\[
\frac{\,}{a\ \mathbf{Match}\ x}
\]

\[
\frac{\overline{a}\ \mathbf{Match}\ \overline{patc}}{d\,\overline{a}\ \mathbf{Match}\ \left(d\,\overline{patc}\right)::x_{p}}
\]

\[
\frac{\overline{a}\ \mathbf{Match}\ \overline{patc}}{\left(d\,\overline{a}\right)::kcast\ \mathbf{Match}\ \left(d\,\overline{patc}\right)::x_{p}}
\]

\[
\frac{\overline{a}\ \mathbf{Match}\ \overline{patc}}{\left(d\,\overline{a}\right)::kcast\ \mathbf{Match}\ \left(d\,\overline{patc}\right)::x_{p}}
\]

\[
\frac{\,}{.\ \mathbf{Match}\ .}
\]

\[
\frac{b\ \mathbf{Match}\ patc'\quad\overline{a}\ \mathbf{Match}\ \overline{patc}}{b\overline{a}\ \mathbf{Match}\ patc'\overline{patc}}
\]


\caption{Cast Language Matching}
\label{fig:cast-data-match}
\end{figure}

\todo[inline]{double check paths are fully applied when needed}





The Cast language extension defined in this chapter is fairly complex.
Though all the meta-theory of this section is plausible, we have not
fully formalized it, and there is a potential that some subtle errors
exist. To be as clear as possible about the uncertainty around the
meta-theory proposed in this chapter, I will list what would normally
be considered theorems and lemmas as conjectures. \todo{weird place to make this note. add it to the front or back matter?}



We now conjecture the core lemmas that could be used to prove cast
soundness
\begin{conjecture}
substitution of cast terms preserves cast

equivalently the following rule is admissible

\end{conjecture}


% ...
\begin{conjecture}
substitution of cast terms preserves path endpoints

equivalently the following rule is admissible

\end{conjecture}

% ...

Finally we will conjecture the cast soundness.
\begin{conjecture}
The cast system preserves types and path endpoints over normalization
\end{conjecture}

% ...
\begin{conjecture}
a well typed path in an empty context is a value, takes a step, or
produces blame
\end{conjecture}

% ...
\begin{conjecture}
A well typed term in an empty context is a value, takes a step, or
produces blame
\end{conjecture}


\section{Elaborating Eliminations}



% elaboration unification
To make the overall system behave as expected we do not want to expose
users to equality patterns, or force them to manually do the path
bookkeeping. To work around this we extend a standard unification
algorithm to cast patterns with instrumentation to remember paths
that were required for the solution. Then if pattern matching is satisfiable,
compile additional casts into the branch based on its assignments.
Unlisted patterns can be checked to confirm they are unsatisfiable.
If the pattern is unsatisfiable then elaboration can use the proof
of unsatisfiability to construct explicit blame. If an unlisted pattern
cannot be proven ``unreachable'' then we could warn the user, and
like most functional programming languages, blame the incomplete match
if that pattern ever occurs.

\subsection{Preliminaries}

The surface language needs to be enriched with additional location
metadata at each position where the two bidirectional typing modalities
would cause a check in the surface language.

\begin{tabular}{lcll}
$m...$ & $\Coloneqq$ & ... & \tabularnewline
 & | & $\mathsf{case}\,\overline{n_{\ell},}\,\left\{ \overline{|\,\overline{pat\Rightarrow}m_{\ell'}}\right\} $ & data elim. without motive\tabularnewline
 & | & $\mathsf{case}\,\overline{n_{\ell},}\,\left\langle \overline{x\Rightarrow}M_{\ell'}\right\rangle \left\{ \overline{|\,\overline{pat\Rightarrow}m_{\ell''}}\right\} $ & data elim. with motive\tabularnewline
\end{tabular}

The implementation allows additional annotations along the motive,
while this works within the bidirectional framework. The syntax is
not presented here since the theory is already quite complicated.\todo{move note somewhere else}

\subsection{Elaboration}

The biggest extension to the elaboration procedure in Chapter 3 is
the path relevant unification and the insertion of casts to simulate
surface language pattern matching. The unification and casting processes
both work without $k$ assumptions in scope, simplifying the possible
terms that may appear.


\begin{figure}
\[
\frac{\,}{U\left(\emptyset,\emptyset\right)}
\]

\[
\frac{U\left(E,u\right)\quad a\equiv a'}{U\left(\left\{ p:a\thickapprox a'\right\} \cup E,u\right)}
\]

\[
\frac{U\left(E\left[x\coloneqq a\right],u\left[x\coloneqq a\right]\right)}{U\left(\left\{ p:x\thickapprox a\right\} \cup E,u\cup\left\{ p:x\thickapprox a\right\} \right)}
\]

\todo[inline]{actually a little incorrect, needs to use conq to concat the paths}

\[
\frac{U\left(E\left[x\coloneqq a\right],u\left[x\coloneqq a\right]\right)}{U\left(\left\{ p:a\thickapprox x\right\} \cup E,u\cup\left\{ p^{-1}:x\thickapprox a\right\} \right)}
\]

\[
\frac{U\left(\left\{ p:a\thickapprox a'\right\} \cup E,u\right)\quad a\equiv d\overline{b}\quad a'\equiv d\overline{b'}}{U\left(\left\{ Con_{i}p:b_{i}\thickapprox b'_{i}\right\} _{i}\cup E,u\right)}
\]

\todo[inline]{fully applied}

\[
\frac{U\left(\left\{ p:a\thickapprox a'\right\} \cup E,u\right)\quad a\equiv D\overline{b}\quad a'\equiv D\overline{b'}}{U\left(\left\{ TCon_{i}p:b_{i}\thickapprox b'_{i}\right\} _{i}\cup E,u\right)}
\]

\todo[inline]{fully applied}


\todo[inline]{break cycle, make sure x is assignable}

\todo[inline]{double check constraint order}

\todo[inline]{correct vars in 4a}

\caption{Surface Language Unification}
\label{fig:surface-data-unification}
\end{figure}


The elaboration procedure uses the extended unification procedure
to determine the implied type and assignment of each variable. In
the match body casts are made so that variables behave as if they
have the types and assignments consistent with the surface language.
The original casting mechanism is still active, so it is possible
that after all the casting types still don't line up. In this case
primitive casts are still made at their given location.

\todo[inline]{add explicit rules for elaboration?}

The elaboration algorithm is extremely careful to only add casts,
this means erasure is preserved and evaluation will be consistent
with the surface language.

Further the remaining properties from Chapter 3 probably still hold
\begin{conjecture}
Every term well typed in the bidirectional surface language elaborates 
\end{conjecture}

% ..
\begin{conjecture}
Blame never points to something that checked in the bidirectional
system 
\end{conjecture}


\section{Discussion and Future Work}


\subsection{Types invariance along paths}

It turns out that the system defined in Chapter 3 had the advantage
of only dealing with equalities in the type universe. Extending to
equalities over arbitrary type has vastly increased the complexity
of the system. To make the system work paths are untyped, which seems
inelegant. There is nothing currently preventing blame across type.
For instance,

$\left\{ 1\sim_{k,o,\ell}false\right\} $ will generate blame $1\neq false$.
While blame of $Nat\neq Bool$ will certainly result in a better error
message. Several attempts were made to encode the type into the type
assumption, but the resulting systems quickly became too complicated
to work with. Some vestigial typing constraints are still in the system
(such as on the explicit blame) to encourage this cleaner blame.

\subsection{Elaboration is non-deterministic with regard to blame}

Consider the case

\begin{lstlisting}[basicstyle={\ttfamily\small}]
case x <_:Id Nat 2 2 => S 2> {
| refl _ a => s a
}
\end{lstlisting}

that can elaborate to

\begin{lstlisting}[basicstyle={\ttfamily\small}]
case x <_:Id Nat 2 2 => S 2> {
| (refl A a)::p => (s (a::TCon0(p)) :: Cong uncastL(TCon1(p)))
}
\end{lstlisting}

where $p:Id\,A\,a\,a\thickapprox Id\,Nat\,2\,2$, where $TCon_{1}p$
selects the first position $p:Id\,A\,\underline{a}\,a\thickapprox Id\,Nat\,\underline{2}\,2$.
But this could also have elaborated to 

\begin{lstlisting}[basicstyle={\ttfamily\small}]
case x <_:Id Nat 2 2 => S 2> {
| (refl A a)::p => (s (a::TCon0(p)) :: Cong uncastL(TCon2(p)))
}
\end{lstlisting}

relying on $p:Id\,A\,a\,\underline{a}\thickapprox Id\,Nat\,2\,\underline{2}$.
This can make a difference if the scrutinee is 

$refl\ Nat\,2::Id\,Nat\,3\,2::Id\,Nat\,2\,2$

in one case blame will be triggered, in the other it will not. In
this case it is possible to mix the blame from both positions, though
this does not seem to extend in general since the consequences of
inequality are undecidable in general and we intend to allow running
programs if they can maintain their intended types.

\subsection{Extending to Call-by-Value}

As in Chapter 3, the system presented here does the minimal amount
of checking to maintain type safety. This can lead to unexpected results,
for instance consider the surface term 

\begin{lstlisting}[basicstyle={\ttfamily\small}]
case (refl Nat 7 :: Id Nat 2 2) <_ => Nat> {
| refl _ a => a
}
\end{lstlisting}

This will elaborate into 

\begin{lstlisting}[basicstyle={\ttfamily\small}]
case (refl Nat 7 :: Id Nat 2 2) <_ => Nat> {
| (refl A a)::p => a::TCon0(p)
}
\end{lstlisting}

which will evaluate to $7::\mathtt{Nat}$ without generating blame.
And indeed we only ever asserted that the result was of type $\mathtt{Nat}$.

In the implementation, some of this behavior is avoided by requiring
type arguments in a cast be run call-by-value. This restriction will
blame $7\neq2$ before the cast is even evaluated.\todo{expand}

\subsection{Efficiency}

The system defined here is brutally inefficient. 

In theory the system has an arbitrary slow down. As in Chapter 3,
a cast that relies on non-terminating code can itself cause additional
non-termination as paths are resolved.

\todo[inline]{paremetricity}

\todo[inline]{relation to fun-ext}

\todo[inline]{warnings}

\subsection{Relation to UIP}

Pattern matching as outlined in the last Chapter (which follows from
\cite{coquand1992pattern}) implies the \textbf{uniqueness of identity
proofs} (UIP)\footnote{Also called \textbf{axiom k}}. UIP states
that every proof of identity is equal to refl (and thus unique), and
is not provable in many type theories. In univalent type theories
UIP is directly contradicted by the ``non-trivial'' equalities,
required to equate isomorphisms and Id. UIP is derivable in the surface
language by following pattern match 

\begin{lstlisting}[basicstyle={\ttfamily\small}]
case x <pr : Id A a a => Id (Id A a a) pr (refl A a) > {
| refl A a => refl (Id A a a) (refl A a)
}
\end{lstlisting}

This type checks since unification will assign $pr\coloneqq refl\,A\,a$
and under that assumption $refl\ (Id\ A\ a\ a)\ (refl\ A\ a):Id\ (Id\ A\ a\ a)\ (refl\ A\ a)\ (refl\ A\ a)$.
Like univalent type theories, the cast language has its own nontrivial
equalities, so it might seem that the cast language would also contradict
UIP . But it is perfectly compatible, and will elaborate. One interpretation
is that though there are multiple ``proofs'' of identity, we don't
care which one is used. \todo{interpretation + take aways?}


\section{Related work}

This work was previously presented as an extended abstract at the
TyDE workshop\todo[inline]{cite}, the version there reflected a less
plausible meta-theory based on earlier implementation experiments.




% \section{Elaboration}

\todo[inline]{resolve var for the typing context of this section: H? Gamma? some other greek?}

% overview
Even though the \clang{} allows us to optimistically assert equalities, manually noting every cast would be cumbersome.
This bureaucracy is solved with an elaboration procedure that translates (untyped) terms from the \slang{} into the \clang{}.
If the term is well typed in the \slang{}, elaboration will produce a term without blamable errors.
Terms with unproven equality in types are mapped to a cast with enough information to point out the original source when an inequality is witnessed.
 
Elaboration serves a similar role as the \bidir{} type system did in \ch{2}, and uses a similar methodology.
Instead of performing a static equality check when the inference mode and the check mode meet, a runtime cast is inserted asserting the types are equal.

In order to perform elaboration, the \slang{} needs to be enriched with location information, $\ell$, at every position that could result in a type mismatch.
This is done in \Fref{surface-pre-syntax-loc}.
Note that the location tags correspond with the check annotations of the \bidir{} system.
For technical reasons the set of locations is nonempty, and a specific null location ($.$)\todo{- instead of . ?} is designated.
That null location can be used when we need to generate fresh terms, but have no sensible location information available.
All the meta theory from \ch{2} goes through assuming that all locations are indistinguishable and by generating null locations when needed\footnote{
  For instance, the parallel reduction relation will associate all locations,
    $\frac{M\Rrightarrow M'\quad N\Rrightarrow N'}{\left(x:M_{l}\right)\rightarrow N_{l'}\Rrightarrow\left(x:M'_{l''}\right)\rightarrow N'_{l'''}}\,\rulename{\Rrightarrow-fun-ty}$,
    so that the relation does not discriminate over syntaxes that come from different locations.
  While the $\textbf{max}$ function will map terms into the null location,
    $\textbf{max}\left(\left(x:M_{\ell}\right)\rightarrow N_{\ell'}\right)=\textbf{max}\left(\left(x:\textbf{max}\left(M\right)_{.}\right)\rightarrow \textbf{max}\left(N\right)_{.}\right)$.
}.\todo{more explanation on this note? why set max up that way?}
We will avoid writing these annotations when they are unneeded (explicitly in \Fref{surface-pre-syntax-loc-abrev}).

\begin{figure}
\begin{tabular}{lcll}
\multicolumn{4}{l}{source labels,}\tabularnewline
$\ell$ & $\Coloneqq$ & ... & \tabularnewline
& $|$ & $.$ & no source label\tabularnewline
\multicolumn{4}{l}{expressions,}\tabularnewline
$m,n,M,N$ & $\Coloneqq$ & $x$ & variable\tabularnewline
& $|$ & $m::_{\ell}M^{\ell'}$ & annotation\tabularnewline
& $|$ & $\star$ & type universe\tabularnewline
& $|$ & $\left(x:M_{\ell}\right)\rightarrow N_{\ell'}$ & function type\tabularnewline
& $|$ & $\mathsf{fun}\,f\,x\Rightarrow m$ & function\tabularnewline
& $|$ & $m_{\ell}\,n$ & application\tabularnewline
\end{tabular}\caption{\SLang{} Syntax with Locations}
\label{fig:surface-pre-syntax-loc}
\end{figure}

\begin{figure}
\begin{tabular}{lclll}
$m::_{\ell}M^{\ell'}$ & written & $m::_{\ell}M$ & when & $\ell$' is irrelevant\tabularnewline
$m::_{\ell}M$ & written & $m::M$ & when & $\ell$ is irrelevant\tabularnewline
$\left(x:M_{\ell}\right)\rightarrow N_{\ell'}$ & written & $\left(x:M\right)\rightarrow N$ & when & $\ell$, $\ell'$ are irrelevant\tabularnewline
$m_{\ell}\,n$ & written & $m_{\ensuremath{}}\,n$ & when & $\ell$ is irrelevant\tabularnewline
\end{tabular}

\caption{\SLang{} Abbreviations}
\label{fig:surface-pre-syntax-loc-abrev}
\end{figure}

\subsection{Examples}

Functions will elaborate the expected types to their arguments when they are applied.
% For example,
\begin{example}
$f:\mathbb{B}_c \rightarrow\mathbb{B}_c \vdash f_{\ell}7_c \ :\mathbb{B}_c $ elaborates to $f:\mathbb{B}_c \rightarrow\mathbb{B}_c \vdash f\left(7_c ::_{\mathbb{N}_c ,\ell,.Arg}\mathbb{B}_c \right)\ :\mathbb{B}_c $.
\end{example}
 
\todo{arg is there because f may not be a function type}
 
As with \bidir{} type checking, variable types will be inferred from the typing environment.
% For example,
\begin{example}
$\vdash(\lambda x\Rightarrow 7_c)::_{\ell}\mathbb{B}_c \rightarrow\mathbb{B}_c$ elaborates to $\vdash(\lambda x\Rightarrow7::_{\mathbb{N}_c,\ell,Bod_x}\mathbb{B}_c )$.
\end{example}
\todo{these examples aer only precisly correct if the church terms infer their type}

To keep the theory simple, we allow vacuous casts to be created,
\begin{example}
$f:\mathbb{N}_c \rightarrow\mathbb{B}_c \rightarrow\mathbb{B}_c \vdash f_{\ell}7_{c\ell'}3_c \ :\mathbb{B}_c $ elaborates to $f:\mathbb{N}_c \rightarrow\mathbb{B}_c \rightarrow\mathbb{B}_c \vdash f\left(7_c ::_{\mathbb{N}_c,\ell,Arg}\mathbb{N}_c \right)\left(3_c ::_{\mathbb{N}_c,\ell',Arg}\mathbb{B}_c \right)\ :\mathbb{B}_c$.
\end{example}

\todo{dependent type example where cast muddles type}

Unlike in gradual typing, we cannot elaborate arbitrary untyped syntax.
The underlying type of a cast needs to be known so that a function type can swap its argument type at application.
For instance, $\lambda x\Rightarrow x$ will not elaborate since the intended type is not known.
Fortunately, our experimental testing suggests that a majority of randomly generated terms can be elaborated, compared to the \slang{} where only a small minority of terms would type check.
The programmer can make any term elaborate if they annotate the intended type.
For instance, $\left(\lambda x\Rightarrow x\right)::*\rightarrow*$ will elaborate.

\subsection{Elaboration Procedure}
\todo{review location placment in light of the implementation}

% rules
Like the \bidir{} rules, the rules for elaboration are broken into two judgments:
\begin{itemize}
\item $H\vdash m\overleftarrow{\,:_{\ell,o}\,}A\,\textbf{Elab}\ a$\todo{fix spacing to right of : under arrow}, that generates a cast term $a$ from a surface term $m$ given its expected type $A$ along with a location $\ell$ and observation $o$ that made that assertion.
\item $H\vdash m\,\textbf{Elab}\ a\overrightarrow{\,:\,}A$, that generates a cast term $a$ and its type $A$ from a surface term $m$.
\end{itemize}
The rules for elaboration are presented in \Fref{elaboration}.
Elaboration rules are written in a style of \bidir{} type checking, with arrows pointing in the direction information flows.
However, unlike \bidir{} type checking, when checking an inference in \rulename{\overleftarrow{\textbf{Elab}}-cast}, elaboration adds a cast assertion that the two types are equal.
Thus any conversion checking can be suspended until runtime.
Additionally we will allow the mode to change at the type universe with the \rulename{\overleftarrow{\textbf{Elab}}-conv-\star} rule, to avoid unneeded checks on the type universe.\todo{i belive this rule is no longer needed? but it is convient}
% Without the \rulename{\overleftarrow{\textbf{Elab}}-conv-\star} rule, the \slang{} term $1_c :: \mathbb{N}_c$ could not elaborate becouse runtime checks required to cherck teh well formedness of $\mathbb{N}_c$ would result in a term $\mathbb{N}_c :: \star$.
As formulated here, the elaboration procedure is terminating.

\begin{figure}
\[
\frac{
  x:A\in H
}{
  H\vdash x\,\textbf{Elab}\,x\overrightarrow{\,:\,}A
}
\rulename{\overrightarrow{\textbf{Elab}}-var}
\]

\[
\frac{\,}{H\vdash\star\,\textbf{Elab}\,\star\overrightarrow{\,:\,}\star}
\rulename{\overrightarrow{\textbf{Elab}}-\star}
\]

\[
\frac{
  H\vdash M\overleftarrow{\,:_{\ell,.}\,}\star\textbf{Elab}\ A\quad H,x:A\vdash N\overleftarrow{\,:_{\ell',.}\,}\star\textbf{Elab}\ B
}{
  H\vdash\left(\left(x:M_{\ell}\right)\rightarrow N_{\ell'}\right)\textbf{Elab}\left(\left(x:A\right)\rightarrow B\right)\overrightarrow{\,:\,}\star
}
\rulename{\overrightarrow{\textbf{Elab}}-fun-ty}
\]

\[
\frac{H\vdash m\,\textbf{Elab}\ b\overrightarrow{\,:\,}\left(x:A\right)\rightarrow B\quad H\vdash n\overleftarrow{\,:_{\ell,Arg}\,}A\,\textbf{Elab}\,a}{H\vdash\left(m_{\ell}\,n\right)\textbf{Elab}\left(b\,a\right)\overrightarrow{\,:\,}B\left[x\coloneqq a\right]}\rulename{\overrightarrow{\textbf{Elab}}-fun-app}
\]

\[
\frac{H\vdash M\overleftarrow{\,:_{\ell',.}\,}\star\,\textbf{Elab}\ A\quad H\vdash m\overleftarrow{\,:_{\ell,.}\,}A\,\textbf{Elab}\ a}{H\vdash\left(m::_{\ell}M^{\ell'}\right)\textbf{Elab}\,a\overrightarrow{\,:\,}A}
\rulename{\overrightarrow{\textbf{Elab}}-::}
\]

\[
\frac{H,f:\left(x:A\right)\rightarrow B,x:A\vdash m\overleftarrow{\,:_{\ell,o.Bod_x}\,}B\,\textbf{Elab}\ b}{H\vdash\left(\mathsf{fun}\,f\,x\Rightarrow m\right)\overleftarrow{\,:_{\ell,o}\,}\left(x:A\right)\rightarrow B\,\textbf{Elab}\left(\mathsf{fun}\,f\,x\Rightarrow b\right)}\rulename{\overleftarrow{\textbf{Elab}}-fun}
\]

\[
\frac{H\vdash m\,\textbf{Elab}\ a\overrightarrow{\,:\,}A}{H\vdash m\overleftarrow{\,:_{\ell,o}\,}B\ \textbf{Elab}\left(a::_{A,\ell,o}B\right)}\rulename{\overleftarrow{\textbf{Elab}}-cast}
\]

\[
\frac{H\vdash m\,\textbf{Elab}\ a\overrightarrow{\,:\,}\star}{H\vdash m\overleftarrow{\,:_{\ell,o}\,}\star\ \textbf{Elab}\,a}
\rulename{\overleftarrow{\textbf{Elab}}-conv-\star}
\]

\todo[inline]{which syntax looks the best? on the left when input, or alway on the right like the typing judgment}
\todo[inline]{macro this stntax}

\caption{Elaboration}
\label{fig:elaboration}
\end{figure}



There are several desirable properties of elaboration that can be shown with the help of an erasure function (defined in \ref{fig:erasure}).
Erasure is defined over all syntactic forms, removing annotations, locations, and casts.

\begin{figure}
\begin{tabular}{ccc}
$|x|$ & = & $x$\tabularnewline
$|\star|$ & = & $\star$\tabularnewline
$|m::_{\ell}M|$ & = & $|m|$\tabularnewline
$|\left(x:M_{\ell}\right)\rightarrow N_{\ell'}|$ & = & $\left(x:|M|\right)\rightarrow|N|$\tabularnewline
$|m_{\ell}\,n|$ & = & $|m|\,|n|$\tabularnewline
$|\mathsf{fun}\,f\,x\Rightarrow m|$ & = & $\mathsf{fun}\,f\,x\Rightarrow|m|$\tabularnewline
$|\lozenge|$ & = & $\lozenge$\tabularnewline
$|\Gamma,x:A|$ & = & $|\Gamma|,x:|A|$\tabularnewline
$|a::_{A,\ell,o}B|$ & = & $|a|$\tabularnewline
$|\left(x:A\right)\rightarrow B|$ & = & $\left(x:|A|\right)\rightarrow|B|$\tabularnewline
$|\mathsf{fun}\,f\,x\Rightarrow b|$ & = & $\mathsf{fun}\,f\,x\Rightarrow|b|$\tabularnewline
$|b\,a|$ & = & $|b|\,|a|$\tabularnewline
$|H,x:M|$ & = & $|H|,x:|M|$\tabularnewline
\end{tabular}
\caption{Erasure}
\label{fig:erasure}
\end{figure}

\begin{thm} Elaborated terms preserve erasure.
 
If $H\vdash m\,\textbf{Elab}\ a\overrightarrow{\,:\,}A$ then $|m|=|a|$.
 
If $H\vdash m\,a\overleftarrow{\,:_{\ell,o}\,}A\,\textbf{Elab}\,a$ then $|m|=|a|$.
\end{thm}
\begin{proof}
By mutual induction on the $\textbf{Elab}$ derivations.
\end{proof}

It follows that whenever an elaborated cast term evaluates, the corresponding surface term evaluates consistently.
Explicitly,
\begin{thm} \Slang{} and \clang{} have consistent evaluation.
 
If $H\vdash m\,\textbf{Elab}\ a\overrightarrow{\,:\,}A$, and $a\rightsquigarrow_{*}\star$ then $m\rightsquigarrow_{*}\star$.
 
If $H\vdash m\overleftarrow{\,:_{\ell,o}\,}A\,\textbf{Elab}\ a$, and $a\rightsquigarrow_{*}(x:A)\rightarrow B$ then there exists $N$ and $M$ such that $m\rightsquigarrow_{*}(x:N)\rightarrow M$.
\end{thm}

\begin{proof}
Since $a\rightsquigarrow_{*}a'$ implies $|a|\rightsquigarrow_{*}|a'|$ and $m\rightsquigarrow_{*}m'$ implies $|m|\rightsquigarrow_{*}|m'|$.
\end{proof}

Elaborated terms are well-cast in a well formed context.
We will use $H\ \textbf{ok}$ to mean for all $x$, $x : A \in H$ then $H \vdash A : \star$.

\begin{thm} Elaborated terms are well-cast.
 
For any $H\,\textbf{ok}$, $H\vdash a\,\textbf{Elab}\,m\overrightarrow{\,:\,}A$ then $H\vdash a:A$, $H\vdash A:\star$.
 
For any $H\,\textbf{ok}$, $H\vdash A:\star$, $H\vdash m\overleftarrow{\,:_{\ell,o}\,}A\,\textbf{Elab}\,a$ then $H\vdash a:A$.

For any $H\,\textbf{ok}$, $H\vdash M\overleftarrow{\,:_{\ell,o}\,}\star\,\textbf{Elab}\,A$ then $H\vdash A:\star$.
\end{thm}
\begin{proof}
By mutual induction on $\textbf{Elab}$ derivations.
\todo{double check}
\end{proof}

Some additional properties are conjectured to hold, though they have not yet been proven.

\begin{conjecture}
Every term well typed in the \bidir{} \slang{} elaborates.
 
If $\Gamma\vdash$, then there exists $H$ such that $\Gamma\,\textbf{Elab}\,H$
 
$\Gamma\vdash m\overrightarrow{\,:\,}M$ then there exists $H$, $a$ and $A$ such that $\Gamma\,\textbf{Elab}\,H$, $H\vdash m\,\textbf{Elab}\ a\overrightarrow{\,:\,}A$
 
$\Gamma\vdash m\overleftarrow{\,:\,}M$ and given $\ell$, $o$ then there exists $H$, $a$and $A$ such that $\Gamma\,\textbf{Elab}\,H$, $H\vdash\textbf{Elab}\,a\,m\overleftarrow{\,:_{\ell o}\,}A$
\end{conjecture}
 
Which if true would lead to the corollary
\begin{conjecture}
Blame never points to something that checked in the \bidir{} system.
 
If $\vdash m\overrightarrow{\,:\,}M$, and $\vdash\textbf{Elab}\ m\,a\overrightarrow{\,:\,}A$, then for no $a\rightsquigarrow_{*}a'$ will \blame{a'}{\ell}{o} occur.
 
\todo{revise precisely with labels}
\end{conjecture}
 
These properties are inspired by the gradual guarantee\cite{siek_et_al:LIPIcs:2015:5031} for gradual typing.
% \section{Suitable Warnings}
 
As presented here, not every cast corresponds to a reasonable warning.
For instance, $\left(\lambda x\Rightarrow x\right)::_{\star\rightarrow\star}\star\rightarrow\star$ is a possible output from elaboration.
By the rules given the cast will not reduce without input, it will never cause blame.
In fact since the user only interacts with the \slang{}, any cast $a::_{A}B$ where $|A|\equiv|B|$ will not produce an understandable warning.
% A reasonable first attempt would be to simply remove the casts of the form $a::_{A}A$, but this ignores the possibility that casts themselves may contain casts\todo{example}.
% Currently the implementation leaves most casts intact and filters our equivalent casts from the warnings shown to the user
 
% \todo{as fig}
 
% \begin{tabular}{llllll}
% $Warns($ & $a::_{A,\ensuremath{\ell},o}B$ & $)=$ & $\left\{ (a,\ensuremath{\ell},o,B)\right\} \cup Warns(a)\cup Warns(A)\cup Warns(B)$ & if & $|A|\cancel{\equiv}|B|$\tabularnewline
% $Warns($ & $a::_{A,\ensuremath{\ell},o}B$ & $)=$ & $Warns(a)\cup Warns(A)\cup Warns(B)$ & if & $|A|\equiv|B|$\tabularnewline
% $Warns($ & $\star$ & $)=$ & $\emptyset$ &  & \tabularnewline
% $Warns($ & $x$ & $)=$ & $\emptyset$ &  & \tabularnewline
% $Warns($ & $\left(x:A\right)\rightarrow B$ & $)=$ & $Warns(A)\cup Warns(B)$ &  & \tabularnewline
% $Warns($ & $\mathsf{fun}\,f\,x\Rightarrow b$ & $)=$ & $Warns(b)$ &  & \tabularnewline
% $Warns($ & $b\,a$ & $)=$ & $Warns(a)\cup Warns(b)$ &  & \tabularnewline
% \end{tabular}
 
% Since the $\equiv$ relation is undecidable an approximation can be used in practice.
% Removing impossible casts should be considered like a compiler optimization.
In \ch{5} casts will be separated from the assertions that they contain, and it will be more clear how to extract warnings.

% \section{Related Work}
 
% \subsection{Dependent types and equality}
 
% \todo{revise as bulleted list starting with ETT}
 
% Difficulties in dependently typed equality have motivated many research projects \cite{HoTTbook,sjoberg2015programming,cockx2021taming}.
% However, these impressive efforts currently require a high level of expertise from programmers.
% Further, since program equivalence is undecidable in general, no system will be able to statically verify every ``obvious'' equality for arbitrary user defined data types and functions.
% In the meantime systems should trust the programmer when they use an unverified equality, and use that advanced research to suppress warnings.
 
\subsection{\Bidir{} Placement of Casts}
\todo{better title}
 
This is not the first work to use \bidir{} type checking to place errors.
The Haskell compiler, GHC, supplements Hindley-Minler style type checking with bidirectionality to localize error messages.
This approach was extended in \cite{10.1145/2364527.2364554} which weakens the regular type checking to allow runtime casts\footnote{
 Available with the $\mathtt{-fdefer-type-errors}$ compiler flag.}.
The casts themselves are different from the ones described here since they do not optimistically compute, they will only give errors when reached.
Though more restrictive than our casts, that system enforces parametricity, which makes sense in the context of Haskell. 
% The only other work we are aware of

\todo{talk about the java paper it cites}
 
\todo[inline]{talk about dependent haskell?}

\subsection{Contract Systems}
 
Several of the tricks and notations in this Chapter find their basis in the large amount of work on higher order contracts and gradual types.
Higher order contracts were introduced in \cite{10.1145/581478.581484} as a way to dynamically enforce invariants of software interfaces, specifically higher order functions.
% contracts go back to the 70s apparently, but this seems a reasonable place to start the story, though Bigloo Scheme [28] is cited there.
The notion of blame dates at least that far back.
Swapping the type cast of the input argument of a function type is reminiscent of that paper's use of blame contravariance, though it is presented in a much different way.
% However the contract language of that paper was somewhat limited.
 
Contract semantics were revisited in \cite{10.1145/1925844.1926410,10.1007/978-3-642-28869-2_11} where a more specific correctness criteria based on blame is presented.

Contract systems still generally rely on users annotating their intentions explicitly.
Similar to how programmers might include $\mathtt{assert}$s in an imperative language.
In this thesis annotations are added automatically though elaboration based on type annotations.
 
While there are similarities between contract systems and the \csys{} outlined here, the \csys{} is designed to address only issues with definitional equality in a dependent type theory.
Since contract systems are generally used in untyped languages with contracts written in the host language, definitional equality simply isn't applicable in the vast majority of contract systems.

\todo[inline]{discuss why contracts are a bad fit for this problem}

\subsubsection{Gradual Types}
 
Types can be viewed as a very specific form of contracts that are usually enforced statically.
\textbf{Gradual type systems} allow for a mixing of the static type checking and dynamic type assertions.
Often type information can be inferred using standard techniques, allowing programmers to write fewer annotations.
\todo[inline]{cite first paper, DLS2008?}\todo[inline]{talk about type imprecision}
 
Gradual type systems usually achieve this by adding a $?$ meta character into the type language to denote imprecise typing information.
The first popular account of gradual type semantics appeared in \cite{siek_et_al:LIPIcs:2015:5031} with the alliterative ``gradual guarantee'' which has inspired some of the properties targeted in this Chapter.
 
% , which informally asserts that "runtime checks will not change the expected behavior", "runtime checks will not change the expected behavior" and "well typed code won't be blamed"
% and then like an endless back and forth over criteria
 
Additionally some of the formalism from this Chapter were inspired by the ``Abstracting gradual typing'' methodology \cite{10.1145/2837614.2837670}, where static evidence annotations become runtime checks.
% Unlike some impressive attempts to gradualize the polymorphic lambda calculus \cite{10.1145/3110283}, our system does not attempt to enforce any parametric properties of the base language. %example?
% It is unclear if such a restriction would be desirable for a dependently typed language in practice.
 
This thesis borrows some notational conventions from gradual typing such as the $a::A$ construct for type assertions.
 
A system for gradual dependent types has been proposed in \cite{10.1145/3341692}.
That paper is largely concerned with establishing a decidable type checking procedure via an approximate term normalization.
However, that system retains the conventional style of definitional equality, so that it is possible, in principle, to get $\Vect{}\,(1+x)\neq \Vect{}\,(x+1)$ as a runtime error.
Additionally it is unclear if adding the $?$ meta-symbol into an already very complicated type theory is easier or harder from the programmer's perspective.
 
The common motivation for gradual type systems is to gradually convert a code base from untyped to (usually simply) typed code.
% This chapter has a much tighter scope than the other work cited here, dealing only with equational assumptions.
However, anyone choosing to use a dependent type system has already bought into the usefulness of types in general and will probably not want fragments of completely untyped code.
Gradually converting untyped code to include dependent types is far less plausible than gradually converting untyped code to use simple types.
Especially considering that most real-life codebases will use effects, while adding effects into a simply typed programming language is straightforward, mixing dependent types and effects is a complicated area of ongoing research.

While the gradual typing goals of mixing static certainty with runtime checks are similar to our work here, the approach and details are different.
Instead of trying to strengthen untyped languages by adding types, we take a dependent type system and weaken it with a cast operator.
This leads to different trade-offs in the design space.
For instance, we cannot support completely unannotated code, but we do not need to complicate the type language with a $?$ meta-symbol for uncertainty.

One might characterize this work in this Chapter as gradualizing only the definitional equality relation with a degenerate notion of imprecision.
 
\subsubsection{Blame}
 
Blame is one of the key ideas explored in the contract type and gradual types literature\cite{10.1007/978-3-642-00590-9_1,wadler:LIPIcs:2015:5033,10.1145/3110283}.
Often the reasonableness of a system can be judged by the way blame is handled\cite{wadler:LIPIcs:2015:5033}.
% Blame is treated in \cite{wadler:LIPIcs:2015:5033} very similarly to the presentation in this chapter.
This Chapter goes beyond blaming a source location and also tracks a witnessing observation that can also be made.
 
\subsection{Refinement Style Approaches}
 
This thesis describes a \fullSp{} dependently typed language.
This means computation can appear uniformly in both term and type position.
An alternative approach to dependent types is found in \textbf{refinement type systems}.
 
Refinement type systems restrict type dependency, possibly to specific base types such as $\mathtt{int}$ or $\mathtt{bool}$.
Under this restriction, it is straightforward to check type level equalities and additional properties hold at runtime.
 
One approach which explores this is \textbf{hybrid type checking} \cite{10.1145/1111037.1111059,10.1145/1667048.1667051} which performs ``static analysis where possible, ... dynamic checks where necessary''.
% This is a very similar goal to what has been proposed in this Chapter.
However, there are several differences in that work: they have a simply typed system, static warnings for programmers are not considered, and type checking can reject "clearly ill-typed programs".
For the system defined in this thesis there is no clear boundary between clearly ill-typed programs and subtly ill-typed programs, so we treat all potential inequalities uniformly with a static warning and a runtime check.

\todo{need to talk about soft typeing?}
% 'Hybrid type checking is inspired by prior work on soft typing [Fagan 1990;
% Wright and Catwright 1994; Aiken et al. 1994; Flanagan et al. 1996], but it
% extends soft typing by rejecting many ill-typed programs, in the spirit of static
% type checkers.'

Another notable example is \cite{10.1007/1-4020-8141-3_34} which describes a refinement system that limits predicates to base types.
Another example is \cite{10.1145/3093333.3009856}, a refinement system treated in a specifically gradual way.
A refinement type system with higher order features is gradualized in \cite{c4be73a0daf74c9aa4d13483a2c4dd0e}.
\todo{why is full spectrum better?}
 
\todo{cite my abstract}
 
% consider also citing https://www.youtube.com/watch?v=gIYMERs7AZQ https://www.youtube.com/watch?v=EGKeWg2ES0A
