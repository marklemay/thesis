\chapter{Data in the \CLang{}}
\label{chapter:CastData}
\thispagestyle{myheadings}
 
%Surprisingly, the \csys{} can be extended with a pattern matching construct without unification.
\Ch{3} showed how to use the \ac{TAS} and the \bidir{} system as a guide to build a dependently typed language with runtime equality.
The \ac{TAS} inspired the \csys{}, where the properties and lemmas of the \ac{TAS} can be extended with casts.
While the \bidir{} system suggested how to localize uncertain assumptions that can be repurposed by elaboration as equality checks.
 
In this Chapter we will extend these systems for dependently indexed data and pattern matching.
This will turn out to be more complicated than the system in \ch{3}, for two reasons.
First, equality was only testable at types in \ch{3} which allowed for some syntactic and semantic shortcuts.
In the presence of dependent data, equality needs to be testable at terms, which will not necessarily have the same type.
Second, the subtleties of pattern matching will need to be dealt with.
While the intuition built up in \ch{3} still holds, the \clang{} will need to be revised.

As before, we will take the (conjectured) \slang{} of \ch{4} and construct a \clang{} with corresponding features.
Though it is difficult to formalize a \ac{TAS} and corresponding \bidir{} system that has pattern matching, we will assume the unification of pattern matching belongs in the \bidir{} system since it exists only to establish static correctness and is not needed for evaluation.
Accordingly we will extend elaboration with a form of unification.
Because we will not need to deal with unification in the \clang{}, the \clang{} can provide evidence that the \csys{} is \textbf{cast sound}. \todo{"show" if we ever get a proof}
While the lack of a formal \ac{TAS} and \bidir{} system in \ch{4} will make the other properties of \ch{3} impossible to prove here, we will design the system with an eye towards preserving them.

% \todo{formulate case consistently}
% Elaboration should satisfy the additional correctness\todo{``gradual correctness''?} properties relative to a type assignment system and \bidir{} system.
% In this case we will target the conjectured TAS from Chapter 4, with a first order unification style pattern matching.

Despite these caveats, there is an interesting interpretation of data and pattern matching when extended to the \csys{}.

In a conventionally typed language, the normal forms of data terms have a valid data constructor in the head position (justifying the syntax of pattern matching).
In the \clang{}, the normal form of data can have casts applied to an expression.
If the casts are blameless then the constructor in the head position will match the data type.
% Casts are wrapped around terms during the elaboration procedure, and will accumulate during evaluation.
In the \clang{} pattern matching is extended with a path variable that can represent evidence of equality, then that evidence can be extracted and used in the body of the branch.
\todo{needs an example figure?}

As in conventional pattern matching, since the type constructor is known, it is possible to check the coverage against all possible constructors.
If every constructor is accounted for, only blameable \scruts{} are possible.
Quantifying over evidence of equality allows blame to be redirected, so if the program gets stuck in a pattern branch it can blame the original faulty assumption.

To account for ``unreachable patterns'' that are not stated in the \slang, we can record the proof of inequality for use at runtime.
Since in the \clang{}, it is possible for a \case{} expression to reduce into one of those ``unreachable'' branches.
If this happens blame will be reflected back onto a specific problematic assumption of the input.
This will involve extending the \clang{} with operations to manipulate equality evidence directly.
Proofs of inequality will be able to appear in terms.
 
% During elaboration, after a pattern is unified, we will inject the proofs of equality where needed so that they cast check. 
% This will require that our notion of paths support congruence.


\section{Examples}
 
Consider some of the following examples of how \slang{} pattern matches might elaborate.
 
\subsection{Head}
 
In the \slang{} the first element of $x$ can be extracted with,
 
\begin{lstlisting}[basicstyle={\ttfamily\small}]
case x <_:Vec Bool (S n) => Bool> {
| Cons _ a _ _ => a
}
\end{lstlisting}

Where $x$ has the apparent type $\Vect{}\,\Bool{}\,(\Suc{}\,n)$.

What can go wrong in the presence of casts?
\begin{itemize}
\item
A blamable cast may have made $x$ appear to be a \Vect{} even when it is not.
For instance, $\True{}::_{\ell}\Vect{}\,\Bool{}\,3$ (in \ch{3} notation).
\item
The vector may be empty but cast to look like it is inhabited.
For instance, $\mathtt{Nil}\,\mathtt{Bool}\,::_{\ell}\mathtt{Vec}\,\mathtt{Bool}\,5$.
\item
The vector may contain elements that are not $\mathtt{Bool}$.
For instance, $\mathtt{Cons}\,\Nat{}\,3\,...\,...::_{\ell}\mathtt{Vec}\,\mathtt{Bool}\,5$.
\end{itemize}
 
To handle these issues, elaboration can generate the following cast langngae term,
 
\begin{lstlisting}[basicstyle={\ttfamily\small}]
case x {
| (Cons A a y _) :: p => a::(TCon_0(p))
| (Nil A) :: p => !TCon_1(p)
}
\end{lstlisting}
\todo{subscripts in listings}
 
The elaborated \case{} expression covers all possible constructors for the data type constructor $\mathtt{Vec}$, including patterns that did appear in the surface term.
Then unification solves the constraints to help elaborate the body.
 
In the first branch, the pattern captures typed variables, $A:\star$, $a:A$, $y:\Nat{}$, while $p$ is a path variable that contains evidence that the type of $\mathtt{Cons\,A\,a\,y\,-}$ is $\mathtt{Vec\,Bool\,(S\,n)}$.
So we will say, $p:\mathtt{Vec}\ A\ (\Suc{}\,y)\approx\mathtt{Vec}\,\mathtt{Bool}\,(\Suc{}\,n)$.
$TCon_{0}\,p$ extracts the 0th argument from the type constructor $p:\mathtt{Vec}\ \underline{A}\ (\Suc{}\,y)\approx\mathtt{Vec}\,\underline{\mathtt{Bool}}\,(\Suc{}\,n)$ resulting in the type $TCon_{0}p\ :\ A\approx\mathtt{Bool}$.
The body of the branch casts $a$ along $TCon_{0}\,p$ to $\mathtt{Bool}$.
Casts will need to be generalized from \ch{3} to contain evidence of equality.
 
In the the second branch, the pattern match gives $A:\star$, $p:\mathtt{Vec}\ A\ \mathtt{Z}\approx\mathtt{Vec}\,\mathtt{Bool}\,(\Suc{}\,y)$.
The body of that branch encodes the contradiction using explicit blame syntax by observing $\mathtt{Z}\neq\Suc{}\,y$ with $TCon_{1}p$.
Any match in that branch must be blameable.
 
Since there is no assertion made in either branch, no warnings will be reported for this elaborated \case{} term.
Any failure that arises will be redirected to the \scrut{}, which must have made a blameable assertion.
 
Again consider the ways $x$ could go wrong,
\begin{itemize}
\item
If the user tries to eliminate $x=\mathtt{True}::\mathtt{Vec}\,\mathtt{Bool}\,3$, the type constructor is not matched so the faulty assumption can be blamed automatically.
\item
If the \scrut{} is an empty \Vect{}, we will fall into the $\mathtt{Nil}$ branch, which will reflect the underlying faulty assumption, via the explicit blame syntax.
\item
If the \Vect{} is inhabited by an incorrect type, such as $\mathtt{Cons}\,\Nat{}\,3\,...\,...::_{\ell}\mathtt{Vec}\,\mathtt{Bool}\,5$, the \case{} will return $3::_{\ell,...}\mathtt{Bool}$\todo{fill in ..., typeing jusdment and undeline the other?} with a cast that rests on the blamable assertion of $\mathtt{Vec}\,\Nat{}\,5\approx\mathtt{Vec}\,\mathtt{Bool}\,\,5$.
When exactly this blame will surface depends on the evaluation and checking strategies.
In the implemented language \cbv{} and check-by-value are used at runtime and the blame will surface before the pattern match.
Using a \whnf{} strategy the blame will be embedded in the resulting term and discovered whenever that term is eliminated. % , as in the implemented elaborator, 
\end{itemize}
 
\subsection{Sum}
 
The body of a pattern match may need to make use of type level facts discovered from the pattern match.
For instance, in the \slang{} we can  sum the two numbers in a \Vect{} of length $2$ with
 
\begin{lstlisting}[basicstyle={\ttfamily\small}]
case x <_:Vec Nat 2 => Nat> {
| Cons _ i _ (Cons _ j _ _) => i+j
}
\end{lstlisting}
 
The elaboration procedure will produce
 
\begin{lstlisting}[basicstyle={\ttfamily\small}]
case x {
| (Cons Nat' i n' (Cons Nat'' j n'' rest):: p1):: p2 =>
  i::(TCon_0(p2)) + j::(TCon_0(p1) U TCon_0(p2))
| (Nil Nat') :: p =>
  !TCon_1(p)
| (Cons Nat' i n' (Nil Nat''):: p1):: p2 =>
  !(TCon_1(p1) U DCon_0(TCon_1(p2)))
}
\end{lstlisting}
 
\begin{itemize}
\item
In the first branch we have the variables in scope, $Nat':\star$, $Nat'':\star$, $i:Nat'$, $j:Nat''$, $p1:\mathtt{Vec}\ Nat''\ (\Suc{}\ n'')\approx\mathtt{Vec}\ Nat'\ n'$, and $p2:\mathtt{Vec}\ Nat''\ (\Suc{}\ n')\approx\mathtt{Vec}\ \Nat{}\ 2$.
\todo{rename Nat' -> N, Nat'' -> N' }
\begin{itemize}
\item
This means the elaborator can construct $TCon_0(p2):Nat'\approx\Nat{}$, and $TCon_0(p1):Nat''\approx Nat'$.
Thes facts can be combined to show $TCon_0(p1) \cup TCon_0(p2):Nat''\approx\Nat{}$.
\item
The elaborator knows what the type of every sub expression is supposed to be, so casts can be injected using evidence from the pattern.
\end{itemize}
\item
In the 2nd branch we have, $p:\mathtt{Vec}\ Nat''\ 0\approx\mathtt{Vec}\ Nat'\ 2$.
\begin{itemize}
\item
Which is contradictory, by $TCon_1(p):0\approx2$.
\end{itemize}
\item
\todo{note that Dcon needs to eat the S constructor}
In the 3rd branch, $p1:\mathtt{Vec}\ Nat''\ 0\approx\mathtt{Vec}\ Nat'\ n'$, $p2:\mathtt{Vec}\ Nat'\ (\Suc{}\ n')\approx\mathtt{Vec}\ \Nat{}\ 2$.
\begin{itemize}
\item
Which is unsatisfiable by $TCon_1(p1) \cup DCon_0(TCon_1(p2)):0\approx1$.
We don't need to know which sub path is problematic beforehand, only that the combination causes trouble.
If this branch is reached, we can observe a problem in at least one path.
\end{itemize}
\end{itemize}
 
\subsection{Missing Branches}
\todo{this example is a little meh, better with something that needs path evidence like ID}
What about unstated branches that cannot be excluded with type information?
% For instance, variables may not be used in the body of a branch.
Consider this partial pattern match where $\mathtt{rept}\ :\ (x: \Nat{}) \rightarrow \Vect{}\,\Bool{}\,x$,
 
\begin{lstlisting}[basicstyle={\ttfamily\small}]
case x <x: Nat => Vec Bool x> {
| 2 => rept 2
}
\end{lstlisting}
 
will elaborate to
\begin{lstlisting}[basicstyle={\ttfamily\small}]
case x  {
| S (S (Z :: _) :: _) :: _ => rept 2
? Z :: _
? S (Z :: _) :: _
? S (S (S _ :: _) :: _)
}
\end{lstlisting}
 
Substitution can confirm that the explicit branch has exactly the type of the motive and does not need a cast\footnote{
  While it is possible that blame was embedded in the $\mathtt{(S (S (Z :: -) :: -) :: -)}$ term, the \csys{} will allow $\mathtt{(S (S (Z :: -) :: -) :: -)}\equiv2$.
}.
Additionally the elaborator will form a covering of implicit patterns that handle any possible constructor.
Since the unifier cannot find a contradiction for any of these cases, the user will be warned of possible runtime errors.
 
\subsection{Congruence (embedding equalities in terms)}
This surface expression that takes in a propositional proof that $2=2$ and uses the named witness to generate a vector of length 2, demonstrates some of the subtler possibilities that arise in dependently typed pattern matching.
% This will typecheck in the \slang{}.

\begin{lstlisting}[basicstyle={\ttfamily\small}]
case x <_:Id Nat 2 2 => Vec Bool 2> {
| refl _ a => rep Bool True a
}
\end{lstlisting}
 
This will elaborate to
 
\begin{lstlisting}[basicstyle={\ttfamily\small}]
case x {
| (refl N a)::p =>
 (rep Bool True (a :: (TCon_0(p)))) 
   :: Vec Bool (TCon_1(p))
}
\end{lstlisting}
\todo{standardize on the associativity of pat-match}

In the branch, $N:\star$, $a:N$, and $p:\mathtt{Id}\ N\ a\ a\approx\mathtt{Id}\ \Nat{}\ 2\ 2$.
Since we have $p:\mathtt{Id}\ N\ a\ a\approx\mathtt{Id}\ \Nat{}\ 2\ 2$, we can derive $TCon_0(p):\ N\ \approx\ \mathtt{Nat}$.
Which can be used in $a::(TCon_0(p))$ to cast $a$ from $N$ to \Nat{}.
But then we need evidence that $\Vect{}\, \Bool{}\, (a :: (TCon_0(p)))\ \approx\ \Vect{}\, \Bool{}\, 2$ to avoid a sperous assertion\todo{
  give an example of an assertion to remind people it is possible
}.
\todo[inline]{avoiding these sperous casts is important for the conjectured gradual correctness, well typed \slang{} terms should not generate warnings}
First, we need to select the subterm of interest, $\Vect{}\, \Bool{}\, \underline{(a :: (TCon_0(p)))}\ \approx\ Vect{}\, \Bool{}\, \underline{2}$.
Equality evidence is constructed specifically so that it can be embedded into terms.
If we have evidence, $q$, such that $q\ :\ (a ::(TCon_0(p)))\ \approx\ 2$ then $\Vect{}\, \Bool{}\, \underline{q}\ :\ \Vect{}\, \Bool{}\, \underline{(a :: (TCon_0(p)))}\ \approx\ \Vect{}\, \Bool{}\, \underline{2}$.
% The $Cong$ syntax explicitly embeds a path into a larger expression, here $Cong_{x=> (A : *) -> A -> \mathtt{Vec} \mathtt{Nat} x} ...$ selects the relevant part of the type.
 
The \csys{} will only require that terms are equated up to a definitional equality that disregards casts so instead of needing to show $a :: (TCon_0(p))\ \approx\ 2$, we only have to show $a\ \approx\ 2$.
Which we have in $TCon_1(p)\ :\ a\ \approx\ 2$ and $TCon_2(p)\ :\ a\ \approx\ 2$.
The elaborator can choose either to get a well cast term, and while the pattern will behave consistently on blameless terms, different behavior is possible when blame is discoverable.

For instance, given the elaboration above,
 
\begin{itemize}
  \item if $x$ is $\mathtt{refl}\,\Nat{}\,2\ ::\ \Id{}\,\Nat{}\,0\,2\ ::\ \Id{}\,\Nat{}\,2\,2$ then blame will be discoverable from the $TCon_1$ observation.
  \item if $x$ is $\mathtt{refl}\,\Nat{}\,2\ ::\ \Id{}\,\Nat{}\,2\,0\ ::\ \Id{}\,\Nat{}\,2\,2$ then blame will not be discoverable and a blameless \Vect{} is constructed.
\end{itemize}
In general there is no way around this, equality evidence may be constructable in subtle ways.
Not everything can be checked.

\todo[inline]{Example: translate out to motive}

% \subsection{Transitivity}

% \begin{figure}
%   \begin{lstlisting}[basicstyle=\ttfamily\small]
%   -- surface language term
%   trans : (A : *) -> (x : A) -> (y : A) -> (z : A)
%         -> (xy : Id A x y) -> (yz : Id A y z) -> Id A x z
%   trans A x y z xy yz =
%   case xy, yz < _ => _ => Id A x z > {
%     | (refl A' a') => (refl A'' a'') => (refl A' a')
%   } ;
  
%   -- elaborated cast language term
%   trans : (A : *) -> (x : A) -> (y : A) -> (z : A)
%         -> (xy : Id A x y) -> (yz : Id A y z) -> Id A x z
%   case xy, yz {
%     | (refl A' a')::v => (refl A'' a'')::w => 
%       (refl A' a'):: (Id Aeq aeq aeq)
%   } ;
% \end{lstlisting}

% \caption{Transitivity Example}
% \label{fig:cast-trans}
% \end{figure}

% \todo[inline]{may need to expand scruts if we are using flex vars}

% For another example consider the \slang{} function that validates the transitivity of the \Id{} type, and its \clang{} elaboration in \Fref{cast-trans}.
% % For the \slang{} term the constraints ($\text{Id A' a' a'}=\text{Id A x y}$,$\text{Id A'' a'' a''}\approx\text{Id A y z}$) will be solved and the branch type checks under the substitutions implied by those equalities ($A=A'=A''$, $a'=x=y=a''=z$).
% In the elaborated cast term the variables are not directly equated, instead assertion variables are added to scope to build evidence for these equalities ($v:\text{Id A' a' a'}\approx\text{Id A x y}$, $w:\text{Id A' a' a'}\approx\text{Id A x y}$).
% Elaboration based unification will generate terms that correspond to the equalities discovered by normal unification 
%   ($Aeq=\left(TCon_{0}\ v\right)\cup^{\star}\left(TCon_{0}\ w\right)\sqsupseteq A,A',A''$;
%   and $aeq=\left(TCon_{1}\ v\right)\cup^{\star}\left(TCon_{2}\ v\right)\cup^{\star}\left(TCon_{1}\ w\right)\cup^{\star}\left(TCon_{2}\ w\right)\sqsupseteq a',x,y,a'',z$).
% These assertions are embedded into the constructor allowing $\text{Id A' a' a'}=\text{Id A x z}$.
% Again, the $\mathtt{trans}$ function itself is blameless, any blame surfaces from the use of the term must come from a blamable input.


\subsection{Peeking}
\todo{move to the end?}

% Finally case expressions will be blamed if an incompatible constructor appears.
% For instance, if head is called with $\True{} ::(\Bool{} \ \sim_{\ell}^{\star} \Vect{} \ \Nat{}\ 1)$ then $\ell$ will be blamed immediately, since $\True{} $ does not match the correct type of the constructor $Cons$.
% Since the type constructor is known, it is possible to check the coverage of the patterns.
% If every constructor is accounted for, only blameable data remains.
% Quantifying over casts allows blame to be redirected, so if the program gets stuck in a pattern branch it can blame the malformed input.
% This extension seems to preserve cast soundness.

\begin{figure}
\begin{lstlisting}[basicstyle=\ttfamily\small]
peek : Id Nat 0 1 -> Nat
peek x =
case x <_: Id Nat 0 1 => Nat> {
  | (refl _ x :: w) => x :: (TCon_0 w)
}

-- under weak head evaluation
peek (refl 4 :: Id Nat 0 1) = 4
\end{lstlisting}
\caption{Cast Pattern Matching}
\label{fig:cast-peek}
\end{figure}

% As noted, the \clang{} will enforce a minimal amount of checking, 
Another example of a term that might potentially lead to unexpected behavior is the peek function defined in \Fref{cast-peek}.
$\mathtt{peek}$ will ignore several discrepancies in the index of the \Id{} type, if run in \whnf{}\footnote{
  The example can be extended to \cbv{} with functions, 
  $\mathtt{peek'} : \Id{} (\mathtt{Unit} \rightarrow \Nat{}) (\lambda - \Rightarrow 0) (\lambda - \Rightarrow 1)  \rightarrow  \mathtt{Unit} \rightarrow \Nat{}$.
}.
As in \ch{3}, our formalism uses a minimal amount of checking to maintain cast soundness, though more eager checking is implemented in the prototype.
\section{Syntax}
%% overview

%% pre-syntax
The syntax for the cast language can be seen in \Fref{cast-data-pre-syntax}.
There are several differences to note from \ch{3}.

\begin{figure}
  \begin{tabular}{lcll}

\multicolumn{3}{l}{variables,} & \tabularnewline
  $x,f,p,q$ &  &  & \tabularnewline
\multicolumn{3}{l}{pattern, } & \tabularnewline
  $patc$ & $\Coloneqq$ & $x$ & \tabularnewline
    & $|$ & $\left(d\overline{patc}::p\right)$ & \tabularnewline
  \multicolumn{3}{l}{cast expressions,} & \tabularnewline
  $a,b$,$A,B,C,L$ & $\Coloneqq$ & $x$ & \tabularnewline
    & $|$ & $a::C$ & cast\tabularnewline
    & $|$ & $\star$ & \tabularnewline
    & $|$ & $\left(x:A\right)\rightarrow B$ & \tabularnewline
    & $|$ & $\mathsf{fun}\,f\,x\Rightarrow b$ & \tabularnewline
    & $|$ & $b\,a$ & \tabularnewline
    & $|$ & $a\sim_{\ell,o}^{C}b$ & assertion\tabularnewline
    & $|$ & $a\cup^{C}b$ & union\tabularnewline
    & $|$ & $!_{L}^{L'}$ & force blame\tabularnewline
    & $|$ & $D_{\Delta}$ & type cons.\tabularnewline
    & $|$ & $d_{\Delta\rightarrow D\overline{a}}$ & data cons.\tabularnewline
    & $|$ & $\mathsf{case}\,\overline{a,}\,\left\{ \overline{|\,\overline{patc\Rightarrow}b}\ \overline{?_{\ell}\,\overline{patc}} \right\} $ & data elim.\tabularnewline
    & $|$ & $TCon_{i}\ L$ & \tabularnewline
    & $|$ & $DCon_{i}\ L$ & \tabularnewline
  \multicolumn{4}{l}{observations,}\tabularnewline
  o & $\Coloneqq$ & . & \tabularnewline
    & $|$ & $o.Arg$ & \tabularnewline
    & $|$ & $o.App_{a}$ & \tabularnewline
    & $|$ & $o.Bod_{a}$ & \tabularnewline
    & $|$ & $o.TCon_{i}$ & type cons. index\tabularnewline
    & $|$ & $o.DCon_{i}$ & data cons. arg.\tabularnewline
  \multicolumn{4}{l}{contexts,}\tabularnewline
    $\varGamma$ & $\Coloneqq$ & . & \tabularnewline
      & $|$ & $x:A$ & \tabularnewline
      & $|$ & $x_{p}:a:A\approx b:B$ & \tabularnewline
      & $|$ & $\Gamma,\mathsf{data}\,D\,:\,\Delta\rightarrow\star\,\left\{ \overline{|\,d\,:\,\varTheta\rightarrow D\overline{a}}\right\} $ & data definition\tabularnewline
      & $|$ & $\Gamma,\mathsf{data}\,D\,:\,\Delta\rightarrow\star$ & abstract data\tabularnewline
  \end{tabular}
% time permitting factor the location to where . is
\caption{Cast Language Syntax}
\label{fig:cast-data-pre-syntax}
\end{figure}

\todo{abreviations}

In \ch{3} casts did double duty in both asserting an equality and changing the type of an underlying term.
Now casts will hold evidence that may change the type of a term, and equality assertions get a new syntactic form with $a\sim_{\ell,o}^{C}b$.
For instance, if we wanted to assume $\Nat{} $ and $\Bool{} $ are the same we could write $\Nat{} \sim_{\ell}^{\star}\Bool{} $ given a location $\ell$ to blame, since $\star$ shares the type of both $\Nat{} $ and $\Bool{} $.
The cast operation will allow $1$ to be used as a $\Bool{} $ by casting the assumption, $1::\left(\Nat{} \sim_{\ell}^{\star}\Bool{} \right)$.

The presence of assertions allows an expression to have different interpretations.
Assertions that two terms are the same is written as $\left\{ a\sim_{\ell,o}^{C}b\right\}$ and will evaluate $a$ and $b$ in parallel until a head constructor is reached on each branch. 
For instance, $\left(1 \sim_{\ell}^{\Nat{}} 2 \right)$, can be interperted as $1$, $2$ or evidence that $1=2$.
We call the concrete interpretations \textbf{endpoint}s, and they will be formalized in the next section.
If the constructor is the same it will commute out of the term.
If the head constructor is different the term will get stuck with the information for the final blame message.
For instance, $\left(1 \sim_{\ell}^{\Nat{}} 2 \right)\rightsquigarrow_{*} S\left(0 \sim_{\ell.DCon_{0}}^{\Nat{}} 1\right) $.

% Since this equational construct is already needed for functions, we
% will use it to handle all equality assertions. For instance, $\left\{ \left(\lambda x\Rightarrow S\,x\right)Z\sim_{k,o,\ell}2+2\right\} \rightsquigarrow_{*}\left\{ S\,Z\sim_{k,o,\ell}S\,\left(1+2\right)\right\} \rightsquigarrow_{*}S\,\left\{ Z\sim_{k,o.DCon_{0},\ell}1+2\right\} \rightsquigarrow_{*}S\,\left\{ Z\sim_{k,o.DCon_{0},\ell}S\,2\right\} $.
% We compute past the first $S$ constructor and blame the predecessor
% for not being equal.

Assertions can be chained together with $\cup$ when two expressions that share an endpoint, and evidence that the types are the same.
For instance if we assumed $\Nat{} \sim_{\ell}^{\star}\Bool{} $ and $\Bool{} \sim_{\ell'}^{\star}String$, then we can associate $\Nat{} $ and $String$, with $\left(\Nat{} \sim_{\ell}^{\star}\Bool{} \right)\cup^{\star}\left(\Bool{} \sim_{\ell'}^{\star}String\right)$.
This chaining is needed for when multiple equalities are chained together by unification.

Additionally, elaboration needs a way to point out that when a pattern is un-unifiable.
This is what the explicit blame syntax ($!$) is for.
For instance, if the unifier could derive $1=2$ from evidence $p_{1}:1=x$ and $p_{2}:2=x$ then elaboration can force blame with $!_{p_{1}:\cup^{\Nat{}}p_{2}}^{\Nat{}}$.
Holding the type information isn't strictly needed, but allows us to provvide better error messages.
 
We also allow the forms of the data syntax form \ch{4}, though with slightly different annotations.
Data constructors and data type constructors now carry their explicit types, to emphasize that the type information will be needed for some reductions\footnote{
  a more efficient implementation would probly not incude them in the term syntax}.
Motives are no longer needed on $\mathsf{case}$ expressions, and type/cast tracking will be handled implicitly.
In addition to the handled patterns, $\mathsf{case}$ expressions also will contain a covering of unhadled patterns, for use in warnings and errors.
Further, we allow two new observations, to observe the indices of data type constructors and term constructors respectively.
These observations correspond directly to injectivity steps of the unification procedure.
 
As alluded to in the examples we extend pattern matching with a new path variable position.
These variables are bound from the extended notion of patterns, and contain evidence that the data expression has the expected type.
% This is justified by the normal forms of data terms in the cast language.
% While in the surface language a normal form of data must have the data constructor in head position (justifying the surface language pattern syntax), in the cast language the normal form of data may accumulate a cast.

Pattern matching across dependent data types will allow for observations that were impossible before.
For instance it is now possible to observe specific arguments in type constructors and term constructors.
Since function terms can appear directly in data type constructors it is now possible to observe functions directly.
Observations have been extented to allow the term level checking that is now possible.

\todo{App example}



\section{Endpoint Rules}

Care must be taken so that typing is still sensible when an expression could have multiple interpretations.
To do this we construct a new \csys{} to include term level endpoint information that will generalize the cast relation of \ch{3}.

\subsection{Function Fragment}
The rules for non-data terms are listed in \Fref{cast-endpoint-rules}.

\begin{figure}

\begin{align*}
 \frac{x:A\in\varGamma}{\varGamma\vdash x\sqsupseteq x:A}\rulename{\sqsupseteq-var}
  \quad & \quad 
  \frac{\ }{\varGamma\vdash\star\sqsupseteq\star:\star}\rulename{\sqsupseteq-\star}
  \\
\end{align*}

\[
\frac{
\varGamma\vdash A\sqsupseteq A':\star \quad
\varGamma,x:A'\vdash B\sqsupseteq B':\star
}{
  \varGamma\vdash\left(x:A\right)\rightarrow B\sqsupseteq\left(x:A'\right)\rightarrow B':\star
}\rulename{\sqsupseteq-\mathsf{fun}-ty}
\]

\[
\frac{
\varGamma\vdash A:\star\quad\varGamma,x:A\vdash B:\star \quad
\varGamma,x:A\vdash b\sqsupseteq b':B
}{
  \varGamma\ \vdash\ \mathsf{fun}\,f\,x\Rightarrow b\ \sqsupseteq\ \mathsf{fun}\,f\,x\Rightarrow b\ :\ \left(x:A\right)\rightarrow B
}\rulename{\sqsupseteq-\mathsf{fun}}
\]
\[
\frac{
\varGamma\vdash b\sqsupseteq b':\left(x:A'\right)\rightarrow B' \quad
\varGamma\vdash a\sqsupseteq a':A'
}{
  \varGamma\vdash b\,a\sqsupseteq b'\,a':B'\left[x\coloneqq a'\right]
}\rulename{\sqsupseteq-\mathsf{fun}-app}
\]

\[
\frac{\begin{array}{c}
  \varGamma\vdash L\sqsupseteq A:\star\\
  \varGamma\vdash L\sqsupseteq B:\star\\
  \varGamma\vdash a\sqsupseteq a':A
  \end{array}}{\varGamma\ \vdash\ a::L\ \sqsupseteq\ a'::L\ :\ B}\rulename{\sqsupseteq-::}
\]

\begin{align*}
  \frac{\begin{array}{c}
  \varGamma\vdash a\sqsupseteq a':A'\\
  \varGamma\vdash L\sqsupseteq A':\star\\
  \varGamma\vdash L\sqsupseteq C:\star
  \end{array}}{\varGamma\vdash a\sim_{\ell,o}^{L}b\sqsupseteq a'::L\ :\ C}\rulename{\sqsupseteq-\sim L}
  \quad & \quad 
  \frac{\begin{array}{c}
  \varGamma\vdash b\sqsupseteq b':B'\\
  \varGamma\vdash L\sqsupseteq B':\star\\
  \varGamma\vdash L\sqsupseteq C:\star
  \end{array}}{\varGamma\vdash a\sim_{\ell,o}^{L}b\sqsupseteq b'::L\ :\ C}\rulename{\sqsupseteq-\sim R}
\\
\end{align*}

\begin{align*}
  \frac{\begin{array}{c}
  \varGamma\vdash a\sqsupseteq a'::L:C\\
  \varGamma\vdash a\sqsupseteq c'::L:C\\
  \varGamma\vdash b\sqsupseteq c'::L:C
  \end{array}}{\varGamma\vdash a\cup^{L}b\sqsupseteq a'::L\ :\ C}\rulename{\sqsupseteq-\cup L}
  \quad & \quad 
  \frac{\begin{array}{c}
  \varGamma\vdash b\sqsupseteq b'::L:C\\
  \varGamma\vdash a\sqsupseteq c'::L:C\\
  \varGamma\vdash b\sqsupseteq c'::L:C
  \end{array}}{\varGamma\vdash a\cup^{L}b\sqsupseteq b'::L\ :\ C}\rulename{\sqsupseteq-\cup R}
\\
\end{align*}

\begin{align*}
  \frac{\begin{array}{c}
  \varGamma\vdash a\sqsupseteq a':A'\\
  A'\equiv B
  \end{array}}{\varGamma\vdash a\sqsupseteq a':B}\rulename{\sqsupseteq-conv-ty}
  \quad & \quad 
  \frac{\begin{array}{c}
  \varGamma\vdash a\sqsupseteq a':A'\\
  a'\equiv b\\
  \varGamma\vdash b:A'
  \end{array}}{\varGamma\vdash a\sqsupseteq b:A'}\rulename{\sqsupseteq-conv-trm}
\\
\end{align*}

\caption{Endpoints (non-data)}
\label{fig:cast-endpoint-rules}
\end{figure}

The \rulename{\sqsupseteq-var} and \rulename{\sqsupseteq-\star} rules extend the usual type assignment judgments to endpoints.
The \rulename{\sqsupseteq-\mathsf{fun}-ty} and \rulename{\sqsupseteq-\mathsf{fun}-app} rules makes sure different endpoints are used in a type consistent way.
For example, $\left\{ \lambda x.x+1\sim not\right\} \left\{ 1\sim \False{} \right\} \sqsupseteq \True{} ,2$ and $\left\{ 1\sim2\right\} +\left\{ 10\sim3\right\} \sqsupseteq4,5,11,12$.
The $\rulename{\sqsupseteq-::}$ rule allows using a term at a different type if the cast has compatible endpoints.
The next rules allow endpoints to be extracted out of the left or right of $\sim$, and $\cup$ if they are well formed.\todo{separate into multiple paragraphs?}

In addition to the usual type conversion rule we also have a term conversion rule.
Term conversion is important so that $\cup$ can associate terms under reduction.
For instance, $\left(x\sim2\right)\cup\left(1+1\sim y\right)\sqsupseteq x, y$.
Additionally term conversion will allow for the equivalence of type cast information.
For instance, $1::(\Nat{} \sim \Bool) \sqsupseteq 1 : \Nat{}$.

Endpoints can be used to recover the more familiar notions in \Fref{cast-data-endpoint-rules-def}.
A term is well-cast when it is its own endpoint (formalized in the \rulename{ty-def} rule).
Therefore a well-cast term can only have assert and unify syntax in proper positions of casts.
Also we can suggestively write two endpoints as an equality with $\approx$.
 
\begin{figure}
\[
\frac{\varGamma\vdash a\sqsupseteq a:A}{\varGamma\vdash a:A}\rulename{ty-def}
\]
\[
\frac{\varGamma\vdash a\sqsupseteq b:B \quad \varGamma\vdash a\sqsupseteq c:C}{
  \varGamma\ \vdash\ a\ :\ \ b\,:\,B\ \approx\ c\,:\,C\
  }\rulename{\approx-def}
\]
\caption{Definitions}
\label{fig:cast-data-endpoint-rules-def}
\end{figure}
 
As usual, the system needs a suitable definition of $\equiv$.
We use an equivalence relation that respects reductions and substitutions while ignoring casts.
Ignoring casts keeps terms from being made distinct by different casts, and prevents reductions from getting stuck for the purpose of equality.
Note that unlike \ch{3} this means equivalence will not preserve blame and we will need to rely on a more specific reduction relation that preserves both equivalence and blame.
\todo{constructing such an equivalence rigorously...}

\subsection{Data Endpoints}

The rules for data endpoints are listed in \Fref{cast-Data-Endpoint-Rules}.

\begin{figure}

\begin{align*}
  \frac{
  p\ :\ a:A\approx b:B\ \in\ \varGamma
}{
  \varGamma\vdash p\sqsupseteq a:A
  }\rulename{\sqsupseteq-var-L} \quad & \quad \frac{
    p\ :\ a:A\approx b:B\ \in\ \varGamma
  }{
    \varGamma\vdash p\sqsupseteq b:B
    }\rulename{\sqsupseteq-var-R} \\
\end{align*}

\begin{align*}
\frac{\mathsf{data}\ D\ :\ \Delta\rightarrow\star\ \in\ \varGamma
}{
\varGamma\vdash D_{\Delta}\sqsupseteq D_{\Delta}:\Delta\rightarrow\star
}\rulename{\sqsupseteq-TCon} \quad & \quad \frac{
d\ :\ \Delta\rightarrow D\overline{a}\ \in\ \varGamma
}{
\varGamma\vdash d_{\Delta\rightarrow D\overline{a}}\sqsupseteq d_{\Delta\rightarrow D\overline{a}}:\Delta\rightarrow D\overline{a}
}\rulename{\sqsupseteq-Con}\\
\end{align*}

\[
\frac{\begin{array}{c}
\varGamma\vdash\overline{a}\sqsupseteq\overline{a'}:\Delta\\
\varGamma\vdash\Delta\ \mathbf{ok}\\
\varGamma,\Delta\vdash B':\star\\
\forall\overline{p}\in\overline{|\,\overline{patc}},\:\varGamma\vdash\overline{p}:\Delta\\
\forall\overline{p}\in\overline{|\,\overline{patc}},\:\varGamma,\left(\overline{p}:\Delta\right)\vdash b\sqsupseteq b':B'\\
\forall\overline{p}\in\overline{?\,\overline{patc'}},\:\varGamma\vdash\overline{p}:\Delta\\
\varGamma\vdash\overline{|\,\overline{patc}}\overline{?\,\overline{patc'}}:\Delta\ \mathbf{Complete}
\end{array}}{
  \varGamma\vdash \case{} \,\overline{a,}\,\left\{ \overline{|\,\overline{patc\Rightarrow}b} \overline{?\,\overline{patc'}} \right\} \sqsupseteq \case{} \,\overline{a',}\,\left\{ \overline{|\,\overline{patc\Rightarrow}b'} \overline{?\,\overline{patc'}} \right\} :B'\left[\Delta\coloneqq\overline{a'}\right]
  }\rulename{\sqsupseteq-\case{}}
\]
\[
\frac{\begin{array}{c}
\varGamma\vdash L'\sqsupseteq A:\star\\
\varGamma\vdash L'\sqsupseteq B:\star\\
\varGamma\vdash L\sqsupseteq a:A\\
\varGamma\vdash L\sqsupseteq b:B\\
\mathbf{head}\,a\neq\mathbf{head}\,b\\
\varGamma\vdash B:\star
\end{array}}{\varGamma\vdash!_{L}^{L'}\sqsupseteq!_{L}^{L'}\ :\ B
}\rulename{\sqsupseteq-!}
\]
\[
\frac{\begin{array}{c}
\varGamma\vdash b\sqsupseteq D_{\Delta}\overline{a}:\star\end{array}\quad\mathbf{length}\,\Delta=\mathbf{length}\,\overline{a}}{\varGamma\vdash TCon_{i}\ b\sqsupseteq\Delta@_{i}\overline{a}
}\rulename{\sqsupseteq-TCon}
\]
\[
\frac{\begin{array}{c}
\varGamma\vdash b\sqsupseteq d_{\Delta\rightarrow D\overline{b}}\overline{a}\end{array}:D\overline{c}\quad\mathbf{length}\,\Delta=\mathbf{length}\,\overline{a}}{\varGamma\vdash DCon_{i}\ b\sqsupseteq\Delta@_{i}\overline{a}
}\rulename{\sqsupseteq-DCon}
\]
  
\caption{Cast Data Endpoint Rules}
\label{fig:cast-Data-Endpoint-Rules}
\end{figure}

\todo{supressing checks on telescopes}

Assertion variables have endpoints at each side of the equality given from the typing judgment.
Data type constructors and data term constructors have the function type implied by their definitions (and annotations that match).

The \rulename{\sqsupseteq-\case{}} states that a \case{} expression has a corresponding \case{} expression endpoint:
\begin{itemize}
  \item The endpoints of the \scruts{} correspond to an appropriate telescope that is compatible with the patterns\footnote{
    The endpoint and typing judgment can be extended to lists and telescopes.
  }.
  \item A motive $B'$ must exist under the context extended with that telescope.
  \item Every pattern must cast-check against the telescope.
  \item The body of every pattern must have an endpoint consistent with the motive and pattern.
  \item The constructors of the patterns must form a complete covering of the telescope.
\end{itemize}
  
In \rulename{\sqsupseteq-!} direct blame allows evidence with obviously contradictory endpoints to inhabit any type.

Data types and data terms can be inspected with \rulename{\sqsupseteq-TCon} and \rulename{\sqsupseteq-DCon}.
However the prior indices are needed to compute the types, so an index meta-function $@_{i}$ is defined (in \Fref{cast-Data-index-1}).
For instance, $TCon_1 (\Id{}\, (\Nat{}\sim\Bool{})\,(1\sim\True{})\,(1\sim\True{}))\ \sqsupseteq\  1\ :\ \Nat{}$ while also  $... \sqsupseteq\  \True{}\ :\ \Bool{}$.

\begin{figure}

\begin{tabular}{lll}
  $\left(\left(x:A\right)\rightarrow\Delta\right)@_{0}\left(a,\overline{b}\right)$ & = & $a:A$\tabularnewline
  $\left(\left(x:A\right)\rightarrow\Delta\right)@_{i}\left(a,\overline{b}\right)$ & = & $\left(\Delta\left[x\coloneqq a\right]\right)@_{\left(i-1\right)}\overline{b}$\tabularnewline
\end{tabular}
\caption{Typed Index Function}
\label{fig:cast-Data-index-1}
\end{figure}
\section{Reductions}
 
\subsection{Function Fragment}
\todo{rule names might help here}
A selection of reduction operations is listed in \Fref{cast-data-step}.

Function reduction happens as usual.
 
% Reduction over a cast collects the argument and body evidence and uses an argument cast to insure that the argument will be of the correct type.
Type universes reduce away in the following three rules.
 
Function types commute through $\sim$ and $\cup$ when their type annotations are the type universe.
 
When the type position is resolved to function type, arguments can be applied under $::$ and into $\sim$ and $\cup$.
This allows data types to be treated like functions.
For instance, $(\lambda x \Rightarrow\mathtt{S} x) \sim\mathtt{S}$ will never cause a blameable error.
 
Finally, there are reductions to consolidate cast bookkeeping.
In addition to the listed reductions, a reduction can happen recursively in any single sub-position.

% \todo{these reductions are listed?}
% In this thesis we have taken an extremely extensional perspective, terms are only different if an observation recognizes a difference.
% For instance this approach justifies equating the functions $\lambda x\Rightarrow x+1=\lambda x\Rightarrow1+x$ without proof, even though they are usually definitionally distinct.
% Therefore we will only blame inequality across functions if two functions that were asserted to be equal return different observations for ``the same'' input.
% Tracking that two functions should be equal becomes complicated, the system must be sensible under context, functions can take other higher order inputs, and function terms can be copied freely.


\begin{figure}
\[
\frac{\ }{\left(\mathsf{fun}\,f\,x\Rightarrow b\right)a\rightsquigarrow b\left[f\coloneqq\left(\mathsf{fun}\,f\,x\Rightarrow b\right),x\coloneqq a\right]}
\]

\[
\frac{\ }{A::\star\rightsquigarrow A}
\]

\[
\frac{\ }{\star\sim_{\ell,o}^{\star}\star\rightsquigarrow\star}
\]

\[
\frac{\ }{\star\cup^{\star}\star\rightsquigarrow\star}
\]

\[
\frac{\ }{\left(\left(\left(x:A\right)\rightarrow B\right)\sim_{\ell,o}^{\star}\left(\left(x:A'\right)\rightarrow B'\right)\right)\rightsquigarrow\left(x:\left(A\sim_{\ell,o.Arg}^{\star}A'\right)\right)\rightarrow\left(B\sim_{\ell,o.Bod_{x}}^{\star}B'\right)}
\]

\[
\frac{\ }{\left(\left(\left(x:A\right)\rightarrow B\right)\cup^{\star}\left(\left(x:A'\right)\rightarrow B'\right)\right)\rightsquigarrow\left(x:\left(A\cup^{\star}A'\right)\right)\rightarrow\left(B\cup^{\star}B'\right)}
\]

\[
\frac{\ }{\left(b::\left(\left(x:A\right)\rightarrow B\right)\right)a\rightsquigarrow\left(b\left(a::A\right)\right)::\ B\left[x\coloneqq a::A\right]}
\]

\[
\frac{\ }{\left(c\sim_{\ell,o}^{\left(x:A\right)\rightarrow B}b\right)a\rightsquigarrow\left(c\left(a::A\right)\sim_{\ell,o.App_{a}}^{B\left[x\coloneqq a::A\right]}b\left(a::A\right)\right)}
\]

\[
\frac{\ }{\left(c\cup^{\left(x:A\right)\rightarrow B}b\right)a\rightsquigarrow\left(c\left(a::A\right)\right)\cup^{B\left[x\coloneqq a::A\right]}\left(b\left(a::A\right)\right)}
\]

\[
\frac{\ }{\left(a::L'\right)\sim_{\ell,o}^{L}b\rightsquigarrow a\sim_{\ell,o}^{L'\cup^{\star}L}b}
\]

\[
\frac{\ }{a\sim_{\ell,o}^{L}\left(b::L'\right)\rightsquigarrow a\sim_{\ell,o}^{L\cup^{\star}L'}b}
\]

\[
\frac{\ }{\left(a::L'\right)\cup^{L}b\rightsquigarrow a\cup^{L'\cup^{\star}L}b}
\]

\[
\frac{\ }{a\cup^{L}\left(b::L'\right)\rightsquigarrow a\cup^{L\cup^{\star}L'}b}
\]

\[
\frac{\ }{\left(\left(a::L\right)::L'\right)\rightsquigarrow a::\left(L\cup^{\star}L'\right)}
\]
\caption{Cast Language Small Step (select rules)}
\label{fig:cast-data-step}
\end{figure}

\subsection{Data Reductions}

Some reductions for data are listed in \Fref{Data-Reductions}.
Elimination of data types is delegated to the $\mathbf{Match}$ judgment (that is unlisted). \todo{can reconstitute its type if their is no cast}
$TCon_{i}$, and $DCon_{i}$ observations reduce to the expected index.
Data types and data terms consolidate over $\sim$ and $\cup$ when a head constructor is shared and the type annotation has resolved, this needs to be computed by applying the indices point-wise against the typing telescope (see \Fref{Pointwise-Applications}).
This is why the telescope annotation is explicit in the formalism (so reductions can happen independent of the typing context). 

The important properties of reduction are 
\begin{itemize}
% \item Paths reduce into a stack of zero or more $Assert_{k\Rightarrow A}$s
\item Assertions emit observably consistent constructors, and record the needed observations
\item Sameness assertions will get stuck on inconsistent constructors
% \item Casts can commute out of sameness assertions with proper index tracking
% \item function application can commute around $kcasts$ , similar to Chapter 3, but will keep $k$ assumptions properly indexed
\end{itemize}
% Matching is defined in \ref{fig:cast-data-match}.
% Note that uncast terms are equivalent to refl cast terms.

\begin{figure}
\[
\frac{
  \overline{a}\ \left(\overline{|\,\overline{patc\Rightarrow}b}\right)\ \mathbf{Match}\ b'
}{
  \mathsf{case}\,\overline{a,}\,\left\{ \overline{|\,\overline{patc\Rightarrow}b}\, \overline{?\,\overline{patc}} \right\} \rightsquigarrow b'
}
\]

\[
\frac{\ }{TCon_{i}\ \left(D_{\Delta}\overline{a}\right)\rightsquigarrow a_{i}}
\]

\[
\frac{\ }{DCon_{i}\ \left(d_{\Delta\rightarrow D\overline{e}}\overline{a}\right)\rightsquigarrow a_{i}}
\]

\[
\frac{
  \mathbf{length}\,\Delta=\mathbf{length}\,\overline{a}=\mathbf{length}\,\overline{b}
}{
  \left(\left(D_{\Delta}\overline{a}\right)\sim_{\ell,o}^{\star}\left(D_{\Delta}\overline{b}\right)\right)\rightsquigarrow D_{\Delta}\left(\Delta@^{T\sim\ell,o}\left(\overline{a},\overline{b}\right)\right)}
\]

\[
\frac{
  \mathbf{length}\,\Delta=\mathbf{length}\,\overline{a}=\mathbf{length}\,\overline{b}
}{
  \left(\left(d_{\Delta\rightarrow D\overline{e}}\overline{a}\right)\sim_{\ell,o}^{C}\left(d_{\Delta\rightarrow D\overline{e}}\overline{b}\right)\right)\rightsquigarrow d_{\Delta\rightarrow D\overline{e}}\left(\Delta@^{\mathbf{D}\sim\ell,o}\left(\overline{a},\overline{b}\right)\right)::C}
\]

\[
\frac{
  \mathbf{length}\,\Delta=\mathbf{length}\,\overline{a}=\mathbf{length}\,\overline{b}
  }{\left(\left(D_{\Delta}\overline{a}\right)\cup^{\star}\left(D_{\Delta}\overline{b}\right)\right)\rightsquigarrow D_{\Delta}\left(\Delta@^{T\cup}\left(\overline{a},\overline{b}\right)\right)}
\]

\[
\frac{
  \mathbf{length}\,\Delta=\mathbf{length}\,\overline{a}=\mathbf{length}\,\overline{b}
}{
  \left(\left(d_{\Delta\rightarrow D\overline{e}}\overline{a}\right)\sim_{\ell,o}^{C}\left(d_{\Delta\rightarrow D\overline{e}}\overline{b}\right)\right)\rightsquigarrow d_{\Delta\rightarrow D\overline{e}}\left(\Delta@^{\mathbf{D}\cup}\left(\overline{a},\overline{b}\right)\right)::C}
\]
\caption{Data Reductions}
\label{fig:Data-Reductions}
\end{figure}


\begin{figure}
\begin{tabular}{llll}
$\Delta@_{0}^{k\sim\ell,o}\left(\overline{a},\overline{b}\right)$ & = & $\Delta@^{k\sim\ell,o}\left(\overline{a},\overline{b}\right)$ \tabularnewline % & shorthand that begins the recursion\tabularnewline
$\left(\left(x:A\right)\rightarrow\Delta\right)@_{i}^{k\sim\ell,o}\left(\left(a,\overline{a}\right),\left(b,\overline{b}\right)\right)$ & = &
\makecell[l]{$\left(a\sim_{\ell,o.TCon_{i}}^{A}b\right),$\\
  $\left(\Delta\left[x\coloneqq\left(a\sim_{\ell,o.kCon_{i}}^{A}b\right)\right]\right)@_{i+1}^{\sim\ell,o}\left(\overline{a},\overline{b}\right)$}
   \tabularnewline % & non-empty list\tabularnewline
$\left(.\right)@_{i}^{k\sim\ell,o}\left(.,.\right)$ & = & $.$  \tabularnewline % & empty list\tabularnewline
$\Delta@_{0}^{k\cup}\left(\overline{a},\overline{b}\right)$ & = & $\Delta@^{k\cup}\left(\overline{a},\overline{b}\right)$  \tabularnewline % & shorthand that begins the recursion\tabularnewline
$\left(\left(x:A\right)\rightarrow\Delta\right)@_{i}^{k\cup}\left(\left(a,\overline{a}\right),\left(b,\overline{b}\right)\right)$ & = & $\left(a\cup^{A}b\right),\left(\Delta\left[x\coloneqq\left(a\cup^{A}b\right)\right]\right)@_{i+1}^{k\cup}\left(\overline{a},\overline{b}\right)$ \tabularnewline %  & non-empty list\tabularnewline
$\left(.\right)@_{i}^{k\cup}\left(.,.\right)$ & = & $.$ \tabularnewline %  & empty list\tabularnewline
\end{tabular}
\caption{Pointwise Indexing}
\label{fig:Pointwise-Applications}
\end{figure}

\subsection{Endpoint Preservation}

The system preserves typed endpoints over reductions.
 
\[
\frac{a\rightsquigarrow b\quad\varGamma\vdash a\sqsupseteq a':A'}{\varGamma\vdash b\sqsupseteq a':A'}
\]
 
For the fragment without data, this can be shown with some modifications to the argument in \ch{3}.
We conjecture that the proof could be extended to support data.


\section{Cast soundness}

Additionally, For the fragment without data, we can show cast soundness.
And we conjecture that the proof could be extended to support data.
 
If $\varGamma\vdash a\sqsupseteq a':A'$, $\varGamma\ \mathbf{Empty}$ then either $a\ \mathbf{Consistent}$, $a\ \mathbf{Blame}_{\ell,o}$, or $a\rightsquigarrow b$.
Again preservation can be shown by extending the usual \ac{TAS} preservation argument over the new constructs.

The $\mathbf{Consistent}$ judgment (\Fref{cast-val}) generalizes being a value, to elements over $\cup$ and $\sim$.
This only matters for functions, since they are treated extentionally (so that data constructors can be treated as functions).
This relation can be restrited further to be more like a conventional value judgment, for instance inforcing the arguments of a data type are consistent.

\begin{figure}
\[
\frac{\ }{\star\ \mathbf{Consistent}}
\]

\[
\frac{\ }{\left(x:A\right)\rightarrow B\ \mathbf{Consistent}}
\]

\[
\frac{\ }{\left(\mathsf{fun}\,f\,x\Rightarrow b\right)\ \mathbf{Consistent_{\mathsf{fun}}}}
\]

\[
\frac{
  a\ \mathbf{Consistent_{\mathsf{fun}}}\quad b\ \mathbf{Consistent_{\mathsf{fun}}}
}{
  a\sim_{\ell,o}^{\left(x:A\right)\rightarrow B}b\ \mathbf{Consistent_{\mathsf{fun}}}}
\]

\[
\frac{
  a\ \mathbf{Consistent_{\mathsf{fun}}}\quad b\ \mathbf{Consistent_{\mathsf{fun}}}
}{
  a\cup^{\left(x:A\right)\rightarrow B}b\ \mathbf{Consistent_{\mathsf{fun}}}
}
\]

\[
\frac{
  \mathbf{length}\,\Delta < \mathbf{length}\,\overline{a}
}{
  D_{\Delta}\overline{a}\ \mathbf{Consistent_{\mathsf{fun}}}
}
\]

\[
\frac{
  \mathbf{length}\,\Delta < \mathbf{length}\,\overline{a}
}{
  d_{\Delta\rightarrow D\overline{a}}\ \mathbf{Consistent_{\mathsf{fun}}}
}
\]

\[
\frac{
  \mathbf{length}\,\Delta = \mathbf{length}\,\overline{a}
}{
  D_{\Delta}\overline{a}\ \mathbf{Consistent}
}
\]

\[
\frac{
  \mathbf{length}\,\Delta = \mathbf{length}\,\overline{a}
}{
  d_{\Delta\rightarrow D\overline{a}}\ \mathbf{Consistent}
}
\]

\caption{Consistent}
\label{fig:cast-data-val}
\end{figure}


The $a\ \mathbf{Blame}_{\ell,o}$ judgment means a witness of error can be extracted from the term $a$ pointing to the original source location $\ell$ with observation $o$.
The only important rule is,

\[
\frac{\mathbf{head}\ a\neq\mathbf{head}\ b\quad}{a\sim_{\ell,o}^{L}b\ \mathbf{Blame}_{\ell,o}}
\]

Blame can be recursively extracted out of every sub expression.
For instance, $\left(\left(1\sim_{\ell,app_{1}}^{\Nat{} }0\right)+2\right)\ \mathbf{Blame}_{\ell,app_{1}}$.
% In the implementation type errors are preffered.

There are two new sources of blame from the case construct.
The cast language records every ``unhandled'' branch and if a scrutinee hits one of those branches the case will be blamed for in-exhaustiveness\footnote{
  This runtime error is conventional in ML style languages, and is even
how Agda handles incomplete matches }.
If a scrutinee list primitively contradicts the pattern coverage blame can be extracted from the scrutinee. 
Since our type system will ensure complete coverage (based only on constructors) if a scrutinee escapes the complete pattern match in an empty context, it must be that there was a cast blamable cast to the head constructor.
% We have elided most of the structural rules that extract blame from terms, paths, and casts.
% We have left the structural rule for explicit blame for emphasis.



% \begin{figure}
% \todo[inline]{REDO THIS}

% \[
% \frac{\textbf{Blame}\:\ensuremath{\ell}\,o\ p}{\left(d'\ \overline{patc}\right)::p\ \mathbf{!Match}\ \ensuremath{\ell}\,o\,\left(d\ \overline{patc}\right)::p}
% \]

% \todo[inline]{This figure is not very helpful?}

% \caption{Cast Language Blame}
% \label{fig:cast-data-NoMatch}
% \end{figure}
\section{Elaboration}

\subsection{Unification}

To handle elaboration over pattern matching, we will need to extend unification from \ch{4} to accommodate casts (\Fref{cast-data-unification}).
We will now track not just equational constraints, but also why the constraints hold, and why the types are the same.
For instance, in the constraint notation $1 \approx_{C}^{L} \True{}$, means we have the constraint $1 = \True{}$ because of $C$ and $\Nat{} = \Bool{}$ because of $L$.
The $a \approx_{C}^{L} b$ constraint can be thought of as stating the term $C$ has endpoints $a$, $b$.
Solutions record the reasoning behind an assignment.
For instance, $x\coloneqq_{C} 3$ means $x$ will be assigned $3$ because of $C$.

\begin{figure}
\[
\frac{\,}{U\left(\emptyset,\emptyset\right)}
\]

\[
\frac{
  U\left(E,S\right)\quad a\equiv a'
}{
  U\left(\left\{ a \approx_{C}^{L} a' \right\} \cup E,S\right)
}
\]

\[
\frac{
  U\left(E\left[x\coloneqq_{C} a :: L\right],S\left[x\coloneqq_{C} a :: L\right]\right)
}{
  U\left(\left\{ x \approx_{C}^{L} a\right\} \cup E,\left\{x\coloneqq_{C} a :: L\right\} \cup S \right) 
}
\]

\[
\frac{
  U\left(\Delta @^{\approx TCon_{-}C}( \overline{b},\overline{b'} ) \cup E,a\right)\quad
  a\rightsquigarrow^* D_{\Delta}\overline{b}
  \quad   a' \rightsquigarrow^* D_{\Delta}\overline{b'}
}{U\left(\left\{ a \approx_{C}^{L} a'\right\} \cup E,a\right)}
\]

\[
\frac{
  U\left(\Delta @^{\approx DCon_{-}C}( \overline{b},\overline{b'} ) \cup E,a\right)\quad
  a\rightsquigarrow^* d_{\Delta\rightarrow D\overline{d}}\overline{b}
  \quad   a' \rightsquigarrow^* d_{\Delta\rightarrow D\overline{d}}\overline{b'}
}{U\left(\left\{ a \approx_{C}^{L} a'\right\} \cup E,a\right)}
\]

\[
\frac{
  U\left( \left\{ a \approx_{C}^{L\cup^{\star}D} b \right\}\cup E, S \right)
}{
  U\left(\left\{ a \approx_{C}^{L} b::D \right\} \cup E, S \right) 
}
\]

\caption{\CLang{} Unification Rules (select)}
\label{fig:cast-data-unification}
\end{figure}

\todo{backport notation}

When terms are substituted over equations, $E$, the terms are substituted into terms and causes are substituted into causes.
For instance, 
  $\left(x \approx_{TCon_{1}x}^{\Nat{}}x+x\right) \left[x\coloneqq_{p} 3\right]$ $=$ $3 \sim_{TCon_{1}p}^{\Nat{}}3+3$,
  which is sensible when constraints are considered as a notation for endpoints.

Additionally term constructor and type constructor injectivity is handled through an indexing operator $@^{...}$ similar to evaluation.
Finally we add rules to peel off casts so that unification remembers why terms have the appropriate type, and so unification will not get stuck.

\subsection{Elaboration}
Without data, elaboration can be extended from \ch{3} by asserting a cast where an infer meets a check.

\[
\frac{\varGamma\vdash m\,\textbf{Elab}\ a\overrightarrow{\,:\,}A}{\varGamma\vdash m\overleftarrow{\,:_{\ell,o}\,}B\ \textbf{Elab}\left(a::\left(A\sim_{\ell,o}^{\star}B\right)\right)}
\rulename{\overleftarrow{\textbf{Elab}}-cast}
\]

\todo{this is actually a simplification, since there may be ambient equalities that need to be supported}
However the situation becomes more complicated with pattern matched data.
With the information extracted from unification, terms can be elaborated with blameless casts that mimic the behavior of pattern matching in \ch{4}.
The information can also be used to redirect blame from ``impossible'' branches.
However the elaboration procedure als needs to be extended with a scope of assignments to cast into and out of making a formal presentation difficult.

% 
\section{OLD JUNK}


\subsubsection{Need to remove a cast}
\todo{better name}
 
Consider this surface language expression that extracts the last element from a non-empty list.
Assume the function $last:(n:Nat)\rightarrow Vec\,A\,(S\,N)\rightarrow A$ is in scope.
\todo{or just define the recursive function}
 
\begin{lstlisting}[basicstyle={\ttfamily\small}]
case v <_: Vec A (S x) => A > {
| Cons _ a (Z) _ => a
| Cons _ _ (S n) rest => last n rest
}
\end{lstlisting}
 
This will elaborate into
 
\begin{lstlisting}[basicstyle={\ttfamily\small}]
case v <_: Vec A (S x) => A > {
| (Cons A' a' (Z)::q rest) :: p => a' :: (inTC1(p))-1
| (Cons A' a' (S n)::q rest)::p => last n (rest :: p')
| (Nil A')::p => !TCon1(p)
}
\end{lstlisting}
 
\todo[inline]{prettier expressions, rev to -1, in general it might be clearer if cast language is always in math mode}
 
In the 2nd branch we have
  $A':\star$, $a':A'$, $n:\mathtt{Nat}$,
  $q:\mathtt{Nat}\approx \mathtt{Nat}$,
  $rest:\mathtt{Vec}\,A'\,\left(\left(\mathtt{S}\,n\right)::q\right)$,
  and $p:\mathtt{Vec}\,A'\,\left(\mathtt{S}\left(\left(\mathtt{S}\,n\right)::q\right)\right)\approx \mathtt{Vec}\,A\,\left(\mathtt{S}\,x\right)$.
Elaboration cannot unify a solution unless we can remove casts, otherwise it becomes impossible to construct a path from $n \approx x$
  from $\mathtt{Vec}\,A'\,\left(\left(\mathtt{S}\,n\right)\underline{::q}\right)\approx \mathtt{Vec}\,A\,\left(\mathtt{S}\,x\right)$, since the cast blocks
  $\mathtt{S}\,n = \mathtt{S}\,x$ that would be derivable by unification in the surface language.
We will need an operator that can remove casts from the endpoints of paths that arise from unification.
We will call these operators $uncastL$ and $uncastR$ and they will be derivable in the cast language.
With these operations we can match the process of surface level unification so that
 
$p'\ =\ Cong_{x\Rightarrow \mathtt{Vec}\,A'\,x}\left(UncastR\left(refl\right)\right)\circ Cong_{x\Rightarrow \mathtt{Vec}\,x\,\left(\mathtt{S}\,n\right)}\left(TCon_{0}p\right):\ \mathtt{Vec}\,A'\,\left(\left(\mathtt{S}\,n\right)::q\right)\ \approx\ \mathtt{Vec}\,A\,\left(\mathtt{S}\,n\right)$
 
Where
$UncastR\left(refl\right): \left(\mathtt{S}\,n\right)::q  \approx \mathtt{S}\,n $
 
In the first branch we have,
  $A':\star$, $a':A'$,
  $q:\mathtt{Nat}\approx \mathtt{Nat}$,
  $rest:\mathtt{Vec}\,A'\,\left(\mathtt{Z}::q\right)$,
  and $p:\mathtt{Vec}\,A'\,\left(\mathtt{S}\left(\mathtt{Z}::q\right)\right)\approx \mathtt{Vec}\,A\,\left(\mathtt{S}\,x\right)$.
Unification can proceed to derive $TCon_{1}\left(Con_{0}\left(p\right)\right)^{-1}:\ x\approx \mathtt{Z}::q$
and $TCon_{1}\left(p\right)^{-1}:\ A\approx A'$.
 
In the final branch we have $p:\mathtt{Vec}\,A'\,\mathtt{Z}\approx \mathtt{Vec}\,A'\,\left(\mathtt{S}\,x\right)$,
which is contradicted by $TCon_{1}p:\mathtt{Z}\approx \mathtt{S}\,x$

\todo[inline]{Substantail changes will be made made to the formulaitons bellow}


\todo[inline]{Example back to pattern matching?}


\begin{figure}

term reductions
\[
\frac{p\rightsquigarrow p'}{!_{p}\rightsquigarrow!_{p'}}
\]

\todo[inline]{could remove more then just assertions}

\[
\frac{\,}{\left\{ \star\sim_{k,o,\ell}\star\right\} \rightsquigarrow\star}
\]

\[
\frac{\,}{\left\{ \left(x:A\right)\rightarrow B\sim_{k,o,\ell}\left(x:A'\right)\rightarrow B'\right\} \rightsquigarrow\left(x:\left\{ A\sim_{k,o.arg,\ell}A'\right\} \right)\rightarrow\left\{ B\sim_{k,o.bod\left[x\right],\ell}B'\right\} }
\]

\[
\frac{\,}{\left\{ \mathsf{fun}\,f\,x\Rightarrow b\sim_{k,o,\ell}\mathsf{fun}\,f\,x\Rightarrow b'\right\} \rightsquigarrow\mathsf{fun}\,f\,x\Rightarrow\left\{ b\sim_{k,o.app\left[x\right],\ell}b'\right\} }
\]

\[
\frac{\,}{\left\{ d\overline{a}\sim_{k,o,\ell}d\overline{a'}\right\} \rightsquigarrow d\overline{\left\{ a_{i}\sim_{k,o.DCon[i],\ell}a'_{i}\right\} }}
\]

\[
\frac{\,}{\left\{ D\overline{a}\sim_{k,o,\ell}D\overline{a'}\right\} \rightsquigarrow D\overline{\left\{ a_{i}\sim_{k,o.TCon[i],\ell}a'_{i}\right\} }}
\]

\todo[inline]{double check}

\[
\frac{\overline{a}\ \mathbf{Match}\ \overline{patc}_{i}}{\mathsf{case}\,\overline{a,}\,\left\{ \overline{|\,\overline{patc\Rightarrow}b_{i}}\overline{|\,\overline{patc'\Rightarrow}!_{\ell}}\right\} \rightsquigarrow b_{i}\left[patc_{i}\coloneqq\overline{a}\right]}
\]

\caption{Summery of Cast Language Reductions}
\label{fig:cast-data-red}
\end{figure}

This extension to the syntax induces many more reduction rules. We
include a summery of selected reduction rules in \ref{fig:cast-data-red}.
We do not show the value restrictions to avoid clutter\footnote{there are also multiple ways to lay them out. For instance we could
evaluate paths left to right or right to left.}. 


\begin{figure}
\todo[inline]{swap, pattern on the left?}

\todo[inline]{record substitutions}

\[
\frac{\,}{a\ \mathbf{Match}\ x}
\]

\[
\frac{\overline{a}\ \mathbf{Match}\ \overline{patc}}{d\,\overline{a}\ \mathbf{Match}\ \left(d\,\overline{patc}\right)::x_{p}}
\]

\[
\frac{\overline{a}\ \mathbf{Match}\ \overline{patc}}{\left(d\,\overline{a}\right)::kcast\ \mathbf{Match}\ \left(d\,\overline{patc}\right)::x_{p}}
\]

\[
\frac{\overline{a}\ \mathbf{Match}\ \overline{patc}}{\left(d\,\overline{a}\right)::kcast\ \mathbf{Match}\ \left(d\,\overline{patc}\right)::x_{p}}
\]

\[
\frac{\,}{.\ \mathbf{Match}\ .}
\]

\[
\frac{b\ \mathbf{Match}\ patc'\quad\overline{a}\ \mathbf{Match}\ \overline{patc}}{b\overline{a}\ \mathbf{Match}\ patc'\overline{patc}}
\]


\caption{Cast Language Matching}
\label{fig:cast-data-match}
\end{figure}

\todo[inline]{double check paths are fully applied when needed}





The Cast language extension defined in this chapter is fairly complex.
Though all the meta-theory of this section is plausible, we have not
fully formalized it, and there is a potential that some subtle errors
exist. To be as clear as possible about the uncertainty around the
meta-theory proposed in this chapter, I will list what would normally
be considered theorems and lemmas as conjectures. \todo{weird place to make this note. add it to the front or back matter?}



We now conjecture the core lemmas that could be used to prove cast
soundness
\begin{conjecture}
substitution of cast terms preserves cast

equivalently the following rule is admissible

\end{conjecture}


% ...
\begin{conjecture}
substitution of cast terms preserves path endpoints

equivalently the following rule is admissible

\end{conjecture}

% ...

Finally we will conjecture the cast soundness.
\begin{conjecture}
The cast system preserves types and path endpoints over normalization
\end{conjecture}

% ...
\begin{conjecture}
a well typed path in an empty context is a value, takes a step, or
produces blame
\end{conjecture}

% ...
\begin{conjecture}
A well typed term in an empty context is a value, takes a step, or
produces blame
\end{conjecture}


\section{Elaborating Eliminations}



% elaboration unification
To make the overall system behave as expected we do not want to expose
users to equality patterns, or force them to manually do the path
bookkeeping. To work around this we extend a standard unification
algorithm to cast patterns with instrumentation to remember paths
that were required for the solution. Then if pattern matching is satisfiable,
compile additional casts into the branch based on its assignments.
Unlisted patterns can be checked to confirm they are unsatisfiable.
If the pattern is unsatisfiable then elaboration can use the proof
of unsatisfiability to construct explicit blame. If an unlisted pattern
cannot be proven ``unreachable'' then we could warn the user, and
like most functional programming languages, blame the incomplete match
if that pattern ever occurs.

\subsection{Preliminaries}

The surface language needs to be enriched with additional location
metadata at each position where the two bidirectional typing modalities
would cause a check in the surface language.

\begin{tabular}{lcll}
$m...$ & $\Coloneqq$ & ... & \tabularnewline
 & | & $\mathsf{case}\,\overline{n_{\ell},}\,\left\{ \overline{|\,\overline{pat\Rightarrow}m_{\ell'}}\right\} $ & data elim. without motive\tabularnewline
 & | & $\mathsf{case}\,\overline{n_{\ell},}\,\left\langle \overline{x\Rightarrow}M_{\ell'}\right\rangle \left\{ \overline{|\,\overline{pat\Rightarrow}m_{\ell''}}\right\} $ & data elim. with motive\tabularnewline
\end{tabular}

The implementation allows additional annotations along the motive,
while this works within the bidirectional framework. The syntax is
not presented here since the theory is already quite complicated.\todo{move note somewhere else}

\subsection{Elaboration}

The biggest extension to the elaboration procedure in Chapter 3 is
the path relevant unification and the insertion of casts to simulate
surface language pattern matching. The unification and casting processes
both work without $k$ assumptions in scope, simplifying the possible
terms that may appear.


\begin{figure}
\[
\frac{\,}{U\left(\emptyset,\emptyset\right)}
\]

\[
\frac{U\left(E,u\right)\quad a\equiv a'}{U\left(\left\{ p:a\thickapprox a'\right\} \cup E,u\right)}
\]

\[
\frac{U\left(E\left[x\coloneqq a\right],u\left[x\coloneqq a\right]\right)}{U\left(\left\{ p:x\thickapprox a\right\} \cup E,u\cup\left\{ p:x\thickapprox a\right\} \right)}
\]

\todo[inline]{actually a little incorrect, needs to use conq to concat the paths}

\[
\frac{U\left(E\left[x\coloneqq a\right],u\left[x\coloneqq a\right]\right)}{U\left(\left\{ p:a\thickapprox x\right\} \cup E,u\cup\left\{ p^{-1}:x\thickapprox a\right\} \right)}
\]

\[
\frac{U\left(\left\{ p:a\thickapprox a'\right\} \cup E,u\right)\quad a\equiv d\overline{b}\quad a'\equiv d\overline{b'}}{U\left(\left\{ Con_{i}p:b_{i}\thickapprox b'_{i}\right\} _{i}\cup E,u\right)}
\]

\todo[inline]{fully applied}

\[
\frac{U\left(\left\{ p:a\thickapprox a'\right\} \cup E,u\right)\quad a\equiv D\overline{b}\quad a'\equiv D\overline{b'}}{U\left(\left\{ TCon_{i}p:b_{i}\thickapprox b'_{i}\right\} _{i}\cup E,u\right)}
\]

\todo[inline]{fully applied}


\todo[inline]{break cycle, make sure x is assignable}

\todo[inline]{double check constraint order}

\todo[inline]{correct vars in 4a}

\caption{Surface Language Unification}
\label{fig:surface-data-unification}
\end{figure}


The elaboration procedure uses the extended unification procedure
to determine the implied type and assignment of each variable. In
the match body casts are made so that variables behave as if they
have the types and assignments consistent with the surface language.
The original casting mechanism is still active, so it is possible
that after all the casting types still don't line up. In this case
primitive casts are still made at their given location.

\todo[inline]{add explicit rules for elaboration?}

The elaboration algorithm is extremely careful to only add casts,
this means erasure is preserved and evaluation will be consistent
with the surface language.

Further the remaining properties from Chapter 3 probably still hold
\begin{conjecture}
Every term well typed in the bidirectional surface language elaborates 
\end{conjecture}

% ..
\begin{conjecture}
Blame never points to something that checked in the bidirectional
system 
\end{conjecture}


\section{Discussion and Future Work}


\subsection{Types invariance along paths}

It turns out that the system defined in Chapter 3 had the advantage
of only dealing with equalities in the type universe. Extending to
equalities over arbitrary type has vastly increased the complexity
of the system. To make the system work paths are untyped, which seems
inelegant. There is nothing currently preventing blame across type.
For instance,

$\left\{ 1\sim_{k,o,\ell}false\right\} $ will generate blame $1\neq false$.
While blame of $Nat\neq Bool$ will certainly result in a better error
message. Several attempts were made to encode the type into the type
assumption, but the resulting systems quickly became too complicated
to work with. Some vestigial typing constraints are still in the system
(such as on the explicit blame) to encourage this cleaner blame.

\subsection{Elaboration is non-deterministic with regard to blame}

Consider the case

\begin{lstlisting}[basicstyle={\ttfamily\small}]
case x <_:Id Nat 2 2 => S 2> {
| refl _ a => s a
}
\end{lstlisting}

that can elaborate to

\begin{lstlisting}[basicstyle={\ttfamily\small}]
case x <_:Id Nat 2 2 => S 2> {
| (refl A a)::p => (s (a::TCon0(p)) :: Cong uncastL(TCon1(p)))
}
\end{lstlisting}

where $p:Id\,A\,a\,a\thickapprox Id\,Nat\,2\,2$, where $TCon_{1}p$
selects the first position $p:Id\,A\,\underline{a}\,a\thickapprox Id\,Nat\,\underline{2}\,2$.
But this could also have elaborated to 

\begin{lstlisting}[basicstyle={\ttfamily\small}]
case x <_:Id Nat 2 2 => S 2> {
| (refl A a)::p => (s (a::TCon0(p)) :: Cong uncastL(TCon2(p)))
}
\end{lstlisting}

relying on $p:Id\,A\,a\,\underline{a}\thickapprox Id\,Nat\,2\,\underline{2}$.
This can make a difference if the scrutinee is 

$refl\ Nat\,2::Id\,Nat\,3\,2::Id\,Nat\,2\,2$

in one case blame will be triggered, in the other it will not. In
this case it is possible to mix the blame from both positions, though
this does not seem to extend in general since the consequences of
inequality are undecidable in general and we intend to allow running
programs if they can maintain their intended types.

\subsection{Extending to Call-by-Value}

As in Chapter 3, the system presented here does the minimal amount
of checking to maintain type safety. This can lead to unexpected results,
for instance consider the surface term 

\begin{lstlisting}[basicstyle={\ttfamily\small}]
case (refl Nat 7 :: Id Nat 2 2) <_ => Nat> {
| refl _ a => a
}
\end{lstlisting}

This will elaborate into 

\begin{lstlisting}[basicstyle={\ttfamily\small}]
case (refl Nat 7 :: Id Nat 2 2) <_ => Nat> {
| (refl A a)::p => a::TCon0(p)
}
\end{lstlisting}

which will evaluate to $7::\mathtt{Nat}$ without generating blame.
And indeed we only ever asserted that the result was of type $\mathtt{Nat}$.

In the implementation, some of this behavior is avoided by requiring
type arguments in a cast be run call-by-value. This restriction will
blame $7\neq2$ before the cast is even evaluated.\todo{expand}

\subsection{Efficiency}

The system defined here is brutally inefficient. 

In theory the system has an arbitrary slow down. As in Chapter 3,
a cast that relies on non-terminating code can itself cause additional
non-termination as paths are resolved.

\todo[inline]{paremetricity}

\todo[inline]{relation to fun-ext}

\todo[inline]{warnings}

\subsection{Relation to UIP}

Pattern matching as outlined in the last Chapter (which follows from
\cite{coquand1992pattern}) implies the \textbf{uniqueness of identity
proofs} (UIP)\footnote{Also called \textbf{axiom k}}. UIP states
that every proof of identity is equal to refl (and thus unique), and
is not provable in many type theories. In univalent type theories
UIP is directly contradicted by the ``non-trivial'' equalities,
required to equate isomorphisms and Id. UIP is derivable in the surface
language by following pattern match 

\begin{lstlisting}[basicstyle={\ttfamily\small}]
case x <pr : Id A a a => Id (Id A a a) pr (refl A a) > {
| refl A a => refl (Id A a a) (refl A a)
}
\end{lstlisting}

This type checks since unification will assign $pr\coloneqq refl\,A\,a$
and under that assumption $refl\ (Id\ A\ a\ a)\ (refl\ A\ a):Id\ (Id\ A\ a\ a)\ (refl\ A\ a)\ (refl\ A\ a)$.
Like univalent type theories, the cast language has its own nontrivial
equalities, so it might seem that the cast language would also contradict
UIP . But it is perfectly compatible, and will elaborate. One interpretation
is that though there are multiple ``proofs'' of identity, we don't
care which one is used. \todo{interpretation + take aways?}


\section{Related work}

This work was previously presented as an extended abstract at the
TyDE workshop\todo[inline]{cite}, the version there reflected a less
plausible meta-theory based on earlier implementation experiments.





% avoids issues like
% \begin{itemize}
%   \item How do same assertions interact with casts? For instance $\left\{ 1::Bool\sim_{k,o,\ell}2::Bool\right\} $. 
%   \item How do sameness assertions cast check? This is difficult, because
%   there is no requirement that a user asserted equality is of the same
%   type. For instance what type should the term $\left\{ 1\sim_{k,o,\ell}True\right\} $
%   have?
% \end{itemize}

% $\thickapprox$
\todo{consider dropping type info from explicit blame, since it will be redundant (if it is doable from the impl)} 
\todo{still some worthwhile stuff to extract from the defunct version of this chapter}
\todo[inline]{technically speaking, telescopes should generalize to the different syntactic classes}

\section{Future Work}
\todo[inline]{flex vars, refining elaborated patterns, elaborating more \scruts{} into the scrut list}

The system presented here improves on several earlier systems and implementation experiments.
There is reason to believe things could be improved further.
For instance, when pattern matching an uncast data the pattern must have a cast variable (so one is synthesized).
It may be more efficient (and cleaner) to have a term that would correspond to reflexivity, as some of our earlier experimental systems had.
% \todo[inline]{Nullery ops}

\todo[inline]{Coq proofs}

% Make equalities visible in the surface syntax
The system here has some simple inspiration that could be extended into pattern matching syntax more generally. 
It seems useful to be able to read equational information out of patterns, especially in settings with rich treatment of equality.
Matching equalities directly could be a useful feature in Agda, or in univalent type theories where manipulating equations is more critical.% such as CTT.
% \todo[inline]{CTT data is related?}

% \todo[inline]{a conclusion}
