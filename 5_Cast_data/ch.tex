\chapter{Data in the Cast Langugae}
\label{chapter:CastData}
\thispagestyle{myheadings}

\section{Introduction}

% Cast pattern matching is easy!
%Surprisingly, the cast system can be extended with a pattern matching construct without unification.
In Chapter 3 we used the TAS system and the bidirectional system to guid delaying equality to runtime.
The TAS corresponds to the cast system, and the bidirectional system corresponds to elaboration.
Specificly, the bidriectional approach localizes the equality checks.
In this chapter we will extend these systems for data and patternmatching.
This will turm out to be much more complicated then the system in chapter 3, since the only equality checks happened at types of the type universe.
In this chapter we will need to equate terms directly, and those terms will not necissarily have the same type.

\todo{Revise, pattern matching is a bidirectional thing}
\todo{formate case consistently}
If we hope to target the properties of Chapter 3 we need the cast language to be \textbf{cast sound}. 
Elaboration should satisfy the additional correctness\todo{``gradual correctness''?} properties relative to a type assignment system and bidirectional system.
In this case we will target the conjectured TAS from Chapter 4, with a first order unification style pattern matching. 

Despite this, there is an intresting interpertation of data and pattern matching when extended to the cast system.

Consider the normal forms of data terms.
While in standard type theory a normal form of data, in an empty context, must have the data constructor in head position (justifying the pattern syntax).
In the cast language, the normal form of data, in an empty context, will have a stack of zero or more casts applied to an expression with a data constructor in head position.
Casts are wrapped around constructors during the elaboration procedure, and will accumulate during evaluation.
If the cast language is extended with a path variable that can represent the stack of equalities, then that stack can be matched and used in the body of the branch.
Since the type constructor is known, it is possible to check the coverage of the patterns against it's constructors.
If every constructor is accounted for, only blameable data remains.
Quantifying over casts allows blame to be redirected, so if the program gets stuck in a pattern branch it can blame the original faulty assumption.

To account for unreachable patterns, we can record the proof of inequality that makes the unification unsatisfiable with explicit syntax, that will expose blame.
It is perfectly possible for a case expression to reduce into such an ``unreachable'' branch if a scrutinee with obsevably incorrect observations is inspected.
If this happens blame will be reflected back onto a specific problematic cast of the input.\todo{define path}

When we internalize the notion of path we need to support path operations that correspond to each of operations used by the unification of the bidirectional type system.
Specifically we need to index into the arguments of data constructors, and data type constructors.
We will also need to be able to reverse paths and concatenate paths since the unification algorithm implicitly uses these properties of equality.
Finally we need to be a able to remove and append casts from and to the endpoints of paths.

During elaboration, after a pattern is unified, we will inject the proofs of equality into the branch terms so that they cast check.
This will require that our notion of paths support congruence.

\subsection{Examples}

\subsubsection{Head}

For instance if the user case matches the head of $x$ where $x$ has type $\mathtt{Vec}\,A\,(\mathtt{S}n)$ in the surface language,

\begin{lstlisting}[basicstyle={\ttfamily\small}]
case x <_:Vec Bool (S n) => Bool> {
| Cons _ a _ _ => a
}
\end{lstlisting}

What can go wrong in the presence of casts?
\begin{itemize}
\item
A faulty cast assumption may have made $x$ appear to be a Vector even when it is not.
For instance, $\mathtt{True}::\mathtt{Vec}\,\mathtt{Bool}\,3$.
\item 
The vector may be empty but cast to look like it is inhabited.
For instance, $\mathtt{Nil}\,\mathtt{Bool}\,::\mathtt{Vec}\,\mathtt{Bool}\,5$.
\item 
The vector may have a type that is not $\mathtt{Bool}$.
For instance, $\mathtt{Cons}\,\mathtt{Nat}\,3\,...\,...::\mathtt{Vec}\,\mathtt{Bool}\,5$.
\end{itemize}

To handle these issues elaboration will generate the following cast term with unification

\begin{lstlisting}[basicstyle={\ttfamily\small}]
case x <_:Vec Bool (S n) => Bool> {
| (Cons A a y _) :: p => a::InTCon0(p)
| (Nil A) :: p => !InTc1(p)
}
\end{lstlisting}

The elaborated case expression covers all possible constructors for the data type constructor $\mathtt{Vec}$. 
In the fist branch $A:\star$, $a:A$, $y:\mathtt{Nat}$, while $p$ is a path variable that contains evidence that the type of $\mathtt{Cons\,A\,a\,y\,-}$ is $\mathtt{Vec Bool (S n)}$.
Explicitly, $p:\mathtt{Vec}\ A\ (\mathtt{S}\,y)\approx\mathtt{Vec}\,\mathtt{Bool}\,(\mathtt{S}\,n)$.
$InT_{0}p$ extracts the 0th argument from the type constructor $p:\mathtt{Vec}\ \underline{A}\ (\mathtt{S}\,y)\approx\mathtt{Vec}\,\underline{\mathtt{Bool}}\,(\mathtt{S}\,n)$ resulting in the type $InT_{0}p:A\approx\mathtt{Bool}$.
The body of the branch casts $a$ along $InT_{0}\,p$ to $\mathtt{Bool}$.
Casts are generalized from Chapter 3 to support arbitrary paths.

In the the second branch, the pattern mactch gives $A:\star$, $p:\mathtt{Vec}\ A\ \mathtt{Z}\approx\mathtt{Vec}\,\mathtt{Bool}\,(\mathtt{S}\,y)$.
The body of the branch encodes the contradiction observed by noting $\mathtt{Z}\neq\mathtt{S}\,y$.

Since there is no equational assertion, no warnings will be generated for this elaborated term.

Again consider the ways $x$ could go wrong,
\begin{itemize}
\item 
If the case tries to eliminate $\mathtt{True}::\mathtt{Vec}\,\mathtt{Bool}\,3$, the type constructor is not matched so the faulty assumption can be blamed automatically.
\item 
If the scrutinee is an empty vector, we will fall into the $\mathtt{Nil}$ branch, which will reflect the underling faulty assumption, via the explicit blame syntax.
\item
If the vector is inhabited by an incorrect type, it will return $3::\mathtt{Bool}$ with the a cast that rests on the faulty assumption of $\mathtt{Vec}\,\mathtt{Nat}\,5=\mathtt{Vec}\,\mathtt{Bool}\,\,5$.
If checking is done aggressively, as in the implementation, this will produce blame immediately.
If minimal checking is done, the blame will embedded in the resulting term and discovered whenever the output is used.
\end{itemize}

\subsubsection{Sum}

Consider the more involved example that sums the 2 numbers in a vector

\begin{lstlisting}[basicstyle={\ttfamily\small}]
case x <_:Vec Nat 2 => Nat> {
| Cons _ i _ (Cons _ j _ _) => i+j
}
\end{lstlisting}

The elaboration procedure will produce

\begin{lstlisting}[basicstyle={\ttfamily\small}]
case x <_:Vec Nat 2 => Nat> {
| (Cons Nat' i n' (Cons Nat'' j n'' rest):: p1):: p2 => 
  i::InTc0(p2) + j::(InTc0(p1).InTc0(p2))
| (Nil Nat') :: p => 
  !InTc1(p)
| (Cons Nat' i n' (Nil Nat''):: p1):: p2 =>
  !InTc1(p1).InTc1(p2) 
}
\end{lstlisting}

\begin{itemize}
\item
In the first branch we have the variables in scope, $p1:\mathtt{Vec}\ Nat''\ (\mathtt{S}\ n'')\approx\mathtt{Vec}\ Nat'\ n'$, $p2:\mathtt{Vec}\ Nat''\ (\mathtt{S}\ n')\approx\mathtt{Vec}\ \mathtt{Nat}\ 2$, $i:Nat'$, $j:Nat''$.
\begin{itemize}
\item 
This means we can deduce $InTc_{0}(p2):Nat'\approx\mathtt{Nat}$, and $InTc_{0}(p1)InTc_{0}(p2):Nat''\approx\mathtt{Nat}$.
\todo[inline]{how should paths be concatenated? ;}
\item 
We know from unification what the type and value of every term is supposed to be, so casts can be injected using evidence from the pattern.
\end{itemize}
\item 
In the 2nd branch we have , $p:\mathtt{Vec}\ Nat''\ 0\approx\mathtt{Vec}\ Nat'\ 2$.
\begin{itemize}
\item 
Which is contredictory, by $InTc_{1}(p):0\approx2$.
\end{itemize}
\item 
In the 3rd branch, $p1:\mathtt{Vec}\ Nat''\ 0\approx\mathtt{Vec}\ Nat'\ n'$, $p2:\mathtt{Vec}\ Nat'\ (\mathtt{S}\ n')\approx\mathtt{Vec}\ \mathtt{Nat}\ 2$. 
\begin{itemize}
\item
Which is unsatisfiable by $InTc_{1}(p1)InTc_{1}(p2):0\approx2$.
We don't need to know which sub path is problematic beforehand, only that the combination causes trouble.
If this branch is reached in an empty context, we can observe a problem in either branch or both.
\end{itemize}
\end{itemize}

\todo[inline]{Example of a var out of scope that gets set, would require cong in all situations}

\subsubsection{Assignment at the type level}

There are more complicated possibilities that need to be addressed.
For instace, this surface expression that takes in a propositional proof that $2=2$ and uses the named witness to generate a vector of length 2.
This will typecheck in the surface Langugae.

\begin{lstlisting}[basicstyle={\ttfamily\small}]
case x <_:Id Nat 2 2 => Vec Bool 2> {
| refl _ a => rep a Bool True
}
\end{lstlisting}

This will elaborate to

\begin{lstlisting}[basicstyle={\ttfamily\small}]
case x <_:Id Nat 2 2 => Vec Bool 2> {
| (refl Nat' a)::p => 
  (rep (a::InTc0(p))):: Cong (x=> (A : *) -> A -> Vec Nat x)
                             (CastL (InTC0(p)) (InTC1 p))
}
\end{lstlisting}

Where $Nat':\star$, $a:Nat'$, and $p:\mathtt{Id}\ Nat'\ a\ a\approx\mathtt{Id}\ \mathtt{Nat}\ 2\ 2$.
Since we have $p:\mathtt{Id}\ Nat'\ a\ a\approx\mathtt{Id}\ \mathtt{Nat}\ 2\ 2$, we can derive $InTC_{0}(p):\ Nat'\ \approx\ \mathtt{Nat}$.
Which can be used in $a::InTC_{0}(p)$ to cast $a$ from $Nat'$ to $\mathtt{Nat}$.
But then we need evidence that $(A:\star)\rightarrow A\rightarrow\mathtt{Vec}\,\mathtt{Nat}\,(a::InTC_{0}(p))\approx(A:\star)\rightarrow A\rightarrow\mathtt{Vec}\,\mathtt{Nat}\,2$ to avoid a sperous cast.
\todo[inline]{avoiding these sperous casts is important for the conjectured gradual correctenss, well typed surface language terms should not generate warnings}
Fist we need congruence to select the subterm that is of intrest, $(A:\star)\rightarrow A\rightarrow\mathtt{Vec}\,\mathtt{Nat}\,\underline{(a::InTC_{0}(p))}\approx(A:\star)\rightarrow A\rightarrow\mathtt{Vec}\,\mathtt{Nat}\,\underline{2}$.
The $Cong$ syntax eplixitly embedds a  path into a larger expression, here $Cong_{x=> (A : *) -> A -> \mathtt{Vec} \mathtt{Nat} x} ...$ selects the relevent part of the type.

Ideally we would have a path from $a::InTC_{0}(p)\ \approx\ 2$, but we only have a path $InTC_{1}(p)\ :\ a\ \approx\ 2$.
We will use the $CastL$ to produce the path $CastL_{InTC_{0}(p)}\left(InTC_{1}p\right):(a::InTC_{0}p))\approx2$.
$CastL$ will be derivable within the extended cast system.

\todo[inline]{Make 2 examples?}

\todo[inline]{Note that an arbitrary choice was was made $InTC_{1}(p)$ $InTC_{2}(p)$}

\subsubsection{Need to remove a cast}
\todo{better name}

Consider this surface language expression that extracts the last element from a non-empty list.
Assume the function $last:(n:Nat)\rightarrow Vec\,A\,(S\,N)\rightarrow A$ is in scope.
\todo{or just define the recursive function}

\begin{lstlisting}[basicstyle={\ttfamily\small}]
case v <_: Vec A (S x) => A > {
| Cons _ a (Z) _ => a
| Cons _ _ (S n) rest => last n rest
}
\end{lstlisting}

This will elaborate into 

\begin{lstlisting}[basicstyle={\ttfamily\small}]
case v <_: Vec A (S x) => A > {
| (Cons A' a' (Z)::q rest) :: p => a' :: (inTC1(p))-1
| (Cons A' a' (S n)::q rest)::p => last n (rest :: p')
| (Nil A')::p => !TCon1(p)
}
\end{lstlisting}

\todo[inline]{prettier expressions, rev to -1, in general it might be clearer if cast langauge is always in math mode}

In the 2nd branch we have
  $A':\star$, $a':A'$, $n:\mathtt{Nat}$,
  $q:\mathtt{Nat}\approx \mathtt{Nat}$,
  $rest:\mathtt{Vec}\,A'\,\left(\left(\mathtt{S}\,n\right)::q\right)$,
  and $p:\mathtt{Vec}\,A'\,\left(\mathtt{S}\left(\left(\mathtt{S}\,n\right)::q\right)\right)\approx \mathtt{Vec}\,A\,\left(\mathtt{S}\,x\right)$.
Elaboration cannot unify a solution unless we can remove casts, otherwise it becomes impossible to construct a path from $n \approx x$
  from $\mathtt{Vec}\,A'\,\left(\left(\mathtt{S}\,n\right)\underline{::q}\right)\approx \mathtt{Vec}\,A\,\left(\mathtt{S}\,x\right)$, since the cast blocks
  $\mathtt{S}\,n = \mathtt{S}\,x$ that would be derivable by unification in the surface language.
We will need an operator that can remove casts from the endpoints of paths that arise from unification.
We will call these operators $uncastL$ and $uncastR$ and they will be derivable in the cast language.
With these operations we can match the process of surface level unification so that

$p'\ =\ Cong_{x\Rightarrow \mathtt{Vec}\,A'\,x}\left(UncastR\left(refl\right)\right)\circ Cong_{x\Rightarrow \mathtt{Vec}\,x\,\left(\mathtt{S}\,n\right)}\left(TCon_{0}p\right):\ \mathtt{Vec}\,A'\,\left(\left(\mathtt{S}\,n\right)::q\right)\ \approx\ \mathtt{Vec}\,A\,\left(\mathtt{S}\,n\right)$

Where 
$UncastR\left(refl\right): \left(\mathtt{S}\,n\right)::q  \approx \mathtt{S}\,n $

In the first branch we have, 
  $A':\star$, $a':A'$,
  $q:\mathtt{Nat}\approx \mathtt{Nat}$,
  $rest:\mathtt{Vec}\,A'\,\left(\mathtt{Z}::q\right)$, 
  and $p:\mathtt{Vec}\,A'\,\left(\mathtt{S}\left(\mathtt{Z}::q\right)\right)\approx \mathtt{Vec}\,A\,\left(\mathtt{S}\,x\right)$.
Unification can proceed to derive $TCon_{1}\left(Con_{0}\left(p\right)\right)^{-1}:\ x\approx \mathtt{Z}::q$
and $TCon_{1}\left(p\right)^{-1}:\ A\approx A'$.

In the final branch we have $p:\mathtt{Vec}\,A'\,\mathtt{Z}\approx \mathtt{Vec}\,A'\,\left(\mathtt{S}\,x\right)$,
which is contradicted by $TCon_{1}p:\mathtt{Z}\approx \mathtt{S}\,x$

\todo[inline]{Example: transitivity, translate out to motive}

\section{Cast Language Pattern Matching}

Pattern matching across dependent data types will allow for observations that were impossible before.
For instance it is now possible to observe specific arguments in type constructors and term constructors.
Since function terms can appear directly in data type constructors it is now possible to observe functions directly, when they are used in congruence.
Unfortunately formalizing this system seems fairly complicated, see \ref{fig:cast-data} for the extended cast language syntax with paths. 

\begin{figure}
\begin{tabular}{lcll}
\multicolumn{4}{l}{path var,}\tabularnewline
$x_{p}$ &  &  & \tabularnewline
\multicolumn{4}{l}{assertion index,}\tabularnewline
$k$ &  &  & \tabularnewline
\multicolumn{4}{l}{assertion assumption,}\tabularnewline
$kin$ & $\Coloneqq$ & $k=left$ | $k=right$ & \tabularnewline
\multicolumn{4}{l}{assertion maps,}\tabularnewline
$kmap$ & $\Coloneqq$ & $\overline{kin,}$ & \tabularnewline
\multicolumn{4}{l}{casts under assumption,}\tabularnewline
$kcast$ & $\Coloneqq$ & $\overline{kmap,p;}$ & \tabularnewline
\multicolumn{4}{l}{path exp.,}\tabularnewline
$p,p'$ & $\Coloneqq$ & $x_{p}$ & \tabularnewline
 & | & $Assert_{k\Rightarrow C}$ & concrete assumption\tabularnewline
 & | & $\overline{p\circ}$ & concatenated paths\tabularnewline
 & | & $cong_{x\Rightarrow a}p$ & congruence\tabularnewline
 & | & $p^{-1}$ & \tabularnewline
 & | & $inTC_{i}\,p$ & \tabularnewline
 & | & $inC_{i}\,p$ & \tabularnewline
\multicolumn{4}{l}{cast pattern,}\tabularnewline
$patc$ & $\Coloneqq$ & $x$ | $\left(d\,\overline{patc}\right)::x_{p}$ & \tabularnewline
\multicolumn{3}{l}{cast expression,} & \tabularnewline
$a$... & $\Coloneqq$ & ... & \tabularnewline
 & | & $D$ & type cons.\tabularnewline
 & | & $d$ & data cons.\tabularnewline
 & | & $\mathsf{case}\,\overline{a,}\,\left\{ \overline{|\,\overline{patc\Rightarrow}b}\overline{|\,\overline{patc'\Rightarrow}!_{\ell}}\right\} $ & data elim.\tabularnewline
 & | & $!_{kcast}$ & force blame\tabularnewline
 & | & $a::kcast$ & cast\tabularnewline
 & | & $\left\{ a\sim_{k,o,\ell}b\right\} $ & assert same\tabularnewline
\multicolumn{4}{l}{observations,}\tabularnewline
o & $\Coloneqq$ & ... & \tabularnewline
 & | & $o.App[a]$ & application\tabularnewline
 & | & $o.TCon_{i}$ & type cons. arg.\tabularnewline
 & | & $o.DCon_{i}$ & data cons. arg.\tabularnewline
 & | & $inEx_{\overline{patc}}[\overline{a}]$ & in-exhaustive pattern match\tabularnewline
\end{tabular}

\todo[inline]{extend H ctxs}

\caption{Cast Language Data}
\label{fig:cast-data}
\end{figure}

\begin{figure}
\begin{tabular}{lclll}
(the empty path) & written & $refl$ &  & \tabularnewline
$a::kmap,p;kmap',q;...$ & written & $a::\begin{array}{cc}
kmap, & p\\
kmap', & q\\
... & ...
\end{array}$ &  & \tabularnewline
$a::\begin{array}{cc}
kmap,k=left, & p\\
kmap,k=right, & p\\
kmap',k=left, & q\\
kmap,k=right, & q\\
... & ...
\end{array}$ & written & $a::\begin{array}{cc}
kmap, & p\\
kmap', & q\\
... & ...
\end{array}$ & when  & $k$ is irrelevant\tabularnewline
\end{tabular}

\caption{Surface Language Abbreviations}
\label{fig:xxxxxxxx}
\end{figure}

In this thesis we have taken an extremely extensional perspective,
terms are only different if an observation recognizes a difference.
For instance this approach justifies equating the functions $\lambda x\Rightarrow x+1=\lambda x\Rightarrow1+x$
without proof, even though they are usually definitionally distinct.
Therefore we will only blame inequality across functions if two functions
that were asserted to be equal return different observations for ``the
same'' input. Tracking that two functions should be equal becomes
complicated, the system must be sensible under context, functions
can take other higher order inputs, and function terms can be copied
freely.

\todo[inline]{Example back to pattern matching?}

The cleanest way I could find to encode a dynamic check for function
equality, was with a new term level construct\footnote{it would also be possible to extend the system with meta variables,
though this seems harder to formalize}. This assertion that two terms are the same is written as $\left\{ a\sim_{k,o,\ell}b\right\} $
and will evaluate $a$and $b$ in parallel until a head constructor
is reached on each branch. If the constructor is the same it will
commute out of the term. If the head constructor is different the
term will get stuck with the information for the final blame message.
For instance, $\left\{ \lambda x\Rightarrow x+1\sim_{k,o,\ell}\lambda x\Rightarrow1+x\right\} \rightsquigarrow_{*}\lambda x\Rightarrow\left\{ x+1\sim_{k,o.App[x],\ell}1+x\right\} $.

Since this equational construct is already needed for functions, we
will use it to handle all equality assertions. For instance, $\left\{ \left(\lambda x\Rightarrow S\,x\right)Z\sim_{k,o,\ell}2+2\right\} \rightsquigarrow_{*}\left\{ S\,Z\sim_{k,o,\ell}S\,\left(1+2\right)\right\} \rightsquigarrow_{*}S\,\left\{ Z\sim_{k,o.DCon_{0},\ell}1+2\right\} \rightsquigarrow_{*}S\,\left\{ Z\sim_{k,o.DCon_{0},\ell}S\,2\right\} $.
We compute past the first $S$ constructor and blame the predecessor
for not being equal.

Unfortunately this dynamic assertion complicates other aspects of
the system. Specifically, 
\begin{itemize}
\item How do same assertions interact with casts? For instance $\left\{ 1::Bool\sim_{k,o,\ell}2::Bool\right\} $. 
\item How do sameness assertions cast check? This is difficult, because
there is no requirement that a user asserted equality is of the same
type. For instance what type should the term $\left\{ 1\sim_{k,o,\ell}True\right\} $
have?
\end{itemize}
Since there will only ever be a fintite number of assumptions, we
can give each assumption a unique index $k$ and consider all computations
and judgments point-wise for every different combinations of $k$s.
We will extend the notion of cast so different casts are possible
for every assignment of $k$s in scope. So 

$\left\{ 1::Bool\sim_{k,o,\ell}2::Bool\right\} \rightsquigarrow_{*}\left\{ 1\sim_{k,o,\ell}2::Bool\right\} ::k=left,Bool\rightsquigarrow_{*}\left\{ 1\sim_{k,o,\ell}2\right\} ::\begin{array}{c}
k=right,Bool\\
k=left,Bool
\end{array}=\left\{ 1\sim_{k,o,\ell}2\right\} ::Bool$ 

where we allow syntactic sugar to summarize the cast when they are
the same over all branches (Figure \ref{fig:xxxxxxxx}).

We will also index typing judgments by the choice of $k$ in scope
so that, $k=left\vdash\left\{ 1\sim_{k,o,\ell}True\right\} :Nat$
and $k=right\vdash\left\{ 1\sim_{k,o,\ell}True\right\} :Bool$.

To control the assumptions in scope, in a closed term every $k$ will
be bound in the $C$ expression of an $Assert_{k\Rightarrow C}$ path.

Now we must consider how patterns would evaluate under assumptions.
The original inspiration was to allow abstraction over casts as represented
by a path, but now casts contain a bundle of paths each indexed by
assumption, that may or may not have the same start and endpoint.
Luckily, we can maintain the operational behavior by allowing uniform
substitution into path variables.

For simplicity of the formalization, we will require that $kmap$s
always uniquely map every $k$ index in scope. We will also assume
that $kcast$ handles all possible mappings of $k$s in scope.

Substitution is outlined in table \ref{fig:cast-data-sub}. The function
$\left\lfloor -\right\rfloor _{k=-}$ filters a term along $k$ taking
it out of scope. For instance $\left\lfloor \left\{ 7\sim_{k,o,\ell}True\right\} \right\rfloor _{k=left}=7$
and $\left\lfloor 3::\begin{array}{ccc}
k=left & j=left & Bool\\
k=left & j=right & Nat\\
k=right & j=left & String\\
k=right & j=right & Unit
\end{array}\right\rfloor _{k=right}=3::\begin{array}{cc}
j=left & String\\
j=right & Unit
\end{array}$. The function $\left\lceil -\right\rceil ^{k}$ puts an assumption
$k$ into scope extending it in every cast. For instance, $\left\lceil 3::\begin{array}{cc}
j=left & String\\
j=right & Unit
\end{array}\right\rceil ^{k}=3::\begin{array}{ccc}
j=left & k=left & String\\
j=left & k=right & String\\
j=right & k=left & Unit\\
j=right & k=right & Unit
\end{array}$ and $\left\lceil !_{x_{p}}\right\rceil ^{k}=!_{\begin{array}{cc}
k=left & x_{p}\\
k=right & x_{p}
\end{array}}$. The subscript $kcast_{kmap}$ selects the appropriate assumption
from the $kcast$. For instance $\begin{array}{ccc}
k=left & j=left & x_{p}\\
k=left & j=right & refl\\
k=right & j=left & y_{p}\\
k=right & j=right & x_{p}
\end{array}_{k=right,j=left}=y_{p}$ . Since we have assumed that every choice in scope is handled, this
result always exists.

\begin{figure}
\begin{tabular}{ccc}
$\mathsf{case}\,\overline{b,}\,\left\{ \overline{|\,\overline{patc\Rightarrow}c}\overline{|\,\overline{patc'\Rightarrow}!_{\ell}}\right\} $ & $\left[x\coloneqq a\right]=$ & $\mathsf{case}\,\overline{b\left[x\coloneqq a\right],}\,\left\{ \overline{|\,\overline{patc\Rightarrow}c\left[x\coloneqq a\right]}\overline{|\,\overline{patc'\Rightarrow}!_{\ell}}\right\} $\tabularnewline
$!_{kcast}$ & $\left[x\coloneqq a\right]=$ & $!_{kcast\left[x\coloneqq a\right]}$\tabularnewline
$b::kcast$ & $\left[x\coloneqq a\right]=$ & $b\left[x\coloneqq a\right]::kcast\left[x\coloneqq a\right]$\tabularnewline
$\left\{ b\sim_{k,o,\ell}c\right\} $ & $\left[x\coloneqq a\right]=$ & $\left\{ b\left[x\coloneqq\left\lfloor a\right\rfloor _{k=left}\right]\sim_{k,o\left[x\coloneqq a\right],\ell}c\left[x\coloneqq\left\lfloor a\right\rfloor _{k=right}\right]\right\} $\tabularnewline
... & ... & ...\tabularnewline
$x_{p}$ & $\left[x\coloneqq a\right]=$ & $x_{p}$\tabularnewline
$Assert_{k\Rightarrow C}$ & $\left[x\coloneqq a\right]=$ & $Assert_{k\Rightarrow C\left[x\coloneqq\left\lceil a\right\rceil ^{k}\right]}$\tabularnewline
$\overline{p\circ}$ & $\left[x\coloneqq a\right]=$ & $\overline{p\left[x\coloneqq a\right]\circ}$\tabularnewline
$cong_{y\Rightarrow b}p$ & $\left[x\coloneqq a\right]=$ & $cong_{y\Rightarrow b\left[x\coloneqq a\right]}p\left[x\coloneqq a\right]$\tabularnewline
$p^{-1}$ & $\left[x\coloneqq a\right]=$ & $p\left[x\coloneqq a\right]^{-1}$\tabularnewline
$inTC_{i}\,p$ & $\left[x\coloneqq a\right]=$ & $inTC_{i}\,\left(p\left[x\coloneqq a\right]\right)$\tabularnewline
$inC_{i}\,p$ & $\left[x\coloneqq a\right]=$ & $inC_{i}\,\left(p\left[x\coloneqq a\right]\right)$\tabularnewline
$\overline{kmap,p;}$ & $\left[x\coloneqq a\right]=$ & $\overline{kmap,p\left[x\coloneqq\left\lfloor a\right\rfloor _{kmap}\right];}$\tabularnewline
... &  & ...\tabularnewline
$x_{p}$ & $\left[x_{p}\coloneqq p\right]=$ & $p$\tabularnewline
\end{tabular}

\begin{tabular}{ccc}
$\mathsf{case}\,\overline{a,}\,\left\{ \overline{|\,\overline{patc\Rightarrow}c}\overline{|\,\overline{patc'\Rightarrow}!_{\ell}}\right\} $ & $\left[x_{p}\coloneqq kcast\right]=$ & $\mathsf{case}\,\overline{a\left[x_{p}\coloneqq kcast\right],}\,\left\{ \overline{|\,\overline{patc\Rightarrow}c\left[x_{p}\coloneqq kcast\right]}\overline{|\,\overline{patc'\Rightarrow}!_{\ell}}\right\} $\tabularnewline
$!_{kcast}$ & $\left[x_{p}\coloneqq kcast\right]=$ & $!_{kcast\left[x_{p}\coloneqq kcast\right]}$ (issue)\tabularnewline
$a::kcast'$ & $\left[x_{p}\coloneqq kcast\right]=$ & $a\left[x_{p}\coloneqq kcast\right]::kcast'\left[x_{p}\coloneqq kcast\right]$\tabularnewline
$\left\{ b\sim_{k,o,\ell}c\right\} $ & $\left[x_{p}\coloneqq kcast\right]=$ & $\left\{ b\left[x\coloneqq\left\lfloor kcast\right\rfloor _{k=left}\right]\sim_{k,o\left[x\coloneqq a\right],\ell}c\left[x\coloneqq\left\lfloor kcast\right\rfloor _{k=right}\right]\right\} $\tabularnewline
... & ... & ...\tabularnewline
$\overline{kmap,p;}$ & $\left[x_{p}\coloneqq kcast\right]=$ & $\overline{kmap,p\left[x_{p}\coloneqq kcast_{kmap}\right];}$\tabularnewline
\end{tabular}

\caption{Cast Language Data Sub}

\label{fig:cast-data-sub}
\end{figure}


\section{Cast Value, Blame, and Reductions}

We need to extend the notion of value, Blame and call-by-value reduction
from Chapter 3.

\begin{figure}
\[
\frac{a\:\mathbf{whnf}\quad b\:\mathbf{whnf}\quad\mathbf{head}\,a\neq\mathbf{head}\,b}{\textbf{Blame}\:\ensuremath{\ell}\,o\,\left\{ a\sim_{k,o,\ell}b\right\} }
\]

\[
\frac{\overline{a}\,\mathbf{Match}\,\overline{patc'}_{j}}{\textbf{Blame}\:\ensuremath{\ell}\,inEx_{\overline{patc'}_{j}}[\overline{a}]\,\mathsf{case}\,\overline{a,}\,\left\{ \overline{|\,\overline{patc\Rightarrow}b}\overline{|\,\overline{patc'\Rightarrow}!_{\ell_{j}}}\right\} }
\]

\todo[inline]{fix syntax?}

\[
\frac{\overline{a}\ \mathbf{!Match}\ \ensuremath{\ell}\,o\,\overline{|\overline{patc'}_{j}}}{\textbf{Blame}\:\ensuremath{\ell}\,o\,\mathsf{case}\,\overline{a,}\,\left\{ \overline{|\,\overline{patc\Rightarrow}-}\right\} }
\]

\[
\frac{!_{kcast}\quad\textbf{Blame}\:\ensuremath{\ell}\,o\ kcast}{\textbf{Blame}\:\ensuremath{\ell}\,o\,!_{kcast}}
\]

\caption{Selection of Cast Language Blame}
\label{fig:cast-data-Blame}
\end{figure}

Blame the blame conditions of Chapter 3 are simplified via the sameness
assertion\ref{fig:cast-data-Blame}. There are two new sources of
blame from the case construct. The cast language records every ``unhandled''
branch and if a scrutinee hits one of those branches the case will
be blamed for in-exhaustiveness \footnote{This runtime error is conventional in ML style languages, and is even
how Agda handles incomplete matches }. If a scrutinee list primitively contradicts the pattern coverage
via the $\mathbf{!Match}$ judgment blame will be extracted from the
scrutinee. Since our type system will ensure complete coverage (based
only on constructors) if a scrutinee escapes the complete pattern
match in an empty context, it must be that there was a cast blamable
cast to the head constructor. We have elided most of the structural
rules that extract blame from terms, paths, and casts. We have left
the structural rule for explicit blame for emphasis.

\begin{figure}
\todo[inline]{REDO THIS}

\[
\frac{\textbf{Blame}\:\ensuremath{\ell}\,o\ p}{\left(d'\ \overline{patc}\right)::p\ \mathbf{!Match}\ \ensuremath{\ell}\,o\,\left(d\ \overline{patc}\right)::p}
\]

\todo[inline]{This figure is not very helpful?}

\caption{Cast Language Blame}
\label{fig:cast-data-NoMatch}
\end{figure}

\begin{figure}
\[
\frac{\,}{\star\ \mathbf{Val}}
\]

\[
\frac{\,}{\left(x:A\right)\rightarrow B\ \mathbf{Val}}
\]

\[
\frac{kcast\ \mathbf{Val}\quad\forall Assert_{k\Rightarrow\left(x:A\right)\rightarrow B}\in kcast}{\left(fun\,f\,x\Rightarrow a\right)::kcast\ \mathbf{Val}}
\]

\todo[inline]{clean up notation above}

\[
\frac{\overline{a}\ \mathbf{Val}}{D\,\overline{a}\ \mathbf{Val}}
\]

\[
\frac{\overline{a}\ \mathbf{Val}\quad kcast\ \mathbf{Val}\quad\exists D.\forall Assert_{k\Rightarrow D\overline{b}}\in kcast}{\left(d\,\overline{a}\right)::kcast\ \mathbf{Val}}
\]

\todo[inline]{clean up notation above}

\[
\frac{C\ \mathbf{Val}\quad C\neq\star}{Assert_{k\Rightarrow C}\ \mathbf{Val}}
\]

\[
\frac{p\ \mathbf{Val}\quad q\ \mathbf{Val}}{p\circ q\ \mathbf{Val}}
\]

\[
\frac{\ }{kmap,refl\ \mathbf{Val}}
\]

\[
\frac{p\ \mathbf{Val}}{kmap,p\ \mathbf{Val}}
\]

\[
\frac{\forall kmap,p\in kcast,\ kmap,p\,\mathbf{Val}}{kcast\ \mathbf{Val}}
\]

type and term constructors need not be fully applied.

\caption{Selection of Cast Language Values}
\label{fig:cast-data-vals}
\end{figure}

The value forms of the language must also be extended, values are
presented in \ref{fig:cast-data-vals}. As before type level values
must have all type level casts reduced, term level values are allowed
casts as long as they have plausible head form.

\begin{figure}
path reductions

\[
\frac{\ }{Assert_{-\Rightarrow\star}\rightsquigarrow refl}
\]

\[
\frac{\ }{cong_{x\Rightarrow a}\left(Assert_{k\Rightarrow c}\right)\rightsquigarrow Assert_{k\Rightarrow a\left[x\coloneqq c\right]}}
\]

\[
\frac{\ }{cong_{x\Rightarrow a}refl\rightsquigarrow refl}
\]

\[
\frac{\ }{cong_{x\Rightarrow a}\left(p\circ q\right)\rightsquigarrow\left(cong_{x\Rightarrow a}p\right)\circ\left(cong_{x\Rightarrow a}q\right)}
\]

\[
\frac{\ }{\left(Assert_{k\Rightarrow C}\right)^{-1}\rightsquigarrow Assert_{k\Rightarrow\mathbf{Swap}_{k}C}}
\]

\[
\frac{\quad}{refl^{-1}\rightsquigarrow refl}
\]

\[
\frac{\quad}{\left(q\circ p\right)^{-1}\rightsquigarrow p^{-1}\circ q^{-1}}
\]

\[
\frac{\ }{inTC_{i}\left(p\circ Assert_{k\Rightarrow D\overline{a}}\right)\rightsquigarrow inTC_{i}\left(p\right)\circ Assert_{k\Rightarrow a_{i}}}
\]

\[
\frac{\ }{inTC_{i}\left(refl\right)\rightsquigarrow refl}
\]

\[
\frac{\ }{inC_{i}\left(p\circ Assert_{k\Rightarrow\left(d\overline{a}\right)::kcast}\right)\rightsquigarrow inC_{i}\left(p\right)\circ Assert_{k\Rightarrow a_{i}}}
\]

\todo[inline]{given suitable condition on the kcast}

\[
\frac{\ }{inC_{i}\left(refl\right)\rightsquigarrow refl}
\]

structural rules

\[
\frac{C\rightsquigarrow C'}{Assert_{k\Rightarrow C}\rightsquigarrow Assert_{k\Rightarrow C'}}
\]

\[
\frac{q\rightsquigarrow q'}{p\circ q\rightsquigarrow p\circ q'}
\]

\[
\frac{p\rightsquigarrow p'}{p\circ q\rightsquigarrow p'\circ q}
\]

assumption reductions

\[
\frac{p\rightsquigarrow p'}{\begin{array}{c}
kcast\\
\overline{kin,}p;\\
kcast'
\end{array}\rightsquigarrow\begin{array}{c}
kcast\\
\overline{kin,}p';\\
kcast'
\end{array}}
\]

\[
\frac{a\rightsquigarrow a'}{a::kcast\rightsquigarrow a'::kcast}
\]

\[
\frac{kcast\rightsquigarrow kcast'}{a::kcast\rightsquigarrow a::kcast'}
\]

term reductions
\[
\frac{p\rightsquigarrow p'}{!_{p}\rightsquigarrow!_{p'}}
\]

\[
\frac{\,}{\left\{ a::\begin{array}{l}
kcast\\
\overline{kin,}p\circ Assert_{k\Rightarrow C};\\
kcast'
\end{array}\sim_{k,o,\ell}b\right\} \rightsquigarrow\left\{ a::\begin{array}{c}
kcast\\
\overline{kin,}p;\\
kcast'
\end{array}\sim_{k,o,\ell}b\right\} ::\overline{kin,}k=left\,Assert_{k\Rightarrow C};}
\]

\todo[inline]{could remove more then just assertions}

\[
\frac{\,}{\left\{ \star\sim_{k,o,\ell}\star\right\} \rightsquigarrow\star}
\]

\[
\frac{\,}{\left\{ \left(x:A\right)\rightarrow B\sim_{k,o,\ell}\left(x:A'\right)\rightarrow B'\right\} \rightsquigarrow\left(x:\left\{ A\sim_{k,o.arg,\ell}A'\right\} \right)\rightarrow\left\{ B\sim_{k,o.bod\left[x\right],\ell}B'\right\} }
\]

\[
\frac{\,}{\left\{ \mathsf{fun}\,f\,x\Rightarrow b\sim_{k,o,\ell}\mathsf{fun}\,f\,x\Rightarrow b'\right\} \rightsquigarrow\mathsf{fun}\,f\,x\Rightarrow\left\{ b\sim_{k,o.app\left[x\right],\ell}b'\right\} }
\]

\[
\frac{\,}{\left\{ d\overline{a}\sim_{k,o,\ell}d\overline{a'}\right\} \rightsquigarrow d\overline{\left\{ a_{i}\sim_{k,o.DCon[i],\ell}a'_{i}\right\} }}
\]

\[
\frac{\,}{\left\{ D\overline{a}\sim_{k,o,\ell}D\overline{a'}\right\} \rightsquigarrow D\overline{\left\{ a_{i}\sim_{k,o.TCon[i],\ell}a'_{i}\right\} }}
\]

\[
\frac{\,}{a::refl\rightsquigarrow a}
\]

\todo[inline]{by shorthand}

\[
\frac{\ }{\left(a::kcast\right)::kcast'\rightsquigarrow a::\left(kcast\circ kcast'\right)}
\]

\todo[inline]{by shorthand}

\[
\frac{\ }{\left(a::\begin{array}{c}
kcast\\
kin,q\ Assert_{k\Rightarrow\left(x:A\right)\rightarrow B};\\
kcast'
\end{array}\right)b\rightsquigarrow\left(\left(a::\begin{array}{c}
kcast\\
kin,q\ Assert_{k\Rightarrow\left(x:A\right)\rightarrow B};\\
kcast'
\end{array}\right)\left(b::kin,Assert_{k\Rightarrow\mathbf{Swap}_{k}A};\right)\right)::kin,Assert_{k\Rightarrow B\left[x\coloneqq\left\lfloor \left(b::kin,Assert_{k\Rightarrow\mathbf{Swap}_{k}A};\right)\right\rfloor _{k=kin}\right]};}
\]

\todo[inline]{double check}

\[
\frac{\overline{a}\ \mathbf{Match}\ \overline{patc}_{i}}{\mathsf{case}\,\overline{a,}\,\left\{ \overline{|\,\overline{patc\Rightarrow}b_{i}}\overline{|\,\overline{patc'\Rightarrow}!_{\ell}}\right\} \rightsquigarrow b_{i}\left[patc_{i}\coloneqq\overline{a}\right]}
\]

\caption{Summery of Cast Language Reductions}
\label{fig:cast-data-red}
\end{figure}

This extension to the syntax induces many more reduction rules. We
include a summery of selected reduction rules in \ref{fig:cast-data-red}.
We do not show the value restrictions to avoid clutter\footnote{there are also multiple ways to lay them out. For instance we could
evaluate paths left to right or right to left.}. The important properties of reduction are 
\begin{itemize}
\item Paths reduce into a stack of zero or more $Assert_{k\Rightarrow A}$s
\item Sameness assertions emit observably consistent constructors, and record
the needed observations
\item Sameness assertions will get stuck on inconsistent constructors
\item Casts can commute out of sameness assertions with proper index tracking
\item function application can commute around $kcasts$ , similar to Chapter
3, but will keep $k$ assumptions properly indexed
\end{itemize}
Matching is defined in \ref{fig:cast-data-match}. Note that uncast
terms are equivalent to refl cast terms.

\begin{figure}
\todo[inline]{swap, pattern on the left?}

\todo[inline]{record substitutions}

\[
\frac{\,}{a\ \mathbf{Match}\ x}
\]

\[
\frac{\overline{a}\ \mathbf{Match}\ \overline{patc}}{d\,\overline{a}\ \mathbf{Match}\ \left(d\,\overline{patc}\right)::x_{p}}
\]

\[
\frac{\overline{a}\ \mathbf{Match}\ \overline{patc}}{\left(d\,\overline{a}\right)::kcast\ \mathbf{Match}\ \left(d\,\overline{patc}\right)::x_{p}}
\]

\[
\frac{\overline{a}\ \mathbf{Match}\ \overline{patc}}{\left(d\,\overline{a}\right)::kcast\ \mathbf{Match}\ \left(d\,\overline{patc}\right)::x_{p}}
\]

\[
\frac{\,}{.\ \mathbf{Match}\ .}
\]

\[
\frac{b\ \mathbf{Match}\ patc'\quad\overline{a}\ \mathbf{Match}\ \overline{patc}}{b\overline{a}\ \mathbf{Match}\ patc'\overline{patc}}
\]

substitution abbreviation

\[
-\left[\left(d\,\overline{patc}\right)::x_{p}\coloneqq d\,\overline{a}\right]=-\left[\overline{patc}\coloneqq\overline{a}\right]\left[x_{p}\coloneqq refl\right]
\]

\[
-\left[\left(d\,\overline{patc}\right)::x_{p}\coloneqq\left(d\,\overline{a}\right)::kcast\right]=-\left[\overline{patc}\coloneqq\overline{a}\right]\left[x_{p}\coloneqq kcast\right]
\]

\caption{Cast Language Matching and sub}
\label{fig:cast-data-match}
\end{figure}

\todo[inline]{double check paths are fully applied when needed}

The Cast language extension defined in this chapter is fairly complex.
Though all the meta-theory of this section is plausible, we have not
fully formalized it, and there is a potential that some subtle errors
exist. To be as clear as possible about the uncertainty around the
meta-theory proposed in this chapter, I will list what would normally
be considered theorems and lemmas as conjectures. \todo{weird place to make this note. add it to the front or back matter?}
\begin{conjecture}
There is a suitable definitional equality $\equiv$ , overloaded to
all syntactic constructs, such that

$\equiv$ is an equivalence

$a\rightsquigarrow_{*}b$ and $a'\rightsquigarrow_{*}b$ implies $a\equiv a'$

$p\rightsquigarrow_{*}q$ and $p'\rightsquigarrow_{*}q$ implies $p\equiv p'$

$kcast\rightsquigarrow_{*}kcast''$ and $kcast'\rightsquigarrow_{*}kcast''$
implies $kcast\equiv kcast''$

if $\mathbf{Head\ }a\neq\mathbf{Head}\ b$ then $a\cancel{\equiv}b$
\todo{settle on a capitalization for head}

if \textup{$kmap$ and $kmap'$ have consistent assignments then}
$kmap\equiv kmap'$

\todo{supports subtitutivity}
\end{conjecture}


\section{Cast System and Pathing}

The cast system needs to maintain the consistency of well cast terms
and also well typed paths. But unlike in Chapter 3 each judgment indexed
by choices of $k$.

The typing and pathing judgments are listed in \ref{fig:cast-data-ty}.
Pathing judgments record the endpoint of paths with the $\thickapprox$
symbol. 

\begin{figure}
\[
\frac{x_{p}:A\thickapprox A'\in H}{HK\vdash x_{p}:A\thickapprox A'}
\]

\[
\frac{HK,k=left\vdash a:A\quad HK,k=right\vdash a:A'}{HK\vdash Assert_{k\Rightarrow a}:\left\lfloor a\right\rfloor _{k=left}\thickapprox\left\lfloor a\right\rfloor _{k=right}}
\]

\todo[inline]{does A=A'?}

\[
\frac{HK\vdash p:A\thickapprox B\quad HK\vdash q:B\thickapprox C}{HK\vdash p\circ q:A\thickapprox C}
\]

\[
\frac{HK\vdash p:b\thickapprox b\quad H,K\vdash b:B\quad H,K\vdash b':B\quad H,x:B,K\vdash A}{HK\vdash cong_{x\Rightarrow a}p:a\left[x\coloneqq b\right]\thickapprox a\left[x\coloneqq b'\right]}
\]

\[
\frac{HK\vdash p:A\thickapprox B}{HK\vdash p^{-1}:B\thickapprox A}
\]

\[
\frac{HK\vdash p:\left(D\,\overline{a}\right)::kcast\ \thickapprox\ \left(D\,\overline{b}\right)::kcast'}{HK\vdash inTC_{i}\,p:a_{i}\thickapprox b_{i}}
\]

\todo[inline]{and fully applied!}

\[
\frac{HK\vdash p:\left(d\,\overline{a}\right)::kcast\ \thickapprox\ \left(d\,\overline{b}\right)::kcast'}{HK\vdash inC_{i}\,p:a_{i}\thickapprox b_{i}}
\]

\todo[inline]{and fully applied!}

\todo[inline]{possibly force a matching type? but then it is unclear what type the
conclusion should have}

\[
\frac{a'\equiv a\quad b'\equiv b\quad HK\vdash p:a\thickapprox b}{HK\vdash p:a'\thickapprox b'}
\]

\todo[inline]{will need to adjust this if moves to a typed conversion rule}

\[
\frac{k=left\in K\quad HK\vdash a:A}{HK\vdash\left\{ a\sim_{k,o,\ell}b\right\} :A}
\]

\[
\frac{k=right\in K\quad HK\vdash b:B}{HK\vdash\left\{ a\sim_{k,o,\ell}b\right\} :B}
\]

\[
\frac{HK\vdash a:A\quad HK\vdash kcast_{K}:A\thickapprox B}{HK\vdash a::kcast\ :B}
\]

\[
\frac{\begin{array}{c}
HK\vdash\overline{a}:\Delta\\
H,\Delta,K\vdash B:\star\\
\forall\:i\,\left(HK\vdash\overline{pat}_{i}:\Delta\quad H,\left(\overline{pat}_{i}:\Delta\right)K\vdash b_{i}:B\right)\\
\forall\:j\,\left(HK\vdash\overline{pat}_{j}:\Delta\right)\\
HK\vdash\overline{\overline{patc}}\,\overline{\overline{patc'}}:\Delta\ \mathbf{complete}
\end{array}}{\begin{array}{c}
HK\vdash\mathsf{case}\,\overline{a,}\,\left\{ \overline{|\,\overline{patc_{i}\Rightarrow}b_{i}}\overline{|\,\overline{patc'_{j}\Rightarrow}!_{\ell}}\right\} \\
:M\left[\Delta\coloneqq\overline{a}\right]
\end{array}}
\]

\todo[inline]{pattern expansion and pattern on context may need further exposition}

\[
\frac{\begin{array}{c}
HK\vdash C:\star\\
HK\vdash kcast_{K}:a\thickapprox a'\\
HK\vdash a:A\quad HK\vdash a':A\\
\mathbf{Head}(a)\neq\mathbf{Head}(a')
\end{array}}{HK\vdash!_{kcast}\ :C}
\]

\todo[inline]{REMOVE TYPE RESTRICTION? the extra type restrictions is intended to
force blame to the type level when needed, though this will not (cannot?)
be invariant over reduction of paths in general}

\caption{Cast Language typing and pathing rules}
\label{fig:cast-data-ty}
\end{figure}

\todo[inline]{technically speaking, telescopes should generalize to the different
syntactic classes}

We now conjecture the core lemmas that could be used to prove cast
soundness
\begin{conjecture}
substitution of cast terms preserves cast

equivalently the following rule is admissible

\[
\frac{HK\vdash a:A\quad x:A\in H\quad HK\vdash b:B}{H\left[x\coloneqq a\right]K\vdash b\left[x\coloneqq a\right]:B\left[x\coloneqq a\right]}
\]
\end{conjecture}

% ...
\begin{conjecture}
substitution of typed path preserves type

equivalently the following rule is admissible

\[
\frac{HK\vdash p:a\thickapprox a'\quad x_{p}:a\thickapprox a'\in H\quad HK\vdash b:B}{H\left[x_{p}\coloneqq p\right]K\vdash b\left[x_{p}\coloneqq p\right]:B\left[x_{p}\coloneqq p\right]}
\]
\end{conjecture}

% ...
\begin{conjecture}
substitution of $kcasts$ preserve cast

equivalently the following rule is admissible

\[
\frac{HK\vdash kcast_{K}:a\thickapprox a'\quad x_{p}:a\thickapprox a'\in H\quad HK\vdash b:B}{H\left[x_{p}\coloneqq kcast_{K}\right]K\vdash b\left[x_{p}\coloneqq kcast\right]:B\left[x_{p}\coloneqq kcast\right]}
\]
\end{conjecture}

% ...
\begin{conjecture}
substitution of cast terms preserves path endpoints

equivalently the following rule is admissible

\[
\frac{HK\vdash a:A\quad x:A\in H\quad HK\vdash p:b\thickapprox b'}{H\left[x\coloneqq a\right]K\vdash p\left[x\coloneqq a\right]:b\left[x\coloneqq a\right]\thickapprox b'\left[x\coloneqq a\right]}
\]
\end{conjecture}

% ...
\begin{conjecture}
substitution of typed paths preserves path endpoints

equivalently the following rule is admissible

\[
\frac{HK\vdash p:a\thickapprox a'\quad x_{p}:a\thickapprox a'\in H\quad HK\vdash q:b\thickapprox b'}{H\left[x_{p}\coloneqq p\right]K\vdash q\left[x_{p}\coloneqq p\right]:b\left[x_{p}\coloneqq p\right]\thickapprox b'\left[x_{p}\coloneqq p\right]}
\]
\end{conjecture}

% ...
\begin{conjecture}
substitution of $kcasts$ preserve cast

equivalently the following rule is admissible

\[
\frac{HK\vdash kcast_{K}:a\thickapprox a'\quad x_{p}:a\thickapprox a'\in H\quad HK\vdash b:B}{H\left[x_{p}\coloneqq kcast_{K}\right]K\vdash q\left[x_{p}\coloneqq kcast_{K}\right]:b\left[x_{p}\coloneqq kcast\right]\thickapprox b'\left[x_{p}\coloneqq kcast\right]}
\]
\end{conjecture}

% ...

Finally we will conjecture the cast soundness.
\begin{conjecture}
The cast system preserves types and path endpoints over normalization
\end{conjecture}

% ...
\begin{conjecture}
a well typed path in an empty context is a value, takes a step, or
produces blame
\end{conjecture}

% ...
\begin{conjecture}
A well typed term in an empty context is a value, takes a step, or
produces blame
\end{conjecture}


\section{Elaborating Eliminations}

% elaboration unification
To make the overall system behave as expected we do not want to expose
users to equality patterns, or force them to manually do the path
bookkeeping. To work around this we extend a standard unification
algorithm to cast patterns with instrumentation to remember paths
that were required for the solution. Then if pattern matching is satisfiable,
compile additional casts into the branch based on its assignments.
Unlisted patterns can be checked to confirm they are unsatisfiable.
If the pattern is unsatisfiable then elaboration can use the proof
of unsatisfiability to construct explicit blame. If an unlisted pattern
cannot be proven ``unreachable'' then we could warn the user, and
like most functional programming languages, blame the incomplete match
if that pattern ever occurs.

\subsection{Preliminaries}

As mentioned in the introduction we will need to add and remove justified
casts from the endpoints of arguments. For instance, we will need
to be able to generate $3\thickapprox x::Nat$ from $3\thickapprox x$,
$x:X$, and $X\thickapprox Nat$. Fortunately the language is already
expressive enough to embed these operations using an $Assert$ that
does not bind a $same$ assertion. 

We will specify the shorthand $CastR_{a}p=Assert_{k\Rightarrow a::k=right,p}:a\thickapprox a::p$,
similarly we can define $CastL$.

We can use use a similar construction to remove casts from an endpoint.
Given a path $p:a::q\thickapprox b$ , we can define the macro $UnCastR_{a}p=Assert_{k\Rightarrow a::k=right,q}\circ p\ :\ a\thickapprox b$,
similarly for $UnCastL$.

The surface language needs to be enriched with additional location
metadata at each position where the two bidirectional typing modalities
would cause a check in the surface language.

\begin{tabular}{lcll}
$m...$ & $\Coloneqq$ & ... & \tabularnewline
 & | & $\mathsf{case}\,\overline{n_{\ell},}\,\left\{ \overline{|\,\overline{pat\Rightarrow}m_{\ell'}}\right\} $ & data elim. without motive\tabularnewline
 & | & $\mathsf{case}\,\overline{n_{\ell},}\,\left\langle \overline{x\Rightarrow}M_{\ell'}\right\rangle \left\{ \overline{|\,\overline{pat\Rightarrow}m_{\ell''}}\right\} $ & data elim. with motive\tabularnewline
\end{tabular}

The implementation allows additional annotations along the motive,
while this works within the bidirectional framework. The syntax is
not presented here since the theory is already quite complicated.\todo{move note somewhere else}

\subsection{Elaboration}

\todo[inline]{"can", "could", weasel words until implementation is finalized}

The biggest extension to the elaboration procedure in Chapter 3 is
the path relevant unification and the insertion of casts to simulate
surface language pattern matching. The unification and casting processes
both work without $k$ assumptions in scope, simplifying the possible
terms that may appear.

\begin{figure}
\[
\frac{\,}{U\left(\emptyset,\emptyset\right)}
\]

\[
\frac{U\left(E,u\right)\quad a\equiv a'}{U\left(\left\{ p:a\thickapprox a'\right\} \cup E,u\right)}
\]

\[
\frac{U\left(E\left[x\coloneqq a\right],u\left[x\coloneqq a\right]\right)}{U\left(\left\{ p:x\thickapprox a\right\} \cup E,u\cup\left\{ p:x\thickapprox a\right\} \right)}
\]

\todo[inline]{actually a little incorrect, needs to use conq to concat the paths}

\[
\frac{U\left(E\left[x\coloneqq a\right],u\left[x\coloneqq a\right]\right)}{U\left(\left\{ p:a\thickapprox x\right\} \cup E,u\cup\left\{ p^{-1}:x\thickapprox a\right\} \right)}
\]

\[
\frac{U\left(\left\{ p:a\thickapprox a'\right\} \cup E,u\right)\quad a\equiv d\overline{b}\quad a'\equiv d\overline{b'}}{U\left(\left\{ Con_{i}p:b_{i}\thickapprox b'_{i}\right\} _{i}\cup E,u\right)}
\]

\todo[inline]{fully applied}

\[
\frac{U\left(\left\{ p:a\thickapprox a'\right\} \cup E,u\right)\quad a\equiv D\overline{b}\quad a'\equiv D\overline{b'}}{U\left(\left\{ TCon_{i}p:b_{i}\thickapprox b'_{i}\right\} _{i}\cup E,u\right)}
\]

\todo[inline]{fully applied}

\[
\frac{U\left(\left\{ p:a::q\thickapprox a'\right\} \cup E,u\right)}{U\left(\left\{ uncastLp:a\thickapprox a\right\} _{i}\cup E,u\right)}
\]

\[
\frac{U\left(\left\{ p:a\thickapprox a'::q\right\} \cup E,u\right)}{U\left(\left\{ uncastRp:a\thickapprox a'\right\} \cup E,u\right)}
\]

\todo[inline]{fully applied}

\todo[inline]{break cycle, make sure x is assignable}

\todo[inline]{double check constraint order}

\todo[inline]{correct vars in 4a}

\caption{Surface Language Unification}
\label{fig:surface-data-unification}
\end{figure}

The elaboration procedure uses the extended unification procedure
to determine the implied type and assignment of each variable. In
the match body casts are made so that variables behave as if they
have the types and assignments consistent with the surface language.
The original casting mechanism is still active, so it is possible
that after all the casting types still don't line up. In this case
primitive casts are still made at their given location.

\todo[inline]{add explicit rules for elaboration?}

The elaboration algorithm is extremely careful to only add casts,
this means erasure is preserved and evaluation will be consistent
with the surface language.

Further the remaining properties from Chapter 3 probably still hold
\begin{conjecture}
Every term well typed in the bidirectional surface language elaborates 
\end{conjecture}

% ..
\begin{conjecture}
Blame never points to something that checked in the bidirectional
system 
\end{conjecture}


\section{Discussion and Future Work}

\subsection{Blame is not tight}

Though the meta theory in this section is plausible, there are some
awkward allowed behaviors. Blame may not be able to zero in as precisely
as it seems is possible, when an assumption interacts with itself.
For instance take the term under assumption $k$,

$\left\{ \lambda x\Rightarrow x\sim_{k,o,\ell}\lambda x\Rightarrow x\right\} \left\{ 1\sim_{k,o,\ell}2\right\} \rightsquigarrow_{*}\left(\lambda x\Rightarrow\left\{ x\sim_{k,o.app[x],\ell}x\right\} \right)\left\{ 1\sim_{k,o,\ell}2\right\} \rightsquigarrow_{*}\left\{ 2\sim_{k,o.app[\left\{ 1\sim_{k,o,\ell}2\right\} ],\ell}0\right\} $
which will mistakenly give blame to the function when it is more reasonable
to blame the argument. This situations is more complicated if we want
to avoid blame when the two sides are mutually consistent $\left\{ \lambda x\Rightarrow x\sim_{k,o,\ell}\lambda x\Rightarrow Not\ x\right\} \left\{ true\sim_{k,o,\ell}false\right\} \rightsquigarrow_{*}\left(\lambda x\Rightarrow\left\{ x\sim_{k,o.app[x],\ell}Not\ x\right\} \right)\left\{ true\sim_{k,o,\ell}false\right\} \rightsquigarrow_{*}\left\{ true\sim_{k,o.app[\left\{ true\sim_{k,o,\ell}false\right\} ],\ell}true\right\} $.

\subsection{Types invariance along paths}

It turns out that the system defined in Chapter 3 had the advantage
of only dealing with equalities in the type universe. Extending to
equalities over arbitrary type has vastly increased the complexity
of the system. To make the system work paths are untyped, which seems
inelegant. There is nothing currently preventing blame across type.
For instance,

$\left\{ 1\sim_{k,o,\ell}false\right\} $ will generate blame $1\neq false$.
While blame of $Nat\neq Bool$ will certainly result in a better error
message. Several attempts were made to encode the type into the type
assumption, but the resulting systems quickly became too complicated
to work with. Some vestigial typing constraints are still in the system
(such as on the explicit blame) to encourage this cleaner blame. \todo{remove or revise}

\subsection{Elaboration is non-deterministic with regard to blame}

Consider the case

\begin{lstlisting}[basicstyle={\ttfamily\small}]
case x <_:Id Nat 2 2 => S 2> {
| refl _ a => s a
}
\end{lstlisting}

that can elaborate to

\begin{lstlisting}[basicstyle={\ttfamily\small}]
case x <_:Id Nat 2 2 => S 2> {
| (refl A a)::p => (s (a::TCon0(p)) :: Cong uncastL(TCon1(p)))
}
\end{lstlisting}

where $p:Id\,A\,a\,a\thickapprox Id\,Nat\,2\,2$, where $TCon_{1}p$
selects the first position $p:Id\,A\,\underline{a}\,a\thickapprox Id\,Nat\,\underline{2}\,2$.
But this could also have elaborated to 

\begin{lstlisting}[basicstyle={\ttfamily\small}]
case x <_:Id Nat 2 2 => S 2> {
| (refl A a)::p => (s (a::TCon0(p)) :: Cong uncastL(TCon2(p)))
}
\end{lstlisting}

relying on $p:Id\,A\,a\,\underline{a}\thickapprox Id\,Nat\,2\,\underline{2}$.
This can make a difference if the scrutinee is 

$refl\ Nat\,2::Id\,Nat\,3\,2::Id\,Nat\,2\,2$

in one case blame will be triggered, in the other it will not. In
this case it is possible to mix the blame from both positions, though
this does not seem to extend in general since the consequences of
inequality are undecidable in general and we intend to allow running
programs if they can maintain their intended types.

\subsection{Extending to Call-by-Value}

As in Chapter 3, the system presented here does the minimal amount
of checking to maintain type safety. This can lead to unexpected results,
for instance consider the surface term 

\begin{lstlisting}[basicstyle={\ttfamily\small}]
case (refl Nat 7 :: Id Nat 2 2) <_ => Nat> {
| refl _ a => a
}
\end{lstlisting}

This will elaborate into 

\begin{lstlisting}[basicstyle={\ttfamily\small}]
case (refl Nat 7 :: Id Nat 2 2) <_ => Nat> {
| (refl A a)::p => a::TCon0(p)
}
\end{lstlisting}

which will evaluate to $7::\mathtt{Nat}$ without generating blame.
And indeed we only ever asserted that the result was of type $\mathtt{Nat}$.

In the implementation, some of this behavior is avoided by requiring
type arguments in a cast be run call-by-value. This restriction will
blame $7\neq2$ before the cast is even evaluated.\todo{expand}

\subsection{Efficiency}

The system defined here is brutally inefficient. 

In theory the system has an arbitrary slow down. As in Chapter 3,
a cast that relies on non-terminating code can itself cause additional
non-termination as paths are resolved.

As written there are many redundant computations, and trying every
combination of assumptions is very inefficient. Currently the implementation
is quite slow, though there are several ways to speed things up. Caching
redundant computation would help. Having a smarter embedding of $k$
assignments would remove redundant work directly. To some extent $Cong$
and $Assert$ can be made multi-arity to allow fewer jumps. But most
helpful of all would be simplifying away unneeded casts. More advanced
options include using proof search to find casts that will never cause
an error.

\todo[inline]{paremetricity}

\todo[inline]{relation to fun-ext}

\todo[inline]{warnings}

\subsection{Relation to UIP}

Pattern matching as outlined in the last Chapter (which follows from
\cite{coquand1992pattern}) implies the \textbf{uniqueness of identity
proofs} (UIP)\footnote{Also called \textbf{axiom k}}. UIP states
that every proof of identity is equal to refl (and thus unique), and
is not provable in many type theories. In univalent type theories
UIP is directly contradicted by the ``non-trivial'' equalities,
required to equate isomorphisms and Id. UIP is derivable in the surface
language by following pattern match 

\begin{lstlisting}[basicstyle={\ttfamily\small}]
case x <pr : Id A a a => Id (Id A a a) pr (refl A a) > {
| refl A a => refl (Id A a a) (refl A a)
}
\end{lstlisting}

This type checks since unification will assign $pr\coloneqq refl\,A\,a$
and under that assumption $refl\ (Id\ A\ a\ a)\ (refl\ A\ a):Id\ (Id\ A\ a\ a)\ (refl\ A\ a)\ (refl\ A\ a)$.
Like univalent type theories, the cast language has its own nontrivial
equalities, so it might seem that the cast language would also contradict
UIP . But it is perfectly compatible, and will elaborate. One interpretation
is that though there are multiple ``proofs'' of identity, we don't
care which one is used. \todo{interpretation + take aways?}

\subsection{Future work}

\todo[inline]{Make equalities visible in the surface syntax}

The system here has some simple inspiration that could be extended
into pattern matching syntax more generally. It seems useful to be
able to read equational information out of patterns, especially in
settings with rich treatment of equality. Matching equalities directly
could be a semi-useful feature in Agda, or in univalent type theories
such as CTT.

\section{Related work}

This work was previously presented as an extended abstract at the
TyDE workshop\todo[inline]{cite}, the version there reflected a less
plausible meta-theory based on earlier implementation experiments.

\todo[inline]{CTT data is related?}
