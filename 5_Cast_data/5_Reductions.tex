\section{Reductions}
 
\subsection{Function Fragment}
\todo{rule names might help here}
A selection of reduction operations is listed in \Fref{cast-data-step}.

Function reduction happens as usual.
% Reduction over a cast collects the argument and body evidence and uses an argument cast to insure that the argument will be of the correct type.
Type universes reduce away in the following three rules.
Function types commute through $\sim$ and $\cup$ when their type annotations are the type universe.
When the type position is resolved to function type, arguments can be applied under $::$ and into $\sim$ and $\cup$.
This allows data types to be treated like functions.
For instance, $(\lambda x \Rightarrow\mathtt{S} x) \sim\mathtt{S}$ will never cause a blameable error.
 
Finally, there are reductions to consolidate cast bookkeeping.
In addition to the listed reductions, a reduction can happen recursively in any sub-position.

% \todo{these reductions are listed?}
% In this thesis we have taken an extremely extensional perspective, terms are only different if an observation recognizes a difference.
% For instance this approach justifies equating the functions $\lambda x\Rightarrow x+1=\lambda x\Rightarrow1+x$ without proof, even though they are usually definitionally distinct.
% Therefore we will only blame inequality across functions if two functions that were asserted to be equal return different observations for ``the same'' input.
% Tracking that two functions should be equal becomes complicated, the system must be sensible under context, functions can take other higher order inputs, and function terms can be copied freely.


\begin{figure}
\[
\frac{\ }{\left(\mathsf{fun}\,f\,x\Rightarrow b\right)a\rightsquigarrow b\left[f\coloneqq\left(\mathsf{fun}\,f\,x\Rightarrow b\right),x\coloneqq a\right]}
\]

\[
\frac{\ }{A::\star\rightsquigarrow A}
\]

\[
\frac{\ }{\star\sim_{\ell,o}^{\star}\star\rightsquigarrow\star}
\]

\[
\frac{\ }{\star\cup^{\star}\star\rightsquigarrow\star}
\]

\[
\frac{\ }{\left(\left(\left(x:A\right)\rightarrow B\right)\sim_{\ell,o}^{\star}\left(\left(x:A'\right)\rightarrow B'\right)\right)\rightsquigarrow\left(x:\left(A\sim_{\ell,o.Arg}^{\star}A'\right)\right)\rightarrow\left(B\sim_{\ell,o.Bod_{x}}^{\star}B'\right)}
\]

\[
\frac{\ }{\left(\left(\left(x:A\right)\rightarrow B\right)\cup^{\star}\left(\left(x:A'\right)\rightarrow B'\right)\right)\rightsquigarrow\left(x:\left(A\cup^{\star}A'\right)\right)\rightarrow\left(B\cup^{\star}B'\right)}
\]

\[
\frac{\ }{\left(b::\left(\left(x:A\right)\rightarrow B\right)\right)a\rightsquigarrow\left(b\left(a::A\right)\right)::\ B\left[x\coloneqq a::A\right]}
\]

\[
\frac{\ }{\left(c\sim_{\ell,o}^{\left(x:A\right)\rightarrow B}b\right)a\rightsquigarrow\left(c\left(a::A\right)\sim_{\ell,o.App_{a}}^{B\left[x\coloneqq a::A\right]}b\left(a::A\right)\right)}
\]

\[
\frac{\ }{\left(c\cup^{\left(x:A\right)\rightarrow B}b\right)a\rightsquigarrow\left(c\left(a::A\right)\right)\cup^{B\left[x\coloneqq a::A\right]}\left(b\left(a::A\right)\right)}
\]

\[
\frac{\ }{\left(a::L'\right)\sim_{\ell,o}^{L}b\rightsquigarrow a\sim_{\ell,o}^{L'\cup^{\star}L}b}
\]

\[
\frac{\ }{a\sim_{\ell,o}^{L}\left(b::L'\right)\rightsquigarrow a\sim_{\ell,o}^{L\cup^{\star}L'}b}
\]

\[
\frac{\ }{\left(a::L'\right)\cup^{L}b\rightsquigarrow a\cup^{L'\cup^{\star}L}b}
\]

\[
\frac{\ }{a\cup^{L}\left(b::L'\right)\rightsquigarrow a\cup^{L\cup^{\star}L'}b}
\]

\[
\frac{\ }{\left(\left(a::L\right)::L'\right)\rightsquigarrow a::\left(L\cup^{\star}L'\right)}
\]
\caption{Cast Language Small Step (select rules)}
\label{fig:cast-data-step}
\end{figure}

\subsection{Data Reductions}

Some reductions for data are listed in \Fref{Data-Reductions}.
Elimination of data types is delegated to the $\mathbf{Match}$ judgment (that is unlisted). \todo{can reconstitute its type if their is no cast}
$TCon_{i}$, and $DCon_{i}$ observations reduce to the expected index.
Data types and data terms consolidate over $\sim$ and $\cup$ when a head constructor is shared and the type annotation has resolved, this needs to be computed by applying the indices point-wise against the typing telescope (see \Fref{Pointwise-Applications}).
This is why the telescope annotation is explicit in the formalism (so reductions can happen independent of a context). 

The important properties of reduction are 
\begin{itemize}
% \item Paths reduce into a stack of zero or more $Assert_{k\Rightarrow A}$s
\item Assertions emit observably consistent constructors, and record the needed observations
\item Sameness assertions will get stuck on inconsistent constructors
% \item Casts can commute out of sameness assertions with proper index tracking
% \item function application can commute around $kcasts$ , similar to Chapter 3, but will keep $k$ assumptions properly indexed
\end{itemize}
% Matching is defined in \ref{fig:cast-data-match}.
% Note that uncast terms are equivalent to refl cast terms.

\begin{figure}
\[
\frac{
  \overline{a}\ \left(\overline{|\,\overline{patc\Rightarrow}b}\right)\ \mathbf{Match}\ b'
}{
  \mathsf{case}\,\overline{a,}\,\left\{ \overline{|\,\overline{patc\Rightarrow}b}\, \overline{?\,\overline{patc}} \right\} \rightsquigarrow b'
}
\]

\[
\frac{\ }{TCon_{i}\ \left(D_{\Delta}\overline{a}\right)\rightsquigarrow a_{i}}
\]

\[
\frac{\ }{DCon_{i}\ \left(d_{\Delta\rightarrow D\overline{e}}\overline{a}\right)\rightsquigarrow a_{i}}
\]

\[
\frac{
  \mathbf{length}\,\Delta=\mathbf{length}\,\overline{a}=\mathbf{length}\,\overline{b}
}{
  \left(\left(D_{\Delta}\overline{a}\right)\sim_{\ell,o}^{\star}\left(D_{\Delta}\overline{b}\right)\right)\rightsquigarrow D_{\Delta}\left(\Delta@^{T\sim\ell,o}\left(\overline{a},\overline{b}\right)\right)}
\]

\[
\frac{
  \mathbf{length}\,\Delta=\mathbf{length}\,\overline{a}=\mathbf{length}\,\overline{b}
}{
  \left(\left(d_{\Delta\rightarrow D\overline{e}}\overline{a}\right)\sim_{\ell,o}^{C}\left(d_{\Delta\rightarrow D\overline{e}}\overline{b}\right)\right)\rightsquigarrow d_{\Delta\rightarrow D\overline{e}}\left(\Delta@^{\mathbf{D}\sim\ell,o}\left(\overline{a},\overline{b}\right)\right)::C}
\]

\[
\frac{
  \mathbf{length}\,\Delta=\mathbf{length}\,\overline{a}=\mathbf{length}\,\overline{b}
  }{\left(\left(D_{\Delta}\overline{a}\right)\cup^{\star}\left(D_{\Delta}\overline{b}\right)\right)\rightsquigarrow D_{\Delta}\left(\Delta@^{T\cup}\left(\overline{a},\overline{b}\right)\right)}
\]

\[
\frac{
  \mathbf{length}\,\Delta=\mathbf{length}\,\overline{a}=\mathbf{length}\,\overline{b}
}{
  \left(\left(d_{\Delta\rightarrow D\overline{e}}\overline{a}\right)\sim_{\ell,o}^{C}\left(d_{\Delta\rightarrow D\overline{e}}\overline{b}\right)\right)\rightsquigarrow d_{\Delta\rightarrow D\overline{e}}\left(\Delta@^{\mathbf{D}\cup}\left(\overline{a},\overline{b}\right)\right)::C}
\]
\caption{Data Reductions}
\label{fig:Data-Reductions}
\end{figure}


\begin{figure}
\begin{tabular}{llll}
$\Delta@_{0}^{k\sim\ell,o}\left(\overline{a},\overline{b}\right)$ & = & $\Delta@^{k\sim\ell,o}\left(\overline{a},\overline{b}\right)$ \tabularnewline % & shorthand that begins the recursion\tabularnewline
$\left(\left(x:A\right)\rightarrow\Delta\right)@_{i}^{k\sim\ell,o}\left(\left(a,\overline{a}\right),\left(b,\overline{b}\right)\right)$ & = &
\makecell[l]{$\left(a\sim_{\ell,o.TCon_{i}}^{A}b\right),$\\
  $\left(\Delta\left[x\coloneqq\left(a\sim_{\ell,o.kCon_{i}}^{A}b\right)\right]\right)@_{i+1}^{\sim\ell,o}\left(\overline{a},\overline{b}\right)$}
   \tabularnewline % & non-empty list\tabularnewline
$\left(.\right)@_{i}^{k\sim\ell,o}\left(.,.\right)$ & = & $.$  \tabularnewline % & empty list\tabularnewline
$\Delta@_{0}^{k\cup}\left(\overline{a},\overline{b}\right)$ & = & $\Delta@^{k\cup}\left(\overline{a},\overline{b}\right)$  \tabularnewline % & shorthand that begins the recursion\tabularnewline
$\left(\left(x:A\right)\rightarrow\Delta\right)@_{i}^{k\cup}\left(\left(a,\overline{a}\right),\left(b,\overline{b}\right)\right)$ & = & $\left(a\cup^{A}b\right),\left(\Delta\left[x\coloneqq\left(a\cup^{A}b\right)\right]\right)@_{i+1}^{k\cup}\left(\overline{a},\overline{b}\right)$ \tabularnewline %  & non-empty list\tabularnewline
$\left(.\right)@_{i}^{k\cup}\left(.,.\right)$ & = & $.$ \tabularnewline %  & empty list\tabularnewline
\end{tabular}
\caption{Pointwise Indexing}
\label{fig:Pointwise-Applications}
\end{figure}

\subsection{Endpoint Preservation}

The system preserves typed endpoints over reductions.
 
\[
\frac{a\rightsquigarrow b\quad\varGamma\vdash a\sqsupseteq a':A'}{\varGamma\vdash b\sqsupseteq a':A'}
\]
 
For the fragment without data, this can be shown with some modifications to the argument in \ch{3}.
We conjecture that the proof could be extended to support data.

