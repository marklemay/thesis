\section{Endpoint Rules}

Care must be taken so that typing is still sensible when an expression could have multiple interpretations.
To do this we construct a new \csys{} to include term level endpoint information.

\subsection{Function Fragment}
The rules for non-data terms are listed in \Fref{cast-endpoint-rules}.

\begin{figure}

\begin{align*}
 \frac{x:A\in\varGamma}{\varGamma\vdash x\sqsupseteq x:A}\rulename{\sqsupseteq-var}
  \quad & \quad 
  \frac{\ }{\varGamma\vdash\star\sqsupseteq\star:\star}\rulename{\sqsupseteq-\star}
  \\
\end{align*}

\[
\frac{
\varGamma\vdash A\sqsupseteq A':\star \quad
\varGamma,A'\vdash B\sqsupseteq B':\star
}{
  \varGamma\vdash\left(x:A\right)\rightarrow B\sqsupseteq\left(x:A'\right)\rightarrow B':\star
}\rulename{\sqsupseteq-\mathsf{fun}-ty}
\]

\[
\frac{
\varGamma\vdash A:\star\quad\varGamma,A\vdash B:\star \quad
\varGamma,A\vdash b\sqsupseteq b':B
}{
  \varGamma\ \vdash\ \mathsf{fun}\,f\,x\Rightarrow b\ \sqsupseteq\ \mathsf{fun}\,f\,x\Rightarrow b\ :\ \left(x:A\right)\rightarrow B
}\rulename{\sqsupseteq-\mathsf{fun}}
\]
\[
\frac{
\varGamma\vdash b\sqsupseteq b':\left(x:A'\right)\rightarrow B' \quad
\varGamma\vdash a\sqsupseteq a':A'
}{
  \varGamma\vdash b\,a\sqsupseteq b'\,a':B'\left[x\coloneqq a'\right]
}\rulename{\sqsupseteq-\mathsf{fun}-app}
\]

\[
\frac{\begin{array}{c}
  \varGamma\vdash L\sqsupseteq A:\star\\
  \varGamma\vdash L\sqsupseteq B:\star\\
  \varGamma\vdash a\sqsupseteq a':A
  \end{array}}{\varGamma\ \vdash\ a::L\ \sqsupseteq\ a'::L\ :\ B}\rulename{\sqsupseteq-::}
\]

\begin{align*}
  \frac{\begin{array}{c}
  \varGamma\vdash a\sqsupseteq a':A'\\
  \varGamma\vdash L\sqsupseteq A':\star\\
  \varGamma\vdash L\sqsupseteq C:\star
  \end{array}}{\varGamma\vdash a\sim_{\ell,o}^{L}b\sqsupseteq a'::L\ :\ C}\rulename{\sqsupseteq-\sim L}
  \quad & \quad 
  \frac{\begin{array}{c}
  \varGamma\vdash b\sqsupseteq b':B'\\
  \varGamma\vdash L\sqsupseteq B':\star\\
  \varGamma\vdash L\sqsupseteq C:\star
  \end{array}}{\varGamma\vdash a\sim_{\ell,o}^{L}b\sqsupseteq b'::L\ :\ C}\rulename{\sqsupseteq-\sim R}
\\
\end{align*}

\begin{align*}
  \frac{\begin{array}{c}
  \varGamma\vdash a\sqsupseteq a'::L:C\\
  \varGamma\vdash a\sqsupseteq c'::L:C\\
  \varGamma\vdash b\sqsupseteq c'::L:C
  \end{array}}{\varGamma\vdash a\cup^{L}b\sqsupseteq a'::L\ :\ C}\rulename{\sqsupseteq-\cup L}
  \quad & \quad 
  \frac{\begin{array}{c}
  \varGamma\vdash b\sqsupseteq b'::L:C\\
  \varGamma\vdash a\sqsupseteq c'::L:C\\
  \varGamma\vdash b\sqsupseteq c'::L:C
  \end{array}}{\varGamma\vdash a\cup^{L}b\sqsupseteq b'::L\ :\ C}\rulename{\sqsupseteq-\cup R}
\\
\end{align*}

\begin{align*}
  \frac{\begin{array}{c}
  \varGamma\vdash a\sqsupseteq a':A'\\
  A'\equiv B
  \end{array}}{\varGamma\vdash a\sqsupseteq a':B}\rulename{\sqsupseteq-conv-ty}
  \quad & \quad 
  \frac{\begin{array}{c}
  \varGamma\vdash a\sqsupseteq a':A'\\
  a'\equiv b\\
  \varGamma\vdash b:A'
  \end{array}}{\varGamma\vdash a\sqsupseteq b:A'}\rulename{\sqsupseteq-conv-trm}
\\
\end{align*}

\[
\frac{\varGamma\vdash a\sqsupseteq a:A}{\varGamma\vdash a:A}\rulename{ty-def}
\]
\caption{Endpoints (non-data)}
\label{fig:cast-endpoint-rules}
\end{figure}
 
The first rules extend the usual type assignment judgments to endpoints.
The \rulename{\sqsupseteq-\mathsf{fun}-app} rule makes sure different endpoints are used in a type consistent way.
For example, $\left\{ \lambda x.x+1\sim not\right\} \left\{ 1\sim \False{} \right\} \sqsupseteq \True{} ,2$ and $\left\{ 1\sim2\right\} +\left\{ 10\sim3\right\} \sqsupseteq4,5,11,12$.
The $\rulename{\sqsupseteq-::}$ rule allows using a term at a different type if the cast has compatible endpoints.
The next rules allow endpoints to be extracted out of the left or right of $\sim$, and $\cup$ if they maintain enough information to track why those endpoints have the same type.

In addition to the usual type conversion rule we also have a term conversion rule.
Term conversion is important so that $\cup$ can associate terms under reduction.
For instance, $\left(x\sim2\right)\cup\left(1+1\sim y\right)\sqsupseteq x, y$.

We can recover a typing judgment when a term is its own endpoint (formalized in the \rulename{ty-def} rule).
 
As usual, the system needs a suitable definition of $\equiv$.
We use a transitive reflexive equivalence relation that respects reductions and ignores casts.
Ignoring casts keeps terms from being made distinct by different casts, and prevents reductions from getting stuck for the purpose of equality.
Note that unlike \ch{3} this means equivelence will not preserve Blame and we will need to rely on a more specific reduction relation that preseves both equivalence and blame.
\todo{constructing such an equivelence rigorously...}


\subsection{Data Endpoints}

The Rules for data endpoints are listed in \Fref{cast-Data-Endpoint-Rules}.

\begin{figure}
\[
\frac{
  p\ :\ a:A\approx b:B\ \in\ \varGamma
}{
  \varGamma\vdash x\sqsupseteq a:A
  }
\]
\[
\frac{
  p\ :\ a:A\approx b:B\ \in\ \varGamma
}{\varGamma\vdash x\sqsupseteq b:B}
\]
\[
\frac{\mathsf{data}\ D\ :\ \Delta\rightarrow\star\ \in\ \varGamma}{\varGamma\vdash D_{\Delta}\sqsupseteq D_{\Delta}:\Delta\rightarrow\star}
\]
\[
\frac{d\ :\ \Delta\rightarrow D\overline{a}\ \in\ \varGamma}{\varGamma\vdash d_{\Delta\rightarrow D\overline{a}}\sqsupseteq d_{\Delta\rightarrow D\overline{a}}:\Delta\rightarrow D\overline{a}}
\]
\[
\frac{\begin{array}{c}
\varGamma\vdash\overline{a}\sqsupseteq\overline{a'}:\Delta\\
\varGamma\vdash\Delta\ \mathbf{ok}\\
\varGamma,\Delta\vdash B':\star\\
\forall\overline{p}\in\overline{|\,\overline{patc}},\:\varGamma\vdash\overline{p}:\Delta\\
\forall\overline{p}\in\overline{|\,\overline{patc}},\:\varGamma,\left(\overline{p}:\Delta\right)\vdash b\sqsupseteq b':B'\\
\forall\overline{p}\in\overline{?\,\overline{patc'}},\:\varGamma\vdash\overline{p}:\Delta\\
\varGamma\vdash\overline{|\,\overline{patc}}\overline{?\,\overline{patc'}}:\Delta\ \mathbf{Complete}
\end{array}}{
  \varGamma\vdash\mathsf{case}\,\overline{a,}\,\left\{ \overline{|\,\overline{patc\Rightarrow}b} \overline{?\,\overline{patc'}} \right\} \sqsupseteq\mathsf{case}\,\overline{a',}\,\left\{ \overline{|\,\overline{patc\Rightarrow}b'} \overline{?\,\overline{patc'}} \right\} :B'\left[\Delta\coloneqq\overline{a'}\right]}
\]
\[
\frac{\begin{array}{c}
\varGamma\vdash L'\sqsupseteq A:\star\\
\varGamma\vdash L'\sqsupseteq B:\star\\
\varGamma\vdash L\sqsupseteq a:A\\
\varGamma\vdash L\sqsupseteq b:B\\
\mathbf{head}\,a\neq\mathbf{head}\,b\\
\varGamma\vdash B:\star
\end{array}}{\varGamma\vdash!_{L}^{L'}\sqsupseteq!_{L}^{L'}\ :\ B}
\]
\[
\frac{\begin{array}{c}
\varGamma\vdash b\sqsupseteq D_{\Delta}\overline{a}:\star\end{array}\quad\mathbf{length}\,\Delta=\mathbf{length}\,\overline{a}}{\varGamma\vdash TCon_{i}\ b\sqsupseteq\Delta@_{i}\overline{a}}
\]
\[
\frac{\begin{array}{c}
\varGamma\vdash b\sqsupseteq d_{\Delta\rightarrow D\overline{b}}\overline{a}\end{array}:D\overline{c}\quad\mathbf{length}\,\Delta=\mathbf{length}\,\overline{a}}{\varGamma\vdash DCon_{i}\ b\sqsupseteq\Delta@_{i}\overline{a}}
\]

\caption{Cast Data Endpoint Rules}
\label{fig:cast-Data-Endpoint-Rules}
\end{figure}

Assertion variables have endpoints at each side of the equality given from the typing judgment.
Data type constructors and data term constructors have the function type implied by their definitions (and annotations that match).

To find the endpoints of a $\mathsf{case}$ expression first, find the endpoints of the scrutinees and a pattern that it types under (the endpoint and typing judgment can be extended to lists and telescopes).
A motive $B'$ must exist under the context extended with the telescope.
Every pattern must type check against the telescope.
The body of every pattern must have an endpoint consistent with the motive and pattern.
Finally the constructors of the patterns must form a complete covering.
  
Direct blame allows evidence with contradictory endpoints to inhabit any type.

Data types and data terms can be inspected, however the prior indices are needed to compute the type using the index meta-function $@_{i}$ (see \Fref{Pointwise-Applications}).

\begin{figure}
\begin{tabular}{lll}
$\left(\left(x:A\right)\rightarrow\Delta\right)@_{0}\left(a,\overline{b}\right)$ & = & $a:A$\tabularnewline
$\left(\left(x:A\right)\rightarrow\Delta\right)@_{i}\left(a,\overline{b}\right)$ & = & $\left(\Delta\left[x\coloneqq a\right]\right)@_{\left(i-1\right)}\overline{b}$\tabularnewline
\end{tabular}

\begin{tabular}{llll}
$\Delta@_{0}^{k\sim\ell,o}\left(\overline{a},\overline{b}\right)$ & = & $\Delta@^{k\sim\ell,o}\left(\overline{a},\overline{b}\right)$ \tabularnewline % & shorthand that begins the recursion\tabularnewline
$\left(\left(x:A\right)\rightarrow\Delta\right)@_{i}^{k\sim\ell,o}\left(\left(a,\overline{a}\right),\left(b,\overline{b}\right)\right)$ & = & $\left(a\sim_{\ell,o.TCon_{i}}^{A}b\right),\left(\Delta\left[x\coloneqq\left(a\sim_{\ell,o.kCon_{i}}^{A}b\right)\right]\right)@_{i+1}^{\sim\ell,o}\left(\overline{a},\overline{b}\right)$  \tabularnewline % & non-empty list\tabularnewline
$\left(.\right)@_{i}^{k\sim\ell,o}\left(.,.\right)$ & = & $.$  \tabularnewline % & empty list\tabularnewline
$\Delta@_{0}^{k\cup}\left(\overline{a},\overline{b}\right)$ & = & $\Delta@^{k\cup}\left(\overline{a},\overline{b}\right)$  \tabularnewline % & shorthand that begins the recursion\tabularnewline
$\left(\left(x:A\right)\rightarrow\Delta\right)@_{i}^{k\cup}\left(\left(a,\overline{a}\right),\left(b,\overline{b}\right)\right)$ & = & $\left(a\cup^{A}b\right),\left(\Delta\left[x\coloneqq\left(a\cup^{A}b\right)\right]\right)@_{i+1}^{k\cup}\left(\overline{a},\overline{b}\right)$ \tabularnewline %  & non-empty list\tabularnewline
$\left(.\right)@_{i}^{k\cup}\left(.,.\right)$ & = & $.$ \tabularnewline %  & empty list\tabularnewline
\end{tabular}
\caption{Pointwise Applications}
\label{fig:Pointwise-Applications}
\end{figure}