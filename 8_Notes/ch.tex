\chapter{Notes and Future Work}
\label{chapter:Notes}
\thispagestyle{myheadings}
 
The content of this thesis was achieved through much trial and error.
There were several experiments that while interesting and promising, did not cleanly fit into the narrative of the first 5 Chapters.
This Chapter contains an idealized\footnote{
  For instance, automated testing procedure was originally specified and implemented on a different language than the \clang{} described in this thesis.
  Accordingly the notations have changed, to be sensible in the context of the rest of the thesis. % , often improving/simplifing the presentation.
  Further, automated testing procedure has not yet been reimplemented on the current version of the prototype.
} review of what was tried, and hopefully provide hints about how one might productively try again.
 
The goal was always to make a dependently typed language as approachable as python\footnote{
  We have perhaps succeeded in making a dependently typed language as approachable as Haskell.
}.
While I believe a \fullSp{} dependently typed language with runtime definitional equality checking, dependent pattern matching and with no restrictions on recursion are part of that practical language, there are still lingering usability issues.
For those who share the dream of more reliable software through easier to use dependently typed languages, here are some lines of work for your consideration.

\section{Symbolic Execution}

One of the advantages of type checking is the immediacy of feedback.
We have outlined here a system that will give warning messages immediately, but requires evaluation to give the detailed error messages that are most helpful when correcting a program.
This is especially important if the user wants to use the system as a proof language, and will not generally execute their proofs.
A symbolic evaluation system recaptures some of that quicker feed back, by having a system that passively tries to find errors.
This ideal workflow appears in \Fref{notes-workflow}.

Since this procedure operates over the cast language, we must decide what constitutes a reasonable testing environment

\begin{figure}
\begin{lstlisting}
            edit program
             |      
             |      
             v
 testing  <- Elaboration                    ^
    | ^      |              |               | runtime error
     -       | no warnings  | type warnings |
             v              v               |
                run program   ---------------
\end{lstlisting}

\todo[inline]{better graphics}

\caption{Ideal Workflow}
\label{fig:notes-workflow}
\end{figure}

A one hole context\todo{cite? other people don't} can be defined for the cast languages presented in this thesis $C[-]$ and we can say that an error is observable if $\vdash a:A$ and $\vdash C[a]:B$ and $C[a]\rightsquigarrow_{*}b$ and $\textbf{Blame}\:\ell \,o\,b$ for some $\ell \in lables\left(a\right)$ but $\ell \notin lables\left(C[-]\right)$.

This process is semi-decidable in general (by enumerating all well typed syntax).
But testing every term is infeasibly inefficient for functions and types.
An approximate approach can build partial contexts based on fixing observations.
These contexts are correct up to some constraint.\todo{??}
For instance, 
\begin{itemize}
\item
  If we have a term that claims to inhabit the ``empty'' type, $\lambda x\Rightarrow\star::_{\ell }x\quad:\left(x:\star\right)\rightarrow x$, 
    then we can observe an error by applying $x$ of type $x:\star$ to the term and observing $x=-\rightarrow-$.
  This observation  would correspond to the context $[\lambda x\Rightarrow\star::_{\ell }x]\left(\star\rightarrow\star\right)\rightsquigarrow_{*}\star::_{\star,\ell }\left(\star\rightarrow\star\right)$ and $\textbf{Blame}\:\ell \,.\,\star::_{\star,\ell }\left(\star\rightarrow\star\right)$
\item
  This reasoning can be extended to higher order functions $\lambda f\Rightarrow\star::_{\ell }f\star::\star\quad:\left(\star\rightarrow\star\right)\rightarrow\star$, 
    then we can observe an error by applying $f$ of type $f:\star\rightarrow\star$ to the term and $f\star=-\rightarrow-$.
  Which would correspond to the context $[\lambda f\Rightarrow\star::_{\ell }f\star::\star]\left(\lambda x\Rightarrow\left(\star\rightarrow\star\right)\right)\rightsquigarrow_{*}\star::_{\star,\ell }\left(\star\rightarrow\star\right)$ and $\textbf{Blame}\:\ell \,.\,\star::_{\star,\ell }\left(\star\rightarrow\star\right)$
\item Observing a higher order input directly, $\lambda\,f\Rightarrow f\left(!!\right)\quad:\left(\star\rightarrow\star\right)\rightarrow\star$,
and can observe an error by inspecting its input, which would correspond
to the context $[\lambda\,f\Rightarrow f\left(!!\right)]\left(\lambda x\Rightarrow x\right)\rightsquigarrow_{*}!!$
which is blamable. Where $!!$ will stand inform any blamable term
(here $!!=\star::\left(\star\rightarrow\star\right)::\star$) 
\item Similarly with dependent types, $!!\rightarrow\star\quad:\star$,
can observe an error by inspecting its input, which would correspond
to the context $\left(\left(\lambda x\Rightarrow x::\star\right)::[!!\rightarrow\star]::\left(\star\rightarrow\star\right)\right)\star\rightsquigarrow_{*}\star::[!!]::\star$
which is blamable.
\item Similarly we will allow extraction from the dependent body of a function
type $\left(b:\mathbb{B}_{c}\right)\rightarrow b\,\star\,!!\,\star\quad:\star$,
by symbolically applying $b$ where $b:\mathbb{B}_{c},b\,\star=y$
leaving $y\,!!\,\star$ which can observe blame via $y$'s first argument. 
\item data can be observed incrementally, and observed paths along data
will confirm that the data conductors are consistent.
\end{itemize}
The procedure insists that the following constraints hold, 
\begin{itemize}
\item variables and assignments are type correct
\item observable different outputs must come from observably different inputs
(in the case of dependent function types, the argument should be considered
as an input)
\item observations do not over-specify behavior. For instance, $f:\mathtt{Nat}\rightarrow\mathtt{Nat},f2=x,f3=x$
would not be allowed since $f2=f3$ and is over specified.
\end{itemize}
This seems to be good because the procedure,
\begin{itemize}
\item can guide toward the labels of interest, for instance we can move
to labels that have not yet observed a concrete error. Terms without
labels can be skipped entirely.
\item can choose assignments strategically avoiding or activating blame
as desired
\item Since examples are built up partially the partial contexts can avoid
introducing their own blame by construction
\item handle higher order functions, recursions, and self reference gracefully.
For instance, $f:Nat\rightarrow Nat$, $f\left(f\,0\right)=1$ and
$f\left(3\right)=3$ if there is an assignment that implies $f\,0\neq3$
\end{itemize}
However the procedure is unsound, it will flag errors that are not
possible to realize in a context,
\begin{itemize}
\item Since there is no way for a terms within the cast language to ``observe''
a distinction between the type formers plausible environments cannot
always be realized back to a term that would witness the bad behavior.
For instance, the environment $F:\star\rightarrow\star,F\,\star=\star\rightarrow\star,F\,\left(\star\rightarrow\star\right)=\star$
, which might be needed to explore the term with casts like $\lambda F\Rightarrow...::\star::F\left(\star\rightarrow\star\right)......::\left(\star\rightarrow\star\right)::F\,\star\ :\left(\star\rightarrow\star\right)\rightarrow...$,
cannot be realized as a closed term. In this way the environment is
stronger then the cast language. The environment reflects a term language
that has a type case construct\todo{cite}. 
\end{itemize}
\todo[inline]{add non termination as a source of unrealizability}
\begin{itemize}
\item Additionally, working symbolically can move past evaluations that
would block blame in a context. For instance that procedure outlined
above would allow $!!$ to be reached in $\left(\lambda xf\Rightarrow f(!!)\right)loop\quad:\star\rightarrow\star$
even though there is no context that would allow this to cause blame.
\end{itemize}
\todo[inline]{ins't as crisp an example as I would like}
\begin{itemize}
\item Subtly a version of parallel-or can be generated via assignment even
though such a term is unconstructable in the language, $por:\mathbb{B}\rightarrow\mathbb{B}\rightarrow\mathbb{B},por\ loop\ tt=tt,por\ tt\ loop=tt,por\ ff\ ff=ff$.
Here all assignments are well typed, and each output can be differentiated
by a different input. \todo{G. Plotkin, LCF considered as a programming language, Theoretical
Computer Science 5 (1977) 223\{255. as cited in https://alleystoughton.us/research/por.pdf} 
\end{itemize}

\subsection{Related and Future Work}

While formalizing a complete and efficient testing procedure along
these lines is still future work. There are likely insights to be
gained from the large body of research on symbolic execution, especially
work that deals with typed higher order functions. A fully formal
account would deal with a formal semantics of the cast language. 

\subsubsection{Testing}

Many of the testing strategies for typed functional programming trace
their heritage to \textbf{property based} testing in QuickCheck \cite{quickcheck}.
Property based testing involves writing functions that encode the
properties of interest, and then randomly testing those functions.
\begin{itemize}
\item QuickChick \footnote{https://github.com/QuickChick/QuickChick} \cite{denes2014quickchick}\cite{lampropoulos2017generating,lampropoulos2017beginner,lampropoulos2018random}
uses type-level predicates to construct generators with soundness
and completeness properties, but without support for higher order
functions. However, testing requires building types classes that establish
the properties needed by the testing framework such as decidable equality.
This is presumably out of reach of novice Coq users.
\begin{itemize}
\item Current work in this area uses coverage guided techniques in \cite{lampropoulos2019coverage}
like those in symbolic execution. More recently Benjamin Pierce has
used Afl on compiled Coq code as a way to generate counter examples\footnote{https://www.youtube.com/watch?v=dfZ94N0hS4I}\todo{review this, maybe there's a paper or draft now?}.
\end{itemize}
\item \cite{dybjer2003combining} added QuickCheck style testing to Agda
1.
\end{itemize}

\subsubsection{Symbolic Execution}

Symbolic evaluation is a technique to efficiently extract errors from
programs. Usually this happens in the context of an imperative language
with the assistance of an SMT solver. Symbolic evaluation can be supplemented
with other techniques and a rich literature exists on the topic. 

The situation described in this chapter is unusual from the perspective
of symbolic execution:
\begin{itemize}
\item the number of blamable source positions is limited by the location
tags. Thus the search is error guided, rather then coverage guided.
\item The language is dependently typed. Often the language under test is
untyped.
\item The language needs higher order execution. often the research in this
area focuses on base types that are efficiently handleable with an
SMT solver such as integer aritmetic.
\end{itemize}
This limits the prior work to relatively few papers
\begin{itemize}
\item A Symbolic execution engine for Haskell is presented in \cite{10.1145/3314221.3314618},
but at the time of publication it did not support higher order functions.
\item A system for handling higher order functions is presented in \cite{nguyen2017higher},
however the system is designed for Racket and is untyped. Additionally
it seems that there might be a state space explosion in the presence
of higher order functions.
\item \cite{10.1007/978-3-030-72019-3_23} extended and corrected some issues
with \cite{nguyen2017higher}, but still works in a untyped environment.
The authors note that there is still a lot of room to improve performance.
\item Closest to the goal here, \cite{lin_et_al:LIPIcs:2020:12349} uses
game semantics to build a symbolic execution engine for a subset of
ML with some nice theoretical properties.
\item An version of the above procedure was experimented with in the extended
abstract \todo{cite}, however conjectures made in that preliminary
work were false (the procedure was unsound). The underlying languages
has improved substantially since that work.
\end{itemize}
The subtle unsoundness hints that the approach presented here could
be revised in terms of games semantics (perhaps along lines like \cite{lin_et_al:LIPIcs:2020:12349}).
Though game semantics for dependent types is a complicated subject
in and of itself \todo{cite}. Additionally it seems helpful to allow
programmers to program their own solvers that could greatly increase
the search speed.
 
\section{Runtime Proof Search}

Just as ``obvious'' equalities are missing from the definitional relation, ``obvious'' proofs and programs are not always conveniently available to the programmer.
For instance, in Agda it is possible to write a sorting function quickly using simple types.
With effort is it possible to prove that sorting procedure correct by rewriting it with the necessarily dependently typed invariants.
However, very little is offered in between.
The problem is magnified if module boundaries hide the implementation details of a function, since those details are often exactly what is needed to make a proof!
This is especially important for larger scale software where a library may require proof terms that while true are not provable from the exports of other libraries.

The solution proposed here is additional syntax that will search for a term of the type when resolved at runtime.
Given the sorting function 

\[
\mathtt{sort}:\mathtt{List}\,\Nat{}\rightarrow\mathtt{List}\,\Nat{}
\]

and given the first order predicate that 

\[
\mathtt{IsSorted}:\mathtt{List}\,\Nat{}\rightarrow*
\]

then it is possible to assert that $\mathtt{sort}$ behaves as expected with

\[
\lambda x.?:\left(x:\mathtt{List}\,\Nat{}\right)\rightarrow\mathtt{IsSorted}\left(\mathtt{sort}x\right)
\]

This term will act like any other function at runtime, given a $\mathtt{List}$ input the function will verify that the $\mathtt{sort}$ correctly handled that input, or the term will give an error, or non-terminate.

Additionally, this would allow prototyping form first order specification.
For instance,

\begin{align*}
data\ \mathtt{Mult} & :\Nat{}\rightarrow\Nat{}\rightarrow\Nat{}\rightarrow*\ where\\
\mathtt{base} & :\left(x:\Nat{}\right)\rightarrow\mathtt{Mult}\ 0\ x\ 0\\
\mathtt{suc} & :\left(x\,y\,z:\Nat{}\right)\rightarrow\mathtt{Mult}\,x\,y\,z\rightarrow\mathtt{Mult}\,\left(1+x\right)\,y\,(y+z)
\end{align*}

can be used to prototype

\[
\mathtt{div}=\lambda z.\lambda x.\mathtt{fst}\left(?:\sum y:\Nat{}.\mathtt{Mult}x\,y\,z\right)
\]

% The symbolic execution described above can precompute many of these solutions in advance.
The testing system in the last section could direct the computation of these solutions in advance.
In some cases it is possible to find and report a contradiction. 

Experiments along these lines have been limited to ground data types, and fix an arbitrary solution for every type problem.
Ground data types do not need to worry about the path equalities since all the constructors will be concrete.

Non ground data can be very hard to work with when functions, function types or universes are considered.
For instance,

\[
?:\sum f:\Nat{}\rightarrow\Nat{}.\mathtt{Id}\left(f,\lambda x.x+1\right)\&\mathtt{Id}\left(f,\lambda x.1+x\right)
\]

It is tempting to make the $?$ operator sensitive to more than just the type.
For instance,

\begin{lstlisting}
n : Nat;
n = ?;

pr : Id Nat n 1;
pr = refl Nat 1;
\end{lstlisting}

Will likely give the warning message ``$\mathtt{n\ =?=\ 1\ in\ Id\ Nat\ \uline{n}\ 1}$ ''.
It will then likely give the runtime error ``$\mathtt{0=!=1'}$'.
Since the only information to solve $\mathtt{?}$ is the type \Nat{} and an arbitrary term of type \Nat{} will probably be solved with 0.
Most users would expect the context to be considered and $n$ to be solved with $1$.

However constraints assigned in this manner can be extremely non-local.
For instance,

\begin{lstlisting}[basicstyle={\ttfamily}]
n : Nat;
n = ?;

...

pr : Id Nat n 1;
pr = refl Nat 1;

...

pr2 : Id Nat n 2;
pr2 = refl Nat 2;
\end{lstlisting}

And things become even more complicated when solving is interleaved with computation.
For instance,

\begin{lstlisting}[basicstyle={\ttfamily}]
n : Nat;
n = ?;

prf : Nat -> Nat ;
prf x = (\ _ => x) (refl Nat x : Id Nat n x);
\end{lstlisting}


\subsection{Prior Work}

Proof search is often used for static term generation in dependently typed languages (for instance Coq tactics).
A first order theorem prover is attached to Agda in \cite{norell2007towards}.
However it is rare to make those features available at runtime. 

Logic programing languages such as Prolog\footnote{\url{https://www.swi-prolog.org}}, Datalog\footnote{\url{https://docs.racket-lang.org/datalog}}, and miniKanren\footnote{\url{http://minikanren.org}} use ``proof search'' as their primary method of computation.
Dependent data types can be seen as a kind of logical programming predicate\todo{cite the Curry lang, bird lang?}.
The Twelf project\footnote{\url{http://twelf.org/wiki/Main\_Page}} makes use of runtime proof search and has some support for dependent types, but the underlying theory cannot be considered \fullSp{}.
% The Caledon Language\footnote{https://github.com/mmirman/caledon} attemped to merge logic programing with dependent types for metaprogramming.
The Curry Language\footnote{\url{https://curry.pages.ps.informatik.uni-kiel.de/curry-lang.org/}} performs logic programming in a Haskell-like language.
Gradual dependent type research is working towards a similar goal \cite{10.1145/3341692}, but \cite{10.1145/3495528} has a good explanation of why extending graduality to dependent indexed types is difficult.

\todo{probly better via an effect sys}
% though it is unclear how that work handles the non locality of constraints given their local $?$ operator.


\section{Future work}

\todo[inline]{Convenience: implicit function arguments}

\subsection{Effects}

The last and biggest hurdle to bring dependent types into a mainstream programming language is by providing a reasonable way to handle effects.
Though dependent types and effects have been studied I am not aware of any \fullSp{} system that has implemented those theories.
It is not even completely clear how best to add an effect system into Haskell, the closest ``mainstream'' language to the one studied here.\todo{cites!!}

While trying carefully to avoid effects in this thesis, we still have encountered 2 important effects: blame based error and non-termination.

\subsubsection{Errors}

The current system implements blame based runtime errors and static warnings in a unique way.
There is no control flow for errors built into the reduction or \cbv{} relations, and there is no way to handle an error within the program.
Every potential error is linked to a static warning.
There are a few features that would be good to experiment with.

Ideally we could allow users to provide proofs of equality to remove warnings by having them define and annotate an appropriate identity type.
This would allow the language to act more like an Extensional Type Theory.
Programmers could justify these proofs as a way to remove runtime checks and make code (and testing) faster.
Just as with \ac{ETT} many desirable properties such as function extensionality would still not be provable.
We have pushed this to future work since there are already many explored strategies for dealing with equality proofs in an Intentional Type Theory that are suitable for avoiding warnings in the current implementation.

Currently blame based errors aren't handled\footnote{Or caught.}.
Programmers may want to use the information from a bad cast to build the final output, it might even be possible to capture a well typed term that witnesses the inequality.
For instance,
\todo{fix example}
\begin{lstlisting}
f : Vec String 1 -> String;
f x = case x {
  Cons _ a _ _ => a
}

h : (x : Nat) -> String;
h x =
  handle{
    f (rep String "hi" x)
  } pr : x != 1 => "whoops" ;
\end{lstlisting}

Though additional research would be needed for exactly the form the contradictions should take if they are made available to the handler.

Handling effects in a dependent type system is subtle\footnote{
  Everything about dependent types is subtle.
} since the handling construct can observe differences that should not otherwise be apparent.
This is most clearly seen in the generalization of Herbelin's paradox\todo{independent cite} presented in \cite{10.1145/3371126}.
The problem is that the value of a \Bool{} term may depend on effects that cause logical unsoundness (or worse).
The paradox can be presented in our system with an additional handling construct,

\begin{lstlisting}
h : (u : Unit) -> Bool;
h u =
  handle{
    case u {
      | tt => true
    }
  } _ => false ;

hIsTrue : (u : Unit) -> Id Bool true (h u);
hIsTrue u =
  case u <u -> Id Bool true (h u)>{
    | tt => refl Bool true
  };

hIsTrue !! : Id Bool true false
\end{lstlisting}

Interestingly this term is not as bad as the paradox would be in other settings.
A warning is given so we would not expect logical soundness.
If evaluated in \whnf{} the term will produce blame witnessing the static warning given.

\todo{doe ther errors kind of work like those Error TT papers?} 

\subsubsection{Non-termination}

Non-termination is allowed, but it would be better to have it work in the same framework as equational warnings, namely warn when non-termination is possible, and try to find slow running code via symbolic execution (runtime errors for non termination are famously not possible).
Then we could say without caveat ``programs without warnings are proofs''.
It might be possible for users to supply their own external proofs of termination\cite{casinghino2014combining}, or termination metrics\todo{Ats, ``Termination casts: A flexible approach to termination with general recursion''}.

\subsubsection{Other effects}

One of the difficulties of an effect system for dependent types is expressing the definitional equalities of the effect modality.
Is $\mathtt{print"hello";print"world"}$ $\equiv$ $\mathtt{print"helloworld"}$ at type $\mathtt{IO\ Unit}$?
By delaying equality checks until runtime these issues can be avoided until the research space is better explored.
Effects risk making computation mathematically inelegant.
In this thesis we avoided this inelegance for an error effect with a blame system.
Something analogous could perhaps be applied to more interesting effect systems.

Both the symbolic execution and search above could be considered in terms of an effect in an effect system.
Proof search could be localized though an effect modality, avoiding the non local examples above.

\section{User studies}
 
The main proposition of this work is that it will make dependent types easier to learn and use.
This should be demonstrated empirically with user studies.
 
\section{Semantics}
 
The semantics explored in this thesis have been operational, this has led to serviceable, but cumbersome, proofs.
Ideally less syntactic semantics of a typed language in this style should be explored.
 
For instance, the entire system is designed with an unformalized notion of observational equivalence\todo{foot also called...} in mind.
While there has been some exploration into untyped observational equivalence for dependent types in \cite{sjoberg2015dependently,jia2010dependent}, it uses untyped contextual equivalence, which is a weak relation\todo{
  Ian A. Mason and Carolyn L. Talcott. Equivalence in functional languages
  with effects. Journal of Functional Programming, 1(3):287--327, 1991}.
A version of typed operational equivalence is considered in \cite{VAKAR2018401} though they consider definitional distinctions observable. % (is does notseem reasonable to consider
There is a difficult circularity when trying to define typed observational equivalence in a dependently typed setting.
A good of dependently typed observational equivalence would be an interesting and helpful direction for further study.

\todo[inline]{cite OTT if only as a point of contrast}